%%%% From amt:/lisp/cake/doc

% Include these definitions in a LaTeX file to document Lisp
%
% General form for Lisp documentation, a la Steele.
% Arg 1: Name
% Arg 2: List of argument names and lambda-list keywords
% Arg 3: Type (e.g., Function, Macro)
% Produces roughly:
% {\tt name} {\em arg1 arg2 ...} \hspace*{\fill}[{\em type}]
\makeatletter
\newcommand{\lispdoc}[3]{\par\bigskip
  \tripartite{{\tt #1}}{{\em\lowercase{#2}}}{[{\em #3\/}]}\index{{#1}{#2}{#3}}\par
  \nopagebreak\smallskip
  \global\@nobreaktrue
  \everypar{\if@nobreak\global\@nobreakfalse{\setbox0=\lastbox}\fi\everypar{}}}
\makeatother
\newcommand{\indexentry}[1]{\indexdoc#1}
\newcommand{\indexdoc}[4]{%
  \tripartite{\makebox[.35in][l]{#4}{\tt #1}}{{\em\lowercase{#2}}}{[{\em #3\/}]}\par}
%
\newlength{\namespace}
\newlength{\typespace}
\newlength{\argspace}
\newcommand{\tripartite}[3]{{\hyphenpenalty=10000\exhyphenpenalty=10000
  \settowidth{\namespace}{#1}
  \settowidth{\typespace}{#3}
  \setlength{\argspace}{\textwidth}
  \addtolength{\argspace}{-1\namespace}
  \addtolength{\argspace}{-1\typespace}
  \addtolength{\argspace}{-1\tabcolsep}
  \noindent\begin{tabular*}{\textwidth}[t]{@{}l@{~}p{\argspace}@{\extracolsep{\fill}}r@{}}
  #1\ & 
  {\begin{minipage}[t]{\argspace}\raggedright\noindent\vphantom{p}#2\vphantom{p}\end{minipage}}
  & #3 \\
  \end{tabular*}}}
%
% Specializations of lispdoc
% For Functions:
\newcommand{\fndoc}[2]{\lispdoc{#1}{#2}{Function}}
\newcommand{\sfndoc}[2]{\fndoc{#1}{#2}\marginpar{Symbolics Only}}
% For Inline Functions
\newcommand{\indoc}[2]{\lispdoc{#1}{#2}{Inline}}
\newcommand{\sindoc}{\marginpar{*}\indoc}
% For New Inline Functions (Previously Macros)
\newcommand{\newindoc}[2]{\lispdoc{#1}{#2}{Inline}\draftnote{Was a macro}}
\newcommand{\snewindoc}{\marginpar{*}\newindoc}
% For Keywords
\newcommand{\keydoc}[2]{\lispdoc{#1}{#2}{Keyword}}
\newcommand{\skeydoc}{\marginpar{*}\keydoc}
% For Macros
\newcommand{\macdoc}[2]{\lispdoc{#1}{#2}{Macro}}
\newcommand{\smacdoc}{\marginpar{*}\macdoc}
% For Variables
\newcommand{\vardoc}[1]{\lispdoc{#1}{\hspace{\fill}}{Variable}}
\newcommand{\svardoc}{\marginpar{*}\vardoc}
% For Constants
\newcommand{\condoc}[1]{\lispdoc{#1}{\hspace{\fill}}{Constant}}
\newcommand{\scondoc}{\marginpar{*}\condoc}
% For Prefixes
\newcommand{\predoc}[1]{\lispdoc{#1}{\hspace{\fill}}{Prefix}}
% For Properties 
\newcommand{\propdoc}[2]{\lispdoc{#1}{#2}{Property}}
% For gluing two def lines together in a row
\newcommand{\doubledoc}{\par\addvspace{-3ex}}
%
% For typesetting the values of optional and keyword arguments
% Puts argname and default value in roman parentheses, separated by nb space.
\newcommand{\optarg}[2]{\mbox{{\rm (}{#1}~{#2\/}{\rm )}}}
% Puts argname in roman square braces
\newcommand{\optmacarg}[1]{\mbox{{\rm [}#1\/{\rm ]}}}
%
% For Kleene closures of arguments (e.g., defstruct on p. 307 of Steele)
% Surrounds arg with {...} and trails with superscript + or *.
\newcommand{\kleeneplus}[1]{\mbox{{\rm \{}#1\/{\rm \}$^{+}$}}}
\newcommand{\kleenestar}[1]{\mbox{{\rm \{}#1\/{\rm \}$^{*}$}}}
%
% Abbreviations for Common Lisp lambda-list keywords
\newcommand{\optional}{{\tt \&optional}\/}
\newcommand{\rest}{{\tt \&rest}\/}
\newcommand{\key}{{\tt \&key}\/}
\newcommand{\aok}{{\tt \&allow-other-keys}\/}
\newcommand{\allow}{{\tt \&allow-other-keys}\/}
\newcommand{\aux}{{\tt \&aux}\/}
\newcommand{\body}{{\tt \&body}\/}
\newcommand{\whole}{{\tt \&whole}\/}
\newcommand{\env}{{\tt \&environment}\/}
\newcommand{\returns}{$\rightarrow$\/}
\newcommand{\plus}{{\tt \&Plus}\/}
\newcommand{\keys}{\tt\ }
