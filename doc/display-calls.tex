%% Figure depicting the calls used to manipulate pane stacks

\begin{figure}
\newlength{\oldunitlength}
\setlength{\oldunitlength}{\unitlength}
\setlength{\unitlength}{0.015in}

\begin{center}
\begin{picture}(400,270)(4,0)

\put(70,260){\oval(80,20)}
\put(70,260){\makebox(0,0){pop-picture}}
\put(76,247){\vector(3,-4){18}}  	%to destroy

\put(330,260){\oval(80,20)}
\put(330,260){\makebox(0,0){display}}
\put(330,243){\circle{14}}
\put(330,243){\makebox(0,0){?}}
\put(320,245){\vector(-3,-1){78}}  	%to move-picture
\put(320,237){\vector(-1,-1){64}}  	%to push-picture
\put(292,148){\line(2,5){34}}	  	%to draw-pane
%\put(296,158){\line(2,5){30}}	  	%to draw-pane
\put(292,148){\vector(-3,-1){75}}  	%to draw-pane

\put(100,210){\makebox(0,0){\framebox(80,20){destroy (picture)}}}
\put(108,197){\vector(1,-1){24}}  	%to remove-picture

\put(200,210){\oval(80,20)}
\put(200,210){\makebox(0,0){move-picture}}
\put(189,197){\vector(-2,-1){48}}  	%to remove-picture
\put(203,197){\vector(1,-1){24}}  	%to push-picture

\put(40,160){\oval(80,20)}
\put(40,160){\makebox(0,0){cycle-pane}}
\put(52,147){\vector(3,-1){106}} 	%to draw-pane

\put(135,160){\makebox(0,0){\framebox(80,20){remove-picture}}}
\put(144,147){\vector(4,-3){32}} 	%to draw-pane

\put(230,160){\makebox(0,0){\framebox(80,20){push-picture}}}
\put(219,147){\vector(-3,-4){18}} 	%to draw-pane

\put(360,160){\oval(80,20)}
\put(360,160){\makebox(0,0){refresh}}
\put(348,147){\vector(-3,-1){106}} 	%to draw-pane

\put(200,110){\makebox(0,0){\framebox(80,20){draw-pane}}}
\put(200,97){\vector(0,-1){24}}

\put(360,60){\oval(80,20)}
\put(360,60){\makebox(0,0){drag-picture}}
\put(348,47){\vector(-3,-1){75}} 	%to render

\put(200,60){\makebox(0,0){\framebox(80,20){present}}}
\put(187,47){\vector(-1,-1){24}} %to compute-picture
\put(213,47){\vector(1,-1){24}} %to render

\put(150,10){\makebox(0,0){\framebox(80,20){compute-picture}}}
\put(250,10){\makebox(0,0){\framebox(80,20){render}}}

\end{picture}
\end{center}

\caption{Chart indicating the function call hierarchy of functions
used to manipulate the picture stacks of OBVIUS panes.  Functions in
oval boxes are top-level functions.  Those in rectangular boxes are
internal.  The question mark enclosed in a circle indicates that the
{\protect\tt display} function calls one of the three indicated functions.}
\label{fig:display-calls}

\setlength{\unitlength}{\oldunitlength}
\end{figure}
