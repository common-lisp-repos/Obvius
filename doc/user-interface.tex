%%%%%%%%%%%%%%%%%%%%%%%%%%%%%%%%%%%%%%%%%%%%%%%%%%%%%%%%%%%%%%%%%%%%%%%%%%%%%%%%%%

\section{User Interface}
\label{sec:userface}

This section describes the user interface provided for OBVIUS.

\subsection{The Mouse}
\label{sec:mouse}

Mouse clicks within OBVIUS panes are captured and interpreted by
OBVIUS.  WARNING: Some X11 or OpenWindows window managers override
the mouse bindings in obvius.  In particular, twm allows the user to
bind mouse-clicks in their .twmrc file.  If your mouse is not
responding -- check your window manager manual pages.

All mouse-handling code is in the file {\bf x-mouse.lisp}.  The
bindings can be altered within this file.  The mouse can be used to
cycle through the stacks of pictures, zoom, pan, and select images,
among other things.

The following is a list of the current mouse bindings, organized
according to the ``bucky'' keys (Shift, Control, and Meta).  All
actions are performed on the pane containing the mouse at the time of
the button click.

\begin{tabular}{|l|ccc|}
\hline
buckies & left button       & middle button  & right  button       \\
\hline
Raw	& select-viewable   & refresh        & select-pane/position-message    \\
Control & previous-picture  & next-picture   & move-to-here        \\
Shift 	& zoom-in           & zoom-out       & numerical-magnifier \\
Meta	& describe-viewable & picture-parameters & picture-menu    \\
C-Sh	& pop-picture       & center         & drag                \\
C-Sh-M	& destroy-viewable  & unbound        & hardcopy             \\
\hline
\end{tabular}

The following are picture-specific mouse bindings for gray pictures:

\begin{tabular}{|l|ccc|}
\hline
buckies & left button       & middle button  & right  button       \\
\hline
M-C	& boost-contrast  & reduce-contrast  & histogram    \\
M-Sh	& x-slice           & y-slice        & crop           \\
\hline
\end{tabular}

The following are picture-specific mouse bindings for flipbooks:

\begin{tabular}{|l|ccc|}
\hline
buckies & left button       & middle button  & right  button       \\
\hline
M-C	& previous-frame  & next-frame  & show-movie    \\
\hline
\end{tabular}


\begin{description}

\item [Raw] (no bucky keys):  
\begin{itemize}
\item Left: {\bf Select viewable}.  Selecting a viewable that is bound
to a symbol prints that symbol into an emacs buffer at current prompt
position (the ``point''), where it is then ready to be evaluated.  
Of course, if the viewables have symbols bound to them, you can simply
type the symbol name yourself.

If the selected viewable has a nil name or a string name, then OBVIUS
prints a more cryptic functional expression like (obvius::pid 1) that
will also evaluate to the viewable.  This allows easy, interactive
manipulation of viewable objects.  Examples are given in the OBVIUS
tutorial.

You can also access unnamed viewables via the variables *, **, and ***
that are bound to the last three values returned by lisp. Also, the
variables @, @@, and @@@ are bound to the last three viewables
selected with a raw left mouse-click.

This mouse-click also selects the pane as the current pane.

\item Center: {\bf Refresh}.  The top picture on the picture stack is
recomputed (if necessary) and then redisplayed.  Calls (refresh pane).

\item Right: {\bf Select pane/Position Message}.  The pane containing
the mouse is set to be the current pane and some information that
depends on the picture type is printed in the Emacs minibuffer.  For
example, clicking right on a grey picture of an image gives the
(x,y)-position and the floating-point intensity value of the {\em
image} (not the picture).  This information is updated dynamically if
you drag the mouse.  Note that many other mouse clicks will also
select the pane.
\end{itemize}

\item [Control]:
\begin{itemize}
\item  Left: {\bf Previous picture}.  The stack of pictures in the
pane is cycled.  The pane is selected.  Calls (cycle-stack pane nil).
\item  Middle: {\bf Next picture}. The stack of pictures in the pane
is cycled in the opposite direction as Control-Left.  This pane is
selected.  Calls (cycle-stack pane t).
\item  Right: {\bf Move to here}.  The picture on top of the current
pane is moved to the top of this pane.  Selects this pane.
\end{itemize}

\item [Shift]:
\begin{itemize}
\item  Left: {\bf Zoom in}.  Pixel-replication is used to produce an
enlarged version of the image.  The zoomed picture can be dragged
using C-Sh-middle.
\item Middle: {\bf Zoom out}.  This reduces the magnification of a
picture.  This will subsample below the resolution of the original
image.
\item  Right: Numerical magnifying glass  [Currently unimplemented].
\end{itemize}

\item [Meta]:
\begin{itemize} 
\item Left: {\bf Describe viewable}.  This prints out a description of the
viewable in the lisp listener. 
\item  Middle: {\bf Edit picture parameters}.  Pops up a dialog to view and edit
the parameters of the picture.
\item  Right: {\bf Picture menu}.  Pops up a menu to select (i.e., cycle to
the top) any picture.  Faster than using next-picture or
previous-picture if the pane has a large stack.
\end{itemize}

\item [Control-shift]:
\begin{itemize}
\item Left: {\bf Pop}.  This mouse click calls (pop-picture pane),
which is described in section \ref{sec:pictures}.  The picture on the
top of the picture stack is destroyed. The viewable is not explicitly
destroyed, although {\em it will no longer be accessible if it did not
have a symbol name slot}.
\item  Middle: {\bf Center}.  Recenters the picture in the pane if it
has been dragged.
\item  Right: {\bf Drag}.  Allows interactive re-positioning of
pictures within the pane.
\end{itemize}

\item [Control-meta-shift]:
\begin{itemize}
\item Left: {\bf Destroy}.  This mouse click calls the method destroy
on the viewable.  All static memory used by the viewable its pictures
is relinquished, the pictures are removed from the picture stacks, and if
the viewable has a symbol name, it is unbound.  If the viewable has
inferiors that are otherwise unaccessable, then the inferiors are also
destroyed.  If the viewable has superiors, then this causes a
continuable error that allows the user to destroy both the viewable
its superior.
\item Right: {\bf Hardcopy}.  Prints the picture to a postscript
printer.  The global parameter *default-printer* specifies which
printer it goes to.
\end{itemize}

\item [Meta-Shift]:  These bindings are {\em picture specific}!

\noindent
For gray pictures:
\begin{itemize}
\item  Left: {\bf X-slice}. This produces a new viewable, a
one-dimensional ``slice'' of a two-dimensional image, and pushes a
one-d-graph picture of it onto the stack of the current pane.  The
y-coordinate of the slice is selected by the position of the
mouse-click in the original picture.
\item  Middle: {\bf Y-slice}. Orthogonal to x-slice.
\item Right: {\bf Interactive Crop}. Hold down the mouse and drag to
define a square subregion of a picture.  Releasing mouse creates a new
viewable, cropped from the original.
\end{itemize}

\item [Meta-Control]:  These bindings are {\em picture specific}!

\noindent
For gray pictures:
\begin{itemize}
\item Left: {\bf Boost-contrast}.
\item Middle: {\bf Reduce-Contrast}.
\item  Right: {\bf Histogram}.  This computes a histogram viewable, with
a bin size which is controlled by the global parameter {\bf
*default-number-of-bins*}, and pushes a one-d-graph picture of it onto
the stack of the current pane.  This calls the function make-histogram
which is described in section \ref{sec:images}.
\end{itemize}

\noindent
For flipbook pictures:
\begin{itemize}
\item Left: {\bf Previous Frame}.  Cycle through flip-book picture of a sequence.
\item Middle: {\bf Next Frame}. Cycle through flip-book picture of a sequence.
\item Right: {\bf Display Sequence}.  Display all frames of a
sequence.  See Section \ref{sec:flipbook} for a description of the
parameters that control delay between frames, etc.  Any mouse button
will stop the display.
\end{itemize}

\end{description}


\subsection{The Control Panel}
\index{control panel}
The control panel displays status information that tells you what the
OBVIUS listener (top-level process) is doing.  It also displays
on-line, passive (i.e., you don't have to do anything to get it),
mouse documentation.  This means that when you move the mouse into an
OBVIUS pane, the relevant mouse bindings are echoed in the message
line; if you press a bucky key (shift, meta, or control), the
documentation is updated accordingly.

To create a control panel, use the command 
\lfun{make-control-panel}{\&key left right top bottom}
You may wish to put this in your {\tt .obvius} initialization file.
To destroy the control panel, call \ldef{destroy-control-panel}.

The control panel also has several menu buttons.  Holding the right
mouse button down on a menu button reveals the pull-down menu.  Some
of the menu choices bring up dialog boxes.  The dialog box is created
by releasing the mouse button on that choice.  Some of the menu
choices are set up to reveal sub-menus.  The sub-menus are revealed by
moving the mouse to the right while continuing to hold down the right
button.

Within a dialog box, click left to move the cursor around and type as
usual.  Many of the dialog boxes are interfaces to function calls.
The functions are executed by clicking either the {\tt Execute} button
or the {\tt OK} button.  When this is done OBVIUS prints an expression
in the *lisp* buffer and evaluates it.  Clicking left on an OBVIUS
pane selects a viewable for the function argument, by inserting it at
the cursor position in the dialog box.

At any given moment, there can only be a single function dialog box.
If you try to bring up another one, OBVIUS will destroy the first one.
You can add functions to the menus by adding them to particular global
variables in the file {\tt x-control-panel.lisp}.  FOr example, you
could add a new function {\tt foo} to the  ``Misc'' menu, but adding
the symbol {\tt 'foo} to the value of the variable
\lsym{obvius::*obvius-misc-functions*}. Changes to these
variables are {\em dynamically} incoporated into the menus.

\index{menus}
The menus on the control panel are:
\begin{description}

\item [Parameters]: This menu allows you to adjust parameters of the
current picture, the screen, the default slots of pictures and
viewables, and global parameters.  The items on the menu are described
in more detail below.

\item [Statistics]: Menu of operations for computing viewable
statistics (e.g., mean or variance of an image or image-sequence).
Each menu choice brings up a dialog box for entering the function
arguments.

\item [Arith Ops]: Menu of statistics operations (e.g., mean,
variance, maximum, minimum, etc.).

\item [Filter Ops]: Menu of filtering operations (e.g., apply-filter,
fft, etc.).

\item [Geom Ops]: Menu of geometric operations (e.g., crop, paste,
slice, flip-x, etc.).

\item [Compare]: Menu of comparison operations (e.g., greater-than,
less-than, mean-square-error, etc.).

\item [Synth]: Menu of synthetic image functions (e.g., make-impulse,
make-sine-grating, make-zone-plate, etc.).

\item [Matrix]: Menu of matrix operations (e.g., matrix-multiply,
singular-value-decomposition, etc.).

\item [Misc]: Menu of miscillaneous operations (e.g., print-values,
hardcopy, etc.), and a submenu of optional modules that may be loaded.
\end{description}

\subsection{Current Picture, Picture Defaults and Viewable Defaults menus}
\begin{description}
\item [Current Picture]: Brings up a dialog box to change parameters
of the current picture (picture on top of the current pane).  Buttons
at the bottom allow user to Apply the changes to the picture (the
display will be updated), revert to the current slot values, revert to
the default slot values, or quit the dialog.

\item [Viewables]: Reveals a sub-menu that allows you to change
the default parameters of a particular viewable class.  The parameters
for each viewable class are described in section \ref{sec:viewables}.
The default parameters are used to initialize a new viewable when it
is first created.  Changing the default parameter values with this
menu has no effect on viewables that have already been created.

\item [Pictures]: Reveals a sub-menu that allows you to change
default parameters of a particular picture class.  The parameters for
each picture class are described in section \ref{sec:viewables}.  The
default parameters are used to initialize a new picture when it is
first created.  Changing the default parameter values with this menu
has no effect on pictures that have already been created.  The
parameters of the current picture may be changed with the {\tt Current
Picture Parameters} menu (mentioned above).
\end{description}


\subsection{Current Screen Parameters}

The user can change the following screen parameters from the {\tt
Parameters} menu on the control panel (see above):
\begin{description}
\item [Gray-shades]: Number of grays to allocate in the color map.  If
OBVIUS fails to allocate the requested number, it tries again with a
smaller number.
\item [Gray-gamma]: Default gamma value in the allocated gray-shades.
\item [Gray-dither]: If gray-shades is less than gray-dither, then
gray pictures will automatically be dithered.
\item [Foreground]: Foreground color used for graphics (e.g., it is
the color used for each new graph pictures).
\item [Background]: Background color of each pane.
\end{description}

\subsection{Global Parameters}

There are a number of global parameters that affect how OBVIUS
operates.  The user can change the values of these globals from the
{\tt Parameters} menu on the control panel (see above).  The function
of each parameter is describe below:

\begin{description}
\item \lsym{*default-printer*}: Name of a PostScript printer.

\item \lsym{*max-print-vals*}: Maximum number of values which the
print-values method prints in a line of the image.  See print-values,
Section \ref{sec:images}.

\item \lsym{*y-print-range*} and \lsym{*x-print-range*}: Default range of
values in an image to be printed by the print-values function.  Should
be a list of two numbers between 0 and 1.  See print-values, Section
\ref{sec:images}.

\item \lsym{*auto-bind-loaded-images*}: If t, automatically binds images
loaded from files to symbols matching their filenames.  See
load-image, Section \ref{sec:datfile}.

\item \lsym{*auto-update-pictures*}: If non-nil, out-of-date pictures will
be updated when they are revealed.  If nil (the default), user must
explicitly refresh the pane to update an out-of-date picture.

\item \lsym{*auto-destroy-orphans*}: Slightly risky parameter causes
viewables to be destroyed when 1) their names are rebound if they have no
pictures or superiors, 2) their only picture is popped off of the pane
and they do not have a symbol name or superiors, 3) their only
superior is destroyed, and they do not have a symbol name or any
pictures.  These three cases are the situations in which OBVIUS can
prevent the creation of orphans.  Note that this does not prevent the
user from creating their own orphans (e.g., make a new image with no
name and don't display it).  See also section~\ref{sec:memory} on
memory management.

\item \lsym{*auto-display-viewables*}: If non-nil, viewables returned to the
top-level listener are automatically displayed on the currently
selected pane.  This is the normal mode of operation.

\item \lsym{*heap-growth-rate*}: Number of elements to expand the heap
when out of memory.  See Section \ref{sec:memory}.

\item \lsym{*auto-expand-heap*}: When non-nil, expand heap automatically
when out of memory.  See Section \ref{sec:memory}.

\item \lsym{*div-by-zero-result*}: Default result used by /-0 and div
functions when the divisor is zero.  See Section \ref{sec:operations}.
\end{description}


\subsection{Running OBVIUS on a remote host}

The networking capabilities of X Windows (or OpenWindows) make it
possible to open OBVIUS windows on a machine other than the one on
which the OBVIUS process resides. Like other X clients, OBVIUS
examines the value of the shell environment variable, DISPLAY, in
order to determine which screen to open its windows on. If your X
server finds more than one screen, you will be able to open obvius
panes on any of them using the {\tt :screen} keyword argument to {\bf
new-pane}. This will typically be the case if you have two monitors
attached to one workstation, or if your framebuffer has a separate
overlay plane.  The list of screens that OBVIUS knows about is kept in
the (unexported) global variable {\tt obvius::*screen-list*}.

Because X is a server-based window system, each bitblt requires that
data be transferred between the OBVIUS (lisp) process, and the X
server process. Displaying a picture for the first time will be
delayed by the time it takes for this transfer, but the remote
(server) copy of the data is retained as an X pixmap, and subsequent
redisplay, via refresh or cycling a pane, is free of this penalty.


\subsection{Screens and Windows}

OBVIUS maintains a list of {\bf screens} on which its panes reside.
This mechanism makes it possible to have OBVIUS panes on several
screens which may have different capabilities at the same time, and
leaves open the possibility of using different window systems on the
various screens. System-independent code which defines windows and
screens is in pane.lisp, while system-dependent window code for X is
in the file {\bf x-window.lisp}.

OBVIUS panes are described in section \ref{sec:pictures}.  These panes
are built on top of LispView window objects and thus inherit all of
the attributes of those windows.  Accordingly, panes can be moved,
resized, refreshed, hidden, closed, or opened in the same manner as
any other X11 window.  NOTE: if you are using X11 (as opposed to
OpenWindows), you should not destroy an OBVIUS pane from the window
manager as this will destroy the client process (i.e., Lisp)!!  Use
the OBVIUS destroy method to destroy a pane (this can be invoked from
the {\tt misc} menu on the control panel).

In LispView, mouse events are handled by a separate Lisp process.
Multiprocessing poses particular difficulties for OBVIUS because some
of the mouse clicks may result in lots of computation, or they may
need to destructively modify data structures that are being
simultaneously accessed by the main process.  OBVIUS has two different
solutions for handling these problems:
\begin{enumerate}
\item If the mouse click involves only modification of pictures,
the computation {\em is} performed by the mouse process.  To avoid
data structure collisions (e.g., the mouse process and main process
both trying to display something in the same pane), OBVIUS ``locks''
the pane being modified, using the macro
\lsym{with-locked-pane} to give exclusive access to one process (see
{\bf pane.lisp} for examples of use).

\item If the mouse click requires substantial computation or
creation of viewables, OBVIUS does not let the mouse process handle
the event directly.  Rather, OBVIUS hands off compute-intensive
evaluation to the main lisp process via an evaluation queue.  The
main lisp process polls the eval queue each time it goes through its
read-evel-print loop.  The upshot of this is that some mouse events
are handled immediately (e.g., cycle stack) whereas others are not
(e.g., make-slice).  If the event is going to be handled in a delayed
fashion, OBVIUS will notify you by putting up the status message
``Enqueued form: <form>''.
\end{enumerate}


\subsection{Emacs and OBVIUS}

The user interface of OBVIUS is greatly enhanced when it is run as a
sub-process of the Emacs editor.  The following capabilities are available:
\begin{itemize}
\item Quick entry of viewable arguments to commands [left
mouse click (select-viewable)].
\item Passive mouse documentation and system messages in the Emacs
minibuffer (the bottom line of the Emacs window) (Note: this is only
true in the absence of the contol panel).
\item Interactive Lisp shell buffer with command history mechanism.
\item Evaluation of lisp expressions from Emacs file buffers [{\tt C-c
e}], and In-package compilation of lisp functions from Emacs file
buffers [{\tt C-c c}].
\item Function argument lists [{\tt C-c C-a}], documentation strings [{\tt
C-c C-d}], symbol apropos [{\tt C-c C-h}], symbol description [{\tt
C-c C-i}], macro expansion [{\tt C-c C-m}], and source file editing
[{\tt C-c .}].
\item Filename completion for load-image requests [{\tt C-c l}].
\item Complete record of interaction with OBVIUS (in the form of an
Emacs buffer).
\end{itemize}

The extensions to Emacs that provide this functionality are contained
in the \abox{emacs-source} files.  For more information, see the file
\abox{emacs-source}/cl-shell.doc.

\subsection{Help}
\label{sec:help}

The following helpful functions are provided:
\begin{description}
\item ({\bf obvius-parameters} \&key doc) \\
Prints out a list of the system
parameters and their current values.  If the optional {\em doc}
argument is non-nil, the documentation of each parameter will also be
printed. Documentation of each parameter is also
available via a call to the Common Lisp documentation function:
(documentation '\abox{symbol} 'variable).
\item ({\bf obvius-commands}) \\
Prints out a list of all OBVIUS functions which are exported.
\item ({\bf obvius-variables}) \\ 
Currently not implemented.
\end{description}
In addition, the source code for any OBVIUS function may be found in
using the ``C-c.'' command.  Also, the Common Lisp function {\tt
(apropos <piece-of-symbol-or-string>)} will search for all symbols
containing the given symbol or substring.
