%%% Keep this file consistent with the INSTALL file in the top directory.

\section{Installation}	
\label{sec:install}

This section gives step-by-step instructions for installing OBVIUS on:

\begin{itemize}
\item{{\bf Sun workstations}}

With Lucid (Sun) Common Lisp 4.1, with CLOS and LispView.  Currently,
OBVIUS runs comfortably on Suns with 8 bit frame buffers (color or
gray-scale monitors), or monochrome (1bit) frame buffers.

\item{{\bf SGI workstations}}

With Lucid (SGI) Common Lisp 4.1 (beta) with CLOS.  In order to run
OBVIUS, you will need at least this version of Common Lisp from Lucid;
older versions like version 3.0.2+ will not work.  The current version
of OBVIUS for this platform was developed on both the Indigo and the
Personal Iris.  It supports both 8 and 24 bit frame buffers.

\end{itemize}


\subsection{Requirements}

You need a fairly large swap space (virtual memory, paging area) in
order to use OBVIUS to its fullest potential (we generally allot 50
Megabytes or more).  20 Megabytes is a bare minimum to get OBVIUS to
load; this amount of swap space will not allow you to do very much in
the way of image processing as the image storage space restrictions
will be overly stringent.

If you are accustomed to working with Lisp, these requirements will
not be very remarkable.  If they do seem somewhat voracious, remember
that if you do image processing with individual binary programs and
shell scripts, as many people do, you will end up storing all of your
results as files on the disk and may soon frequently require easily
this much storage space for your intermediate, temporary results.

The OBVIUS directory, which contains Lisp and C source and binary
files and documentation, occupies around 4Mb of disk space.  The
complete lisp image (including Lucid 4.0, CLOS, OBVIUS, possibly
LispView) will occupy at least 15Mb, so you should have that much disk
space to devote to OBVIUS.

In order to make use of the user interface features, you should also
have GNU emacs. You can obtain GNU emacs via anonymous ftp from
prep.ai.mit.edu.

OBVIUS 3.0 runs under the X window system.  X was developed at MIT and
there are public domain servers available for a number of different
architectures.  You can ftp a copy of X from expo.lcs.mit.edu.  Or,
you can get Openwindows (available from Sun), that includes both the X
and News window systems.

You also need LispView, an object-oriented lisp-interface to X, if you
are running OBVIUS on a Sun.  LispView is available from Sun
Microsystems with Sun (Lucid) Common Lisp, version 4.0.  LispView is
not required for the SGI platform as OBVIUS uses the native SGI
graphics library (GL).  The lisp-interface to the GL is included in
the OBVIUS distribution.


\subsection{Installation Instructions}

The symbol \abox{obv} will be used throughout this document to
indicate the pathname of the primary OBVIUS directory.

\begin{enumerate}

\item 
Decide where the OBVIUS source directory tree (approx. 3-5 Mbytes)
will live in your filesystem.  We will refer to this directory as
\abox{obv}.  On our machines, this is /usr/local/obvius. Also decide
where the OBVIUS executable binary file (the lisp image, approximately
15 Mbytes) will live. We'll refer to this directory as
\abox{bin-path}.  On our system, this is /usr/local/bin.  This
directory must be in each OBVIUS user's PATH environment variable.

\item  
Extract the OBVIUS directory tree from the tape or from the
compressed tarfile (you have probably already done this!):
\begin{itemize}
\item	{\tt mkdir \abox{obv}}
\item	{\tt cd \abox{obv}/..}
\item	If you are using a tape, insert the tape and wait for the
whirring to stop.  If you are starting from a compressed tar file
(obvius.tar.Z), put the tarfile in the current directory and
uncompress it using {\tt uncompress obvius.tar.Z}.
\item	For a tape, execute {\tt tar xvf /dev/nrst8}.  From a
tarfile, execute {\tt tar xvf obvius.tar}.
\end{itemize}
The distribution contains several source subdirectories (lisp-source,
c-source and emacs-source), a documentation subdirectory (doc), and a
set of top-level files (README, INSTALL, lucid-defsys.lisp, etc).  You
may move the individual source subdirectories anywhere in your file
system.  Also, decide where you want the Lisp and C binaries to live
(see next instruction).

\item	
In the directory \abox{obv}/c-source, there are two makefiles:
Makefile-sun4 (for sun 4's) and Makefile-sgi (for SGI's).  Edit these
makefiles to refer to the appropriate binary pathnames.  The default
is for the binaries to go in subdirectories sun4-bin and sgi-bin of
the \abox{obv} directory.  Also edit the {\tt INCLUDES} line in these
makefiles to refer to the appropriate pathnames for the Xlib.h and X.h
include files (this is required for the Sun platform only).  Create
the necessary binary directories using mkdir.

\item
In file \abox{obv}/site-paths.lisp, edit the definitions for
*obvius-directory-path*, *lisp-source-path*, *c-source-path*,
*binary-path*.  Usually these will be \abox{obv}/,
\abox{obv}/lisp-source/, \abox{obv}/c-source/, and the appropriate
binary subdirectory for the given machine (must be consistent with the
directory chosen in the previous item).  Also, edit the defvar for
*obvius-image-path*.

\item  
In file \abox{obv}/lucid-site-paths.lisp, edit the defvars for
*temp-ps-directory* (usually /tmp/), *default-printer* (should be the
name of a postscript printer), and *print-command-string* (this will
probably need to be changed depending on your printing configuration).

\item	
Edit the file \abox{obv}/lucid-site-init.lisp.  This file should load
any optional modules that you want to be part of the basic OBVIUS lisp
image at your site.  The list of modules appears in the
lucid-defsys.lisp file.

\item 
Start up a lisp world that includes CLOS.  Use ``lisp -n'' to start
lisp without loading your personal lisp-init.lisp file.

\item
If you are running on a Sun platform, load LispView by typing: {\tt
(load "<lispview>/lispview.sbin")}.  You can get this compiled
lispview file (lispview.sbin) via anonymous ftp from
white.stanford.edu.

If you are compiling LispView from scratch, you must add the following
lines to the file "build.lisp":
\begin{verbatim}
   ...
   (in-package "USER")

   ;; add these lines to get lispview to complie correctly
   (proclaim '(optimize (compilation-speed 0) (speed 3) (safety 1)))
   (proclaim '(notinline make-instance))
   ...
\end{verbatim}

\item 
To compile and load OBVIUS, type {\tt (load
"\abox{obv}/lucid-defsys")} and then type {\tt (obvius::run-obvius)}.
This takes a while.  Note that if you do not have a lot of swap space,
lisp may die during the compilation.  Just quit the lisp and start
again.  It will load the files that have already been compiled and
then compile the rest of them.

\item To save an OBVIUS lisp image:
\begin{itemize}
\item 
You should have already compiled all of the OBVIUS source (see
previous instruction), and tested to see that it works (take a look at
<obv>/obvius-tutorial.lisp and try evaluating some of the expressions).

\item Start up a lisp world that includes CLOS.

\item Load LispView if you are running on a Sun platform by typing
{\tt (load "\abox{lispview}/lispview.sbin")}.

\item 
Type {\tt (load "\abox{obv}/lucid-defsys")} and then
type {\tt (obvius::make-obvius)}.  WARNING: We have found that
attempting a ``disksave'' over an NFS connection sometimes fails.  If
your OBVIUS image does not work correctly, then try making it from
your file server.
\end{itemize}

\item
OBVIUS will run without Gnu Emacs, but there are many advantages to
running with it.  If you will be using emacs, you will need to install
the emacs extensions which are included in the tarfile in the
emacs-source subdirectory.  Copy the directory of emacs-extensions
whereever you like in your filesystem.  If you prefer, you can put
them in with your standard distribution.  We will refer to this
directory as \abox{emacs-extensions}.  On our machines, this is {\tt
"<obv>/emacs-source"}.  Add the following lines to your .emacs file:
\begin{verbatim}
(setq load-path (cons \abox{emacs-extensions} load-path)) \\
(setq *obvius-program* \abox{obvius-image}) \\
(autoload 'run-obvius  "cl-obvius" "" t) \\
\end{verbatim}
The symbol {\tt \abox{obvius-image}} should be the pathname for the lisp
image containing OBVIUS.  To start OBVIUS from a saved OBVIUS lisp
image, run emacs and type {\tt M-x run-obvius}.

If you have not yet made an OBVIUS lisp image, or want to run from the
compiled binary files, you may put the following definition in your
.emacs file:
\begin{verbatim}
(defun run-development-obvius ()
  (interactive)
  (load "cl-obvius")
  (let ((*cl-replacement-prompt*  "OBVIUS-3.0> "))
    (run-cl "<lisp-view>"))
  (setq default-directory "<obv>/lisp-source/") ;VISCI
  (cl-send-string "(load \"<obv>/lucid-defsys\")\n") ;VISCI
  (cl-send-string "(obvius::run-obvius)\n")
  (cl-add-obvius-key-bindings))	;add bindings to cl-shell-mode and lisp-mode
\end{verbatim}
where {\tt <lisp-view>} refers to the pathname of a lisp world
containing LispView and CLOS.  To run OBVIUS, you may type {\tt M-x
run-development-obvius} in Emacs.

\item 
If you want to use the same emacs environment for running Common Lisp,
you may want to put the following lines in your .emacs file as well:
\begin{verbatim}
(setq *cl-program* \abox{lisp-image}) \\
(autoload 'run-cl "cl-shell" "" t) \\
\end{verbatim}
where \abox{lisp-image} refers to the pathname for a lisp-image.  On
our machines, this is in {\tt /usr/local/bin/Lucid-4.0}.  To start up
Common Lisp inside emacs, type {\tt M-x run-cl}.  Documentation for
the cl-shell environment is in the file
\abox{emacs-extensions}/cl-shell.doc.  Further extensions to emacs are
included with the source, and examples of how to use these may be
found in the file {\tt <obv>/emacs-source/example.emacs}.

\end{enumerate}

