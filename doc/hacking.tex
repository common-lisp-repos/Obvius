\section{Hacking}
\label{sec:hacking}

This section is provided for users who want to write their own
extensions to OBVIUS.  As OBVIUS becomes larger and more people work
on it, it becomes increasingly important to observe the existing
programming conventions.  It is generally a good idea to use existing
code (copy it and modify it) as a prototype for whatever you are
doing.  In the following sections, we try to describe and motivate
these conventions.

%%%%%%%%%%%%%%%%%%%%%%%%%%%%%%%%%%%%%%%%%%%%%%%%%%%%%%%%%%%%%%%%%%%%%%%%%%
\subsection{Writing Additional Image Processing Functions}
\label{sec:writing-ops}

Since OBVIUS was originally conceived as an image processing package,
it contains a library of image processing functions in the file {\bf
imops.lisp}.  Here is a simple example:
\begin{tt} \begin{verbatim}
(defmethod add ((val number) (im image) &key ->)
  (with-result ((result ->) im 'add val im)
    (add (data im) val :-> (data result))
    result))
\end{verbatim} \end{tt}
Note the result keyword argument, \lsym{:->}.  All functions that
return viewables should take this keyword argument.  The standard
handling of this argument is described in
section~\ref{sec:operations}.  OBVIUS provides a macro, {\tt
with-result} that makes it easy to provide standard handling of this
result argument.

\mysubsubsec{The with-result macro}
The \lsym{with-result} macro should be used by {\em all} operations
returning viewables.  It provides a simple mechanism for creating
functions with uniform result argument handling.  The syntax is as
follows: 
\begin{verbatim}
(with-result ((res-sym res-arg) model . history)
   . body)
\end{verbatim}
The programmer should think of this macro as being very similar to the
Common Lisp {\tt let*} macro.  The forms of the {\tt body} are
executed within a lexical environment where the symbol {\tt res-sym}
is bound to a ``result'' viewable.  This result viewable is protected
from the garbage collector, and from auto-destruction of orphans
(i.e., it is placed on the \lsym{*protected-viewables*} list).  The
macro returns the value (or multiple-values) returned by the {\tt
body}.  The {\tt res-arg}, {\tt model}, and {\tt history} arguments
are used to construct the result viewable.

The {\tt model} argument must either be a plist of keyword initargs
specifying the type and slots of the result, or an existing viewable
which is used as a model for the result.  If a plist is used, the
first two elements should be the keyword \lsym{:class} and the symbol
name of a viewable class.  The remaining keyword/value pairs should be
those you would pass to {\tt make-instance} in order to create the
viewable.  That is, they are a set of initialization arguments
(initargs) for the object.

The {\tt res-arg} is typically the result argument, \res, as passed
by the user.  If this is an existing viewable, OBVIUS will check that
it is consistent with the specification given by the {\tt model} and
then bind {\tt res-sym} to it.  If it is a name (nil, a symbol, or a
string), then OBVIUS will create a new viewable according to the
specification of the {\tt model} and give it this name.

The {\tt history} argument is typically the symbol name of the function
({\tt 'add} in the example above) followed by the arguments to the
function (except for the result argument itself).  This list is saved
in the history slot of the viewable.  If no {\tt history} argument is
passed, then the history of the result viewable will be set according
to the operations performed in the {\tt body}. 

The macro {\tt with-result} calls the methods \lsym{set-result} and
\lsym{set-not-current}, both of which are described below in section
\ref{sec:new-viewable}.  It also calls
\lsym{set-history} to set the history slot of the resulting image
according to the {\tt history} argument.  Examples of usage can be
found in {\bf imops.lisp}.  Calling {\tt macroexpand-1} on these
examples may help to clarify the use of the macro.

\mysubsubsec{Temporary Viewables}
Often, you will need to generate intermediate viewables within the
body of a function.  These viewables are not needed once execution of
the function is completed: they should be destroyed so that their
static arrays are returned to the system for re-allocation.  OBVIUS
provides a macro for this purpose: \\
\lfun{with-local-viewables}{ image-list . body}
This macro has the same syntax as the Common Lisp {\tt let*} macro.
It performs a local binding (using {\tt let*}), executes the body
within an {\tt unwind-protect} and then (silently) destroys all
viewables that were locally bound, thus freeing any heap memory
belonging to those viewables.   The returned value (or multiple values)
is that of the {\tt body}.

It is an error to return a temporary viewable from the body of {\tt
with-local-viewables}.  You may, however, return inferior viewables of
temporary viewables (even if \lsym{*auto-destroy-orphans*} is
non-nil).  For example, you may create a temporary complex image by
computing an fft, and then return the real-part of this complex image.
The complex image will be destroyed, but the real-part will be
preserved.  See the sections on destroying viewables and orphans
(section~\ref{sec:organization}) and the next section
(section~\ref{sec:using-memory}) for a more complete explanation of
\lsym{*auto-destroy-orphans*}.

If you need to create temporary static arrays (i.e., arrays allocated from
OBVIUS's static heap), use the the macro {\tt with-static-arrays},
described in the next section.

\mysubsubsec{Efficient Looping}
In the {\tt add} method given above, we call another method on the
data arrays of the argument image and the result image.  The methods
on arrays are typically designed to call C functions, which run at
least 20-30\% on Sun SPARC machines.  Despite this speed improvement,
it is somewhat inconvenient to have to write C code every time you
want a new image operation.  It is often simpler just to write the
code in LISP.  Several macros, defined in the file {\bf
lucid-image-loops.lisp} are provided to make it easier to write new
image processing functions in LISP:
\begin{description}
\item\lfun{loop-over-image-pixels}{ image-list . body}
Provides efficient looping mechanism for images.  The syntax is like
the Common Lisp {\tt let*} macro.  
See examples in the
files {\bf lucid-image-loops.lisp}, and {\bf imops.lisp}.

\item\lfun{loop-over-image-positions}{ image-list (ypos xpos) . body}
Provides efficient looping mechanism for images, with access to the current
coordinates.  See examples in the
files {\bf lucid-image-loops.lisp}, and {\bf synth.lisp}.
\end{description}

\mysubsubsec{Calling Foreign Functions}
\index{foreign functions}
If you do need to call C code (or some other language) from OBVIUS,
you can look at the foreign function definitions given in {\tt
lucid-ffi.lisp}.  Communication between Lisp and C can be complicated,
and you should  be aware of a few difficulties:
\begin{itemize}
\item If you pass non-static data structures to C code, you must be
careful that the Lisp Garbage Collector is not invoked.  See the
discussion of this in section~\ref{sec:using-memory} on memory
management.
\item If the C code encounters a floating point exception (eg, underflow,
overflow, divide-by-zero, log-of-negative-number... ), then Lisp
\index{foreign functions,floating-point exceptions}
will (by default) generate a continuable error.  This is appropriate
if it only happens once, but if it happens in a loop over a large data
structure, it is disastrous!  The user can turn off this error
handling by evaluating {\tt (setf (enabled-floating-point-traps)
nil)}, but then they will not know when an exception has occured.
Much of the C code in obvius handles floating exceptions more
gracefully (see, for example, \lsym{internal-div}), returning the
number of exceptions encountered within the loop.
\end{itemize}

%%%%%%%%%%%%%%%%%%%%%%%%%%%%%%%%%%%%%%%%%%%%%%%%%%%%%%%%%%%%%%%%%%%%%%%%%%
\subsection{The Memory Manager}
\label{sec:using-memory}

OBVIUS objects that contain large arrays should make use of OBVIUS's
static array allocator, defined in the {\bf memory.lisp} file.  In
Lucid Common Lisp, the ``static'' storage area is a portion of memory
that is not managed by the Lisp garbage-collector.  The reason for
using static arrays (as opposed to normal ``consed'' arrays) is
three-fold:
\begin{enumerate}
\item 
Lucid requires that arrays passed to foreign functions are static, and
begin on even word boundaries.  
\index{foreign functions, passing arrays to}
\index{foreign functions, and garbage collection}
Technically, this is only a problem if
the Lisp garbage collector is invoked while in the foreign function.
In this case, the arrays will be relocated by the garbage collector,
and the foreign function will write into the wrong portions of memory.
This can situation can arise if the foreign function calls a Lisp
function (a ``callback'' ), {\em or
if the foreign function calls} \lsym{malloc}! 
\item We want to prevent large arrays from being garbage collected.
\item Keeping images together in memory improves virtual memory swapping behavior.
\end{enumerate}
The OBVIUS static array allocation function ({\tt allocate-array})
works by displacing sub-arrays into large static vectors.  OBVIUS
keeps track of the space that has been allocated, and provides a
function to de-allocate arrays ({\tt free-array}).  The disadvantage
of this explicit memory allocation scheme is that arrays allocated in
this manner must be freed by the user or the programmer.

The following memory allocator functions are exported:
\begin{description}

\item\lfun{allocate-array}{ dims \&key element-type initial-element}
This function is similar to the \lsym{malloc} function in C.  Returns
an array of the specified dimensions and type, allocated from OBVIUS's
static heap.  The args are the same as for Common Lisp {\tt
make-array}.  The array returned is actually displaced to a large
static array.  Nevertheless, it is guaranteed to begin on an even word
boundary.  Remember that these arrays are not garbage collected and so
must be explicitly freed.  If the memory allocator does not have a
large enough block of the appropriate type of static memory, and if
\lsym{*auto-expand-heap*} is nil, a continuable error is given,
allowing the user to expand the heap by \lsym{*heap-growth-rate*}
elements.  If {\tt *auto-expand-heap*} is non-nil, this expansion will
happen automatically.  NOTE that Lucid is not very well protected
against over-expanding the static memory area and may crash if you
expand beyond the available process memory space.

\item\lfun{free-array}{ arr}
This function is similar to the \lsym{free} function in C.  Frees
storage space for an array that has been allocated using
allocate-array.  After evaluating (free-array A), it will still be
possible to access A, but the contents may be overwritten at any time.
It is also possible to free sub-arrays of an allocated array.  This
function gives a warning if the argument is not an array, or if it was
not allocated from the static heap.

\item\lfun{with-static-arrays}{ arr-list . body}
This macro acts just like the Common Lisp {\tt let*} macro except that
all of the static-arrays bound by it are freed (silently) before
exiting.  If a let* was used, the local variables would become unbound
on exiting, but the static storage space would not be freed.

\item\lfun{expand-heap}{ type-or-heap \&optional size return-block}
Expand the static heap of the given type (can also pass the heap
itself) by the given number of elements.  Any valid {\tt
:element-type} argument for the argument of the Common Lisp make-array
function may be used as a type.  Note that image arrays are of type
{\tt 'single-float}.  Default value for {\tt size} is {\tt
*heap-growth-rate*}.  If {\tt return-block} is nil (the default),
returns t.  Otherwise returns the newly allocated static array.

\item\lfun{heap-status}{}
Print out information about the static array allocator.  For each type
of static heap, prints the number of used elements, the number of free
elements, the size of the largest block, and the number of contiguous
blocks (i.e. fragmentation).

\item\lfun{run-heap-diagnostics}{}
Use this to verify the integrity of the static heap memory allocator.
Prints out all free arrays, giving warnings if inconsistencies are
found.

\item\lfun{ogc}{ \&key verbose}
This is the OBVIUS garbage collector, which can only be called from
top-level.  The \lsym{scrounge!} macro provides a top-level interface
(it just calls ogc with keyword {\tt :verbose} non-nil).  When keyword
{\tt :verbose} is non-nil, it prints out information about the space
reclaimed, the current heap status, and the (preserved) named
viewables.  {\tt ogc} preserves all static arrays that are owned by
pictures on picture stacks, and {\em non-orphaned} viewables (see the
discussion in the next section).  That is, all viewables that are
displayed as pictures, all viewables that are bound to symbols in the
current package or the user package, the viewables on the list
\lsym{*protected-viewables*}, and all of the inferior and superior
viewables of the aformentioned viewables (phew!).  All arrays that are
not preserved are returned to the heap for reallocation.  Therefore,
it is VERY IMPORTANT that the pointers from pictures to viewables, and
from viewables to their superiors and inferiors be maintained.  In
addition, users should be discouraged from creating their own data
structures containing ``inaccessible'' viewables, since these will not
be preserved during a garbage collection.  ASIDE: If Lucid ever
provides a finalize-instance method (called when all references to a
CLOS object have been lost and it is about to be garbage collected) we
will be able to get rid of this stuff and just have the
finalize-instance on viewables free their static arrays!

\end{description}

\mysubsubsec{Orphans and OGC}
\index{orphans}
In order to keep track of which allocated arrays are needed, OBVIUS
contains a fairly large amount of code devoted to maintaining
bi-directional pointers between viewables and pictures, and between
viewables and their superiors and inferiors.  For example, when
viewables are destroyed, their static space is de-allocated, and their
pictures are destroyed.  At any given time, OBVIUS can access
viewables through their pictures (which must reside on panes), through
the Lisp symbol-table (if they have symbol names), and through their
inferiors and superiors.  Any viewable that is not accessible in this
manner is called an ``orphan''.  The OBVIUS garbage collector,
\lsym{ogc}, will reclaim all allocated memory of orphans.

There are three standard situations in which an existing viewable can
become an orphan.  These are described in the section on ``orphans''
(in section~\ref{sec:organization}).  If the parameter
\lsym{*auto-destroy-orphans*} is nil, then the static space belonging
to these orphans will remain allocated.  In this case, the only way to
recover the static space belonging to these orphans is by running the
OBVIUS garbage collector {\tt ogc} (described below).

If the global parameter \lsym{*auto-destroy-orphans*} is non-nil, the
orphans created in these three situations will be {\em automatically
destroyed}.  This is a little bit dangerous for new users: it is easy
to write code that that locally creates orphans that one later expects
to use.  For example, you create a temporary compound-viewable (e.g.,
an image-pair), you bind a local variable to one of its inferiors
(e.g., the x-component).  Later in the function, outside of the scope
of the {\tt with-local-viewables} macro, you perform computations on
the inferior.  But this inferior was destroyed upon exiting the {\tt
with-local-viewables} macro!  You can prevent such destruction of
orphans by 1) setting \lsym{*auto-destroy-orphans*} to nil, or 2)
pushing the orphan onto the list in the special variable
\lsym{*protected-viewables*}.  The \lsym{with-local-viewables} macro
places its return value(s) on the {\tt *protected-viewables*} (see the
definition, in {\bf viewable.lisp}.

It is important to understand, however, that setting
\lsym{*auto-destroy-orphans*} to a non-nil value only ensures that
non-orphan viewables that become orphaned will be destroyed.  {\em It
is still possible to create orphans.} For example, suppose the user
writes a function that returns a list of viewables.  If I call this
function, and I don't hold onto the list returned by the function,
then these viewables will be orphans: OBVIUS will have no way of
accessing them.  But OBVIUS cannot ``know'' about the creation of such
orphans and thus cannot automatically destroy them.  As before, the
only way to recover the static space belonging to these orphans is by
running the OBVIUS garbage collector {\tt ogc} (described below).

Note also that this is a double-edged sword: if I call the function
and bind the resulting list to a global variable, then these viewables
are not accessible by OBVIUS (and are thus considered orphans), but I
(the user) {\em still have access to them} (through the global
variable).  Calling {\tt ogc} will therefore {\em de-allocate the
storage belonging to these viewables}.  The user must therefore be
careful not to create data structures containing ``hidden'' (orphaned)
viewables.  To return multiple viewables from a function, you should
use the Common Lisp multiple-values facility.  See
section~\ref{sec:memory} on memory management for more information.

\mysubsubsec{Troubleshooting Memory Problems}
If you find that viewables are being destroyed or over-written when
you don't want them to be destroyed, there are severa things to
examine.  First, try running your code with the variable
\lsym{*auto-destroy-orphans*} set to nil.  If this fixes the problem,
then you must figure out where in your code you are ``orphan'ing''
viewables (see the previous section).  You should get warnings when
you orphan viewables and the \lsym{*auto-destroy-orphans*} is nil.

You may also be creating new viewables that are orphans upon creation.
In this case the problems would not occur until you run the OBVIUS
garbage collector (\lsym{ogc}).  Look for places in the code that
store viewables in top-level data structures that OBVIUS doesn't know
about (i.e., viewable-matrices are ok, but a list of viewables is
not).

Finally, if the viewables your are losing are {\em not} orphans, it
must be that OBVIUS does not know about the static arrays of those
viewables.  Make sure that the \lsym{static-arrays-of} method returns
alist of all of the appropriate arrays.  These are the arrays that
will be protected from garbage collection.

\index{foreign functions, and garbage collection}
Further memory problems may arise because of foreign function calls.
If you call a foreign function with non-static Lisp data structures,
and the garbage collector is activated while you are in the foreign
code, the Lisp data structures will be moved by the garbage collector.
The foreign function, however, will still have the pointer to the old
data structure and may continue to read or write data from the old
memory locations.
There are several situations in which the garbage collector might be
called while in foreign code:
\begin{enumerate}
\item if the foreign code performs callbacks (that is, calls a Lisp
function)
\item if the ephemeral-garbage-collection (egc) is enabled
\item if the foreign code calls {\tt malloc}
\end{enumerate}
Even worse, this behavior may be different depending on whether your
code is interpreted or compiled, and which which version of the
compiler (i.e., production or development) you used!!  If your
def-foreign function definitions are {\em interpreted}, egc will often
run while you are in the C code!

The programmer should be aware of these situations, since these type
of bugs are often not noticed when testing code on small examples.
Once the code is in place and being used, these problems are very
difficult to track down!  Here are some possible solutions to the
problem:
\begin{enumerate}
\item write lisp interface code that {\em always} passes static
objects to your foreign code.  You can use the OBVIUS
\lsym{allocate-array} function to allocate static arrays.
\item Avoid the three items above that might cause garbage collection in lisp
while executing foreign code.  This means that all temporary space
used in the C code must be {\em allocated by Common Lisp} and passed
in to the foreign code.
\item Wrapping a \lsym{with-interrupts-deferred} macro around calls to the
foreign code can prevent garbage collection due to the second and
third items above.
\end{enumerate}

%%%%%%%%%%%%%%%%%%%%%%%%%%%%%%%%%%%%%%%%%%%%%%%%%%%%%%%%%%%%%%%%%%%%%%%%%%
\subsection{Adding New Viewable Subclasses}
\label{sec:new-viewable}

Use the CLOS function {\tt defclass} or the OBVIUS function
\lsym{def-simple-class} to define new types of viewables.  {\tt
def-simple-class} just calls defclass with args to set up the
accessors automatically.  The following functions/methods must be
provided for each viewable (for an example, look at the file {\bf
image-pair.lisp}).  In some cases, the default methods (defined in the
file {\bf viewable.lisp}) inherited from the generic viewable class
will suffice.  Typically, viewable classes and their methods are
placed in a file called
\abox{viewable-class}.lisp.
\begin{description}

\item {\tt (def-simple-class <viewable-class> ...)}
The class name is usually exported.

\item\lfun{settable-parameters}{ viewable-class-name}
This method, which should dispatch on a symbol eql to the viewable
class name, should return a list of the class slots which are user
adjustable.  It is used by the user interface code to create the
dialog boxes for editing the viewable slots.  Note that it should
append a list of the settable slots for this class with the result of
a call to the inherited method, since the superclasses may also
provide settable slots.  The default method returns {\tt
`(display-type)}.

\item\lfun{initialize-instance}{ vbl \&rest initargs}
This is the standard CLOS initialization function.  Any static arrays
needed by the viewable should be allocated here using the
\lsym{allocate-array} function (described in the previous section).
This method must also fill the superiors-of slots of the inferior
viewables if the viewable is a compound viewable (ie.  contains other
viewables).

\item\lfun{set-result}{ vbl model}
This method is called by the \lsym{with-result} macro (which should be
used by all functions that return viewables) as described above in
section \ref{sec:writing-ops}.  If the programmer is planning to write
routines for performing computations on instances of the new viewable
class, they should provide set-result methods to standardize the
result argument (:\res) behavior. The default methods are defined in
the {\bf viewable.lisp} file. There are four standard method dispatch
cases:
\begin{enumerate}
\item Both arguments are viewables.  In this case, the method checks that 
the viewables are of the same type, and calls \lsym{check-size} to
make sure they are the same size.  The default method is usually
sufficient here.
\item The {\tt vbl} argument is a name (nil/symbol/string), {\tt model} is a viewable.  
This method creates a new viewable, as specified by the {\tt model},
with the given name.  It should copy relevant slots from the model
into the result.  This method may need to be redefined for viewable
subclasses.
\item The {\tt vbl} argument is a viewable, {\tt model} is a property list (plist).  In
this case, the user has passed a viewable as a result argument, and it
must be checked for consistency with the initargs specified by the
plist.  This should be redefined for viewable subclasses to do this
error-checking.
\item The {\tt vbl} argument is a name (nil/symbol/string), and the
{\tt model} is a plist.  A new viewable is created using the plist as
a list of initargs in a call to the CLOS function
\lsym{make-instance}.  The default method defined in {\bf
viewable.lisp} is usually sufficient for this case.
\end{enumerate}

\item\lfun{make-<viewable-class>}{ \&rest initargs}
The top-level constructor function for the viewable.  This function
creates an instance of the viewable and fills in its slots.  This is
redundant -- the user could just as well call the CLOS function
\lsym{make-instance}.  It is provided to make the initargs explicit
(see, for example, the definition of \lsym{make-image}).  It should
make use of the functionality captured by the {\tt
initialize-instance} and \lsym{set-result} methods, by calling
\lsym{with-result}.  This function should
be exported.

\item\lfun{<viewable-class>-p}{ thing}
The predicate for the class.  Typically a macro.  This should be exported.

\item\lfun{print-object}{ vbl stream}
Use this to modify the printed form of this class.

\item\lfun{print-values}{ vbl}
This should print out ``values'' of the viewable.  It need not be
defined for all viewable subclasses.

\item\lfun{check-size}{ vbl \&rest vbl-list}
This method is called by set-result (case 1) and should ensure
that a list of viewables has comparable ``dimensions'' so that point
operations performed on them will work.  The meaning of ``dimensions''
depends on the sub-class of viewable.  The method should return a
viewable if the sizes are consistent, and signal an error otherwise.
See example in image.lisp.  The default method does no checking and
returns the first argument.

\item\lfun{static-arrays-of}{ vbl}
A method that returns a list of the static arrays allocated when the
viewable was created.  The list should {\em not} include static arrays
belonging to the inferiors of the viewable.  These arrays will be
freed when the viewable is destroyed.  They are also the arrays that
are preserved when the OBVIUS garbage collector (\lsym{ogc}) is
called.  The default method returns an empty list (nil).

\item\lfun{set-not-current}{ vbl}
This method is called when the viewable when it is destructively
modified.  It is called by the \lsym{with-result} macro.  This method
does not usually have to be written as the inherited one is usually
sufficient.  The inherited method increments the {\tt current} slot
and calls {\tt set-not-current} on the superiors and pictures of the
viewable (see viewable.lisp).

\item\lfun{inferiors-of}{ vbl}
A method that returns a list of the inferiors of the viewable.  This
must be defined for all compound viewables: it is used by the
\lsym{destroy} and \lsym{ogc} functions. The default (inherited)
method returns nil, and is thus appropriate for atomic (non-compound)
viewables.

\item\lfun{notify-of-inferior-destruction}{ vbl}
A method that is called on a compound viewable when one of its
inferior viewables is being destroyed.  The default (inherited) method
signals a continuable error.  If continued, it destroys both the
superior viewable and the inferior (see viewable.lisp)!  Some
viewables may allow destruction of inferiors, but they should erase
all pointers to those inferiors so as not to confuse the system memory
manager.  See the examples in {\bf sequence.lisp} and {\bf
gaussian-pyramid.lisp}.

\item\lfun{notify-of-superior-destruction}{ inf-vbl sup-vbl}
This method is called on a viewable when one of its superiors is being
destroyed.  It need not be rewritten for viewable subclasses.  The
standard method removes the superior from the \lsym{superiors-of}
list, and if \lsym{*auto-destroy-orphans*} is non-nil and the {\tt
inf-vbl} has been orphaned, it will be destroyed.

The description of the functions and methods above should make it
clear that the objects in OBVIUS are often connected by bi-directional
pointers.  This bi-directionality is necessary for the proper
functioning of the OBVIUS garbage collector (\lsym{ogc}).
These must be maintained every time an object is created or
destroyed.  This is especially difficult in the case of compound
viewables such as complex images and image sequences.  In order to
localize the source of such problems, it is important that the
\lsym{initialize-instance} method set up these pointers properly, and
that viewables should only be destroyed using the standard destroy
function (defined in the file viewable.lisp).  The superiors-of and
pictures-of slots must be carefully maintained.  The superiors-of slot
is used by set-not-current, destroy, and the garbage collector.  The
inferiors-of slot (method) is used by destroy and the garbage
collector. The pictures-of method is used by destroy and the
garbage-collector.
 
\item\lmeth{present}{viewable}{ pic viewable pane}
Described in section~\ref{sec:new-picture}.

\item {\bf Inferior Accessors} \\
In general, compound viewables should provide accessor methods to
access their sub-viewables.  The convention is that these take a :\res
keyword argument.  If no result is passed, they return the
sub-viewable, in the style of a standard accessor function.  If a
result is passed, the sub-viewable is copied into that result. If a
symbol is passed, a new viewable (bound to that symbol) is created and
the sub-viewable is copied into it.  For examples, see the definition
of \lsym{frame} in {\bf sequence.lisp} or \lsym{x-component} in {\bf
image-pair.lisp}.

\item {\bf History} \\
The \lsym{history} slot of a viewable contains the a history object.
This object preserves information about how the viewable was created.
It keeps track of the symbol names of intermediat viewables, so that
the user can print out an abbreviated version of the history.  The
history of a viewable is set with the \lsym{set-history} function:
{\tt (set-history vbl \&rest hist-list)}.  This destructively replaces
viewable elements of hist-list by their histories.  This function is
normally called by the \lsym{with-result} macro.
\end{description}


%%%%%%%%%%%%%%%%%%%%%%%%%%%%%%%%%%%%%%%%%%%%%%%%%%%%%%%%%%%%%%%%%%%%%%%%%%
\subsection{Multi-Processing}
\label{sec:multi-processing}

Starting with version 2.2, OBVIUS takes advantage of Lucid's
multi-processing capabilities.  The main lisp process runs a top-level
``listener'' or ``read-eval-print-loop'' (\lsym{repl}), described
below.  The source code for this is in the file {\bf repl.lisp}.

A secondary process, the \lsym{event-dispatch-process} created by
LispView, handles window system events (mouse clicks, window exposure,
etc).  NOTE: if this process dies, you will no longer be able to use
the mouse.  If you encounter an error and are given a choice of
killing this process, AVOID DOING SO!

\mysubsubsec{Mouse Events}
In order to avoid data-access collisions between processes, OBVIUS
mouse events are handled in one of two ways:
\begin{enumerate}
\item  If the a mouse event
requires processing that will only effect the display (i.e., pictures
and panes), and the processing is relatively simple (e.g., requiring
less then one second), then that event is handled immediately by the
mouse process.  In order to ensure that no other process tries to
access the same windows, all functions that are to be executed in
immediate mode should be called within the context of the macro
\lsym{with-locked-pane}.  This macro sets up a ``lock'' on the a given
pane, such that other processes executing a {\tt
with-locked-pane} macro will be forced to wait before accessing the pane.
\item More extensive event handling is placed on a queue, for
execution by the listener's read-eval-print-loop (described below).
This queue is executed in FIFO (first in first out).  Execution will
be delayed if the listener is not at top-level.
\end{enumerate}
Note that if there is an error, the debugger inhibits multi-processing
and therefore mouse events will not be handled.

\mysubsubsec{Repl}
The OBVIUS read-eval-print-loop has several purposes:
\begin{itemize}
\item Reads forms from *standard-input*, evaluating them and printing
their value(s) to *standard-output*.  If the value(s) begin printed
are viewables and \lsym{*auto-display-viewables*} is non-nil, then
these viewables are automatically displayed on the current pane.
\item Sets the standard Common Lisp variables, {\tt *, **, ***, +, ++,
+++, -, /, //, ///}, as described in Steele.
\item Handles enqueued mouse events (see above).
\item Sends messages to the Status line in the OBVIUS control panel,
indicating when the listener is at top level, computing, or in the
debugger. 
\end{itemize}

\mysubsubsec{Adding Menus and Dialog Boxes}
\index{menus,modifying} \index{dialog boxes}
The menus on the control panel are cynamically computed at
``click-time''.  This makes their reponse time slower, but makes it
very easy to modify them.  

Adding new functions to the function menus is easy.  You need only add
the function name (a symbol) to one of the global variables {\tt
obv::*obvius-<type>-functions*} where {\tt <type> } is one of {\tt
\{stats, arith, geom, filter, compare, synth, matrix, misc\} }.  These
menus are recomputed at ``click-time'' from these variables, so
modifying will result in an immediate change in the menu.  Function
dialog boxes are computed automatically (although this is sometimes a
bit flakey).

The Global parameters dialog is also computed at ``click-time'', and
will contain all exported symbols beginning and ending with an
asterisk (\*) character.

Similarly, if you define a new viewable or picture sub-class, it will
be automatically added (in alphabetical order) to the ``Viewable
defaults'' or ``Picture defaults'' menus.

The ``Modules'' sub-menu (on the ``misc'' menu), is created on the fly
to contain a list of all currently unloaded modules.  If you load a
module, it will no longer appear on this menu.  If you define a new
module (by adding to the obv::*obvius-module-plist*), it will appear
on this menu.

%%%%%%%%%%%%%%%%%%%%%%%%%%%%%%%%%%%%%%%%%%%%%%%%%%%%%%%%%%%%%%%%%%%%%%%%%%
\subsection{Display of Pictures}
\label{sec:auto-display}

%% Figure depicting the calls used to manipulate pane stacks

\begin{figure}
\newlength{\oldunitlength}
\setlength{\oldunitlength}{\unitlength}
\setlength{\unitlength}{0.015in}

\begin{center}
\begin{picture}(400,270)(4,0)

\put(70,260){\oval(80,20)}
\put(70,260){\makebox(0,0){pop-picture}}
\put(76,247){\vector(3,-4){18}}  	%to destroy

\put(330,260){\oval(80,20)}
\put(330,260){\makebox(0,0){display}}
\put(330,243){\circle{14}}
\put(330,243){\makebox(0,0){?}}
\put(320,245){\vector(-3,-1){78}}  	%to move-picture
\put(320,237){\vector(-1,-1){64}}  	%to push-picture
\put(292,148){\line(2,5){34}}	  	%to draw-pane
%\put(296,158){\line(2,5){30}}	  	%to draw-pane
\put(292,148){\vector(-3,-1){75}}  	%to draw-pane

\put(100,210){\makebox(0,0){\framebox(80,20){destroy (picture)}}}
\put(108,197){\vector(1,-1){24}}  	%to remove-picture

\put(200,210){\oval(80,20)}
\put(200,210){\makebox(0,0){move-picture}}
\put(189,197){\vector(-2,-1){48}}  	%to remove-picture
\put(203,197){\vector(1,-1){24}}  	%to push-picture

\put(40,160){\oval(80,20)}
\put(40,160){\makebox(0,0){cycle-pane}}
\put(52,147){\vector(3,-1){106}} 	%to draw-pane

\put(135,160){\makebox(0,0){\framebox(80,20){remove-picture}}}
\put(144,147){\vector(4,-3){32}} 	%to draw-pane

\put(230,160){\makebox(0,0){\framebox(80,20){push-picture}}}
\put(219,147){\vector(-3,-4){18}} 	%to draw-pane

\put(360,160){\oval(80,20)}
\put(360,160){\makebox(0,0){refresh}}
\put(348,147){\vector(-3,-1){106}} 	%to draw-pane

\put(200,110){\makebox(0,0){\framebox(80,20){draw-pane}}}
\put(200,97){\vector(0,-1){24}}

\put(360,60){\oval(80,20)}
\put(360,60){\makebox(0,0){drag-picture}}
\put(348,47){\vector(-3,-1){75}} 	%to render

\put(200,60){\makebox(0,0){\framebox(80,20){present}}}
\put(187,47){\vector(-1,-1){24}} %to compute-picture
\put(213,47){\vector(1,-1){24}} %to render

\put(150,10){\makebox(0,0){\framebox(80,20){compute-picture}}}
\put(250,10){\makebox(0,0){\framebox(80,20){render}}}

\end{picture}
\end{center}

\caption{Chart indicating the function call hierarchy of functions
used to manipulate the picture stacks of OBVIUS panes.  Functions in
oval boxes are top-level functions.  Those in rectangular boxes are
internal.  The question mark enclosed in a circle indicates that the
{\protect\tt display} function calls one of the three indicated functions.}
\label{fig:display-calls}

\setlength{\unitlength}{\oldunitlength}
\end{figure}


The most important convention in OBVIUS is the chain of command for
display of viewables, and manipulation of picture-stacks.  This is
illustrated in figure~\ref{fig:display-calls}.  Typically, the
top-level listener loop (\lfun{repl}, described above) will call the
\lsym{display} function when viewable(s) are returned
to top level.  Other code (e.g. a function called from the mouse
process) can achieve the same behavior by calling the function
\lsym{print-top-level-values}.

Since OBVIUS allows destructive modification of viewables, it often
happens that a picture will become out-of-date with respect to its
viewable.  OBVIUS keeps track of this by maintaining a \lsym{current}
slot for viewables and pictures.  This slot holds a version number.
If the version number of the picture is equal to that of the viewable,
it is up-to-date.  Software that modifies a viewable should call
\lsym{set-not-current} (or use the \lsym{with-result} macro). 
{\tt Set-not-current} increments the \lsym{current} slot of the
viewable.  Below, decribe some of the
functions used to manipulate and display pictures of viewables, and
the use of the {\tt current} slot.
\begin{description}

\item\lfun{display}{ viewable \&optional display-type \&rest initargs \&key pane make-new}
This is the exported method for creating a picture of a viewable.
This assumes {\em only} that the viewable exists.  The default value
for the :pane keyword is the current pane.  If there are no panes, the
function creates one using new-pane.  The default value for the
:display-type keyword is the display-type slot of the viewable.  If
this :display-type keyword is nil, then nothing is displayed.
Otherwise, the value of the :display-type keyword must be a picture
class (or the name of one) suitable for passing to make-instance.
When the :display-type keyword is non-nil, OBVIUS searches through the
picture stacks for a picture of the specified type.  If no picture of
this type exists or if the {\tt :make-new} parameter is non-nil, a new
one is created using {\tt make-instance}.  The {\tt initargs} are
passed along to make-instance, so the user can specify initialization
arguements for the picture.  If a picture of this type exists on one
of the panes, it is moved to the top of the current pane, if the
variable \lsym{*preserve-picture-pane*} is non-nil.  Otherwise it is
moved to the top of its own pane.  The picture will be updated if it
is out of date with the viewable and the variable
\lsym{*auto-update-pictures*} is non-nil.  In either case, the picture
is put onto the pane using either {\tt push-picture} or {\tt
move-picture}.  If it is already displayed in the desired pane, {\tt
display} just calls \lsym{draw-pane}.

\item\lfun{refresh}{ \&optional (pane *current-pane*)}
This provides an entrypoint to the draw-pane method that {\em always}
re-computes the internal representation of the picture, whether or not
it is out-of-date with respect to the viewable.

\item\lfun{move-picture}{ pic \&optional (pane *current-pane*)}
This function performs bookkeeping manipulations of the picture-stack
of the pane and the pane-of the picture (i.e., it is responsible for
keeping consistent pointer references).  If the picture is being moved
to another screen, then its \lsym{system-dependent-frob} may have to
be converted to be consistent with the new type of screen.  It then
calls {\tt draw-pane} to update the screen.

\item\lfun{cycle-pane}{ pane \&optional (num 1)}
This function performs bookkeeping manipulations of the picture-stack
of the pane, rotating the list {\tt num} places (note: num can be
negative).  It then calls {\tt draw-pane} to update the display.

\item\lfun{pop-picture}{ pane}
This function calls \lsym{destroy} on the top (visible) picture of the
pane stack.

\item\lfun{draw-pane}{ pane \&key clear}
This method is the bottleneck for bringing the appearance of the pane
up-to-date with the picture on top of the stack.  In addition to being
called the functions illustrated in figure~\ref{fig:display-calls}, it
is called internally by the window system when it receives ``damage''
or ``exposure'' events.  If the {\tt
:clear} argument is non-nil, {\tt draw-pane} clears the pane and sets
the title-bar to the string {\tt "displaying ..."}.  When the pane is
not empty, {\tt draw-pane} calls the {\tt present} method (described
in the next section) on the picture, viewable and pane, and then sets
the title bar string.  If an error occurs in the call to {\tt
present}, {\tt draw-pane} will pop the bad picture from the
picture-stack when the user returns from the debugger.

\end{description}

%%%%%%%%%%%%%%%%%%%%%%%%%%%%%%%%%%%%%%%%%%%%%%%%%%%%%%%%%%%%%%%%%%%%%%%%%%
\subsection{Adding a New Picture Subclass}
\label{sec:new-picture}

Pictures are typically created by the {\tt display} function
(defined in the file {\bf viewable.lisp}).  Note
that joint methods on viewables and pictures are defined in the
picture file by convention since the viewable files are loaded first.

In creating a new picture type, the programmer should first write the
low level code that fill's the \lsym{system-dependent-frob} slot.
This slot holds a system-dependent representation of the picture that
can be efficiently ``rendered'' to the screen or printer.  For
example, the frob might be an X window system offscreen bitmap, or an
object containing lists of lines that need to be drawn.  This
abstraction also makes porting to other window systems or display
devices easier.

The \lsym{drawable} picture subclass, which provides a parent class
for \lsym{graph}s and \lsym{vector-field}s is a good example.  The
frob for this class will be either an offscreen bitmap object (called
a bltable), or an object containing a description of lines to be drawn
directly in the pane.  The \lsym{retain-bitmap} determines which of
these types of frob is used.  The code in {\bf graph.lisp} draws the
graph into the frob using \lsym{draw-line} and \lsym{draw-text}
commands, which are defined for both type of frob.  Note also that
these are described for \lsym{postscript-bltable} frobs, which are
used when sending output to  a postscript printer.

After writing the low-level code that draws the picture, the
programmer must connect the picture with the rest of OBVIUS.  In order
to do this, the following functions must be provided for each new
picture subclass (for examples, see the files {\bf picture.lisp},
{\bf gray.lisp}, and {\bf drawing.lisp}):
\begin{description}

\item\lfun{defclass}{ \abox{picture-class} ...}
The class name is usually exported.

\item\lfun{settable-parameters}{ picture-class-name}
This method, which should dispatch on a symbol equal to the picture
class name, should return a list of the class slots which are user
adjustable.  It is used by the user interface code to automatically
create dialog boxes for editing pictures.  Note that it should append
a list of the settable slots for this class with the result of a call
to the inherited method, since the superclasses may also provide
settable slots.

\item\lfun{reset-picture-defaults}{ pic vbl \&rest initargs}
This method should compute appropriate values for all of the picture
slots.  It is called by the CLOS method \lsym{shared-initialize} which
is called by both \lsym{initialize-instance} {\em and}
\lsym{reinitialize-instance}.  That is, it is called whenever the
slots of the object are being set.  In order to modify slots of a
picture, the user will typically either use a dialog box, or call
\lsym{setp}.  Both of these call {\tt reinitialize-instance}.  The
{\tt reset-picture-defaults} method should take care of all automatic
computation of parameters (e.g., if the user sets the parameter to
{\tt :auto}), and should also handle parameter dependencies (i.e.,
when the modification of one parameter requires adjustment of
another).  {\em It must return a modifed list of initargs.}  See, for
example, the method for gray pictures.

\item\lfun{set-not-current}{ pic}
Called when the underlying viewable is modified.  The default
(inherited) method puts a double asterisk (ala emacs) in the title bar
if the picture is at the top of a picture stack.  Specialized versions
may need to destroy certain internal data, or reset certain
parameters.  See, for example, the method on gray pictures.

\item\lfun{static-arrays-of}{ pic}
Returns the static arrays used by the picture.  Note that if any
of the arrays are shared with a viewable, these should {\em not} be
returned.  They should be returned only by the static-arrays-of method
on the viewable (for example, see the code for gray pictures of
images, which resets the :scale and :pedestal slots ).  The standard
method just returns the {\tt static-arrays-of} the \lsym{system-dependent-frob}.

\item\lfun{drag-picture}{ pic dy dx}
This method is called while the user is dragging the picture with the
mouse.  It should adjust the {\tt x-offset} and {\tt y-offset} slots
of the picture and redraw the picture in the pane.  The default one
does this by calling draw-pane, which is inefficient in most cases.
Therefore, efficient versions are typically written for subclasses.

\item\lfun{zoom-picture}{ pic factor y x}
Called by the zoom mouse clicks.  {\tt factor} is the amount to zoom
by, and the {\tt y} and {\tt x} parameters indicate the location of
the mouse when it was clicked.  The default version in {\bf
picture.lisp} is often sufficient.

\item\lfun{position-message}{ pic vbl pane x y}
This should present useful information to the user when they click the
mouse on the pane at the given coordinates.  It typically calls
\lsym{pane-coord-to-viewable-coord} to compute the viewable coordinates, and
then \lsym{status-message} to put the message into the OBVIUS status line.
The default version does nothing.

\item\lfun{frob-coord-to-viewable-coord}{ pic frob-y frob-x}
This should convert the coordinates of the frob (these are typically
the same as the pane coordinates, but with the offset and zoom
removed) to those of the viewable.  It is called by the 
\lsym{pane-coord-to-viewable-coord} method.

\item\lfun{title-bar-string}{ pic}
Returns a string to be put in the title bar of the pane displaying the
picture.  This should give to the user any information necessary for
proper interpretation of the visual display.  The default version
returns the name of the viewable.

\item\lmeth{present}{picture}{ picture viewable pane}
This is a triple-method on picture, viewable, and pane sub-classes
which does the actual displaying.  The pane dispatch is meant to allow
specialization for different types of screen.  The typical method
computes the internal system-dependent picture representation by
calling the {\tt compute-picture} method, and then calls {\tt render}
to display the picture in the pane.  See {\bf gray.lisp} for an example.

\item\lfun{compute-picture}{ picture viewable}
This method computes the window-system-dependent intermediate picture
representation (the ``system-dependent-frob'').  It typically calls a
system-dependent picture-specific method {\tt draw-\abox{thing}} to
actually fill the frob.  Note that {\tt compute-picture} is called by
flipbooks and pasteups on the sub-pictures.  See {\bf gray.lisp} and
{\bf x-blt.lisp} for an example of how this is done.

\item\lfun{render}{ pane frob y-offset x-offset zoom}
This method displays the system-dependent intermediate picture
representation (the ``frob'') in the pane, with the given x-offset,
y-offset and zoom.

\end{description}

%%%%%%%%%%%%%%%%%%%%%%%%%%%%%%%%%%%%%%%%%%%%%%%%%%%%%%%%%%%%%%%%%%%%%%%%%%
\subsection{Porting To Other Machines, Window Systems, or Lisp Implementations} 

[***** This section should be elaborated.  Describe the strategy of
the window-system interface layering.  In particular, the generic
screen, pane, and frob stuff.
*****]

Porting to another window system would require replacing the source
files beginning with an ``x-'' prefix, and would probably require a
moderate amount of work.  We have deliberately kept our use of
LispView simple, and have tried to provide a layer of generic
interface to the window system.  All references to LispView code are
package-prefixed with the lispview: prefix.  The generic OBVIUS window
system interface is accessed in the OBVIUS source files (e.g. {\bf
pane.lisp} and {\bf drawing.lisp}).  It relies on four
system-dependent types of object: screen, pane, drawable, bltable.
The screen and pane objects should inherit from the system-independent
versions defined in pane.lisp (see, for an example, the definitions of
x-screen and x-pane in x-window.lisp).  The function make-pane should
be provided.  The drawable and bltable objects do not inherit from any
system-independent objects, but make-bltable and make-drawable
functions must be provided.

As for porting to another Lisp implementation, there are three
Lucid-specific facilities which are heavily used in OBVIUS: 1) the
foreign function interface, and 2) the multi-processing capability.
We also make extensive use of Lucid's CLOS implementation, and 3) the
static array allocation facility.  Calls to lucid-specific functions
are package-prefixed with {\tt LCL:} (general lucid extensions), {\tt
CLOS:} (Lucid's implementation of CLOS), or {\tt MP:} (Lucid's
multi-processing facility).  Changing the syntax of the foreign
function calls (assuming the alternate Lisp Implementation provides a
foreign function interface) would probably be relatively simple.  The
multi-processing is actually only used indirectly through the LispView
interface to X.  Only the code in {\bf repl.lisp} relies on the
multi-processing capabilities of Lucid (which are modularized in the
MP package).  Porting LispView to run in another Lisp would probably
be difficult.  It might be easier to rewrite the window system
interface from scratch (see above).

Porting to other machines (assuming the X window system and Lucid
Common Lisp are both available) necessitates either porting LispView
to the new machine, or providing your own X windows interface.  Rumor
has it that Unipress is planning to port LispView to other hardware
platforms, but it is not certain when this will happen.
