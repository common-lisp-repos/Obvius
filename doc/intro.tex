\section{Introduction}


OBVIUS (Object-Based Vision and Image Understanding System) is an
image-processing system based on Common Lisp and CLOS (Common Lisp
Object System).  The system provides a flexible interactive user
interface for working with images, image-sequences, and other
pictorially displayable objects.  By using Lisp as its primary
language, the system is able to take advantage of the interpretive
lisp environment (the ``listener''), object-oriented programming, and
the extensibility provided by incremental compilation.

The basic functionality of OBVIUS is to present certain lisp objects
to the user pictorially.  These objects are refered to as
``viewables''. \index{{\ptt viewable}} Some examples of viewables are
monochrome images, color images, binary images, complex images, image
sequences, image pyramids, filters and discrete functions.  A
``picture'' \index{{\ptt picture}} is the displayable representation of a
viewable.  Note that a given viewable may be compatible with several
different picture types.  For example, a floating point image may be
displayed as an eight bit grayscale picture, as a one bit dithered
picture, or as a 3D surface plot.  OBVIUS also provides postscript
hardcopy output of pictures.

In the typical mode of interaction, the user types an expression to
the lisp listener and it returns a viewable, which is automatically
displayed as a picture in a window.  Each window contains a circular
stack of pictures.  Standard stack manipulation operations are
provided via mouse clicks (e.g., cycle, pop, and push).  Commonly used
operations such as histogram and zoom are also provided via mouse
clicks.

OBVIUS provides a library of image processing routines (e.g., point
operations, image statistics, convolutions, fourier transforms).  All
of the operations are defined on all of the viewable types.  The
low-level floating point operations are implemented in C for speed.
OBVIUS also provides a library of functions for synthesizing images.
In addition, it is straightforward to add new operations and new
viewable and picture types.

OBVIUS currently runs on Sun workstations in Lucid v4.0 Common Lisp.
We hope to have it running on other machines in the future (e.g.,
DecStations, SGI workstations, Apple Macintoshes, etc).  

This manual describes the overall organizational strategy of the
system, and provides a description of the user interface.  Throughout,
we assume the reader is familiar with Emacs, Common Lisp, and some
form of object-oriented programming environment.  The manual is
written primarily for Sun users -- users of other machines should see
the Appendices for machine dependencies.  Section \ref{sec:start} and
the OBVIUS tutorial \index{tutorial} (in the file {\tt
<obv>/tutorials/obvius/obvius-tutorial.lisp}) are provided for new
users to aquaint themselves with the system without getting bogged
down in hairy implementational details.  The middle sections
\ref{sec:organization} through \ref{sec:userface} are a more detailed
description of the system for more experienced users.  Section
\ref{sec:hacking} is provided for programmers (``hackers'').

\newpage

\section{Getting Started}
\label{sec:start}

We assume in this section that you have installed OBVIUS on your
system (if you have not yet created an OBVIUS image, instructions for
doing so are given in appendix~\ref{sec:install}).  We also assume
that you are running the X window system (version 11, release 4 or
higher), or Sun's OpenWindows (version 2.0 or higher).  You may use a
window manager of your choice.  We have used OBVIUS with ``twm'',
``uwm'', ``mwm'', and ``olwm''.

We recommend that you run OBVIUS from within the Gnu emacs editor,
version 18.49 or later (Gnu emacs is avaiable for anonymous ftp from
prep.ai.mit.edu).  The tutorial (in the file {\tt
<obv>/tutorials/obvius/obvius-tutorial.lisp}) assumes that you are
doing this.  OBVIUS can, however, be run from a Unix shell.

If you are running from Emacs, you should add the following
lines to your ``{\tt .emacs}'' startup file:
\begin{verbatim}
(setq load-path (cons <obv>/emacs-source load-path))
(setq *cl-program* <lisp-image>)
(autoload 'run-cl "cl-shell" nil t)
(setq *obvius-program* <obvius-image>)
(autoload 'run-obvius "cl-obvius" nil t)
\end{verbatim}
The symbol {\tt <obv>} is used throughout this document to indicate the
pathname of the primary OBVIUS directory (on our system, this is ``{\tt
/usr/local/obvius}'').  The symbol {\tt <lisp-image>} refers to the
pathname for a Lucid 4.0 image that includes CLOS and LispView.  On
our system, this is in ``{\tt /usr/local/bin/lispview}''.

To start lisp, you type {\tt M-x run-cl} from within Emacs.  To start
OBVIUS, you type {\tt M-x run-obvius}. After several seconds, a new
Emacs buffer will appear named {\tt *lisp*} and various system loading
messages will be printed into the buffer.  Eventually, you will get
OBVIUS's prompt: {\tt OBVIUS>}.  This buffer provides an interface to
Lisp that is running as a sub-process of Emacs.  As described in the
next section, hitting the {\tt cr} key in this buffer will send text
to the Lisp process for evaluation, and Lisp output will be printed
into this buffer.  

At this point, new users go through the OBVIUS tutorial in the file
{\tt <obv>/tutorials/obvius/obvius-tutorial.lisp}.  The purpose is to
familiarize new users with the basic functionality of OBVIUS, and with
the user interface.

The remaining part of this section describes customization of your
OBVIUS environment, and is unnecessary for novices.  At run-time,
OBVIUS looks for two startup files in your home directory.  These can
be used for personal customization of your environment.  The first is
named either ``{\tt .obvius-windows}'' or ``{\tt
obvius-window-init.lisp}''.  This file loads the appropriate window
system code.  The default version is in ``{\tt
<obv>/lucid-window-init.lisp}''.

The second startup file is called either {\bf .obvius} or {\bf
obvius-init.lisp}.  You can use this file to create a fixed number of
OBVIUS windows in predefined locations, to initialize default values
for global parameters and to have code loaded automatically.  The
default file is in ``{\tt <obv>/lucid-obvius-init.lisp}''.  This file
contains many (commented out) function calls that you may wish to add
to your personal initialization file.

