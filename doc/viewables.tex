%%%%%%%%%%%%%%%%%%%%%%%%%%%%%%%%%%%%%%%%%%%%%%%%%%%%%%%%%%%%%%%%%%%%%%%%%%%%%%%%%%
\section{Viewables}
\label{sec:viewables}

This section describes the various viewable subclasses.  To add new
viewable classes to OBVIUS, see section \ref{sec:hacking} of this
manual.  The currently defined viewables are are (indentation
indicates inheritance):
\begin{tabbing}
image \= \\
\> one-d-image \\
\> slice \= \\
\> bit-image \\
viewable-matrix \\
\> image-matrix \\
\> viewable-sequence \\
\> \> image- \= sequence \\
\> \> \> color-image \\
\> \> \> image- \= pair \\
\> \> \> \> one-d-image-pair \\
\> \> \> \> one-d-complex-image \\
\> \> \> \> complex-image \\
\> \> \> \> polar-image \\
\> \> complex-discrete-function \\
filter \\
\> separable-filter \\
\> hex-filter \\
discrete-function \\
\> histogram \\
\> log-discrete-function \\
\> periodic-discrete-function \\
pyramid \\
\> gaussian-pyramid \\
\> laplacian-pyramid \\
\> steerable-pyramid \\
\> qmf-pyramid \\
overlay-viewable \\
\> overlay-gray-viewable \\
\end{tabbing}

%%%%%%%%%%%%%%%%%%%%%%%%%%%%%%%%%%%%%%%%%%%%%%%%%%%%%%%%%%%%%%%%%%%%%%%%%%%%%%%%%%

\subsection{Images}
\label{sec:images}
\index{{\ptt image}}

This section describes images and the functions provided for their
manipulation.  OBVIUS has several other related types of viewables
(e.g., bit-image, image-pair, complex-image, polar-image,
image-sequence, and one-d-image) that are also described below.

The following files in the \abox{obv}  directory contain source
code that is relevant to images:
\begin{itemize}
\item {\bf image.lisp} --  contains the definition of the
image class, accessors for the slots of images, and standard methods
and utilities for images.
\item {\bf lucid-image-loops.lisp} -- contains macros such as 
\lsym{loop-over-image-pixels} for efficient image accessing and looping.
\item {\bf imops.lisp} -- contains the definitions for all
of the arithmetic methods on images as well as some other assorted
methods.  See Section \ref{sec:operations} for a complete list.
\item {\bf fft.lisp} -- contains a fast Fourier transform, a Hilbert transform,
and a discrete cosine transform.
\item {\bf synth.lisp} -- contains functions as described below for
generating synthetic images.
\end{itemize}

\mysubsubsec{Make-image}

The function \lsym{make-image} may be used to make a new image:
\begin{verbatim}
(make-image dims &key display-type name)
\end{verbatim}
This function returns a new image of the specified dimensions,
initialized to zeros.  The dims parameter must be a list of two
fixnums, the first specifying pthe y-dimension and the second
specifying the x-dimension.

{\bf Note}: The convention in OBVIUS is always to specify y first and
x second.  This is consistent with the row-major storage of arrays in
both Common Lisp and C.

If only one dimension is specified or if one of the dimensions is 1,
then {\tt make-image} will return a {\tt one-d-image}.  This is a
subclass of image that is displayed as a graph.  Otherwise, its
behavior should be the same as any other image.

The {\tt :display-type} argument must be the name of a picture
sub-class that is compatible with the image.  The default display-type
is either {\tt 'gray} or {\tt 'dither}.  The default may be set from
the {\tt Viewable Default} sub-menu of the {\tt Parameters} menu on
the control panel.

\mysubsubsec{Iref}

``Iref is to images as aref is to arrays''.  It returns the value of the
image at the specified location, and may be used with setf to modify the
image values.  For example,
\begin{verbatim}
(setf (iref im 20 30) 5.1)
\end{verbatim}
sets pixel {\tt (20,30)} to the value {\tt 5.1}.  NOTE: It is not
efficient to use \lsym{iref} in loops, since it is a method and
therefore has substantial dispatch-time overhead (relative to the
operation being performed).  Instead, you should either: 1)
restructure your algorithm to use existing OBVIUS functions, which
call C code, 2) write C code to perform your operation, 3) use the
looping macros \lsym{loop-over-image-pixels},
\lsym{loop-over-image-positions}, or with-images (see section
\ref{sec:hacking}), 4) write lisp code that extracts the data arrays
and uses aref to access them efficiently. \index{{\ptt iref},efficiency}

\mysubsubsec{Print-values}

The \lsym{print-values} function is used to generate a formatted printout
of the values in some part of an image.  For example,
\begin{verbatim}
(print-values image :y 0 :x 0 :y-size 10 :x-size 10)
\end{verbatim}
prints the top-left 10x10 pixels of the image.  The keyword 
parameters {\tt :x} and {\tt :y} specify the location of the upper-left
hand corner of the subimage to be printed, and {\tt :x-size} and {\tt
:y-size} specify the size of that sub-image.  If these keywords are
nil then the subimage is determined by the global parameters
\lsym{*x-print-range*} and \lsym{*y-print-range*}.  Each of these global
parameters is a list of two floats between 0 and 1.  For example if
\lsym{*x-print-range*} is '(0.0 0.5) and \lsym{*y-print-range*} is
'(0.5 1.0) then the lower left quarter of the image will be printed.
The subimage is subsampled so that no more than
\lsym{*max-print-vals*} samples are printed on each line.  The globals
{\tt *max-print-vals*}, {\tt *x-print-range*}, and {\tt
*y-print-range*} parameters may be reset from the {\tt Globals} dialog
box that is accessed from the {\tt Parameters} menu on the control
panel.

The {\tt print-values} function is also defined for arrays, and for
{\tt image-pairs}.  

\mysubsubsec{Pictures of Images}

Images may be displayed as {\tt gray} (see section \ref{sec:gray}),
{\tt dither} (see section \ref{sec:dither}), as {\tt
surface-plot} (see section \ref{sec:surface-plot}), or as 
{\tt contour-plot} (see section \ref{sec:contour-plot}).

%%%%%%%%%%%%%%%%%%%%%%%%%%%%%%%%%%%%%%%%%%%%%%%%%%%%%%%%%%%%%%%%%%%%%%%%%%%%%%%%%%

\subsection{One-d-images and Slices}
\label{sec:one-d-images}

One-d-images are really the same as images except that one of the
dimensions is 1.  So, all of the image operations that are defined for
images work on one-d-images as well.  The only difference is that a
one-d-image is (by defaults) displayed as a {\tt one-d-graph}.  A
one-d-image can be generated using make-image, e.g.,
\begin{verbatim}
(make-image '(100 1) :-> '1d-im)
\end{verbatim}
or any of the synthetic image generating functions.  The \lsym{slice}
viewable class is a subclass of the {\tt one-d-image} class.  The
one-d-image class and the slice class are defined in the file {\bf
image.lisp}.  \lsym{one-d-image}s are typically displayed as
{\tt graph} pictures (see section \ref{sec:graph}).  

%%%%%%%%%%%%%%%%%%%%%%%%%%%%%%%%%%%%%%%%%%%%%%%%%%%%%%%%%%%%%%%%%%%%%%%%%%%%%%%%%%

\subsection{Discrete Functions and Histograms}
\label{sec:discrete-function}

The viewables \lsym{discrete-function}, \lsym{periodic-discrete-function}, and
\lsym{log-discrete-function} are defined in the file {\bf
discrete-function.lisp}.  The class \lsym{histogram}s are a subclass
of {\tt discrete-functions}.  {\tt Discrete-functions} are also used
as lookup tables to do fast nonlinear point operations on images (see
the function \lsym{point-operation} in Section \ref{sec:operations}).

{\tt Discrete-functions} are essentially a lookup table of function
values --- they are made up of a {\tt data} array, an {\tt origin},
and an {\tt increment}.  The first entry in the {\tt data} array is
taken to be the value of the function evaluated at {\tt origin}, the
next entry is taken to be the value of the function evaluated at {\tt
(+ origin increment)}, etc.  The {\tt evaluate} function is used to
evaluate a discrete-function at a particular value.  If that value
falls between samples, then the result is interpolated.

{\tt Periodic-discrete-functions} are the same as {\tt
discrete-functions} except they are interpreted as one full period of
a periodic function.  If the periodic function is defined from 0 to 1,
evaluating at 1.5 is the same as evaluating it at 0.5.  {\tt
Log-discrete-functions} are sampled on a log axis.

The following functions are used to make and evaluate discrete
functions.  
\begin{description}
\item\lfun{make-discrete-function}{ func min max \&key size
interpolator endpoint name \res} Makes a discrete function of the
function {\tt func}, from {\tt min} to {\tt max}, with {\tt :size}
samples.  The default for {\tt :interpolator} is {\tt
'linear-interpolation} which does linear interpolation.  If {\tt
:endpoint} is non-nil then the discrete-function includes a sample at
{\tt max}.  Otherwise, the last sample is slightly below {\tt max}.
An example is given in the OBVIUS tutorial.

\item\lfun{make-periodic-discrete-function}{ func min max \&key size
interpolator endpoint name \res} 
The default {\tt :interpolator} {\tt 'periodic-linear-interpolation}
which does linear interpolation, modulo {\tt min} and {\tt max}.

\item\lfun{make-log-discrete-function}{ func min max \&key size
base interpolator endpoint name \res} 
The default for {\tt :base} is 10.  Example:
\begin{verbatim}
(make-log-discrete-function '(lambda (x) (log x 10)) 
                            0.01 10.0 :base 10 :size 4)
\end{verbatim}
The {\tt discrete-function} includes samples at .01, .1, 1, and 10.

\item\lmeth{evaluate}{discrete-function}{ discrete-function value}
Evaluates the discrete-function function at value.  Interpolates if
value is between two entries in the table.
\end{description}

{\tt Discrete-functions} are displayed by default as {\tt graph}
pictures (see section \ref{sec:graph}), but they may also be displayed
as {\tt polar-plot} pictures (see section \ref{sec:polar-plots}).

%%%%%%%%%%%%%%%%%%%%%%%%%%%%%%%%%%%%%%%%%%%%%%%%%%%%%%%%%%%%%%%%%%%%%%%%%%%%%%%%%%

\subsection{Bit-images}
\label{sec:bit-image}

The \lsym{bit-image} class is a subclass of the {\tt image} class.
{\tt Bit-images}, utilities for using {\tt bit-images}, and operations
on {\tt bit-images} are defined in the file {\bf bit-image.lisp} (see
section \ref{sec:bitmap}).  {\tt Bitmaps} are the only display-type
for {\tt bit-images}.  The threshold functions {\tt greater-than},
{\tt less-than}, {\tt equal-to}, {\tt greater-than-or-equal-to}, and
{\tt less-than-or-equal-to} discussed in Section \ref{sec:operations}
can be used to make {\tt bit-images} from images.

Bit-images are just like images except that they have only one bit of
data at each pixel (the bit-image data is stored using common-lisp bit
arrays).  So, some of the functions and accessors are inherited from
images (e.g., {\tt iref}, {\tt dimensions}, {\tt x-dim}, {\tt y-dim},
{\tt total-size}).  Other methods are functionally equivalent to those
defined on images (e.g., {\tt mul}, {\tt copy}, {\tt similar}, {\tt
crop}, and {\tt paste}).  Other methods are modified somewhat (e.g.,
add is modified to perform a logical-or operation on bit-images).  

%%%%%%%%%%%%%%%%%%%%%%%%%%%%%%%%%%%%%%%%%%%%%%%%%%%%%%%%%%%%%%%%%%%%%%%%%%%%%%%%%%

\subsection{Sequences}
\label{sec:sequence}

The classes \ldef{image-sequence} and \ldef{viewable-sequence} are defined
in the file {\bf sequence.lisp}.  

{\tt Image-sequences} may be used to work with temporal sequences of
images or 3d image data (e.g., volumetric data used in medical
imaging).  Since {\tt image-sequences} are made up of {\tt images},
all of the image operations that are defined for {\tt images} work on
{\tt image-sequences} as well.  Image Sequences may be displayed as
{\tt pasteup} pictures (see section \ref{sec:pasteup}), or as flipbook
pictures (see section \ref{sec:flipbook}).  

A {\tt viewable-sequence} consists of a number of sub-viewables, all
of the same type.  See the example in the OBVIUS tutorial.  General
(non-image) sequences can be displayed as {\tt flipbooks} or as {\tt
overlays}.

%%% *** Needs work!


%%%%%%%%%%%%%%%%%%%%%%%%%%%%%%%%%%%%%%%%%%%%%%%%%%%%%%%%%%%%%%%%%%%%%%%%%%%%%%%%%%

\subsection{Image Pairs, Complex Images, and Polar Images}
\label{sec:image-pair}

The file {\bf image-pair.lisp} contains the definitions of the
\ldef{image-pair}, \ldef{complex-image}, and \ldef{polar-image}
classes, that are compound images formed from a pair of standard
images.  {\tt Image-pairs} are used to store the left and right frames
of a stereo pair or the y- and x-components of a flow field.  {\tt
Complex-images} are used for Fourier transforms.  {\tt Polar-images}
store the magnitude and complex-phase of a {\tt complex-image}.  Since
{\tt image-pairs} are made up of {\tt images}, all of the image
operations that are defined for {\tt images} work on {\tt image-pairs}
as well.

{\tt Horizontal-pasteup} pictures are the default display type for
{\tt image-pairs}.  These are a subclass of \lsym{pasteup} pictures
(see Section \ref{sec:pasteup}).  {\tt Image-pairs} may also be
displayed as \lsym{vector-field}s (see section \ref{sec:vector-field}).

%%%%%%%%%%%%%%%%%%%%%%%%%%%%%%%%%%%%%%%%%%%%%%%%%%%%%%%%%%%%%%%%%%%%%%%%%%%%%%%%%%

\subsection{Filters}
\label{sec:filters}

The viewable subclasses \ldef{filter} and \ldef{separable-filter} are
defined in the file {\bf filter.lisp}.  Currently, all filters are
considered to be finite impulse response (FIR) filters -- infinite
impulse response (IIR) filters will be implemented in a future
release.  Display methods are provided to plot them as \lsym{graph}s
(for one-dimensional filters) or as \lsym{gray}s (for two-dimensional
filters).  The user may also create three-dimensional filters for
application to image sequences: these are currently not displayed.

Filters are made up of a kernel array, an edge-handler, and a grid.
For purposes of efficiency, filters are implemented to allow
subsampling of the output.  The grid slot specifies the subsampling to
be performed in the convolution (see make-filter examples below).  The
edge-handler specifies how to handle the edges of the image while
convolving.  Current choices for edge-handler are nil
(circular-convolution), and the keywords \lsym{:reflect1} (reflect the
image through the edge pixel),
\lsym{:reflect2} (reflect the image about the edge), \lsym{:zero}, and
\lsym{:repeat}.  Separable filters are compound viewables 
containing two filters.

The function \lsym{make-filter} is
used to make filters.  The first argument to these function should be a
vector, a list, or an array specifying the filter kernel.  Additional
keyword arguments may be provided: \lsym{:step-vector},
\lsym{:start-vector}, \lsym{:edge-handler}.  Examples:
\begin{verbatim}
(make-filter '(1.0 2.0 1.0))
(make-filter '((1.0 2.0 1.0) (2.0 4.0 2.0) (1.0 2.0 1.0)))
(make-filter 2darray :step-vector '(2 2))
\end{verbatim}
The first of these makes a 1d filter, and the second makes a 2d
filter.  The last example makes a filter that will downsample by a
factor of two when the \lsym{apply-filter} function is called (or
upsample by a factor of two when the \lsym{expand-filter} function is
called).

The function  \lsym{make-separable-filter}  may be used to create
filters that are applied separably (i.e., two one dimensional filters,
applied in the x- and y- directions).  This function takes two
filters, or two arrays, or two nested lists as arguments, along with
all of the keywords described above.  The first filter will be applied
in the vertical (y) direction.  You can also make higher dimensional
separable filters (although the dimensionality cannot exceed that of
the thing you are applying them to).  For example, you can make a
separable filter out of a 1D and a 2D filter, and apply it to an image
sequence.  The 2D filter may be separable or non-separable.  

The {\tt :edge-handler} argument determines what happens at the
borders of an image when the \lsym{apply-filter} or
\lsym{expand-filter} functions are called.  Any local linear
edge-handling scheme may be implemented.  Some examples are
implemented in the file {\bf \abox{obv}/c-source/edges.c}.  The
associated with each type of edge-handler is a C function that
computes a new filter (set of weights) to be used when computing one
of the edge samples of the convolution.  The arguments to this
edge-handler function are described in the file {\bf
\abox{obv}/c-source/edges.c}.

\mysubsubsec{3D filtering}
Three-dimensional filtering (on image-sequences) is also supported.
To make a 3D filter, just pass a 3D array (or list) to the {\tt
make-filter} function.  To make a 3D separable filter, you can call
{\tt make-separable-filter} with a 2D filter and a 1D filter.

\mysubsubsec{Hexagonal filtering}
Two-dimensional {\tt hex-filter}s are defined on a hexagonal lattice:
half of each row of the samples should be zero.  These are a special
class because the subsampling must be done differently.  They have not
been fully ported from OBVIUS-1.2 yet.

%%%%%%%%%%%%%%%%%%%%%%%%%%%%%%%%%%%%%%%%%%%%%%%%%%%%%%%%%%%%%%%%%%%%%%%%%%%%%%%%%%

\subsection{Pyramids}
\label{sec:pyramid}

The \ldef{gaussian-pyramid} module may be loaded into OBVIUS by evaluating:
\begin{verbatim}
(obv-require 'gaussian-pyramid)
\end{verbatim}
Pyramids are built with the function \lsym{make-gaussian-pyramid} and
reconstructed with the \lsym{collapse} function.  Various subbands may
be accessed with the \lsym{access-band} function.  Other pyramid
operations are listed in in section~\ref{sec:operations}.

Pyramids are currently undocumented.  For now, look at example in the
OBVIUS tutorial, and at the code in {\bf gaussian-pyramid.lisp}.

%%%%%%%%%%%%%%%%%%%%%%%%%%%%%%%%%%%%%%%%%%%%%%%%%%%%%%%%%%%%%%%%%%%%%%%%%%%%%%%%%%

\subsection{Color Images}
\label{sec:color-image}

The \ldef{color-image} class is defined in the file {\bf
color-image.lisp}.  {\tt Color-images} inherit from {\tt
image-sequences}, but the images in the sequence are interpreted as
different color bands (e.g., R, G, and B electron beam gun values).

This is unfinished code.  We are writing a full color analysis package
that keeps track of color-spaces, input devices (e.g., photoreceptors,
cameras, scanners), output devices (e.g., monitors, printers),
spectral sensitivities of these input and output devices, and spectral
properties of surface reflectances.  

For the time being, {\bf color-image.lisp} defines simple color images
that can be displayed as {\tt color-pictures} or as {\tt pasteups}
({\tt pasteups} are the default display type).

%%%%%%%%%%%%%%%%%%%%%%%%%%%%%%%%%%%%%%%%%%%%%%%%%%%%%%%%%%%%%%%%%%%%%%%%%%%%%%%%%%

\subsection{Viewable Matrices}
\label{sec:viewable-matrix}

The \lsym{viewable-matrix} and \lsym{image-matrix} are defined in the
file {\bf viewable-matrix.lisp}.  {\tt Viewable-matrices} contain an
arrays of sub-viewables.  The {\tt data} slot of a {\tt
viewable-matrix} is the array of sub-viewables.  A number of other
viewable types inherit from {\tt viewable-matrices} (e.g.,
\lsym{image-sequence}s are {\tt viewable-matrices} with a 1xN array of
images).

{\tt Viewable-matrices} are particularly useful for parallelizing
pixel-by-pixel matrix operations.  For example, this kind of operation
is used in OBVIUS to convert {\tt color-images} from one color space
to another.  For another example, let's say that you had six images.
At each pixel you want to take the corresponding values out of each of
the six images, put four of those values in a matrix and the other two
in a vector, and then compute the inverse of the matrix times the
vector.  Believe it or not, this type of operation comes up a lot in
visual modeling and image analysis, and it is very easy to do with
{\tt viewable-matrices}.

