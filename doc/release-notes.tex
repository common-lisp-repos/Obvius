\section{Release Notes}
\label{sec:release-notes}

\subsection{Version 2.1 Release Notes}

There have been several major changes since releases 1.0 and 1.2 of
OBVIUS.  We have tried to keep most of the changes internal (i.e.
non-exported), but some of them will affect the top-level user.  The
following are the major changes since release 1.2:

\subsubsection{General Changes}

\begin{itemize}

\item OBVIUS now runs under the X11 or OpenWindows window systems  exclusively.
We no longer support SunView or the Symbolics window systems.

\item OBVIUS now runs under Lucid 4.0.  This version of Lucid provides
an implementation of CLOS (we no longer use PCL), and supports
multi-processing.

\item OBVIUS now uses a standard interface to the window system called
``LispView''.  LispView was written at Sun, and is distributed with
Sun (Lucid) Common Lisp.  We have deliberately kept our interface to
LispView as simple as possible so that it should be easy to port
OBVIUS to other GUI platforms.  Note that LispView has a few minor
bugs when running under the twm window manager.

\end{itemize}

\subsubsection{Specific Notes for OBVIUS Users}

\begin{itemize}

\item The following global parameters have gone away: 
{\tt *SEQ-DISPLAY-TYPE*}, {\tt *DEFAULT-DF-SIZE*}, {\tt
*DEFAULT-NUMBER-OF-BINS*}, {\tt *AUTO-SCALE-IMAGES*}, {\tt
*GAMMA-CORRECT*}, {\tt *GRAY-SCALE*}, {\tt *GRAY-PEDESTAL*}, {\tt
*AUTO-SCALE-COMPOUND-IMAGES*}, {\tt *RETAIN-GRAPHS*}, {\tt
*INDEPENDENT-PARAMETERS*}, {\tt *DEFAULT-PASTEUP-FORMAT*}, {\tt
*AUTO-SCALE-VECTOR-FIELDS*}, {\tt *VECTOR-FIELD-DENSITY*}, {\tt
*VECTOR-FIELD-MAGNIFY*}, {\tt *VECTOR-FIELD-LENGTHEN*}, {\tt
*VECTOR-FIELD-SKIP*}, {\tt *VECTOR-FIELD-ARROW-HEADS*}.  In their
place, we provide a mechanism for the user to adjust the default
values of slots for CLOS classes. The function \lsym{set-default}
allows the user to set the default values of the slots of any class.
The arguments look like: {\tt (set-default <class-name>
<slot-name-or-initarg> <default-form>). } The function
\lsym{get-defaults} is provided to get a list of the current default
values for a class.  For example, to auto-scale gray pictures by
default, you would execute {\tt (set-default 'gray 'scale t)}.  To
auto-scale an existing instance of a gray picture, you would execute 
{\tt (setp scale t)}.

\item The user initialization file may now be called either
\lsym{obvius-init.lisp} or \lsym{.obvius}.  This allows you to use a ``.'' file if
your prefer, and allows old users to keep around their former startup
file (previous version of obvius did not use .obvius).  Also, the user
can now have an \lsym{obvius-window-init.lisp} or
\lsym{.obvius-windows} file.  This one will be loaded before the
window system is initialized and thus allows modification of the
behavior of window system resource allocation etc.  Default versions
of both of these files are included in the lisp-source directory.

\item All pictures now have a \lsym{:zoom} slot which you can setp.
Zoom factors are no longer limited to powers of two.  A zoom value of
one corresponds to the ``natural'' size of the picture.  If the zoom
value is set to :auto, the picture is automagically zoomed to fill the
pane.  If it is a list of length two, the picture is zoomed to these
dimensions.

\item The \lsym{:scale} in gray pictures now has a different meaning.  The
values in the image are {\em divided} by the scale (after subtracting off
the pedestal).  The resulting values should be in the range [0,1],
where 1 will correspond to the brightest intensity of the display, and
0 the darkest.  Values outside this range are clipped.

\item New picture type \lsym{dither} allows display of dithered images
on screens deeper than 1 bit.  It inherits from gray.

\item \lsym{one-d-graph} picture type is now called \lsym{graph}.

\item New picture type \lsym{drawing}.  This is now the parent class
for {\tt graph}s, {\tt vector-field}s, {\tt surface-plot}s, and {\tt
contour-plot}s.

\item New slots have been added to screens.  The \lsym{:gray-gamma} parameter
determines the gamma correction exponent for the gray colormap.  The
\lsym{:gray-shades} parameter corresponds to the number of gray colors in the
colormap.  The \lsym{:gray-dither} parameter determines whether OBVIUS dithers
float array values into gray pictures.  If this holds a number, OBVIUS
will dither if there are less gray shades than this number.  If t,
OBVIUS always dithers.  The \lsym{:foreground} should hold either a
keyword corresponding to one of the colors in {\tt
/usr/lib/X11/rgb.txt}, or a list of three numbers in the interval
[0,1] corresponding to red, green, blue components.  This color is
used as a default for graphs, vector-fields, etc.  The
\lsym{:background} slot is similar.  It determines the color used to
clear the panes.

\item Edge-handler arguments to \lsym{make-filter} have changed.  They are now
keywords instead of strings.  The current list is: {\tt nil (wrap),
:dont-compute, :zero, :repeat, :reflect1, :reflect2, :extend }.

\item The "point-operation" \index{point operations} 
macros ({\tt +. *. /.} etc) are now act like the corresponding Common
Lisp functions (documented in Steele, edition I).  They take any
number of arguments, along with a result arg.  New ones have been
defined.  Eventually, ALL of the numerical functions in Steele should
be created.  Partial list of new operations: {\tt fill!}, {\tt
make-ramp}, {\tt make-1-over-r}, {\tt make-r-squared}, {\tt expt.},
{\tt exp.}, {\tt log.}, {\tt mod.}, {\tt sin.}, {\tt cos.}, {\tt
realpart.}, {\tt imagpart.}, {\tt phase.}, {\tt incf},. {\tt decf.},
{\tt zero-crossings}, {\tt rect-to-polar} {\tt polar-to-rect}, {\tt
transpose}, {\tt max-abs-error}, {\tt minimum-location}, {\tt
point-minimum}, {\tt maximum-location}, {\tt point-maximum}, {\tt
skew}, {\tt kurtosis}.

\item The {\tt *auto-scrounge!} parameter has been eliminated.  A new
parameter, \lsym{*auto-destroy-orphans*}, has been added to allow
OBVIUS to destroy orphaned viewables.  This is described in
section~\ref{sec:organization}.

\item The macro \lsym{with-local-images} has been replaced by
\lsym{with-local-viewables}.  The behavior has been slightly modified:
1) the new macro is capable of returning multiple-values, 2) the macro
signals an error if you try to return one of the local viewables from
the body, 3) you may return inferiors of compound viewables -- these
will not be destroyed (even if the global symbol
\lsym{*auto-destroy-orphans*} is non-nil).

\item Emacs interface has been rewritten as a much cleaner package (in use
since 1/90).  Support for arglists, macroexpansion, describe, etc are
provided.  See the file emacs-source/cl-shell.doc for more
information.

\item Arguments to display have changed.  The new function is much
more powerful and flexible.  The syntax is as follows:
\begin{verbatim}
(display viewable &optional (display-type (display-type vbl))
                  &rest keys
                  &key 
		  (pane (or *current-pane* (new-pane)))
		  make-new
                  &allow-other-keys)
\end{verbatim}
The differences are: 1) \lsym{:display-type} is now an optional
parameter (it used to be a keyword).  2) The \lsym{:make-new}
parameter allows you to force OBVIUS to create a new picture of the
given type (otherwise, OBVIUS may just bring an existing picture of
the given type up to the top of the current pane).  3) You can pass
keywords corresponding to slot names in the picture (as in setp) to
set them up.  Example: {\tt (display <image> 'gray :scale 200
:pedestal 37)}

\item Arguments to \lsym{hardcopy} have changed.  They are now similar
to those for display:
\begin{verbatim}
(hardcopy viewable &optional (display-type (display-type vbl))
                   &rest keys
                   &key
		   path printer y-location x-location y-scale x-scale
		   invert-flag suppress-text make-new
                   &allow-other-keys)
\end{verbatim}
Also, hardcopy now does real postscript output (i.e. line drawing
commands) for drawing pictures (e.g. graphs, vector-fields,
surface-plots), so these will be drawn at the resolution of the
printer!

\item Adopted a convention for matrix multiply operations that vectors
correspond to row-vectors.  We used to automagically interpret vectors
as either row or column depending on the operation.  Now, you get a
continuable error if the method is expecting a column and you give it
a row (or vice versa).  To work with column vectors, make Nx1 arrays.
Or, use the functions vectorize and columnize to convert (via
displaced arrays) between row and columns.

\item Arguments to \lsym{make-sin-grating} have changed (see
description in Section \ref{sec:operations}).

\item Arguments to \lsym{make-histogram} and \lsym{entropy} have been
changed (see description in Section \ref{sec:operations}).

\item Keyword arguments to {\tt make-<vbl>} functions can now be any initargs that
you could pass to a call to make-instance.  They no longer take a
:\res keyword -- you should use the \lsym{:name} keyword to specify a
name.  The purpose of the functions is to 1) emphasize that some args
are required, 2) provide result arg behavior (including history, etc).

\item You can now create sequences of any type of viewable (not just images)
and can display flipbooks of most types of pictures.

\item Big speedup in most synthetic image generation code. (eg.
{\tt make-uniform-random-image, make-sin-grating}, etc).

\item Optional loadable modules: look at \lsym{*obvius-module-plist*}.
Current modules are {\tt :gaussian-pyramid :qmf-pyramid :overlay
:x-magnifier :matrix}.  Load these by executing {\tt
(obv-require <module-name>)}.

\item The \lsym{overlay} module allows you to display two graphs on one plot,
or overlay a one-bit image on top of a gray image.  See
section~\ref{sec:overlays}.

\item Mouse clicks are now received by a secondary process.  Some of
them are handled immediately (i.e., even when the top-level listener
is computing something).  Examples are the simple manipulations of
panes: cycle-pane, move-to-here, etc.  Other operations are put onto a
queue for evaluation by the top-level (listener) process.  See
section~\ref{sec:multi-processing} for more details.

\item Mouse bindings have changed slightly.  The right mouse button
now displays position messages dynamically as you drag the mouse.
Bindings with the M-Sh and M-C buckys down are now picture-specfic and
will change depending on the type of picture in the pane.  See
section~\ref{sec:mouse} for a listing of the bindings.

\item We now provide a more interactive user interface: a control
panel with menus.  To use it, execute {\tt (obv-require
:x-control-panel)} and then {\tt (make-control-panel)}.  Status
messages will appear in the control-panel if it exists, in the Emacs
mini-buffer (if running in cl-shell-mode) or not at all.  Dialog boxes
provide pop-up help: click M-left on an item for pop-up help
information (NOTE: due to a bug in LispView, this feature doesn't work
sometimes.  To ``reset'' it, click in a blank region of the dialog,
and then click M-left on the desired item immediately afterward).

\item New viewable, \lsym{polar-image} stores the magnitude and
complex-phase of a complex-image.  The functions magnitude and
complex-phase do the right thing on polar images.  The arithmetic
functions (e.g., {\tt add}, {\tt sub}, {\tt mul}, and {\tt div}) are
not yet defined correctly for {\tt polar-images}.  Convert them to
complex images first (using {\tt polar-to-complex}), and then do the
arithmetic.  \lsym{polar-to-rect} renamed \lsym{polar-to-complex}, and converts a
polar-image to complex-image.  Likewise, \lsym{rect-to-polar} renamed
\lsym{complex-to-polar}.

\item Defined nearly all image operations on arrays.  Redefined image
methods to call these array methods.

\item Defined nearly all image operations on discrete-functions.
Got rid of peak.  Now it is same as image operations, maximum and
maximum-location.

\item Changed args to \lsym{circular-shift}, \lsym{paste}, and
\lsym{crop} to be keywords.
Defined circular-shift, paste, and crop for vectors and arrays.  For
vectors, ignores the ``y'' keywords.

\item Added ``x'' ``y'' ``x-dim'' and ``y-dim'' keyword arguments to 
\lsym{correlate} for correlating over a restricted range of offsets.

\item Defined complex-discrete-functions.  All the operations for
complex-images are implemented for complex-discrete-functions,
including fft, magnitude, etc.

\item One-d-images can now be made by specifying only the 1 relevant
size, e.g., {\tt (make-sin-grating 64) } 
One-d-images are now stored as vectors (not as 2D arrays).

\end{itemize}

\subsubsection{Notes for OBVIUS Hackers (i.e. non-exported changes)}

\begin{itemize}

\item OBVIUS now uses LispView, an object-oriented Lisp-to-X interface
written by Sun.  Most of the old X/SunView cruftiness has been thrown
out.  The new interface is slower, but it is also more robust and
requires much less maintenance!

\item Lucid-4.0 contains a nearly-up-to-spec version of CLOS.  In
particular, defclass and initialize-instance now adhere to the spec.
We have defined a
simple macro called {\tt def-simple-class} which automatically creates
accessors, an initarg and an initform (nil) for each slot.
We have also moved most of the functionality of the {\tt make-<vbl>} functions
into initialize-instance methods, where it belongs!

\item To avoid multi-processing collisions, any function which might
be called directly from the mouse must be careful to modify the
contents of the pane only within a call to the macro
\lsym{with-locked-pane}.  Also, note that because of the
multi-processing in Lucid4.0, you can no longer compile when in the
debugger.  Alternatively, one can enqueue forms for execution by the
top-level loop using the function {\tt push-onto-eval-queue}.

\item The memory manager has been rewritten to be noticably faster,
more general, and more robust.  A diagnostic tool is provided to check
the integrity of the heap data structures:
\lsym{obvius::run-heap-diagnostics}. Eventually, we would like to
replace this with a Lucid-supported static memory allocator.  Two
global parameters are provided to adjust the behavior of the memory
manager. If \lsym{*auto-expand-heap*} is non-nil, the static heap is
automatically expanded when OBVIUS needs more memory.  The number of
elements by which to expand is determined by the variable \lsym{*heap-growth-rate*}.

\item The \lsym{with-result} macro replaces all 
\lsym{with-result-<vbl>} macros.  It returns the value of the last
expression in the body (including multiple values).  It has been
rewritten to use a model specification similar to what you would pass
to make-instance.  If you pass a model-plist, the first keyword must
be {\tt :class} followed by a CLOS class or the symbolic name of one.
Also, if you don't pass a history list, it will not modify the history
of the result.  This allows the programmer to let the history be that
resulting from the sub-calls made within a function.  See
sections~\ref{sec:writing-ops} and~\ref{sec:new-viewable}.

\item Auto-display is no longer done from the print-object method.  It
is now done directly by the top level loop (\lsym{repl}).  In a
multi-processing environment, it was too messy to determine whether
the printing was being done at ``top-level''.

\item The order of arguments in the methods \lsym{present} and
\lsym{set-result} has changed.

\item The keyword arguments used for the \lsym{set-result} method on
\lsym{viewable-sequence} are a bit different.  You can now pass a
\lsym{:sub-viewable-spec} which will be used to create the sub-viewables.
You must provide a {\tt :length} parameter with this.  

\item The \lsym{gray-lut} (gray lookup-table) in a colormap is now a vector of
length no more than 256 which indexes into the successive gray entries
in the colormap.  These gray values may be gamma-corrected.

\item Some variable names have changed:

\begin{tabular}{ccc}
status-line-message & {\tt -->} & status-message \\
draw-gray           & {\tt -->} & draw-float-array \\
draw-bitmap	    & {\tt -->} & draw-bit-array \\
new-array           & {\tt -->} & allocate-array \\
memory-usage        & {\tt -->} & heap-status \\
*growth-rate* 	    & {\tt -->} & *heap-growth-rate* \\
scrounge!           & {\tt -->} & ogc \\
\end{tabular}

In addition, the function \lsym{default-display-type} is no longer used.

\item Patches are loaded from the patches sub-directory.  Files in
this directory are expected to have filenames like patch-123.lisp.
They are loaded at startup time in numerical order, starting with the
filenumber specified by the parameter \lsym{*starting-patch-file*}.
See the file {\tt lisp-source/patches.lisp} for the code that does the
loading.  We will periodically put new patch files up for anonymous
ftp.  Users will typically have the choice of ftp'ing a whole new
source tree, with the patches installed, or just the patch  files.
The new source tree will have an appropriate {\tt
*starting-patch-file*}, so as not to load unnecessary patches.

\end{itemize}

\subsection{Version 2.2beta Release Notes}

Below is a listing of major changes/additions.  See the Changelog for
detailed changes.
\begin{itemize}
\item Color-images and color-pictures added.

\item Overlays totally rewritten.

\item Viewable-matrices added.  Sequences now inherit from
viewable-matrices. 

\item A huge number of matrix utilites added, and new versions of svd,
matrix-inverse, regrees, etc.

\item Stepit added.

\item List-ops added.

\item Statistics and numerical-recipes added.

\item Plot symbols added to graphs.

\item Polar-plots added.

\item Scatter-plots added.

\item For flipbooks and overlays, picture parameters dialog now allows
you to change parameters of the sub-pictures.

\item Generic-ops added.  When adding new ops, write the methods for
arrays and for generic viewables first.

\item eliminated gamma-correct from gray (and pasteup) pictures.
Wrote new gamma-correct method.

\item args to compile-load changed: Now  takes an :output-file keyword
like the Common Lisp compile-file command.

\end{itemize}


\subsection{Version 2.2 Release Notes}

Below is a listing of major changes/additions.  See the Changelog for
detailed changes.

\begin{itemize}
\item Additional keyword args :optimizations and
:brief-optimize-message have been added to compile-load.
:optimizations is a list that you would pass to (proclaim (optimize
...)).  :brief-optimize-message overrides the Lucid :optimize-message
arg and prints a one line synopsis of the current compiler
optimizations.

\item Arglists have been made more consistent for \lsym{crop},
\lsym{paste}, \lsym{make-slice}, and \lsym{circular-shift}.  
Old keywords were left in for back-compatibility.

\item Overlays totally rewritted again (see obvius-tutorial.lisp for
examples).  Overlays is no longer an optional module.

\item New version of surface plots, that can be printed using
hardcopy.

\item Contour plots added (an optional module).

\item Fast display-sequence code added.

\item Synthetic functions:  Eliminated \lsym{make-normal-random-image}.
Changed names of some functions:
\begin{tabular}{ccc}
\lsym{make-uniform-random-image} & {\tt -->} & make-uniform-noise \\
\lsym{make-gaussian-noise-image} & {\tt -->} & make-gaussian-noise \\
\lsym{make-random-dot-image} & {\tt -->} & make-random-dots \\
\end{tabular}

\item \lsym{load-image} on image sequences now checks for a {\tt
class} key.  Useful for image-pairs or color-images.

\item \lsym{fft} now takes a {\tt :dimensions} keyword.

\item Added display for filters.

\item Default args to \lsym{point-operation} have changed.  The
default binsize is now nil (=> call the function, rather than making a
lookup-table).  This is slower, but gives no sampling surprises in the
result!

\item Added :center, :post-center, and :pre-center keyword for fft
and power-spectrum. 

\item Added \lsym{rotate} and \lsym{resample}.

\item Added \lsym{round.} \lsym{truncate.} \lsym{floor.} methods

\item The \lsym{warp} code is no longer an optional module
(it is included in the source).

\item Pane manipulation functions now exported (useful from within
programs, or if mouse dies): {\tt pop-picture}, {\tt cycle-pane}, {\tt
get-viewable}.

\item Added function \lsym{next-pane}. 

\item Got rid of {\tt destroy-top-picture} (same as {\tt pop-picture}), and
{\tt destroy-pane} which is just {\tt (destroy *current-pane*)}.  Got
rid of {\tt destroy-top-viewable}.  Provided {\tt current-viewable}
instead, which returns viewable on top of current pane.

\item Added \lsym{purge!} function.

\item Fixed {\tt setf iref} and lots of other annoying little bugs.

\item Added \lsym{periodic-discrete-function}s and \lsym{log-discrete-function}s.

\end{itemize}
