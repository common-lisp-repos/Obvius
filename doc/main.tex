\documentstyle[twoside,11pt,spacing,makeidx]{article}
%% draft causes overfull hboxes to be marked
%% spacing loads single,double,and onehalf spacing macros from spacing.sty 
%% twoside makes margins different on odd and even numbered pages

\makeindex

%%% Margins
\oddsidemargin 0.5in     %%0.25in
\evensidemargin 0.0in    %%0.25in   
\topmargin 0.1in           %%-0.1in
\textwidth 6.0in        %%6.25in 
\textheight 8.7in        %%9.0in        
\columnwidth \textwidth

\headheight 0in			% no header
\headsep \headheight

\footheight .3in		% space for page number
\footskip \footheight

\parskip 8pt
\parindent 0.4in                % Was 0.5 in original thesis

\long\def\comment#1{}

%%% Spacing commands (different from those defined in spacing.sty)
\newcommand\doublesp{\renewcommand\baselinestretch{1.7}\large\normalsize}
\newcommand\singlesp{\renewcommand\baselinestretch{1.05}\large\normalsize}
\newcommand\halfsp{\renewcommand\baselinestretch{1.3}\large\normalsize}

%%%  Use before 1st paragraph of a section to force an indentation
\newcommand{\yesindent}{\hspace*{\parindent}}   

%%% Caption width variable, to be used with the new definition...
%%% This is currently ignored.
\newlength{\cw}  
\setlength{\cw}{4.5in}	% Default value

%%% New version of \caption puts things in smaller type and indents them 
%%% to set them off more from the text.
%%% should also allow extra arg to make correct width?
%%% \bcaption is the old caption.
\makeatletter
\long\def\@makecaption#1#2{
	\smallskip
	\setbox\@tempboxa\hbox{\small {\bf #1:} #2}
	\parindent 1.5em  %% How can we use the global value of this???
	\dimen0=\hsize
	\advance\dimen0 by -2\parindent
	\ifdim \wd\@tempboxa >\dimen0
		\begin{singlespace}
		\hbox to \hsize{\hfil 
				\parbox{\dimen0}{\small {\bf #1:} #2} 
				\hfil}
		\end{singlespace} \par
	\else \hbox to \hsize{\hfil \box\@tempboxa \hfil}
	\fi}
	\medskip
\makeatother

%%% Non-numbered subsections, with table of contents entries!
\def\mysec#1{\section*{#1} \addcontentsline{toc}{section}{#1}}
\def\mysubsec#1{\subsection*{#1} \addcontentsline{toc}{subsection}{#1}}
\def\mysubsubsec#1{\subsubsection*{#1} \addcontentsline{toc}{subsubsection}{#1}}

%%% Macros to print things with < and > characters.
%\def\abox#1{\mbox{$\langle$#1$\rangle$}}    % <arg>
\def\abox#1{{\tt <#1>}}
\def\res{{\tt ->}}  %%\mbox{$-$$>$\ }                   % ->

%% lisp function definition: typewriter text, in parens, fun name
%% bold, makes a DEF index entry.
\def\ldef#1{{\tt #1} \index{{\tt #1},DEF}}
\def\lfun#1#2{{\tt ({\bf #1}#2)} \index{{\tt #1},DEF} \\}
\def\lmeth#1#2#3{{\tt ({\bf #1}#3)} \index{{\tt #1},{\it #2},DEF} \\}

%% Lisp symbol: typewriter text, makes index entry, but not for definition!
\def\lsym#1{{\tt #1}\index{{\tt #1}}}    

%%%%%%%%%%%%%%%%%%%%%%%%%% document %%%%%%%%%%%%%%%%%%%%%%%%%%
\begin{document}
\pagenumbering{arabic}

\begin{titlepage}
\hbox{ }

%\vbox to0.2in{\vfil}
\vfil

\begin{center}
{\Huge {\bf OBVIUS:}}\\
\bigskip
\bigskip
{\Large {\bf Object-Based Vision and Image Understanding System}} \\

\vfil
\vfil

{\Large {\bf Version 3.0}} \\
\bigskip
\bigskip
{\large
\copyright 1988, 1989, 1990, 1991 \\
Media Laboratory, MIT. \\
\medskip
\copyright 1994 \\
Leland Stanford Junior University. \\
\bigskip
\bigskip
\today
}
\vfil
\vfil

\hbox to \textwidth{\large \hfil David Heeger  \hfil  Eero Simoncelli \hfil
Eduardo-Jose Chichilnisky}

\vfil

\end{center}
\end{titlepage}

%% Blank page to get even/odd correct...
\newpage
\thispagestyle{empty}
\setcounter{page}{0}
\mbox{}		%dummy, to force page output
\newpage

\tableofcontents

\singlesp

Copyright 1994 by the Leland Stanford Junior University.  Copyright
1991 by the Massachusetts Institute of Technology.  All rights
reserved.
   
Developed by David J. Heeger at Stanford University, Stanford,
California and Eero P. Simoncelli at the University of Pennsylvania
This distribution is approved by Nicholas Negroponte, Director of the
Media Laboratory, MIT and by the Office of Technology Licensing,
Stanford University.

Permission to use, copy, or modify these programs and their
documentation for educational and research purposes only and without
fee is hereby granted, provided that this copyright notice appears on
all copies and supporting documentation.  For any other uses of this
software, in original or modified form, including but not limited to
distribution in whole or in part, specific prior permission must be
obtained from MIT and Stanford University.  These programs shall not
be used, rewritten, or adapted as the basis of a commercial software
or hardware product without first obtaining appropriate licenses from
MIT and Stanford University.  MIT and Stanford University make no
representations about the suitability of this software for any
purpose.  It is provided "as is" without express or implied warranty.

\newpage

\section{Introduction}


OBVIUS (Object-Based Vision and Image Understanding System) is an
image-processing system based on Common Lisp and CLOS (Common Lisp
Object System).  The system provides a flexible interactive user
interface for working with images, image-sequences, and other
pictorially displayable objects.  By using Lisp as its primary
language, the system is able to take advantage of the interpretive
lisp environment (the ``listener''), object-oriented programming, and
the extensibility provided by incremental compilation.

The basic functionality of OBVIUS is to present certain lisp objects
to the user pictorially.  These objects are refered to as
``viewables''. \index{{\ptt viewable}} Some examples of viewables are
monochrome images, color images, binary images, complex images, image
sequences, image pyramids, filters and discrete functions.  A
``picture'' \index{{\ptt picture}} is the displayable representation of a
viewable.  Note that a given viewable may be compatible with several
different picture types.  For example, a floating point image may be
displayed as an eight bit grayscale picture, as a one bit dithered
picture, or as a 3D surface plot.  OBVIUS also provides postscript
hardcopy output of pictures.

In the typical mode of interaction, the user types an expression to
the lisp listener and it returns a viewable, which is automatically
displayed as a picture in a window.  Each window contains a circular
stack of pictures.  Standard stack manipulation operations are
provided via mouse clicks (e.g., cycle, pop, and push).  Commonly used
operations such as histogram and zoom are also provided via mouse
clicks.

OBVIUS provides a library of image processing routines (e.g., point
operations, image statistics, convolutions, fourier transforms).  All
of the operations are defined on all of the viewable types.  The
low-level floating point operations are implemented in C for speed.
OBVIUS also provides a library of functions for synthesizing images.
In addition, it is straightforward to add new operations and new
viewable and picture types.

OBVIUS currently runs on Sun workstations in Lucid v4.0 Common Lisp.
We hope to have it running on other machines in the future (e.g.,
DecStations, SGI workstations, Apple Macintoshes, etc).  

This manual describes the overall organizational strategy of the
system, and provides a description of the user interface.  Throughout,
we assume the reader is familiar with Emacs, Common Lisp, and some
form of object-oriented programming environment.  The manual is
written primarily for Sun users -- users of other machines should see
the Appendices for machine dependencies.  Section \ref{sec:start} and
the OBVIUS tutorial \index{tutorial} (in the file {\tt
<obv>/tutorials/obvius/obvius-tutorial.lisp}) are provided for new
users to aquaint themselves with the system without getting bogged
down in hairy implementational details.  The middle sections
\ref{sec:organization} through \ref{sec:userface} are a more detailed
description of the system for more experienced users.  Section
\ref{sec:hacking} is provided for programmers (``hackers'').

\newpage

\section{Getting Started}
\label{sec:start}

We assume in this section that you have installed OBVIUS on your
system (if you have not yet created an OBVIUS image, instructions for
doing so are given in appendix~\ref{sec:install}).  We also assume
that you are running the X window system (version 11, release 4 or
higher), or Sun's OpenWindows (version 2.0 or higher).  You may use a
window manager of your choice.  We have used OBVIUS with ``twm'',
``uwm'', ``mwm'', and ``olwm''.

We recommend that you run OBVIUS from within the Gnu emacs editor,
version 18.49 or later (Gnu emacs is avaiable for anonymous ftp from
prep.ai.mit.edu).  The tutorial (in the file {\tt
<obv>/tutorials/obvius/obvius-tutorial.lisp}) assumes that you are
doing this.  OBVIUS can, however, be run from a Unix shell.

If you are running from Emacs, you should add the following
lines to your ``{\tt .emacs}'' startup file:
\begin{verbatim}
(setq load-path (cons <obv>/emacs-source load-path))
(setq *cl-program* <lisp-image>)
(autoload 'run-cl "cl-shell" nil t)
(setq *obvius-program* <obvius-image>)
(autoload 'run-obvius "cl-obvius" nil t)
\end{verbatim}
The symbol {\tt <obv>} is used throughout this document to indicate the
pathname of the primary OBVIUS directory (on our system, this is ``{\tt
/usr/local/obvius}'').  The symbol {\tt <lisp-image>} refers to the
pathname for a Lucid 4.0 image that includes CLOS and LispView.  On
our system, this is in ``{\tt /usr/local/bin/lispview}''.

To start lisp, you type {\tt M-x run-cl} from within Emacs.  To start
OBVIUS, you type {\tt M-x run-obvius}. After several seconds, a new
Emacs buffer will appear named {\tt *lisp*} and various system loading
messages will be printed into the buffer.  Eventually, you will get
OBVIUS's prompt: {\tt OBVIUS>}.  This buffer provides an interface to
Lisp that is running as a sub-process of Emacs.  As described in the
next section, hitting the {\tt cr} key in this buffer will send text
to the Lisp process for evaluation, and Lisp output will be printed
into this buffer.  

At this point, new users go through the OBVIUS tutorial in the file
{\tt <obv>/tutorials/obvius/obvius-tutorial.lisp}.  The purpose is to
familiarize new users with the basic functionality of OBVIUS, and with
the user interface.

The remaining part of this section describes customization of your
OBVIUS environment, and is unnecessary for novices.  At run-time,
OBVIUS looks for two startup files in your home directory.  These can
be used for personal customization of your environment.  The first is
named either ``{\tt .obvius-windows}'' or ``{\tt
obvius-window-init.lisp}''.  This file loads the appropriate window
system code.  The default version is in ``{\tt
<obv>/lucid-window-init.lisp}''.

The second startup file is called either {\bf .obvius} or {\bf
obvius-init.lisp}.  You can use this file to create a fixed number of
OBVIUS windows in predefined locations, to initialize default values
for global parameters and to have code loaded automatically.  The
default file is in ``{\tt <obv>/lucid-obvius-init.lisp}''.  This file
contains many (commented out) function calls that you may wish to add
to your personal initialization file.


%%%%%%%%%%%%%%%%%%%%%%%%%%%%%%%%%%%%%%%%%%%%%%%%%%%%%%%%%%%%%%%%%%%%%%%%%%%%%%%%%%
\section{Organization of OBVIUS}
\label{sec:organization}

As described in the introduction, OBVIUS is based on three types of
object: viewables, pictures, and panes.  This section covers these
objects in more detail and also explains the memory management system.
We assume that you have gone through the exercizes of the OBVIUS
tutorial.

%%%%%%%%%%%%%%%%%%%%%%%%%%%%%%%%%%%%%%%%%%%%%%%%%%%%%%%%%%%%%%%%%%%%%%%%%%%%%%%%%%

\subsection{Viewables}

\ldef{viewable}s are the objects that one works with in OBVIUS.
In this section, we describe the general
attributes of viewables.  For a discussion of the particulars of each
viewable subclass, see section
\ref{sec:viewables}.  Every viewable (simple or compound) contains the
following slots:\index{{\ptt viewable},slots of}

\begin{itemize}

\item \lsym{name} -- the name of the viewable can be either a string,
a symbol, or nil.  If the name is bound to a symbol, there is a symbol
with that print-name bound to the viewable, i.e., the symbol may be
used to refer to the viewable.  The name of a viewable is usually set
using the standard lisp {\tt setq} function or using the {\tt :->}
(result) keyword argument.  
NOTE: Using (setf slot-value) to set the
name slot will cripple the ability of OBVIUS to keep track of the
existing viewables.  For this reason, the name slot accessor is not
exported, and should not be used.

\item \lsym{superiors-of} -- Some viewables contain other viewables.
These are known as ``compound'' viewables.  For example, a complex
image contains two images (real and imaginary parts).  Viewables
containing other viewables are called {\em compound} viewables.  Each
compound viewable knows how to access its ``inferior'' viewables, and
each of the inferior viewables has a pointer to the ``superior''
(compound) viewable.  This slot contains a list of all viewables
containing the viewable as a sub-viewable.  For instance, if the
viewable is the real-part of a complex-image, the list will contain
that complex-image.

\item \lsym{display-type} -- this slot contains the name of the 
sub-class of picture \index{{\ptt picture}} that will be used when the viewable is
automatically displayed.  A nil value indicates that the viewable
should {\em not} be displayed automatically.  The slot may be set using the
standard setf syntax: {\tt (setf (display-type \abox{viewable})
'\abox{picture-type})}

\item  \lsym{pictures-of} -- this slot contains a list of all of the
pictures of the viewable.  There may be several different types of
picture of the same viewable, and there may be more than one of a
given picture type.  For example, there may be a grayscale picture
{\em and} a surface plot of an image, or there may be two different
grayscale pictures.

\item \lsym{history} -- this slot contains a history object for the
viewable.  See the section below for more information.

\item \lsym{current} -- this slot contains a number that is incremented
each time a viewable is modified.  It is used to keep track of whether
the pictures of that viewable are up-to-date or not.  There is a
similar slot in the picture class.

\item \lsym{info-list} -- this slot contains a lisp ``plist'' (property
list) of attributes.  For example, if the mean of an image is
computed, it is stored in this p-list so it does not need to be
recomputed.  The same is true for histograms, minimum, maximum, and
other image statistics.  The info-list is set to nil when the viewable
is destructively modified (this is done by the {\tt set-not-current} method).
\end{itemize}

The following functions may be used to access the info-list of a viewable:
\begin{itemize}
\item\lfun{info-list}{ viewable}
Returns the info-list of an image.

\item\lfun{info-get}{ viewable slot}
Returns the value of the slot named {\em slot} in im's info-list.  May be
used with setf to set the value instead of using info-set.

\item\lfun{info-set}{ viewable slot value}
Sets the value of the slot named {\em slot} in im's info-list to have
the value {\em value}.  You can also use {\tt (setf (info-get vbl
slot) val)}.

\end{itemize}

\mysubsubsec{Displaying Viewables}

\index{{\ptt viewable},auto display}
When a viewable is returned to the top-level lisp listener, it will be
displayed automatically if the global parameter
\lsym{*auto-display-viewables*} is non-nil.  If a picture 
\index{{\ptt picture}} of the viewable
already exists, it will be brought to the top of the stack of the
current pane \index{{\ptt pane}} if the variable
\lsym{*preserve-picture-pane*} is nil.  If this variable is non-nil,
then the picture will stay on the same pane, but still be brought to
the top of the stack.  Otherwise, a picture is created of the type
defined by the display-type slot of the viewable and pushed onto the
picture stack of the current pane.  If you wish to explicitly create a
picture of a viewable with a particular set of display parameters, use
the
\lsym{display} function (described in Section \ref{sec:pictures}).
Section \ref{sec:pictures} also describes methods for altering the
parameters of pictures.

\mysubsubsec{Modifying Viewables}

\index{{\ptt viewable},destructive modification}
When a viewable is destuctively modified (e.g. by passing it as the
\lsym{:->} argument to a function) the method \lsym{set-not-current} is
called.  This method tells OBVIUS that the pictures that depend on
that viewable need to be recomputed.  In general, the method does
nothing to the viewable, but calls itself on each of the superiors of
the viewable, and on each of the pictures of the viewable.  The
pictures then either update themselves immediately, or indicate to the
user that they are ``not current'' by placing a double asterisk in the
title bar (this is consistent with the double asterisk Emacs uses to
indicate a buffer has been modified).  The display may be updated by
refreshing the pane via a mouse click (see Section
\ref{sec:mouse}).  Note that modifying a viewable does {\em not} cause
related viewables to be re-computed.  For example, if a histogram is
computed from an image and the image is subsequently modified, the
histogram will not be recomputed.

\mysubsubsec{Destroying Viewables}

A viewable may be destroyed by evaluating the expression {\tt (\lsym{destroy}
\abox{viewable})} or via a mouse click (see Section
\ref{sec:mouse}).  The function \lsym{purge!} destroys all viewables.
When a viewable is destroyed, the following things occur:
\begin{enumerate}
\item Superiors of the viewable are notified.  Note that some
superiors will not allow their inferiors to be destroyed (e.g. a
complex image will trigger a continuable error when the user tries to
destroy its real part).
 
\item Inferiors of the viewable are notified.  Typically, the
inferiors just take the superior off of their {\tt superiors-of}
lists.  If the global parameter
\lsym{*auto-destroy-orphans*} is non-nil, the inferior is destroyed
(see section on orphaned viewables below).  

\item All pictures of the viewable are destroyed (i.e., the {\tt
destroy} method is called on each picture).  The static space used by
the pictures is freed (see section \ref{sec:memory}).

\item The static space of the viewable is freed (see Section
\ref{sec:memory}).

\item If the viewable name slot contains a symbol, it is unbound in
the current package of the global Lisp environment.

\item The name of the viewable is set to the keyword
\lsym{:destroyed}, as an indicator to the user in case they try to
re-use the viewable.
\end{enumerate}

\mysubsubsec{Orphans}
\index{orphans}
A viewable is called an ``orphan'' if it is no longer accessible.  As
far as OBVIUS is concerned, a viewable is considered inaccessible if
it has no superiors, there are no pictures of it, and there is no
global symbol bound to it.  When you run the OBVIUS garbage collector,
\lsym{ogc}, it will recycle all orphaned viewables.  Furthermore, if
the global variable \lsym{*auto-destroy-orphans*} has a non-nil value,
OBVIUS will attempt to destroy orphans when they are created.  Thus,
users of OBVIUS should understand when they are creating orphans!

There are three situations when OBVIUS can prevent the ``orphan'ing'' of
an existing viewable::
\begin{enumerate}
\item The (symbol) name of the viewable is (re-)bound (using {\tt setq} or
{\tt set}) to another value.
\item The only picture of the viewable is popped from a pane (destroyed).
\item The viewable is an inferior of a compound viewable, and that
compound viewble is destroyed.
\end{enumerate}
In each of these cases, if the global parameter
\lsym{*auto-destroy-orphans*} is non-nil, the orphan will be
automatically destroyed.  

{\em Note that this can be dangerous.} For example, if a user writes
code that returns an inferior viewable from within a function, but the
function destroys the superior, the inferior may be destroyed {\em
before} it is returned to top level.  Note also that automatic
destruction of orphans {\em will not} prevent the user from creating
new viewables that are orphans.  For example, one can create a list of
images with no name and avoid dislaying them.  These images will be
orphans and will not be preserved by the OBVIUS garbage-collector
({\tt ogc}), although the user will have access to them.  If you want
to return several images from a function, you should return them as an
image-sequence, or using the Common Lisp multiple-values facility.
See section~\ref{sec:hacking} for a more thorough discussion of this
problem.

\mysubsubsec{History and Constructor}
The \lsym{constructor} function produces a lisp S-expression that
would evaluate to a copy of the viewable.  When you save a viewable to
a datfile (see section~\ref{sec:datfile}, this expression is written
to the descriptor file.  It is also useful as a reminder of where the
viewable came from.  The function {\tt (\lsym{history-sexp} <viewable>
\&key (level 1))} allows the user to print a condensed listing of the
history, substituting symbols for the viewables with symbol name
slots.  The {\tt :level} parameter determines how many nested levels
of history (function calls) will be displayed before the system
attempts to replace them with symbols.  These functions are defined in
the file {\bf viewable.lisp}.


%%%%%%%%%%%%%%%%%%%%%%%%%%%%%%%%%%%%%%%%%%%%%%%%%%%%%%%%%%%%%%%%%%%%%%%%%%%%%%%%%%

\subsection{Pictures and Panes}
\label{sec:pictures}

A \ldef{pane} is a window that contains a stack of pictures.  The
picture on top of the stack is the one that is displayed on the
screen.  Each picture is allowed to reside on the stack of only one
pane.  To create a new pane, the user should call the function 
\lsym{new-pane}. The keyword arguments :left, :bottom, :right, and :top
specify the position in pixels of the lower left and upper right
corners of the pane.

The system maintains a notion of the {\em selected} or {\em current}
pane.  In general, clicking the mouse in a pane will select that pane
(see section \ref{sec:mouse}).  Whenever a viewable is automatically
displayed by the system, it is displayed in the current pane.

A \ldef{picture} is a particular visual representation of a viewable.

\mysubsubsec{Picture Slots and Default Values}
The following slots are found in every picture type:
\begin{itemize}
\item {\bf pane-of} -- the pane in which this picture lives.  Pictures
cannot exist in isolation; they must be on a picture stack.

\item {\bf viewable} -- the viewable of which this is a picture.

\item {\bf system-dependent-frob} -- a window-system dependent
intermediate representation.  The window system knows how to rapidly
``render'' the ``frob'' into a pane.  For gray pictures on X-screens,
the frob is an offscreen pixmap.  For graphs with retain-bitmap
non-nil, it is an offscreen bitmap.  But for graphs with retain-bitmap
nil, it is an object containing lists of graphical objects (such as
lines) to draw.

\item {\bf index} -- a unique registration number identifying the picture.

\item {\bf current} -- a number that is is used to
keep track of whether the picture is up-to-date with its viewable
(given that the viewable may have been modified since the picture was
originally made).  The current slot is reset each time the picture is
updated.  There is a similar slot for the viewable class.

\item {\bf zoom} -- magnification of the picture, relative to its
``natural'' size (this is described in detail below).

\item {\bf overlays} -- this slot contains a list of the viewables that
are overlayed on this picture (see section \ref{sec:overlays}).

\item {\bf y-offset} and {\bf x-offset} -- these slots specify the
position of the picture within the pane (relative to centered position).

\end{itemize}

In addition to those listed above, each picture class has parameters
that control its computation from the underlying viewable.  For
example gray pictures have \lsym{scale} and \lsym{pedestal} parameters
that specify the transformation of the floating-point image array into
a grayscale colormap.  The values of these parameters may be changed
using the {\tt Current Picture} dialog from the {\tt Parameters} menu
on the contral panel, or using the {\tt M-left} mouse click.  The
parameter values may also be reset using the
\lsym{setp} macro, which acts on the picture on top of the current
pane.  The {\tt setp} macro is described in detail below.

When a new picture is created, default values are chosen for the
picture parameters.  The default slot values for each picture class
may be set from the {\tt Picture Defaults} sub-menu of the {\tt
Parameters} menu.  Typically, the default value may be a number or it
may be the keyword{\tt :auto}.  If it is a number, that number will be
used as an initial slot value for all pictures of this type.  If the
default value is not a number, then OBVIUS will choose parameter
values automatically.

For details about the parameters for each picture class, see the
descriptions in Section \ref{sec:viewables}.

\mysubsubsec{Manipulating Pictures and Panes}

In general, the user should not need to directly manipulate pictures
or panes.  OBVIUS is designed with the intention that the user perform
computations on {\em viewables}.  There are, however, several methods
provided that allow the user to alter pictures:
\begin{description}
\item\lsym{*current-pane*} \\
This global symbol is bound to the current pane.  This symbol can be
used when calling functions that take a pane as an argument.

\item\lfun{display}{ \&optional display-type \&key pane make-new \&allow-other-keys}
This function is used to create a picture of a viewable.  The optional
{\tt display-type} argument should be a symbol corresponding to the
name of a picture subclass (eg. {\tt 'gray}, or {\tt 'vector-field}).
The default value is t, meaning to use the \lsym{display-type} slot of
the viewable.  The {\tt :pane} keyword argument specifies the pane in
which to display the picture and defaults to the value of
\lsym{*current-pane*}.  If this is nil, a new pane is created.  The
{\tt display} function first tries to find an existing picture of the
desired type.  If no picture of the right type exists, or if a non-nil
value is passed with the keyword \ldef{:make-new}, a new picture is
created with all default parameter values.  Otherwise, one of the
existing pictures is displayed, and updated if it is out-of-date and
\lsym{*auto-update-pictures*} is non-nil.  
All additional keyword args ({\tt picture-initargs}) are used as
initialization arguments in creating the picture (i.e., they are
passed along to {\tt make-instance}).  They typically correspond to
initargs for the slots of the picture being created.  For example, you
can create a gray picture of an image, with the scale factor set to
1.0, and the zoom factor set to fill the pane, as follows:
\begin{verbatim}
(display <image> 'gray :scale 1.0 :zoom :auto)
\end{verbatim}

\item\lfun{refresh}{ \&optional pane}
This function re-displays the picture on top of the specified pane.
The optional {\tt pane} argument defaults to the value of
\lsym{*current-pane*}.  This function is typically called via a mouse
click (see Section \ref{sec:mouse}).

\item\lfun{next-pane}{}
This function sets the current pane to be the next pane in the list of panes.

\item\lfun{cycle-pane}{ \&optional pane number}
This function cycles the circular picture stack of a pane and
refreshes the display.  The default pane is the value of
\lsym{*current-pane*}.  The optional {\tt number} argument defaults to
a value of 1, and specifies how many pictures to cycle past.  Negative
values cycle in reverse. This function is typically called via a mouse
click (see Section \ref{sec:mouse}).

\item\lfun{current-viewable}{}
Returns the viewable currently visible in the current pane.

\item\lfun{get-viewable}{ \&optional pane}
Returns an S-expression that will evaluate to the viewable currently
visible on top of the pane.  Optional argument {\tt pane} defaults to the value of
\lsym{*current-pane*}.   This function is typically called via a mouse
click (see Section \ref{sec:mouse}).

\item\lfun{pop-picture}{ \&optional pane}
This function removes the picture on the top of a picture stack and
destroys it.  The underlying viewable, however, is left intact.  The
default pane is the current pane.  It is usually called via a mouse
click (see Section \ref{sec:mouse}).

\item\lfun{setp}{ \&rest args}
It is often desirable to modify the parameters that are associated
with the various picture types.  The macro setp has been provided for
this purpose.  The syntax for setp is just like setq except that all
of the symbols to be set should correspond to slot names in the
picture on top of the current pane.  After setting the new slot
values, the picture will be recomputed using the new values and the
display will be refreshed.  Note that some picture slots are not
settable; calling setp on these will have no effect.  Picture
parameters can also be reset using the {\tt Current Picture} option
from the {\tt Parameters} menu.

\item\lfun{getp}{ \&optional slot-name}
This macro is complementary to setp.  It takes a single argument that
must be the name of a slot in the current picture and returns the
value of that slot.  When called with no arguments, it prints a list
of the names and values of {\em all} of the slots of the current
picture.  
\end{description}

\mysubsubsec{Zoom}

Every picture class has a {\tt zoom}\index{{\ptt zoom},slot} parameter that
controls the size of the picture in the pane.  The zoom amount is
relative to the ``natural'' size of the picture (this depends on the
type of picture).  The {\tt zoom} parameter may be set to {\tt
:auto}\index{{\ptt :auto},zoom}, a number, or a list of two numbers.  If it
is {\tt :auto}, then OBVIUS sizes the picture to fill the pane.  If it
is a list, then OBVIUS sizes the picture to have dimensions equal to
that list.  If {\tt zoom} is a number, then the size of the picture
will be this multiple of its natural size.  For example, if a 64x64
image is displayed as a gray picture with zoom = 2, then the picture
will by 128x128 pixels.  If zoom = 1/2, then the picture will be 32x32
pixels.  For some pictures (e.g., gray), zoom will be rounded to the
nearest integer, or reciprocal of an integer.  For others (e.g.
graph), it can take a floating point value.

The value of the zoom parameter may be changed from the {\tt Current
Picture Parameters} menu, using the {\tt setp} macro, or by using the mouse
(see section~\ref{sec:mouse}).

% The default value for {\tt zoom} may be set separately for each
% picture class from the {\tt Picture Defaults} submenu of the {\tt
% Parameters} menu.  If the default is {\tt nil}, then OBVIUS chooses a
% value automatically each time a new picture is made.  Typically, the
% automatic choice is {\tt zoom} = 1.

%%%%%%%%%%%%%%%%%%%%%%%%%%%%%%%%%%%%%%%%%%%%%%%%%%%%%%%%%%%%%%%%%%%%%%%%%%%%%

\subsection{File System Interface}
\label{sec:datfile}

OBVIUS uses a subset of the {\em datfile}\index{datfile} image file format used by
several groups at the Media Lab and in other labs at MIT.  A datfile
is a directory containing one or more files.  One of these files,
named ``data'', contains the data organized in a rectangular matrix
with any number of dimensions.  A separate file in the datfile
directory, named ``descriptor'', contains a pseudo-LISP description of
the data file.  Keyword/value pairs of ASCII strings, enclosed in
matching parentheses, describe the organization, data type, history
and other aspects of the datfile.  The advantage of this scheme is
that information about the image is kept separate from the image data
itself, and is human-readable and easily modified.  No special-format
headers are required.

A typical descriptor file looks like the following (the corresponding
data file is 16K bytes):
\begin{verbatim}
(_data_type unsigned_1) 
(_data_sets 1) 
(_channels 1) 
(_dimensions 256 256) 
\end{verbatim}
The most important keywords are {\tt \_data\_type} and {\tt
\_dimensions}.  The {\tt \_dimensions} keyword specifies the size of the
image (y-dimension is first).  The {\tt \_data\_type} keyword
in the descriptor file specifies the type of the data file elements,
and may take any of the following values (on the left is the
{\tt \_data\_type} specification in the descriptor file, and on the right is
the Common Lisp type specifier):
\begin{itemize}
\item unsigned\_1 = '(unsigned-byte 8)
\item unsigned\_4 = '(unsigned-byte 32)
\item float = 'single-float
\item ascii = 'character
\end{itemize}
The {\tt \_data\_sets} and {\tt \_channels} keywords are currently not
used by OBVIUS, but they must both be specified in the descriptor file
and equal to $1$.  In the future, {\tt \_channels} may be used for color
images (3 channels in an rgb image stored consecutively in the same
data file) and {\tt \_data\_sets} will be used for image sequences (multiple
images from an image sequence stored in the same data file).  

Currently, OBVIUS stores {\tt image-sequences} as a datfile directory
with one descriptor file and several data files named ``data0'',
``data1'', etc.  The keyword {\tt \_data\_files} in the descriptor
file specifies how many data files are present.

The function \lsym{load-image} (see Section \ref{sec:operations}) loads
images from datfiles and the function \lsym{save-image} saves images to
datfiles.  The {\tt load-image} function will also load images from a
file that is a stream of bytes in row-major order with one byte per
pixel.

OBVIUS is designed to load images in various machine formats.  Some
machines (e.g., Suns and Symbolics) store 32-bit numbers in reversed
byte order from each other.  OBVIUS uses the {\tt :byte-order}
keyword in the descriptor of a datfile to determine the byte-order of
the file.  If necessary, {\tt load-image} will reverse the order of
the bytes as it reads in a 32bit image file.  OBVIUS is not currently
set up to load different floating point formats.  Rather, you should
use ascii or 32-bit file format if you have different machines at your
site.

The following describes the functions for load and saving images and
image-sequences:
\begin{description}
\item\lfun{load-image}{ path \&key ysize xsize reverse-bytes \res}
If path is a directory, load-image assumes that path is a datfile, and
{\tt :ysize} and {\tt :xsize} are ignored.  Otherwise it assumes that
path is an image file stored in row-major order with one byte per
pixel.  In such a case the keywords {\tt :ysize} and {\tt :xsize} may
be used to specify the size of the image.  If they are not specified
then load-image assumes that the image is square.  For a datfile,
OBVIUS uses the information in the descriptor file to determine the
element-type, the byte-order, and the dimensions of the data.

First, {\tt load-image} converts the incoming data into floats.  Then
it uses a scale and pedestal, if specified in the descriptor file, to
rescale the image data.

The {\tt :reverse-bytes} argument is only relevant for 32-bit
datfiles.  If {\tt :reverse-bytes} is nil then OBVIUS uses the
byte-order keyword in the descriptor file to determine whether or not
to reverse the bytes.  If the {\tt :reverse-bytes} is non-nil, then
load-image ignores always reverses the bytes (ignoring the byte-order
keyword in the descriptor file).

If :\res is not specified and the global {\tt
*auto-bind-loaded-images*} is non-nil, then the image will be named
(see Section \ref{sec:viewables}) with the file-name or directory-name
of path (not the whole path-name).  See the tutorial for a simple
example.

The {\tt load-image} function calls the {\tt load-image-sequence}
function if the {\tt \_data\_files} keyword is present in the
descriptor file.

\item\lmeth{save-image}{image}{ image path \&key element-type auto-scale reverse-bytes df-keys}
Saves image as a datfile in path.  The {\tt :element-type} keyword specifies
what type of file to save it as.  {\tt :Element-type} may be '(unsigned-byte
8), '(unsigned-byte 32), `single-float, or 'character.  The default is
'(unsigned-byte 8).  WARNING: Floating point word formats are
different for different manufacturers, so it may not be possible to
load an image that was saved from a different machine (use 'character
or '(unsigned=byte 32) instead of 'float). 

If {\tt :element-type} is '(unsigned-byte 32) then the image is
rescaled to fill the 32bit range.  The appropriate scale and pedestal
are written to the descriptor file so that the 32-bit numbers will be
rescaled correctly when the image is reloaded.  If the reverse-bytes
keyword is T (default is nil) and the {\tt :element-type} is
'(unsigned-byte 32), then the byte-order is swapped during write.

If {\tt :element-type} is '(unsigned-byte 8) and {\tt :auto-scale} is
NIL then no rescaling is done.  If {\tt :auto-scale} is T (the
default) then the image is rescaled to fill the range from 0 to 255,
and the appropriate scale and pedestal are written to the descriptor
file so that the 8-bit numbers will be rescaled correctly when the
image is reloaded.

\item\lfun{load-image-sequence}{ path \&key start-index end-index \res}
Loads an image sequence from a datfile directory with multiple data
files name ``data0'', ``data1'', etc.  The keywords :start-index and
:end-index may be used to load a subsequence.  The load-image function
calls the load-image-sequence function if the \_data\_files keyword is
present in the descriptor file.

\item\lmeth{save-image}{image-sequence}{ sequence path \&key element-type auto-scale reverse-bytes df-keys start-index end-index}
Saves a sequence as a datfile directory with multiple data files.  The
keywords are the same as for saving images.  The :start-index and
:end-index keywords may be used to save a subsequence.
\end{description}

%%%%%%%%%%%%%%%%%%%%%%%%%%%%%%%%%%%%%%%%%%%%%%%%%%%%%%%%%%%%%%%%%%%%%%%%%%%%%%%%%%

\subsection{Memory Management}
\label{sec:memory}

Lisp memory space is divided into several different areas.  The most
important distinction is between the dynamic and static areas.  All
user-defined storage space normally resides in the ``dynamic'' areas.
Lucid Lisp provides several functions for examining and manipulating
the dynamic Lisp memory space from top level.  A complete explanation
is available in the Lucid Common Lisp User's Guide.  A few very useful
functions are:
\begin{description}
\item\lfun{change-memory-management}{ \&key growth-limit expand ...}
If a number is supplied for the {\tt :growth-limit} keyword, it specifies
the maximum Lisp process size in 64k segments.  If a number is
supplied for the {\tt :expand} keyword, the Lisp dynamic space is expanded
by that many 64k segments.

\item\lfun{room}{ \&optional info}
This function prints out information about the current state of the
lisp memory map.  If the optional argument {\tt info} is t, more
information is given.
\end{description}

In order to avoid garbage collecting large arrays such as image
arrays, OBVIUS allocates storage for viewables and pictures from the
static heap.  {\em Since these arrays are never garbage collected,
they must be returned explicitly to the system}.  OBVIUS maintains
separate ``heaps'' of static storage space for different array types
(e.g., 'single-float, 'bit, etc).  The memory allocator is designed to
create heaps for any element-type that is a valid argument to the
Common Lisp {\tt make-array} function.  Arrays are allocated from and
returned to the static heaps with the functions {\tt allocate-array}
and {\tt free-array} as described in Section~\ref{sec:hacking}.

Sometimes in the midst of a computation, OBVIUS will need to allocate
an array from one of the static heaps that has no more free space.
The global parameter \lsym{*auto-expand-heap*} controls the behavior in
this situation.  If {\tt *auto-expand-heap*} is non-nil, then OBVIUS
will automatically call the {\tt expand-heap} function (described
below) to increase the size of the appropriate heap.  The amount of expansion is
controlled by the variable \lsym{*heap-growth-rate*}, which should
contain an integer specifying the number of {\em elements} to add to
the heap.  If {\tt *auto-expand-heap*} is nil, then OBVIUS will signal
a continuable error: you can either abort the computation or enter
{\tt :c} to expand the heap by {\tt *heap-growth-rate*} elements.

Several functions are provided for top-level management of the static
heaps:
\begin{description}

\item\lfun{expand-heap}{ type \&optional size}
This function expands the size of OBVIUS's static array area, for the
given type of element.  The type argument can be any array type.
Examples are: 'single-float, '(unsigned-byte 8), and 'bit.  The
optional parameter {\tt size} specifies the number of elements by
which to expand the respective space.  It defaults to the value of the
global parameter {\tt *heap-growth-rate*}.

\item\lfun{heap-status}{}
Prints a description of the current static memory usage.

\item\lfun{ogc}{ \&key verbose}
A primitive garbage collector that offers a means of reclaiming lost
storage space (i.e., allocated arrays that belong to orphaned \index{orphans}
viewables -- see the previous section on orphans for an explanation).
Note that unlike the Common Lisp garbage collector, {\tt ogc} must be
called manually by the user.  {\tt Ogc} reclaims static arrays that do
not belong to: (1) pictures, (2) viewables of pictures, (3) viewables
that are bound to symbols in the current package or the {\tt USER}
package, or (4) inferior or superior viewables of the aformentioned
viewables.  All arrays that are not preserved are returned to the heap
for reallocation.  {\tt Ogc} also prints a list of symbols in the
current package that are currently bound to viewables, so that the
user may explicitly destroy these.  The former name for this
function was \lsym{scrounge!}, which has been retained for
back-compatibility.

{\bf Warnings:} this function should {\em only} be called from top
level, since it will not preserve viewables bound to local variables.
Furthermore, it only preserves symbols that are in the current package
(the name of the current package is the value of the variable {\tt
*package*}) or the {\tt USER} package.  Furthermore, if you have
created viewables that are not accessible through pictures, symbols,
or inferiors or superiors, {\em these viewables will be recycled by
the garbage collector}. 

\end{description}



%%%%%%%%%%%%%%%%%%%%%%%%%%%%%%%%%%%%%%%%%%%%%%%%%%%%%%%%%%%%%%%%%%%%%%%%%%%%%%%%%%
\section{Viewables}
\label{sec:viewables}

This section describes the various viewable subclasses.  To add new
viewable classes to OBVIUS, see section \ref{sec:hacking} of this
manual.  The currently defined viewables are are (indentation
indicates inheritance):
\begin{tabbing}
image \= \\
\> one-d-image \\
\> slice \= \\
\> bit-image \\
viewable-matrix \\
\> image-matrix \\
\> viewable-sequence \\
\> \> image- \= sequence \\
\> \> \> color-image \\
\> \> \> image- \= pair \\
\> \> \> \> one-d-image-pair \\
\> \> \> \> one-d-complex-image \\
\> \> \> \> complex-image \\
\> \> \> \> polar-image \\
\> \> complex-discrete-function \\
filter \\
\> separable-filter \\
\> hex-filter \\
discrete-function \\
\> histogram \\
\> log-discrete-function \\
\> periodic-discrete-function \\
pyramid \\
\> gaussian-pyramid \\
\> laplacian-pyramid \\
\> steerable-pyramid \\
\> qmf-pyramid \\
overlay-viewable \\
\> overlay-gray-viewable \\
\end{tabbing}

%%%%%%%%%%%%%%%%%%%%%%%%%%%%%%%%%%%%%%%%%%%%%%%%%%%%%%%%%%%%%%%%%%%%%%%%%%%%%%%%%%

\subsection{Images}
\label{sec:images}
\index{{\ptt image}}

This section describes images and the functions provided for their
manipulation.  OBVIUS has several other related types of viewables
(e.g., bit-image, image-pair, complex-image, polar-image,
image-sequence, and one-d-image) that are also described below.

The following files in the \abox{obv}  directory contain source
code that is relevant to images:
\begin{itemize}
\item {\bf image.lisp} --  contains the definition of the
image class, accessors for the slots of images, and standard methods
and utilities for images.
\item {\bf lucid-image-loops.lisp} -- contains macros such as 
\lsym{loop-over-image-pixels} for efficient image accessing and looping.
\item {\bf imops.lisp} -- contains the definitions for all
of the arithmetic methods on images as well as some other assorted
methods.  See Section \ref{sec:operations} for a complete list.
\item {\bf fft.lisp} -- contains a fast Fourier transform, a Hilbert transform,
and a discrete cosine transform.
\item {\bf synth.lisp} -- contains functions as described below for
generating synthetic images.
\end{itemize}

\mysubsubsec{Make-image}

The function \lsym{make-image} may be used to make a new image:
\begin{verbatim}
(make-image dims &key display-type name)
\end{verbatim}
This function returns a new image of the specified dimensions,
initialized to zeros.  The dims parameter must be a list of two
fixnums, the first specifying pthe y-dimension and the second
specifying the x-dimension.

{\bf Note}: The convention in OBVIUS is always to specify y first and
x second.  This is consistent with the row-major storage of arrays in
both Common Lisp and C.

If only one dimension is specified or if one of the dimensions is 1,
then {\tt make-image} will return a {\tt one-d-image}.  This is a
subclass of image that is displayed as a graph.  Otherwise, its
behavior should be the same as any other image.

The {\tt :display-type} argument must be the name of a picture
sub-class that is compatible with the image.  The default display-type
is either {\tt 'gray} or {\tt 'dither}.  The default may be set from
the {\tt Viewable Default} sub-menu of the {\tt Parameters} menu on
the control panel.

\mysubsubsec{Iref}

``Iref is to images as aref is to arrays''.  It returns the value of the
image at the specified location, and may be used with setf to modify the
image values.  For example,
\begin{verbatim}
(setf (iref im 20 30) 5.1)
\end{verbatim}
sets pixel {\tt (20,30)} to the value {\tt 5.1}.  NOTE: It is not
efficient to use \lsym{iref} in loops, since it is a method and
therefore has substantial dispatch-time overhead (relative to the
operation being performed).  Instead, you should either: 1)
restructure your algorithm to use existing OBVIUS functions, which
call C code, 2) write C code to perform your operation, 3) use the
looping macros \lsym{loop-over-image-pixels},
\lsym{loop-over-image-positions}, or with-images (see section
\ref{sec:hacking}), 4) write lisp code that extracts the data arrays
and uses aref to access them efficiently. \index{{\ptt iref},efficiency}

\mysubsubsec{Print-values}

The \lsym{print-values} function is used to generate a formatted printout
of the values in some part of an image.  For example,
\begin{verbatim}
(print-values image :y 0 :x 0 :y-size 10 :x-size 10)
\end{verbatim}
prints the top-left 10x10 pixels of the image.  The keyword 
parameters {\tt :x} and {\tt :y} specify the location of the upper-left
hand corner of the subimage to be printed, and {\tt :x-size} and {\tt
:y-size} specify the size of that sub-image.  If these keywords are
nil then the subimage is determined by the global parameters
\lsym{*x-print-range*} and \lsym{*y-print-range*}.  Each of these global
parameters is a list of two floats between 0 and 1.  For example if
\lsym{*x-print-range*} is '(0.0 0.5) and \lsym{*y-print-range*} is
'(0.5 1.0) then the lower left quarter of the image will be printed.
The subimage is subsampled so that no more than
\lsym{*max-print-vals*} samples are printed on each line.  The globals
{\tt *max-print-vals*}, {\tt *x-print-range*}, and {\tt
*y-print-range*} parameters may be reset from the {\tt Globals} dialog
box that is accessed from the {\tt Parameters} menu on the control
panel.

The {\tt print-values} function is also defined for arrays, and for
{\tt image-pairs}.  

\mysubsubsec{Pictures of Images}

Images may be displayed as {\tt gray} (see section \ref{sec:gray}),
{\tt dither} (see section \ref{sec:dither}), as {\tt
surface-plot} (see section \ref{sec:surface-plot}), or as 
{\tt contour-plot} (see section \ref{sec:contour-plot}).

%%%%%%%%%%%%%%%%%%%%%%%%%%%%%%%%%%%%%%%%%%%%%%%%%%%%%%%%%%%%%%%%%%%%%%%%%%%%%%%%%%

\subsection{One-d-images and Slices}
\label{sec:one-d-images}

One-d-images are really the same as images except that one of the
dimensions is 1.  So, all of the image operations that are defined for
images work on one-d-images as well.  The only difference is that a
one-d-image is (by defaults) displayed as a {\tt one-d-graph}.  A
one-d-image can be generated using make-image, e.g.,
\begin{verbatim}
(make-image '(100 1) :-> '1d-im)
\end{verbatim}
or any of the synthetic image generating functions.  The \lsym{slice}
viewable class is a subclass of the {\tt one-d-image} class.  The
one-d-image class and the slice class are defined in the file {\bf
image.lisp}.  \lsym{one-d-image}s are typically displayed as
{\tt graph} pictures (see section \ref{sec:graph}).  

%%%%%%%%%%%%%%%%%%%%%%%%%%%%%%%%%%%%%%%%%%%%%%%%%%%%%%%%%%%%%%%%%%%%%%%%%%%%%%%%%%

\subsection{Discrete Functions and Histograms}
\label{sec:discrete-function}

The viewables \lsym{discrete-function}, \lsym{periodic-discrete-function}, and
\lsym{log-discrete-function} are defined in the file {\bf
discrete-function.lisp}.  The class \lsym{histogram}s are a subclass
of {\tt discrete-functions}.  {\tt Discrete-functions} are also used
as lookup tables to do fast nonlinear point operations on images (see
the function \lsym{point-operation} in Section \ref{sec:operations}).

{\tt Discrete-functions} are essentially a lookup table of function
values --- they are made up of a {\tt data} array, an {\tt origin},
and an {\tt increment}.  The first entry in the {\tt data} array is
taken to be the value of the function evaluated at {\tt origin}, the
next entry is taken to be the value of the function evaluated at {\tt
(+ origin increment)}, etc.  The {\tt evaluate} function is used to
evaluate a discrete-function at a particular value.  If that value
falls between samples, then the result is interpolated.

{\tt Periodic-discrete-functions} are the same as {\tt
discrete-functions} except they are interpreted as one full period of
a periodic function.  If the periodic function is defined from 0 to 1,
evaluating at 1.5 is the same as evaluating it at 0.5.  {\tt
Log-discrete-functions} are sampled on a log axis.

The following functions are used to make and evaluate discrete
functions.  
\begin{description}
\item\lfun{make-discrete-function}{ func min max \&key size
interpolator endpoint name \res} Makes a discrete function of the
function {\tt func}, from {\tt min} to {\tt max}, with {\tt :size}
samples.  The default for {\tt :interpolator} is {\tt
'linear-interpolation} which does linear interpolation.  If {\tt
:endpoint} is non-nil then the discrete-function includes a sample at
{\tt max}.  Otherwise, the last sample is slightly below {\tt max}.
An example is given in the OBVIUS tutorial.

\item\lfun{make-periodic-discrete-function}{ func min max \&key size
interpolator endpoint name \res} 
The default {\tt :interpolator} {\tt 'periodic-linear-interpolation}
which does linear interpolation, modulo {\tt min} and {\tt max}.

\item\lfun{make-log-discrete-function}{ func min max \&key size
base interpolator endpoint name \res} 
The default for {\tt :base} is 10.  Example:
\begin{verbatim}
(make-log-discrete-function '(lambda (x) (log x 10)) 
                            0.01 10.0 :base 10 :size 4)
\end{verbatim}
The {\tt discrete-function} includes samples at .01, .1, 1, and 10.

\item\lmeth{evaluate}{discrete-function}{ discrete-function value}
Evaluates the discrete-function function at value.  Interpolates if
value is between two entries in the table.
\end{description}

{\tt Discrete-functions} are displayed by default as {\tt graph}
pictures (see section \ref{sec:graph}), but they may also be displayed
as {\tt polar-plot} pictures (see section \ref{sec:polar-plots}).

%%%%%%%%%%%%%%%%%%%%%%%%%%%%%%%%%%%%%%%%%%%%%%%%%%%%%%%%%%%%%%%%%%%%%%%%%%%%%%%%%%

\subsection{Bit-images}
\label{sec:bit-image}

The \lsym{bit-image} class is a subclass of the {\tt image} class.
{\tt Bit-images}, utilities for using {\tt bit-images}, and operations
on {\tt bit-images} are defined in the file {\bf bit-image.lisp} (see
section \ref{sec:bitmap}).  {\tt Bitmaps} are the only display-type
for {\tt bit-images}.  The threshold functions {\tt greater-than},
{\tt less-than}, {\tt equal-to}, {\tt greater-than-or-equal-to}, and
{\tt less-than-or-equal-to} discussed in Section \ref{sec:operations}
can be used to make {\tt bit-images} from images.

Bit-images are just like images except that they have only one bit of
data at each pixel (the bit-image data is stored using common-lisp bit
arrays).  So, some of the functions and accessors are inherited from
images (e.g., {\tt iref}, {\tt dimensions}, {\tt x-dim}, {\tt y-dim},
{\tt total-size}).  Other methods are functionally equivalent to those
defined on images (e.g., {\tt mul}, {\tt copy}, {\tt similar}, {\tt
crop}, and {\tt paste}).  Other methods are modified somewhat (e.g.,
add is modified to perform a logical-or operation on bit-images).  

%%%%%%%%%%%%%%%%%%%%%%%%%%%%%%%%%%%%%%%%%%%%%%%%%%%%%%%%%%%%%%%%%%%%%%%%%%%%%%%%%%

\subsection{Sequences}
\label{sec:sequence}

The classes \ldef{image-sequence} and \ldef{viewable-sequence} are defined
in the file {\bf sequence.lisp}.  

{\tt Image-sequences} may be used to work with temporal sequences of
images or 3d image data (e.g., volumetric data used in medical
imaging).  Since {\tt image-sequences} are made up of {\tt images},
all of the image operations that are defined for {\tt images} work on
{\tt image-sequences} as well.  Image Sequences may be displayed as
{\tt pasteup} pictures (see section \ref{sec:pasteup}), or as flipbook
pictures (see section \ref{sec:flipbook}).  

A {\tt viewable-sequence} consists of a number of sub-viewables, all
of the same type.  See the example in the OBVIUS tutorial.  General
(non-image) sequences can be displayed as {\tt flipbooks} or as {\tt
overlays}.

%%% *** Needs work!


%%%%%%%%%%%%%%%%%%%%%%%%%%%%%%%%%%%%%%%%%%%%%%%%%%%%%%%%%%%%%%%%%%%%%%%%%%%%%%%%%%

\subsection{Image Pairs, Complex Images, and Polar Images}
\label{sec:image-pair}

The file {\bf image-pair.lisp} contains the definitions of the
\ldef{image-pair}, \ldef{complex-image}, and \ldef{polar-image}
classes, that are compound images formed from a pair of standard
images.  {\tt Image-pairs} are used to store the left and right frames
of a stereo pair or the y- and x-components of a flow field.  {\tt
Complex-images} are used for Fourier transforms.  {\tt Polar-images}
store the magnitude and complex-phase of a {\tt complex-image}.  Since
{\tt image-pairs} are made up of {\tt images}, all of the image
operations that are defined for {\tt images} work on {\tt image-pairs}
as well.

{\tt Horizontal-pasteup} pictures are the default display type for
{\tt image-pairs}.  These are a subclass of \lsym{pasteup} pictures
(see Section \ref{sec:pasteup}).  {\tt Image-pairs} may also be
displayed as \lsym{vector-field}s (see section \ref{sec:vector-field}).

%%%%%%%%%%%%%%%%%%%%%%%%%%%%%%%%%%%%%%%%%%%%%%%%%%%%%%%%%%%%%%%%%%%%%%%%%%%%%%%%%%

\subsection{Filters}
\label{sec:filters}

The viewable subclasses \ldef{filter} and \ldef{separable-filter} are
defined in the file {\bf filter.lisp}.  Currently, all filters are
considered to be finite impulse response (FIR) filters -- infinite
impulse response (IIR) filters will be implemented in a future
release.  Display methods are provided to plot them as \lsym{graph}s
(for one-dimensional filters) or as \lsym{gray}s (for two-dimensional
filters).  The user may also create three-dimensional filters for
application to image sequences: these are currently not displayed.

Filters are made up of a kernel array, an edge-handler, and a grid.
For purposes of efficiency, filters are implemented to allow
subsampling of the output.  The grid slot specifies the subsampling to
be performed in the convolution (see make-filter examples below).  The
edge-handler specifies how to handle the edges of the image while
convolving.  Current choices for edge-handler are nil
(circular-convolution), and the keywords \lsym{:reflect1} (reflect the
image through the edge pixel),
\lsym{:reflect2} (reflect the image about the edge), \lsym{:zero}, and
\lsym{:repeat}.  Separable filters are compound viewables 
containing two filters.

The function \lsym{make-filter} is
used to make filters.  The first argument to these function should be a
vector, a list, or an array specifying the filter kernel.  Additional
keyword arguments may be provided: \lsym{:step-vector},
\lsym{:start-vector}, \lsym{:edge-handler}.  Examples:
\begin{verbatim}
(make-filter '(1.0 2.0 1.0))
(make-filter '((1.0 2.0 1.0) (2.0 4.0 2.0) (1.0 2.0 1.0)))
(make-filter 2darray :step-vector '(2 2))
\end{verbatim}
The first of these makes a 1d filter, and the second makes a 2d
filter.  The last example makes a filter that will downsample by a
factor of two when the \lsym{apply-filter} function is called (or
upsample by a factor of two when the \lsym{expand-filter} function is
called).

The function  \lsym{make-separable-filter}  may be used to create
filters that are applied separably (i.e., two one dimensional filters,
applied in the x- and y- directions).  This function takes two
filters, or two arrays, or two nested lists as arguments, along with
all of the keywords described above.  The first filter will be applied
in the vertical (y) direction.  You can also make higher dimensional
separable filters (although the dimensionality cannot exceed that of
the thing you are applying them to).  For example, you can make a
separable filter out of a 1D and a 2D filter, and apply it to an image
sequence.  The 2D filter may be separable or non-separable.  

The {\tt :edge-handler} argument determines what happens at the
borders of an image when the \lsym{apply-filter} or
\lsym{expand-filter} functions are called.  Any local linear
edge-handling scheme may be implemented.  Some examples are
implemented in the file {\bf \abox{obv}/c-source/edges.c}.  The
associated with each type of edge-handler is a C function that
computes a new filter (set of weights) to be used when computing one
of the edge samples of the convolution.  The arguments to this
edge-handler function are described in the file {\bf
\abox{obv}/c-source/edges.c}.

\mysubsubsec{3D filtering}
Three-dimensional filtering (on image-sequences) is also supported.
To make a 3D filter, just pass a 3D array (or list) to the {\tt
make-filter} function.  To make a 3D separable filter, you can call
{\tt make-separable-filter} with a 2D filter and a 1D filter.

\mysubsubsec{Hexagonal filtering}
Two-dimensional {\tt hex-filter}s are defined on a hexagonal lattice:
half of each row of the samples should be zero.  These are a special
class because the subsampling must be done differently.  They have not
been fully ported from OBVIUS-1.2 yet.

%%%%%%%%%%%%%%%%%%%%%%%%%%%%%%%%%%%%%%%%%%%%%%%%%%%%%%%%%%%%%%%%%%%%%%%%%%%%%%%%%%

\subsection{Pyramids}
\label{sec:pyramid}

The \ldef{gaussian-pyramid} module may be loaded into OBVIUS by evaluating:
\begin{verbatim}
(obv-require 'gaussian-pyramid)
\end{verbatim}
Pyramids are built with the function \lsym{make-gaussian-pyramid} and
reconstructed with the \lsym{collapse} function.  Various subbands may
be accessed with the \lsym{access-band} function.  Other pyramid
operations are listed in in section~\ref{sec:operations}.

Pyramids are currently undocumented.  For now, look at example in the
OBVIUS tutorial, and at the code in {\bf gaussian-pyramid.lisp}.

%%%%%%%%%%%%%%%%%%%%%%%%%%%%%%%%%%%%%%%%%%%%%%%%%%%%%%%%%%%%%%%%%%%%%%%%%%%%%%%%%%

\subsection{Color Images}
\label{sec:color-image}

The \ldef{color-image} class is defined in the file {\bf
color-image.lisp}.  {\tt Color-images} inherit from {\tt
image-sequences}, but the images in the sequence are interpreted as
different color bands (e.g., R, G, and B electron beam gun values).

This is unfinished code.  We are writing a full color analysis package
that keeps track of color-spaces, input devices (e.g., photoreceptors,
cameras, scanners), output devices (e.g., monitors, printers),
spectral sensitivities of these input and output devices, and spectral
properties of surface reflectances.  

For the time being, {\bf color-image.lisp} defines simple color images
that can be displayed as {\tt color-pictures} or as {\tt pasteups}
({\tt pasteups} are the default display type).

%%%%%%%%%%%%%%%%%%%%%%%%%%%%%%%%%%%%%%%%%%%%%%%%%%%%%%%%%%%%%%%%%%%%%%%%%%%%%%%%%%

\subsection{Viewable Matrices}
\label{sec:viewable-matrix}

The \lsym{viewable-matrix} and \lsym{image-matrix} are defined in the
file {\bf viewable-matrix.lisp}.  {\tt Viewable-matrices} contain an
arrays of sub-viewables.  The {\tt data} slot of a {\tt
viewable-matrix} is the array of sub-viewables.  A number of other
viewable types inherit from {\tt viewable-matrices} (e.g.,
\lsym{image-sequence}s are {\tt viewable-matrices} with a 1xN array of
images).

{\tt Viewable-matrices} are particularly useful for parallelizing
pixel-by-pixel matrix operations.  For example, this kind of operation
is used in OBVIUS to convert {\tt color-images} from one color space
to another.  For another example, let's say that you had six images.
At each pixel you want to take the corresponding values out of each of
the six images, put four of those values in a matrix and the other two
in a vector, and then compute the inverse of the matrix times the
vector.  Believe it or not, this type of operation comes up a lot in
visual modeling and image analysis, and it is very easy to do with
{\tt viewable-matrices}.


%%%%%%%%%%%%%%%%%%%%%%%%%%%%%%%%%%%%%%%%%%%%%%%%%%%%%%%%%%%%%%%%%%%%%%%%%%%%%%%%%%
\section{Pictures}

This section describes the various picture subclasses.  To add new
picture classes to OBVIUS, see section \ref{sec:hacking} of this
manual.  The currently defined picture classes are listed below.  Not
every picture is applicable to every viewable.  For example, {\tt
image}s may be displayed as {\tt gray} or {\tt surface-plot} pictures,
but not as {\tt graph}s.  The indentation indicates inheritance of
classes, and the viewables to which each picture type applies are
listed in parentheses.
\begin{tabbing}
gray \= (image) \\ 
\> pasteup \= (image-matrix, image-sequence, color-image, gaussian-pyramid, \\
\> \> laplacian-pyramid, image-pair, complex-image, polar-image) \\
\> dither (image) \\
color-picture (color-image) \\
bitmap (bit-image) \\
drawing \\
\> graph (image, discrete-function, histogram) \\
\> vector-field (image-pair) \\
\> surface-plot (image) \\
\> contour-plot (image) \\
\> polar-plot (discrete-function, image-pair, viewable-sequence) \\
\> scatter-plot (image-pair) \\
flipbook (viewable-sequence, image-sequence) \\
overlay (viewable-sequence) \\
\end{tabbing} 

%%%%%%%%%%%%%%%%%%%%%%%%%%%%%%%%%%%%%%%%%%%%%%%%%%%%%%%%%%%%%%%%%%%%%%%%%%%%%%%%%%

\subsection{Display}
\label{sec:display}

Pictures of a given type may be created using the {\tt display}
function, which is described in section~\ref{sec:pictures}.

\subsection{Gray Pictures}
\label{sec:gray}

The \lsym{gray} picture subclass is the usual display type for images.
Each gray picture has several parameters that control its computation
from the underlying image.  The parameter values of a {\tt gray}
picture may be changed using the {\tt Current Picture} dialog from the
{\tt Parameters} menu on the control panel.  The parameter values may
also be reset using the \lsym{setp} macro (described in section
\ref{sec:organization}).

The transformation from a (floating point) image pixel value to a gray
lookup table index is given by:
\begin{displaymath}
{\rm round}\left( \frac{x - p}{s} \right),
\end{displaymath}
where $p$ is the picture's \lsym{pedestal} parameter, $s$ is the
\lsym{scale} parameter.

The size of the gray lookup table is determined by the
\lsym{gray-shades} slot of the screen.  The gray lookup table is a
vector of indices into the screen colormap.  The indices are ordered,
from darkest to lightest.  The sequence of gray shades that they refer
to may not be linear: the {\tt gray-gamma} parameter of the screen
controls the set of gray values that are allocated.

OBVIUS will make use of the gray shades available.  The screen
parameter \lsym{gray-dither} is used to decide whether or not to
dither the gray pictures.  If its value is a number, and if there are
fewer than \lsym{gray-dither} grays in the color map, then OBVIUS will
automatically dither the picture.  For example, if there are 8 gray
shades allocated in the color map, and if \lsym{gray-dither} is
greater than 8, then OBVIUS will dither the viewable values using a
simple error-propagating dither algorithm into those 8 gray shades.
If the value of \lsym{gray-dither} is nil, then OBVIUS will never
dither.  If it is non-nil (but not a number), then OBVIUS will always
dither.  All of the screen parameters mentioned ({\tt gray-shades},
{\tt gray-dither}, {\tt gray-gamma}), may be adjusted from the {\tt
Current Screen} item on the {\tt Parameters} menu.

If the value of \lsym{scale} is set to {\tt :auto}\index{{\ptt :auto},scale},
then OBVIUS automaticaly uses the range (max minus min) of the image.
If the value of \lsym{pedestal} is set to {\tt
:auto}\index{{\ptt :auto},pedestal}, then OBVIUS automaticaly uses the
minimum of the image.  Since these are stored on the info list, they
only need be computed once.

In addition, each gray picture has \lsym{zoom}, \lsym{y-offset}, and
\lsym{x-offset} parameters (inherited from the general picture class)
that are described in Section \ref{sec:organization}.  

%%%%%%%%%%%%%%%%%%%%%%%%%%%%%%%%%%%%%%%%%%%%%%%%%%%%%%%%%%%%%%%%%%%%%%%%%%%%%%%%%%

\subsection{Dither Pictures of Images}
\label{sec:dither}

A \ldef{dither} picture is like a {\tt gray} picture, except that
OBVIUS forces it to only use two shades of gray (i.e., black and
white).  The dither class inherits from the gray class.  This is a
separate class for two reasons: 1) The user can request an image to be
displayed in 1-bit dithered format, even on a deeper screen, and 2)
the internal representation for dithers is more efficient (it uses a
1-bit deep bltable).

%%%%%%%%%%%%%%%%%%%%%%%%%%%%%%%%%%%%%%%%%%%%%%%%%%%%%%%%%%%%%%%%%%%%%%%%%%%%%%%%%%

\subsection{Bitmaps}
\label{sec:bitmap}

The \ldef{bitmap} picture subclass, defined in the file {\bf
gray.lisp}, are used for displaying {\tt bit-images}.  The parameters
of a {\tt bitmap} are:
\begin{itemize}
\item \lsym{foreground} specifies the foreground color (e.g., :red or
:green or any other color supported by LispView).
\item \lsym{background} specifies the background color (e.g., :red or
:green or any other color supported by LispView).
\end{itemize}

%%%%%%%%%%%%%%%%%%%%%%%%%%%%%%%%%%%%%%%%%%%%%%%%%%%%%%%%%%%%%%%%%%%%%%%%%%%%%%%%%%

\subsection{Color Pictures}
\label{sec:color-picture}

The \ldef{color-picture} class is defined in the file {\bf
color-picture.lisp}, and may be used for displaying {\tt
color-images}.  The {\tt color-picture} interprets the sub-images of a
{\tt color-image} as corresponding to R, G, and B electron beam gun
values.  The parameters of a color picture are:
\begin{itemize}
\item \lsym{scale} analogous to grays and pasteups.
\item \lsym{pedestal} analogous to grays and pasteups.
\end{itemize}
Example:
\begin{verbatim}
(progn
  (obv-require :color)
  (load-image "/images/clown")
  (change-class clown 'color-image))
(display clown 'color-picture)
\end{verbatim}
In this example, {\tt clown} is a datfile directory with three data
files corresponding to R, G, and B.  {\tt Load-image} loads it as an
{\tt image-sequence} by default, so we use {\tt change-class} to
convert it to a {\tt color-image}.  Then it can be displayed as a {\tt
color-picture}.

Unfortunately, LispView (and consequently OBVIUS) does not (yet)
support 24bit displays, so {\tt color-pictures} are dithered.  The
parameter \lsym{rgb-bits} specifies how many bits to use for dithering
each band.  This parameter may be set in the {\tt Current Screen}
dialog from the {\tt Parameters} button in the control panel.  

%%%%%%%%%%%%%%%%%%%%%%%%%%%%%%%%%%%%%%%%%%%%%%%%%%%%%%%%%%%%%%%%%%%%%%%%%%%%%%%%%%

\subsection{Pasteups}
\label{sec:pasteup}

The \ldef{pasteup} picture subclass are used for displaying {\tt
image-matrices}, {\tt image-sequences}, {\tt color-images}, {\tt
gaussian-pyramids}, {\tt laplacian-pyramids}, {\tt image-pairs}, {\tt
complex-images}, and {\tt polar-images}.  {\tt Pasteups} are defined
in the file {\bf pasteup.lisp}.

A \lsym{pasteup} is a way of laying out (pasting up) a number of
pictures so that they can all be seen simultaneously.  The {\tt
pasteup} class inherits from the {\tt gray} class, and thus these
objects have all of the parameters of {\tt grays}.  In addition, there
is an {\tt independent-parameters} slot that forces all of the
individual sub-pictures to have indentical rescaling parameters.  If
{\tt independent-parameters} is non-nil, then the images that make up
the {\tt image-sequence} are rescaled independently.  Otherwise, all
of the images are rescaled with the same {\tt scale} and {\tt
pedestal}.

The parameter values of a {\tt pasteup} picture may be changed using
the {\tt Current Picture} dialog from the {\tt Parameters} menu, or
using the {\tt setp} macro.  The default values (the values used to
initialize a new picture) for these parameters may be changed using
the {\tt Picture Defaults} menu.  The parameters are:
\begin{itemize}
\item \lsym{independent-parameters}.  If non-nil, then the images are
rescaled independently.

\item \lsym{scale} is ignored if {\tt independent-parameters} is
non-nil.  {\tt Scale} is the same as for gray pictures of images, if
{\tt independent-parameters} is nil.  That is, if {\tt
independent-parameters} is nil and {\tt scale} is not a number, then
OBVIUS automatically uses:
\begin{verbatim}
(range image)
\end{verbatim}

\item \lsym{pedestal} is ignored if {\tt independent-parameters} is
non-nil.  {\tt Scale} is the same as for gray pictures of images, if
{\tt independent-parameters} is nil.  That is, if {\tt
independent-parameters} is nil and {\tt scale} is not a number, then
OBVIUS automatically uses:
\begin{verbatim}
(minimum image)
\end{verbatim}

\item \lsym{border} specifies the spacing between pictures in the
pasteup.

\item \lsym{pasteup-format} may be either {\tt :horizontal}, {\tt
:vertical}, or {\tt :square}, specifying the layout of the
sub-pictures.

\item \lsym{zoom} see Section \ref{sec:pictures}.
\end{itemize}

%%%%%%%%%%%%%%%%%%%%%%%%%%%%%%%%%%%%%%%%%%%%%%%%%%%%%%%%%%%%%%%%%%%%%%%%%%%%%%%%%%

\subsection{Surface Plots}
\label{sec:surface-plot}

The \lsym{surface-plot}  pictures are perspective wire-frame views
of image intensity data (i.e, they are plots of image intensity as a
function of image position).  They are defined in the file {\bf
drawing.lisp}, and the code for creating them is in {\bf
surface-plot.lisp}.

Surface plots have a number of parameters that control the
transformation from the image pixel (world) coordinates to the
display.  The parameter values of a {\tt surface-plot} picture may be
changed using the {\tt Current Picture} dialog from the {\tt
Parameters} menu, or using the {\tt setp} macro.  The default values
(the values used to initialize a new picture) for these parameters may
be changed using the {\tt Picture Defaults} menu.  The parameters are:
\begin{itemize}

\item {\tt Theta} and {\tt phi} specify the orientation of the viewer
relative to the center pixel of the image data.  The viewer always
points toward the center of the intensity surface.  {\tt theta} is the
angle (counter-clockwise) from the y-axis.  {\tt phi} is the elevation
(aximuthal angle) above the x-y plane.

\item \lsym{z-size} determines the height of the surface relative to
its base.  The height (in world coordinates) will be the product of
{\tt z-size} and the average of the image dimensions.

\item {\tt r} specifies the distance of the camera from the center of
the image in world coordinates.  It also controls the perspective
distortion, since OBVIUS attempts to maintain an image-plane size
consistent with the {\tt zoom} parameter.  Thus, {\tt focal-length = r
* zoom}.  If the value is not a number, then OBVIUS automatically
uses:
\[
2\sqrt{\left(\frac{x}{2}\right)^2 + \left(\frac{y}{2}\right)^2 + 
		\left(\frac{z}{2}\right)^2},
\]
where $x$ and $y$ are the image dimensions, and $z$ is the product of
the {\tt z-size} parameter and the average of the image dimensions.
This is twice the minimal safe distance: the camera will be far enough
away not to hit the surface.

\item {\tt Zoom} determines the approximate size of the surface plot (see Section
\ref{sec:organization}.  It is not exact due to perspective distortion.

\item {\tt X-step} and {\tt y-step} specify the sampling rate for the
wireframe plot in x- and y- directions respectively.  If the value is
not a positive integer, then automatically uses:
\begin{verbatim}
(round (/ (x-dim image) 20))
(round (/ (y-dim image) 20))
\end{verbatim}

\item \lsym{box-p} (either {\tt T} or {\tt nil}) specify whether or
not to draw a 3D box around the plot.

\lsym{box-color} the color to use for the box (e.g., :red or :green or
any other color supported by LispView).

\item {\tt retain-bitmap} (either {\tt T} or {\tt nil}) specifies
whether to draw the surface plot to an off-screen bitmap, or redraw it
each time it is displayed.
\end{itemize}

%%%%%%%%%%%%%%%%%%%%%%%%%%%%%%%%%%%%%%%%%%%%%%%%%%%%%%%%%%%%%%%%%%%%%%%%%%%%%%%%%%

\subsection{Contour Plots}
\label{sec:contour-plot}

The \lsym{contour-plot} pictures are defined in the file {\bf
contour.lisp}.  The code works by constructing 3D triangles from
sampling the image in a grid.  Then for each z-value, it constructs a
curve containing segments each of which represents the intersection of
the constant z-value plane with a particular triangle.  So in the end,
the contour contains a list of curves.  This is a lot of computation,
so you should be careful to keep the {\tt num-levels} parameter
relatively small and the {\tt skip} parameter relatively large.

The parameters for contour-plots are:
\begin{itemize}
\item \lsym{z-min} and \lsym{z-max}: are Z-values for the smallest and
largest contour levels.

\item \lsym{num-levels}: Total number of contour levels.

\item \lsym{skip}: The subsampling factor.  If not a
number, the sampling will be chosen automatically.

\item \lsym{retain-bitmap}: (either {\tt T} or {\tt nil}) specifies
whether to keep a copy of the bitmap, or redraw it each time it is
displayed.

\item \lsym{zoom}: determines the size of the graph.  A zoom factor of
1 indicates that the samples of the underlying array should be plotted
one pixel apart in the x-direction.  The height of the graph is
determined relative to the width using the {\tt aspect-ratio} slot.

\item \lsym{line-width} specifies the width of the graph lines (bars)
drawn, in pixels.

\item \lsym{color} specifies the color of the vectors (e.g., :red,
:green, or any other color supported by LispView).
\end{itemize}


%%%%%%%%%%%%%%%%%%%%%%%%%%%%%%%%%%%%%%%%%%%%%%%%%%%%%%%%%%%%%%%%%%%%%%%%%%%%%%%%%%

\subsection{Vector Fields}
\label{sec:vector-field}

\ldef{vector-field} pictures, defined in the file {\bf
drawing.lisp}, may be used to display {\tt image-pairs}.  The first
image in the pair is interpreted as the y-component and the second is
interpreted as the x-component.

The parameter values of a {\tt vector-field} picture may be changed
using the {\tt Current Picture} dialog from the {\tt Parameters} menu,
or using the {\tt setp} macro.  The default values (the values used to
initialize a new picture) for these parameters may be changed using
the {\tt Picture Defaults} menu.  The parameters are:
\begin{itemize}
\item \lsym{skip} determines the subsampling factor.  If not a number, the
sampling will be chosen automatically to give a reasonable vector
density for the given vector-field size.  The default density can be set
from the {\tt Picture Parameters} menu.

\item \lsym{zoom} determines the overall size of the vector-field
picture, relative to the size of the underlying viewable.  If nil,
then the vector field will be the size of the underlying viewable
(i.e., they would line up correctly if the vector field were superposed
on the image).  If t, the zoom will be determined to make the overall
vector field be the size of the pane.  If a dimension list, this will
be the size of the vector field.  

\item \lsym{scale} determines the length of the vectors.  The vector
magnitude will be divided (as for gray) by the scale and the resulting
number will specify the length of the vector, in units corresponding
to the distance between the vector bases.

\item \lsym{arrow-heads} If a number, draws vectors with arrow heads of
that length.  If not a nubmer, draws oriented line segments without
arrow heads (e.g., for displaying orientation fields rather than flow
fields).  With arrow heads, the tail of the vector will correspond to
the appropriate pixel location in the image-pair.  Without arrow
heads, the {\em middle} of the line will correspond to the appropriate
pixel location in the image-pair.

\item \lsym{color} specifies the color of the vectors (e.g., :red,
:green, or any other color supported by LispView).
\end{itemize}


%%%%%%%%%%%%%%%%%%%%%%%%%%%%%%%%%%%%%%%%%%%%%%%%%%%%%%%%%%%%%%%%%%%%%%%%%%%%%%%%%%

\subsection{Graph Pictures}
\label{sec:graph}

The \ldef{graph} pictures are defined in the files {\bf drawing.lisp} and
{\bf graph.lisp}.  The parameter values of a \lsym{graph} picture may
be changed using the {\tt Current Picture} dialog from the {\tt
Parameters} menu, or using the \lsym{setp} macro.  The default values
(the values used to initialize a new picture) for these parameters may
be changed using the {\tt Picture Defaults} menu.  The parameters are:
\begin{itemize}
\item \lsym{graph-type} may be either {\tt :line}, {\tt :point}, 
or {\tt :bar}.  For most viewables, the automatic choice is {\tt
:line}.  For {\tt histograms}, the automatic choice is {\tt :bar}.

\item \lsym{x-range} is a list of length 2 specifying the range of
values covered by the x-axis.  If this is {\tt
:auto}\index{{\ptt :auto},x-range}, then OBVIUS automatically uses {\tt (list
0 (total-size image))}.

\item \lsym{y-range} is a list of length 2 specifying the range of
the y-axis.If this is {\tt :auto}, then OBVIUS
automatically uses {\tt (list (minimum image) (maximum image))}.

\item \lsym{y-axis} is the x-coordinate position of the y-axis.  If
this is nil, no y-axis is drawn.

\item \lsym{x-axis} is the y-coordinate position of the x-axis.  If
this is nil, no x-axis is drawn.

\item \lsym{y-tick-step} and  \lsym{y-tick-step} are the distances 
(in graph coordinates) between tick marks on the y- and x-axes,
respectively.  If set to {\tt :auto}\index{{\ptt :auto},tick-step}, OBVIUS
will choose this to display between 4 and 10 ticks on the axis.

\item \lsym{y-label} \lsym{x-label} [currently not used].

\item \lsym{aspect-ratio} determines the physical (i.e., in screen
pixel coordinates) ratio of height to width for the graphing region.

\item \lsym{retain-bitmap} (either {\tt T} or {\tt nil}) specifies
whether to keep a copy of the bitmap, or redraw it each time it is
displayed.

\item \lsym{zoom} determines the size of the graph.  A zoom factor of
1 indicates that the samples of the underlying array should be plotted
one pixel apart in the x-direction.  The height of the graph is
determined relative to the width using the {\tt aspect-ratio} slot.

\item \lsym{line-width} specifies the width of the graph lines (bars)
drawn, in pixels.

\item \lsym{axis-color} specifies color of the axes (e.g., :red,
:green, or any number of other colors defined by LispView).

\item \lsym{color} color of the graph and plot-symbols (e.g., :red,
:green,or any number of other colors defined by LispView).

\item \lsym{plot-symbol} Either {\tt :circle}, {\tt :square}, or {\tt
nil} (for no plot symbol).

\item \lsym{symbol-size} specifies size of plot symbol, must be
integer between 1 and 10.

\item \lsym{fill-symbol-p} specifies whether or not plot symbol is
filled, must be {\tt T} or {\tt nil}.
\end{itemize}

%%%%%%%%%%%%%%%%%%%%%%%%%%%%%%%%%%%%%%%%%%%%%%%%%%%%%%%%%%%%%%%%%%%%%%%%%%%%%%%%%%

\subsection{Polar Plots}
\label{sec:polar-plots}

The \ldef{polar-plot} pictures, defined in the file {\bf drawing.lisp},
may be used to display {\tt discrete-functions}.  They may also be
used to display {\tt image-pairs} or {\tt viewable-sequences} made up
of pairs of discrete-functions.  

For a single {\tt discrete-function}, the angular coordinate of the
plot is specified (in radians) by the {\tt origin} and {\tt increment}
of the {\tt discrete-function}.  The radial coordinate is specified by
the {\tt discrete-functions}'s data array.  An example is given in the
OBVIUS tutorial.

For a pair (either {\tt image-pair} or {\tt viewable-sequence} pair),
the first viewable of the pair is interpreted as the magnitude (radial
component) and the second is interpreted as the angle (in radians).
Example: 
\begin{verbatim}
(display (make-viewable-sequence
	  (list (make-discrete-function '(lambda (x) (* 100.0 (cos x))) 
                                        0.0 2-pi :size 63)
		(make-discrete-function '(lambda (x) (* 2.0 x)) 
                                        0.0 2-pi :size 63)))
	 'polar-plot)
\end{verbatim}

The parameters of a {\tt polar-plot} are:
\begin{itemize}
\item \lsym{maximum} The range of values plotted radially are between
0 and maximum.

\item \lsym{line-width} specifies the width of the graph lines (bars)
drawn, in pixels.

\item \lsym{axis-color} *** not implemented yet

\item \lsym{color} color of the graph and plot-symbols (e.g., :red,
:green,or any number of other colors defined by LispView).

\item \lsym{plot-symbol} Either {\tt :circle}, {\tt :square}, or {\tt
nil} (for no plot symbol).

\item \lsym{symbol-size} specifies size of plot symbol, must be
integer between 1 and 10.

\item \lsym{fill-symbol-p} specifies whether or not plot symbol is
filled, must be {\tt T} or {\tt nil}.
\end{itemize}


%%%%%%%%%%%%%%%%%%%%%%%%%%%%%%%%%%%%%%%%%%%%%%%%%%%%%%%%%%%%%%%%%%%%%%%%%%%%%%%%%%

\subsection{Scatter Plots}
\label{sec:scatter-plot}

The {\tt scatter-plot} pictures, defined in the file {\bf
drawing.lisp}, may be used to display data that is stored in an {\tt
image-pair}.  The parameters of a {\tt scatter-plot} are the same as
those of a graph.  An example of displaying a scatter-plot is given in
the OBVIUS tutorial.  The {\tt scatter-plot} method may be also used
as a shorthand for generating {\tt scatter-plots}:
\begin{description}
\item\lmeth{make-scatter}{image}{thing1 thing2 \&rest initargs}
Thing1 and thing2 may be images, vectors, or lists.  Makes a
scatter-frob a simple viewable that displays itself as a scatter plot.
\end{description}


%%%%%%%%%%%%%%%%%%%%%%%%%%%%%%%%%%%%%%%%%%%%%%%%%%%%%%%%%%%%%%%%%%%%%%%%%%%%%%%%%%

\subsection{Flipbooks}
\label{sec:flipbook}

The \ldef{flipbook}s are the standard display type for {\tt
viewable-sequences} and {\tt image-sequences}.  {\tt Flipbooks} are
defined in the file {\bf flipbook.lisp}.  {\tt Flipbooks} can be
thought of as a list of sub-pictures.  For example, {\tt flipbooks} of
{\tt image-sequences} contain a list of {\tt gray} pictures.

Mouse bindings are provided to cycle through the subpictures in the
flipbook (see section~\ref{sec:mouse}).  {\tt Flipbook} pictures have
a number of parameters that control the delay between frames when the
pictures are played back.  There is also a way to change the scaling
parameters of the {\tt gray} pictures that make up the {\tt flipbook}.

The parameter values of a {\tt flipbook} picture may be changed using
the {\tt Current Picture} dialog from the {\tt Parameters} menu, or
using the {\tt setp} macro.  The default values (the values used to
initialize a new picture) for these parameters may be changed using
the {\tt Picture Defaults} menu.  The parameters are:
\begin{itemize}
\item \lsym{frame-delay}.  Delay between frames (in approximately
1/60th secs).

\item \lsym{seq-delay}.  Delay between each repeat (in approximately
1/60th secs).

\item \lsym{back-and-forth}.  When t, the flip-book is shown forward
and back during each repeat.

\item \lsym{independent-parameters}.  If non-nil, then the images are
rescaled independently (see above description of {\tt pasteup}
pictures).

\item \lsym{zoom}. Described in section~\ref{sec:pictures}.

\item \lsym{sub-display-type} specifies the display types for the
pictures that make up the {\tt flipbook}.  If T, then OBVIUS chooses
the display type automatically.  For {\tt flipbooks} of {\tt
image-sequences}, the automatic choice is for the sub-pictures to be
{\tt grays}.
\end{itemize}

The {\tt setp} macro may be used to set the parameters of the
sub-pictures that make up a {\tt flipbook}.  For example, 
\begin{verbatim}
(setp scale 100.0)
\end{verbatim}
If {\tt independent-parameters} is non-nil, then this will affect only
the sub-picture that is currently being displayed.  If {\tt
independent-parameters} is nil, then this will change the scale for
all of the sub-pictues.

The parameters for the individual sub-pictures may be set
independently if {\tt independent-parameters} is non-nil.  The
parameters may be set using the {\tt Picture Parameters} dialog in
conjunction with the {\tt previous-frame} (C-M-left) and {\tt
next-frame} (C-M-middle) mouse bindings.  Bring up the current picture
dialog (e.g., from the {\tt Parameters} button on the control panel),
and change the picture parameters of the current sub-picture.  Use the
mouse bindings to change the current sub-picture.  The dialog box is
automatically updated to give the parameter settings for the new
sub-picture.  Then go ahead and change the parameters of the new
sub-picture.

%%%%%%%%%%%%%%%%%%%%%%%%%%%%%%%%%%%%%%%%%%%%%%%%%%%%%%%%%%%%%%%%%%%%%%%%%%%%%%%%%%

\subsection{Overlays}
\label{sec:overlays}

Some {\tt viewable-sequences} can be displayed as \ldef{overlay}
pictures.  The {\tt overlay} class is defined in the file {\bf
overlay.lisp}.  Currently, overlays are supported only for overlaying:
(1) (1) {\tt graphs} of {\tt discrete-functions}, (2) {\tt graphs} of
{\tt one-d-images}, (3) {\tt bitmaps} of {\tt bit-images}, and (4) for
overlaying {\tt bitmaps} of {\tt bit-images} on top of a {\tt gray} of
an {\tt image}.  Examples are given in the OBVIUS tutorial showing how
to display {\tt overlays}.  For more examples, look in the {\bf
overlay.lisp} file.

The parameters for the individual sub-pictures in the overlay may be
set independently using the {\tt Picture Parameters} dialog in
conjunction with the {\tt previous-frame} (C-M-left) and {\tt
next-frame} (C-M-middle) mouse bindings.  Bring up the current picture
dialog (e.g., from the {\tt Parameters} button on the control panel),
and change the picture parameters of the current sub-picture.  Use the
mouse bindings to change the current sub-picture.  The dialog box is
automatically updated to give the parameter settings for the new
sub-picture.  Then go ahead and change the parameters of the new
sub-picture.

%%%%%%%%%%%%%%%%%%%%%%%%%%%%%%%%%%%%%%%%%%%%%%%%%%%%%%%%%%%%%%%%%%%%%%%%%%%%%%%%%%

\subsection{Hardcopy}
\label{sec:hardcopy}

The function \lsym{hardcopy} produces postscript output of OBVIUS
pictures:
\begin{verbatim}
(hardcopy viewable
          &optional display-type
          &key path printer
          x-location y-location x-scale y-scale
          invert-flag suppress-text make-new
	  &allow-other-keys)
\end{verbatim}
\noindent
Note that the syntax is quite similar to that of  the \lsym{display}
function described in section~\ref{sec:display}.
Produces a postscript version of the viewable.  The parameters for
{\tt hardcopy} are similar to those of \lsym{display}: any additional
keywords are used to specify initargs of the picture.  The optional
parameter
\lsym{display-type} specifies the picture class for the postscript
picture.  The function searches for a picture of that display-type in
one of the panes (starting with the top of the current pane).  If one
is found, then it is copied to make the postscript picture.  If none
is found, then it makes a new picture using default parameters.
Additional keywords may be passed to the hardcopy function (similar to
the display function) to override the parameters of the on-screen
picture.  By default, hardcopy passes the keyword-value pair (:zoom
:auto) so that the resulting picture will fill the page. The argument
\lsym{:make-new} will force OBVIUS to make a new picture using
default-parameters.

The keyword argument {\tt :path} specifies a file pathname.  If this
parameter is nil (the default), OBVIUS writes the postscript to a
temporary file, ships it to a printer, and then deletes the temporary
file.  The arguments {\tt :x-location} and :y-location determine the
location of the picture on the page (default is top-left corner).  The
arguments {\tt :x-scale} and {\tt :y-scale} specify by how much to
rescale the size of the the picture (default is 1).  Hardcopy prints
the name of the viewable on the page just below where the picture
appears (unless the argument {\tt :suppress-text} is non-nil, in
which case nothing is printed).  The keyword {\tt :printer} specifies
which printer to use (default is the global {\tt *default-printer*}).

The mouse binding, C-M-Sh-right, calls hardcopy to print a picture to
the default printer.

The following global variables must be set for {\tt hardcopy} to work
properly:
\begin{itemize}
\item \lsym{*temp-ps-directory*} -- Directory to hold temporary
postscript files before they are sent to the printer.  Typical value
is something like: {\tt "/tmp/"}.

\item \lsym{*default-printer*} -- Name of postscript printer, to be
inserted into the {\tt *print-command-string*} (e.g., {\tt "laserwriter1"}).

\item \lsym{*print-command-string*} -- Command string for
sending a postscript file to a printer.  This is used as a format
string (i.e., an argument to the Common Lisp {\tt format} function)
with two arguments: the filename and the printer name.  That is, this
string will be used in the following fashion:
\begin{verbatim}
(format nil *print-command-string* <file> <printer>)
\end{verbatim}
where {\tt <printer>} will default to the value of {\tt
*default-printer*}.  On our system, the value of {\tt
*print-command-string*} is {\tt "cat ~A | lpr -P~A ; rm ~@*~A"}.  The
{\tt ~@*~A} construct allows us to process the filename twice (see
Steele, Common Lisp: The Language).  The result of this call to {\tt
format} will then be run as a shell program.  Thus, the our example
string will cat the first argument (filename) to the {\tt lpr} command
with the appropriate printer argument (second argument), and then call
{\tt rm} on the first argument (deleting the temporary postscript
file).
\end{itemize}


%%%%%%%%%%%%%%%%%%%%%%%%%%%%%%%%%%%%%%%%%%%%%%%%%%%%%%%%%%%%%%%%%%%%%%%%%%%%%%%%%%
\section{OBVIUS Functions}
\label{sec:operations}

OBVIUS provides standard functions for generating viewables, and many
standard image-processing operations for manipulating them.  Many of
these image-processing operations are defined for many of the viewable
classes.  Most of these may be found on the menus in the control
panel.  In addition, you may use the {\tt C-c C-a} command in Emacs to
get the argument list for a given function, and {\tt C-c .} to view
the source code, as explained in Section \ref{sec:help}.  In this
section, we list many of the operations, giving the name of the
function and the argument list.

% The format 
% Below are some examples of the format used to list these
% fuctions:
% \begin{description}
% \item\lfun{add}{ thing1 thing2 \&key \res}
% \item\lfun{square}{ image \&key \res}
% \end{description}
% Here {\tt thing1} and {\tt thing2} may be either viewables or
% numbers. {\tt image} may be an image or it may be a viewable that is
% made up of several images (e.g., image-sequence, image-pair, pyramid,
% etc.)

\mysubsubsec{Result Argument}
By convention, all of the functions that return viewables are written
to take an optional keyword argument \lsym{:->}.  The action taken by
the function depends on the type of argument passed:
\begin{itemize}
\item If no argument (or a nil argument) is passed, a new viewable is
created with a nil name slot, and the result is written into the newly
created viewable.  This is the default behavior: it is wasteful (since
it always creates a new viewable), but avoids destructive modification
which can be dangerous.

\item If the argument is an existing viewable, the result is
written into that viewable, thus destructively replacing the contents
of the original viewable.

\item If the argument is a symbol, a new viewable is created with
the name slot bound to the symbol, the global value of the symbol
(i.e., its value in the current package) is bound to the newly created
viewable, and the result is written into the newly created viewable.

\item If the argument is a string, a new viewable is created with 
the name slot bound to the string, and the result is written 
into the newly created viewable.
\end{itemize}
	

\mysubsubsec{Point Operation Macros}
There are a set of point-operation macros defined to mimic many of the
standard Common Lisp functions.  The names of these macros are the
same as the names of the Common Lisp functions, augmented by a period
(``point'').  The current list of these point-operation macros is as
follows: {\tt +. -. *. /. sqr. sqrt.  expt. exp. log.  abs.  mod.
round. truncate. floor.
sin. cos. min. max. average. realpart. imagpart. phase.  incf. decf.
fill. append. subseq.} 

Where possible, these functions are written to take arguments like
their Common Lisp counterparts. For example, the macro \lsym{+.} takes
multiple viewable arguments and calls \lsym{add} on them iteratively.
As we go through the operations in the following sections, we will
mention equivalent point-operation macros for some of the functions.

\subsection{Image Utilities}

The following are utilities and accessors for images: 
\begin{description}
\item\lmeth{iref}{image}{ image y x}
See Section \ref{sec:images}.

\item\lfun{image-p}{ thing}
Returns t if thing is an image, nil if not.

\item\lmeth{dimensions}{image}{ image}
Returns a list of the dimensions (y-dimension first).

\item\lfun{rank}{ image}
Returns 1 for a {\tt 'one-d-image} and 2 otherwise.

\item\lmeth{x-dim}{image}{ image}

\item\lmeth{y-dim}{image}{ image}

\item\lmeth{total-size}{image}{ image}
The total number of pixels in the image.

\item\lmeth{print-values}{image}{ image \&key y x y-size x-size}
See Section \ref{sec:images}.
\end{description}

\subsection{Generating Images}

The following is a list of functions for making new images, and for
making synthetic images:
\begin{description}
\item\lfun{make-image}{ dims \&key display-type \res}
Returns a new image of the specified dimensions, initialized to
zeros, with the name set to the value of the keyword parameter :\res.
The dims parameter must be a list of two fixnums, the first specifying
the y-dimension and the second specifying the x-dimension.  

\item\lmeth{similar}{image}{ image \&key \res}
Makes a new blank image using the argument {\tt image} as a model
(i.e., the new image will have the same dimensions and display-type as
the model).  Note that {\tt similar} is defined for most viewable
types.

\item\lmeth{copy}{image}{ image \&key \res}
Returns a copy of {\tt image}.

\item\lfun{image-from-array}{ array \&key \res}
Makes an image from an array.  

\item\lfun{make-synthetic-image}{ dims func \&key x-range y-range \res}
This function provides a general facility for mapping a function of
two variables onto an image.  The {\tt dims} argument specifies the
size of the image (in pixels).  The argument {\tt func} should be
something that can be passed to funcall (e.g. a symbol with a function
binding, a lambda expression, or an actual function).  It will be
applied to two floating point numbers {\em and must return a floating
point value}.  The optional keyword arguments specify the range of
argument values over which to map the function (i.e., the pixel values
of the synthetic image will correspond to the evaluation of the
function at discretely sampled points over this range, with the
endpoints set to the start and end of the range).  The defaults
are '(-1.0 1.0) for the {\tt :x-range}, and the {\tt :y-range}
defaults to the value of {\tt :x-range}.  Example:
\begin{verbatim}
(make-synthetic-image '(128 128) '(lambda (y x) x) :-> 'xramp)
\end{verbatim}
Several more examples are provided at the end of the {\bf synth.lisp} file.

\item\lfun{make-impulse}{ dims \&key amplitude \res}

\item\lfun{make-ramp}{ dims \&key orientation pedestal slope origin \res}
Makes a ramp image of the given dimensions.  {\tt :pedestal} is the
value in the center of the image (i.e., at the (d-1)/2 pixel
position).  {\tt :slope} is measured in change per pixel distance.
{\tt :orientation } is in radians relative to the +X axis."  {\tt
origin} is a list of two numbers specifying the (y x) coordinates of
the pixel to which the pedestal applies.

\item\lfun{make-grating}{dims func period \&key orientation phase \res}
Makes an image of a one-dimensional periodic function.  {\tt func}
must be a valid argument for the Common Lisp function {\tt funcall}.
{\tt :period} indicates the desired period of the function in pixels.
{\tt func} will be evaluated at {\tt *default-df-size*} points between
0 and {\tt :period}. Linear interpolation will be used between these
points.  {\tt :phase} should be in the interval [0,{\tt :period}] and
defines the argument to the function at the center of the image.

\item\lfun{make-square-grating}{ dims \&key period orientation phase \res}

\item\lfun{make-saw-tooth-grating}{ dims \&key period orientation phase \res}

\item\lfun{make-sin-grating}{ dims \&key period orientation phase
amplitude dc frequency x-freq y-freq \res}
Specify either: (1) {\tt :period} and {\tt :orientation}; or (2) {\tt
:frequency} and {\tt :orientation}, or (3) {\tt :x-freq} and {\tt :y-freq}.

\item\lfun{make-atan}{ dims \&key center phase \res}
Makes an angle (arc-tangent) image (i.e., each pixel contains its
angular coordinate) between $-\pi$ and $\pi$.

\item\lfun{make-r-squared}{ dims \&key center \res}
Make an image of the function $f(r) = r^2$, where $r$ is the distance
from the point specified by {\tt :center}.  Default center is computed
as $(d-1)/2$, where $d$ will be respective image dimensions in pixels.

\item\lfun{make-1-over-r}{ dims \&key center exponent amplitude zero-val \res}
Computes $f(r) = a r^{-\beta}$ where $r$ is the distance from the center of
the image in pixels, $a$ is specified by the keyword {\tt :amplitude},
$\beta$ is specified by the keyword {\tt :exponent}, and the keyword
{\tt :zero-val} specifies what value to use at $r=0$.

\item\lfun{make-zone-plate}{ dims \&key amplitude phase \res}
Makes a zone plate ($f(r) = \cos(r^2)$) with given {\tt :amplitude}
and {\tt :phase}. Phase is in radians and refers to the origin.

\item\lfun{make-pinwheel}{ dims \&key \res}

\item\lfun{make-disc}{ dims \&key radius origin transition-width \res}

\item\lfun{make-left}{ dims \&key \res}

\item\lfun{make-top}{ dims \&key \res}

\item\lfun{make-uniform-noise}{ dims \&key \res}
Makes image with values randomly distributed between 0 and 1.

\item\lfun{make-gaussian-noise}{ dims \&key mean variance iterations \res}
Approximation to normal distribution that is faster than using {\tt
make-normal-random-image}. Better approximation (and slower) as {\tt
:iterations} is increased. 

\item\lfun{make-random-dots}{ dims \&key density \res}
{\tt :density} is a value between 0 and 1.  Returns a 'bit-image.

\item\lfun{make-fractal}{ dims \&key fractal-dimension \res}
\end{description}


\subsection{Loading and Saving Images and Image-Sequences}

The following functions are used for loading and saving images to and
from datfiles.  Datfiles are described in more detail in section~\ref{sec:datfile}.

\begin{description}
\item\lfun{load-image}{ path \&key ysize xsize reverse-bytes \res}
\item\lmeth{save-image}{image}{ image path \&key element-type auto-scale reverse-bytes df-keys}
\item\lfun{load-image-sequence}{ path \&key start-index end-index \res}
\item\lmeth{save-image}{image-sequence}{ sequence path \&key element-type auto-scale reverse-bytes df-keys start-index end-index}

\item\lfun{save-picture}{ path}
Saves the image displayed in the current pane as an 8-bit datfile.
Rescales the image according to the current scale and pedestal as
specified in the title bar of the pane.  This function only works on
{\tt 'gray} pictures of {\tt 'image} viewables.
\end{description}


\subsection{Operations on Images}

The following operations take images as arguments and return images as
their results.  These functions are also defined for viewables that
are made up of compound images (e.g., image-sequences, image-pairs,
pyramids, etc.)

\begin{description}

\item\lmeth{zero!}{image}{ image}
Sets all the image values to 0.0, destructively modifying the image.

\item\lmeth{fill!}{image}{ image value}
Sets all the image values to {\tt value}, destructively modifying the
image.
\end{description}

\mysubsubsec{Numerical Point Operations}

\begin{description}
\item\lmeth{add}{image}{ thing1 thing2 \&key \res}
equivalent macro \lsym{+.} \\
This method and the next few methods operate on the images point 
by point.  Thing1 and thing2 are either viewables or numbers.

\item\lmeth{sub}{image}{ thing1 thing2 \&key \res}
equivalent macro \lsym{-.}

\item\lmeth{mul}{image}{ thing1 thing2 \&key \res}
equivalent macro \lsym{*.}

\item\lmeth{div}{image}{ thing1 thing2 \&key zero-val suppress-warning \res}
equivalent macro \lsym{/.} \\ 
Attempting to divide by zero prints a 
\index{foreign functions,floating-point exceptions}
warning message describing the number of pixels that were zero, and
sets the result to either the positive or negative value of {\tt
zero-val}, depending on the sign of the numerator.  The default value
for zero-val is given by the global parameter
\lsym{*div-by-zero-result*}.  The warning message can be avoided by
setting the {\tt suppress-warning} keyword to t (default is nil).

\item\lmeth{linear-xform}{image}{ image scale offset \&key \res}
This returns (image * scale) + offset.

\item\lmeth{negate}{image}{ image \&key \res}
equivalent macro \lsym{-.} (unary).
\item\lmeth{square}{image}{ image \&key \res}
equivalent macro \lsym{sqr.} (Note that OBVIUS provides a function on
numbers called \lsym{sqr}).

\item\lmeth{square-root}{image}{ image \&key \res}
equivalent macro \lsym{sqrt.} \\ 
Signed square root on the image, e.g., {\tt (square-root -4.0)} gives -2.0.

\item\lmeth{abs-value}{image}{ image \&key \res}
equivalent macro \lsym{abs.}

\item\lmeth{power}{image}{ image thing \&key \res}
equivalent macro \lsym{expt.} \\
{\tt Thing} may be an image or a number.  If a number, then this
function raises each pixel in {\tt image} that number.  If an image,
then it raises each pixel in {\tt image} to the correponding pixel
value in {\tt thing}.

\item\lmeth{natural-logarithm}{image}{ image \&key zero-val \res}
Attempting to take the log of zero prints an warning message and uses
the value of {\tt zero-val}.  The default value for zero-val is given
by the global parameter *div-by-zero-result*.  
\index{foreign functions,floating-point exceptions}

\item\lmeth{gamma-correct}{image}{im gamma \&key below above binsize \res}
Clips between {\tt below} and {\tt above}, and calls point operation
to gamma correct the rest of the values.  Defaults for {\tt below} and
{\tt above} are the minimum and maximum, respectively.  For that
default case, it computes $m + (M-m) [(x-m)/(M-m)]^\gamma$, where $m$
and $M$ are the minimum and maximum.

\item\lmeth{point-operation}{image}{ image function \&key binsize \res}
{\em function} must take a single float argument and return a float.
If {\em binsize} is a number, this method creates a discrete-function
(a type of viewable) by evaluating the function at intervals of {\em
binsize}.  This is then used as a lookup table and the result image is
computed using linear interpolation to generate values between the
sample points.  If binsize is nil (the default), the function is
applied directly at each pixel (this can be very slow!).  Example:
\begin{verbatim}
(point-operation einstein '(lambda (x) (/ 1.0 (+ 1.0 x))))
\end{verbatim}
If the function being applied is periodic, it is advisable to use the
function {\tt periodic-point-operation} instead.  

\item\lmeth{periodic-point-operation}{image}{ image function period \&key binsize \res}
Apply {\tt function} with given {\tt period} to {\tt image} using a
lookup table and linear interpolation.  This is different from the
{\tt point-operation} function in that the lookup table is built to
cover {\em one period}, with an extra point on the end.  See {\bf
synth.lisp} for useful examples.

\item\lmeth{clip}{image}{ image below above \&key \res}
Returns an image identical to the argument except that values
less than {\em below} are replaced with {\em below} and values greater
than {\em above} are replaced with {\em above}.

\item\lmeth{round.}{image}{ \&key divisor \res}
Round to nearest integer value.  If divisor is passed, divide by this
value before rounding (Same as round function in common lisp).

\item\lmeth{floor.}{image}{ \&key divisor \res}
Compute greatest integer value less than each pixel value.  If divisor
is passed, divide by this value before flooring (Same as floor
function in common lisp).

\item\lmeth{quantize}{image}{ image \&key binsize origin \res}
Image is quantized into bins.  The value of each pixel is set to the
{\em mean} (not the midpoint) of the containing bin.
The default {\tt :binsize} is the range of the image divided by the global
parameter *default-number-of-bins* and the default orgin is the mean
of the entire image.
\end{description}


\mysubsubsec{Geometrical Operations}

\begin{description}
\item\lmeth{circular-shift}{image}{ image \&key offset y x \res}
This translates the pixels of an image, wrapping around at the edges.
It is written so that the result may be the same image as the
original.  Note: parmeters are redunant, since {\tt offset = (y x)}.

\item\lmeth{upsample}{image}{ image \&key start-vector step-vector hex-start \res}
For a full description of these keywords, see the description of
{\tt filters} in Section \ref{sec:filters}.

\item\lmeth{downsample}{image}{ image \&key start-vector step-vector hex-start \res}
For a full description of these keywords, see the description of
{\tt filters} in Section \ref{sec:filters}.

\item\lmeth{subsample}{image}{ image \&key start-vector step-vector hex-start \res}
Same as downsample.

\item\lmeth{resample}{image}{ image \&key dimensions expansion-factor interpolator}
Resample the image.  {\tt dimensions} is the dimension list for the
resulting image.  Alternatively, specify a number or pair of numbers
for {\tt expansion-factor}.  {\tt interpolator} must be either {\tt
:cubic} (the default), or {\tt :linear}.

\item\lmeth{crop}{image}{ image \&key offset y x dimensions y-dim x-dim \res}
Returns the specified (rectangular) sub-image.  Default {\tt :offset}
is {\tt '(0 0)}, and default dimensions are the dimensions of the
image minus the offset. Note that the parameters are redundant: you
can pass {\tt :offset} or {\tt :x} and {\tt :y} separately.

\item\lmeth{paste}{image}{ im big-im \&key offset x y \res}
Pastes the image into a larger image.  Clips if im hangs over the
edge.  Again, parameters are redundant: {\tt offset  = (y x)}.

\item\lfun{side-by-side}{ im1 im2 \&key space \res}
Puts both images next to each other in a new image image twice larger.
The keyword :space leaves the specified number of columns blank to
separate the two images in the result.

\item\lfun{transpose}{ image \&key \res}
Transpose of the image: swap X and Y axes.

\item\lfun{flip-x}{ im \&key \res}
Flips the image around the center vertical axis.

\item\lfun{flip-y}{ im \&key \res}
Flips the image around the center horizontal axis.

\item\lmeth{rotate}{image}{ im \&key angle interpolator origin}
Rotate image about position specified by {\tt origin} (a pair of
numbers defaulting to center of image).  {\tt angle} is specified in
radians and defaults to $\pi/2$.  {\tt interpolator} must be either {\tt
:cubic} (the default), or {\tt :linear}.

\item\lmeth{warp}{image}{ im image-pair \&key interpolator}
Warp the image according to the image-pair.  The image-pair is
interpreted as a field of vectors, one associated with each pixel in
the result image.  Each result pixel is computed by moving to a
position in the image specified by the corresponding vector and
interpolating a value.   {\tt interpolator} must be either {\tt
:cubic} (the default), or {\tt :linear}.

\end{description}

\mysubsubsec{Comparison Point Operations}

\begin{description}
\item\lmeth{greater-than}{image}{ thing1 thing2 \&key \res}
equivalent macro \lsym{>.} \\
Arguments {\tt thing1} and {\tt thing1} can both be images or one of
them can be a number.  If they are both images, then the function does
a point by point comparison.  Returns a {\tt bit-image}.

\item\lmeth{less-than}{image}{ image thing  \&key \res}
equivalent macro \lsym{<.}

\item\lmeth{equal-to}{image}{ image thing \&key result-type \res}
equivalent macro \lsym{=.}

\item\lmeth{greater-than-or-equal-to}{image}{ image thing \&key result-type \res}
equivalent macro \lsym{>=.}

\item\lmeth{less-than-or-equal-to}{image}{ image thing \&key result-type \res}
equivalent macro \lsym{<=.}

\item\lmeth{point-minimum}{image}{ thing1 thing2 \&key \res}
{\tt Thing1} and {\tt thing2} may be images or numbers.  Computes
minimum as a pointop on two inputs.  Each pixel of the result is the
lesser of the two corresponding input values.

\item\lmeth{point-maximum}{image}{ thing1 thing2 \&key \res}

\item\lfun{square-error}{ image1 image2 \&key \res}
Returns the square of the difference of the two images.

\item\lfun{abs-error}{ image1 image2 \&key \res}
Returns the absolute value of the difference of the arguments.

\item\lfun{mean-square-error}{ image1 image2}
\item\lfun{mean-abs-error}{ image1 image2}
\item\lfun{max-abs-error}{ image1 image2}

\item\lfun{zero-crossings}{ image \&key \res}
Computes a {\tt bit-image} containing the zero-crossings of {\tt
image}.

\item\lfun{correlate}{ im1 im2 \&key x y x-dim y-dim \res}
Correlate images, circular-shifting one over top of the other.
{\tt x-dim} and {\tt y-dim} specify to the area over which to correlate
(these default to the dimensions of {\tt im1}).
{\tt x} and {\tt y} specify to the shift at which to begin correlation
(that is, the value corresponding to a shift of {\tt x} and {\tt y}
will appear in the upper-left pixel of the result). These default to zero.

\end{description}


\mysubsubsec{Statistics}

The following operations return numbers.  These functions are also
defined for viewables that are made up of compound images (e.g.,
image-sequences, image-pairs, pyramids, etc.)  All but the last two
store their computed values on the info-list of the viewable.

\begin{description}
\item\lmeth{minimum}{image}{ image}
Equivalent macro \lsym{min.} (unary).  Returns minimum pixel value found in
the image.

\item\lmeth{maximum}{image}{ image}
Equivalent macro \lsym{max.} (unary).

\item\lmeth{minimum-location}{image}{ image}
Returns (y,x) position of the minimum.

\item\lmeth{maximum-location}{image}{ image}
Returns (y,x) position of the minimum.

\item\lmeth{range}{image}{ image}
Returns difference between maximum and minimum.

\item\lmeth{mean}{image}{ image \&key ignore-zeros}
If {\tt ignore-zeros} is non-nil, then the mean is computed without
counting any of the zero values.

\item\lfun{variance}{ image \&key ignore-zeros}
\item\lfun{skew}{ image}
\item\lfun{kurtosis}{ image}

\item\lfun{entropy}{ image \&key binsize origin numbins}
The parameters are the same as for make-histogram.

\end{description}


\mysubsubsec{Fourier Transform}

\begin{description}
\item\lmeth{fft}{image}{ image \&key dimensions inverse center
pre-center post-center \res} Returns a {\tt complex-image} that is the
DFT of the {\tt image} argument.  The algorithm used is a
multidimensional radix-2 FFT (source code in {\bf fft.c}.  The {\tt
image} may be an {\tt image}, an {\tt image-pair}, or a {\tt
complex-image}.  The size of the result may be specified with the {\tt
:dimensions} keyword and should be a list of two numbers, each a power
of 2.  By default, the X (Y) dimension of the result is the smallest
power of two greater than the X (Y) dimension of the image.  Computes
the inverse FFT if the {\tt :inverse} parameter is non-nil.  If the
{\tt :pre-center} argument is non-nil, the image is circular-shifted
before computing the transform.  If the {\tt :post-center} argument is
non-nil, the resulting transform is circular-shifted, placing the
origin (DC value) in the {\em center} of the image.  Both of these
keywords default to the value of the {\tt :center} keyword, thus
allowing the user to pre- and post- center by passing a single
keyword.

\item\lmeth{power-spectrum}{image}{ image \&key center dimensions
\res}
Computes the squared magnitude of the Fourier transform.  See the
explanation of the {\tt :center} and {\tt :dimensions} arguments for
the \lsym{fft} function.

\item\lfun{dct}{}
Not yet implemented.

\item\lfun{hilbert-transform}{ im \&key orientation \res}
This function does a ninety degree phase shift of all frequencies in
the direction of a vector at angle ({\tt orientation}) to the
horizontal.  Frequencies that are low with respect to this vector (ie,
lie near the line perpendicular to this vector) create visual
artifacts.

\item\lfun{derivative}{ im \&key orientation \res}
Multiply the FFT by $j\omega$, then inverse transform.
\end{description}


\mysubsubsec{Filtering Operations}

\begin{description}
\item\lmeth{apply-filter}{image}{ thing1 thing2 \&key \res (direction 0)}
The argument thing1 may be a filter and thing2 an image or vice versa.
This method is also defined on image-sequences.  Correlates the image
with the filter, (this is correlation because the filter is not
flipped) downsampling as specified by the grid of the filter.  Notice
that the result argument (:\res) cannot be equal to the image
argument.  Also, the thing must be at least the size of the filter if
the edge-handler is non-nil.  The {\tt :direction} keyword argument
determines the orientation of the filter.  For example, if you apply a
one dimensional filter to an image, it will be applied in the x
direction.  To apply it in the y direction, set the {\tt :direction}
argument to 1.  See section~\ref{sec:filters} for more information.

\item\lmeth{expand-filter}{image}{ thing1 thing2 \&key (zero t) \res}
The argument thing1 may be a filter and thing2 an image or vice versa.
Upsamples the image (as specified by the filter grid), then convolves
the image with the filter.  If the zero keyword parameter is nil (the
default), the result of the convolution will be added into the result
image; otherwise it replaces the contents of the result image.  See
section~\ref{sec:filters}.

\item\lmeth{gauss-in}{image}{ image \&key resample edge-handler kernel \res}
Does upsampling followed by gaussian convolution (like a zoom-in with
blurring).  The default is to upsample by a factor of two and use a
five-tap separable binomial filter.

\item\lmeth{gauss-out}{image}{ image \&key  resample edge-handler kernel \res}
Does a gaussian convolution and downsampling (like a zoom-out with
averaging).  The default is to use a five-tap separable binomial
filter and then downsample by a factor of two.

\item\lmeth{blur}{image}{ image \&key level kernel edge-handler \res}
Recursive blurring method.  If level is 1 (the default), it does a
gauss-out followed by a gauss-in.  If level is n, it does n
gauss-out's followed by n gauss-in's.
\end{description}


\subsection{Functions for One-d-images}

One-d-images are really the same as images except that one of the
dimensions is 1.  So, nearly all of the image operations that are
defined for images work on one-d-images as well.  There are some image
processing operations that are not implemented for {\tt one-d-images}
including \lsym{flip-x}, \lsym{flip-y},\lsym{upsample}, \lsym{downsample},
\lsym{subsample}, \lsym{transpose}, \lsym{side-by-side}.  The following
functions are particular for one-d-images:
\begin{description}
\item\lfun{one-d-image-p}{ thing}
Returns t if thing is a one-d-image, nil if not.

%%% ***** ???? This is not defined on one-d-images
\item\lmeth{make-slice}{image}{ image \&key x y z display-type \res}
Makes a slice from {\tt image} or {\tt image-pair}.  One (and only
one) of {\tt :x} and {\tt :y} may be specified.  This
function creates a one-dimensional image that is a slice of a two
dimensional image.  This function may be called via a mouse click as
described in Section \ref{sec:mouse}.  The function is also defined on
image sequences.
\end{description}


\subsection{Operations on Bit-Images}

Bit-images are just like images except that they have only one bit of
data at each pixel (the bit-image data is stored using common-lisp bit
arrays).  So, some of the functions and accessors are inherited from
images (e.g., \lsym{iref}, \lsym{dimensions}, \lsym{x-dim}, \lsym{y-dim},
\lsym{total-size}).  Other methods are functionally equivalent to those
defined on images (e.g., \lsym{mul}, \lsym{copy}, \lsym{similar}, {\tt
crop}, and \lsym{paste}).  Other methods are modified somewhat (e.g.,
add is modified to perform a logical-or operation on bit-images).  

Bit-images may be made by performing any of several comparison
operations on images (listed above), e.g., \lsym{greater-than}, {\tt
less-than}, \lsym{equal-to}, \lsym{zero-crossings}.

The following functions are defined for bit-images:
\begin{description}
\item\lfun{bit-image-p}{ thing}
Returns t if thing is a bit-image, nil if not.

\item\lmeth{x-dim}{bit-image}{ im}
\item\lmeth{y-dim}{bit-image}{ im}
\item\lmeth{iref}{bit-image}{ im y x}

\item\lfun{make-bit-image}{ dims \&key display-type \res}
Same arguments as for \lsym{make-image}.

\item\lfun{image-from-bit-image}{ bit-image \&key \res}

\item\lfun{coerce-to-float}{ bit-image \&key \res}
Same as \lsym{image-from-bit-image}.

\item\lmeth{similar}{bit-image}{ image \&key \res}
\item\lmeth{copy}{bit-image}{ image \&key \res}
\item\lmeth{crop}{bit-image}{ image y x y-size x-size \&key \res}
\item\lmeth{paste}{bit-image}{ im big-im x-offset y-offset \&key \res}

\item\lmeth{invert}{bit-image}{ bit-image \&key \res}
equivalent macro \lsym{not.}

\item\lmeth{add}{bit-image}{ bit-image1 bit-image2 \&key \res}
equivalent macro \lsym{or.}

\item\lmeth{mul}{bit-image}{ bit-image1 bit-image2 \&key \res}
equivalent macro \lsym{and.}

\end{description}


\subsection{Operations on Image-Pairs, Complex-Images, and Polar-Images}

Since {\tt image-pairs} are made up of lists of images, many of the
image processing functions described above are defined on image-pairs.
The functions \lsym{mul} and \lsym{div} are redefined for complex images
to perform complex arithmetic.  

The arithmetic functions (e.g., \lsym{add}, \lsym{sub}, \lsym{mul}, and
\lsym{div}) are not yet defined correctly for {\tt polar-images}.
Convert them to complex images first (using \lsym{polar-to-complex}),
and then do the arithmetic.

The following additional functions are defined for image pairs and
complex images:
\begin{description}
\item\lfun{make-image-pair}{ dims \&key display-type \res}
The argument {\tt dims} may be either a dimension list or a list of two
images.  If it is a list of images then the first image in the list is
taken as the y-component (or left-image), and the second is taken as
the x-component (or right-image).

\item\lfun{make-complex-image}{ dims \&key display-type \res}
The argument {\tt dims} may be either a dimension list or a list of
two images.  If it is a list of images then the first image in the
list is taken as the imaginary part, and the second is taken as the
real part.

\item\lfun{make-polar-image}{ dims \&key display-type \res}

\item\lmeth{make-slice}{image-pair}{ image-pair \&key x y display-type \res}
See above description under image operations. 

\item\lfun{image-pair-p}{ thing}
\item\lfun{complex-image-p}{ thing}
\item\lmeth{print-values}{image}{ image-pair \&key y x y-size x-size}
See Section \ref{sec:images}.

\item\lfun{first-image}{ image-pair \&key \res}
If called without a result argument, this acts as a standard accessor
function, returning the y component of the image pair.  If called with
a result argument, this copies the y component into the argument (or
into a new image with the appropriate name if the argument is a symbol
or string).

\item\lfun{second-image}{ image-pair \&key \res}
See description of first-image.

\item\lfun{y-component}{ image-pair \&key \res}
Same as \lsym{first-image}.

\item\lfun{x-component}{ image-pair \&key \res}
Same as \lsym{second-image}.

\item\lfun{left-image}{ image-pair \&key \res}
Same as \lsym{first-image}.

\item\lfun{right-image}{ image-pair \&key \res}
Same as \lsym{second-image}.

\item\lfun{imaginary-part}{ complex-image \&key \res}
Same as \lsym{first-image}.

\item\lfun{real-part}{ complex-image \&key \res}
Same as \lsym{second-image}.

\item\lmeth{magnitude}{polar-image}{ polar-image \&key \res}
Same as \lsym{first-image}.

\item\lmeth{complex-phase}{polar-image}{ polar-image \&key \res}
Same as \lsym{second-image}.

\item\lfun{switch-components}{ image-pair}

\item\lmeth{magnitude}{image-pair}{ image-pair \&key \res}
Returns the square root of the sum of the squares of the two sub-images.

\item\lfun{square-magnitude}{ image-pair \&key \res}
Returns the sum of the squares of the two sub-images.

\item\lmeth{complex-phase}{polar-image}{ image-pair \&key \res}
Returns the arctangent of real-part/imaginary-part (values between -pi
and pi).

\item\lfun{complex-conjugate}{ complex-image \&key \res}
\item\lfun{complex-to-polar}{ complex-image \&key \res}
Returns a {\tt polar-image}.

\item\lfun{polar-to-complex}{ polar-image \&key \res}
Returns a {\tt complex-image}.

\item\lmeth{fft}{complex-image}{ complex-image \&key inverse \res}
Returns a complex image that is the FFT of the image.  The x (y) size
of the result is the smallest power of two greater than the x (y)
dimension of the image.  Computes the inverse FFT if the :inverse
parameter is non-nil.
\end{description}


\subsection{Operations on Discrete Functions}

Most of the image processing functions (listed above) are also defined
for {\tt discrete-functions} and {\tt complex-discrete-functions}.
Most of the functions defined for {\tt complex-images} are also
defined for {\tt complex-discrete-functions}.  In addition, the
following functions are defined for discrete-functions:
\begin{description}
\item\lfun{make-discrete-function}{ func min max \&key size interpolator display-type \res}
Makes a discrete function of {\tt func} from {\tt min} to {\tt max}.
The number of bins is specified by the keyword argument {\tt :size}.
{\tt :interpolator} is the function used to interpolate between
samples.  Example of use is given in Section
\ref{sec:discrete-function}.

\item\lfun{make-histogram}{ image \&key display-type range binsize bincenter size \res}
Returns a histogram of the image values.  May be called via a mouse
click (see Section \ref{sec:mouse}).  The default number of bins ({\tt
size}) may be set from the {\tt viewable defaults} menu.  The default
{\tt binsize} is the range of the image divided by the number of bins.
{\tt Bincenter} is the position of the center of any of the bins,
i.e., {x : x = (bincenter + i * binsize)}, for integer i, is the set
of bin centers.  {\tt Range} can be used to specify a subrange of the
image values.  If range and binsize are both passed, then they
determine the number of bins.  For example:
\begin{verbatim}
(make-histogram im :binsize 1.0 :range '(100 200)}
\end{verbatim}
histograms values between 100 and 200, with binsize 1.0 (i.e., size =
102, including the endpoints).

\item\lfun{evaluate}{ discrete-function value}
Evaluates the function at value.  Interpolates if value is between two
entries in the table.

\item\lfun{distribution-mean}{ discrete-function}
Mean of a df, treated as a probability distribution.  For example, if
you make a histogram from an image, then {\tt (mean image)} and {\tt
(distribution-mean histogram)} are the same.

\item\lfun{distribution-variance}{ discrete-function}
Variance of a df, treated as a probability distribution.

\item\lfun{distribution-mean-and-variance}{ discrete-function}
Returns multiple values.

\item\lfun{range}{ discrete-function}
Returns multiple values (max-min, min, max).

\item\lfun{domain}{ discrete-function}
Returns multiple values, (max-min, min, max).

\item\lfun{make-soft-threshold-df}{cutoff width \&key invert \res}
\end{description}


\subsection{Operations on Sequences}

Since image-sequences are made up of lists of images, many of the
image processing functions described above are also defined for
image-sequences.  In addition, the following functions are particular
for sequences:

\begin{description}
\item\lfun{make-viewable-sequence}{ list \&key length sub-viewable-spec display-type \res}
List must be a list of viewables or nil.  If it is nil, must provide
:length and :sub-viewable-spec.

\item\lfun{make-image-sequence}{ list \&key length sub-viewable-spec display-type \res}
The argument {\tt list} may be either a list of images or a dimension
list or nil.  If it is nil, must pass either (1) :length and
:sub-viewable-spec or (2) :length and :dimensions.  The dim-list can
be (z y x) or (y x).  In the latter case, the :length keyword must
also be provided.  If im-list is 2 integers it is interpreted as
:dimensions.  If im-list is 3 integers, it is interpreted as
:dimensions=(cdr im-list) and :length=(car im-list).

\item\lfun{viewable-sequence-p}{thing}
\item\lfun{image-sequence-p}{thing}

\item\lmeth{viewable-list}{viewable-sequence}{sequence}
Conses a list of the viewables in the sequence.
\item\lmeth{image-list}{image-sequence}{sequence}
Conses a list of the images in the sequence.

\item\lmeth{dimensions}{sequence}{ sequence}
Returns the (y,x) dimensions of the images in the sequence.
\item\lmeth{x-dim}{sequence}{ sequence}
\item\lmeth{y-dim}{sequence}{ sequence}
\item\lfun{sequence-length}{ sequence}
equivalent macro \lsym{length.}

\item\lmeth{frame}{number viewable-sequence}{n sequence \&key \res}
or {\tt (frame sequence n \&key \res)} \\ 
Accessor for the sub-images of a sequence.  Returns the nth frame of
the sequence as an image.  If the result arg is nil, the existing
sub-image is returned.  If it is a symbol, the existing sub-image is
returned, after altering its name.  If an image is passed, then the
appropriate sub-image is {\em copied} into that sub-image.

\item\lfun{sub-sequence}{ seq start-frame \&optional end-frame \&key \res}
\item\lfun{append-sequence}{ seq1 seq2 \&key \res}
equivalent macro \lsym{append.}

\item\lfun{make-iterated-image}{ im func length \&rest func-args}
Makes a sequence of images, starting with {\tt im}, by iteratively
applying {\tt func} (this must be a function which takes an image
argument).

\item\lmeth{make-slice}{sequence}{ seq \&key x y z display-type \res}
Slices through the three-D volume of the sequence.  If one coord is
supplied, this will return a 2D image.  If two are supplied, it will
return a 1D image.

\item\lfun{make-synthetic-image-sequence}{dims func \&key length
x-range y-rante z-range display-type \res}
Analogous to {\tt make-image-sequence}.

\item\lfun{reduce.}{func seq \&key start end from-end initial-value
\res} 
Modeled on common Lisp reduce.

\item\lfun{map.}{func seq \&rest args}
Modeled on common Lisp map.  Args can be more sequences and a \res
argument.
\end{description}


\subsection{Operations on Color-Images}

{\tt Color-images} are just like {\tt image-sequences} except that
the sub-images in the sequence are interpreted as different color
bands.  All of the methods defined for sequences apply to color-images
as well.

The following additional functions are defined for color-images:
\begin{description}
\item\lfun{color-image-p}{ thing}
Returns t if thing is a color-image, nil if not.

\item\lfun{make-color-image}{ list \&key length display-type \res}
The argument {\tt list} may be either a list of images or a dimension
list.  If it is a dimension list then the {\tt :length} keyword must
be specified.

\item\lmeth{div}{color-transform}{ color-image array \&key \res}
Calls {\tt (matrix-mul-transpose color-image array \res \res)}.  See
the section on {\tt viewable-matrix} operations for details.

\item\lmeth{div}{rgb->yiq}{ color-image array \&key \res}
Calls color-transform with standard array for converting from rgg to
yiq.

\item\lmeth{div}{yiq->rgb}{ color-image array \&key \res}
Calls color-transform with standard array for converting from yiq to
rgb.
\end{description}



\subsection{Operations on Viewable-Matrices and Image-Matrices}

Operations on {\tt viewable-Matrices} and {\tt image-matrices} are
defined in the file {\bf viewable-matrix.lisp}.  The {\tt data} slot
of a {\tt viewable-matrix} contains an array of sub-viewables, as
discussed in section \ref{sec:viewable-matrix}.  The majority of the
operations defined on {\tt viewable-matrices} are the standard image
processing operations discussed above, e.g., add, mul, copy, similar,
point-operation, minimum, mean, etc. (see {\bf viewable-matrix.lisp}
for complete list).  Additional functions specific for
viewable-matrices are listed below.  These include some matrix-mul
operations, analogous to those that are defined for arrays.  These
matrix-mul operations are defined for any combination of a
viewable-matrix, an array, and a vector.  In other words, you can
multiply a viewable-matrix by another viewable-matrix or you can
multiply a viewable-matrix by an array (of numbers).  ***Some of the
methods on viewable-matrix/vector are not finished yet.
\begin{description}
\item\lfun{make-viewable-matrix}{ data \&key display-type \res}
Data can be a nested-list of viewables, or an array of viewables.

\item\lfun{make-image-matrix}{ data \&key size sub-viewable-spec display-type \res}
Data can be a nested-list of images, an array of images, a dim-list, or
nil.  If data is nil, must pass either (1) :size and
:sub-viewable-spec or (2) :size and :dimensions.  If data is a
dim-list, then :size must also be specified.

\item\lfun{diagonal-image-matrix}{ im-list \&key display-type \res}
Makes a new image-matrix putting images from im-list along the diagonal.

\item\lfun{viewable-matrix-p}{ thing}
Returns t if thing is a viewable-matrix, nil if not.

\item\lfun{image-matrix-p}{ thing}
Returns t if thing is a image-matrix, nil if not.

\item\lmeth{dimensions}{viewable-matrix}{ vbl-mat}
Returns the dimensions of one of the sub-viewables.

\item\lmeth{x-dim}{viewable-matrix}{ vbl-mat}
Returns the x-dim of one of the sub-viewables.

\item\lmeth{y-dim}{viewable-matrix}{ vbl-mat}
Returns the y-dim of one of the sub-viewables.

\item\lmeth{x-dim}{viewable-matrix}{ vbl-mat}
Returns the x-dim of one of the sub-viewables.

\item\lmeth{y-dim}{viewable-matrix}{ vbl-mat}
Returns the y-dim of one of the sub-viewables.

\item\lmeth{row-dim}{viewable-matrix}{ vbl-mat}
Returns the row-dim of one of the data slot (the number of rows in the
matrix).

\item\lmeth{col-dim}{viewable-matrix}{ vbl-mat}
Returns the col-dim of one of the data slot (the number of cols in the
matrix).

\item\lmeth{size}{viewable-matrix}{ vbl-mat}
Returns a list of (row-dim col-dim).

\item\lmeth{total-size}{viewable-matrix}{ vbl-mat}
Returns (* row-dim col-dim).

\item\lmeth{vectorize}{viewable-matrix}{ vbl-mat \&key size x-offset y-offset \res}

\item\lmeth{columnize}{viewable-matrix}{ vbl-mat \&key size x-offset y-offset \res}

\item\lmeth{matrix-transpose}{viewable-matrix}{ vbl-mat \&key \res}

\item\lmeth{determinant}{viewable-matrix}{ vbl-mat \&key \res}
Works only for 2x2 or 3x3 matrices for the time-being.

\item\lmeth{matrix-inverse}{viewable-matrix}{ vbl-mat \&key \res}
Works only for 2x2 or 3x3 matrices for the time-being.

\item\lmeth{matrix-mul}{viewable-matrix}{ thing1 thing2 \&key \res}
Thing1 and thing2 may be viewable-matrices, arrays, or vectors.

\item\lmeth{matrix-mul-transpose}{viewable-matrix}{ thing1 thing2 \&key \res}

\item\lmeth{matrix-transpose-mul}{viewable-matrix}{ thing1 thing2 \&key \res}

\item\lmeth{dot-product}{viewable-matrix}{ thing1 thing2 \&key \res}

\item\lmeth{normalize}{viewable-matrix}{ thing1 thing2 \&key norm \res}

\item\lmeth{outer-product}{viewable-matrix}{ thing1 thing2 \&key \res}

\end{description}


\subsection{Operations on Filters}

\begin{description}
\item\lfun{make-filter}{ kernel \&key display-type edge-handler start-vector step-vector hex-start \res}
(described in section~\ref{sec:filters}).

\item\lfun{make-separable-filter}{ y-kernel x-kernel \&key display-type edge-handler start-vector step-vector \res}
(described in section~\ref{sec:filters}).

\item\lmeth{x-dim}{filter}{ filter}

\item\lmeth{y-dim}{filter}{ filter}

\item\lmeth{dimensions}{filter}{ filter}

\item\lfun{volume}{ filter}
Equivalent function: \lsym{sum-of}.

\item\lfun{filter-p}{ thing}

\item\lfun{separable-filter-p}{ thing}

\item\lmeth{copy}{filter}{ filter \&key \res}

\item\lmeth{negate}{filter}{ filter \&key \res}

\item\lmeth{mul}{filter}{ filter thing \&key \res}
Thing may be a number or another filter.

\item\lmeth{add}{filter}{ filter thing \&key \res}
Thing may be a number or another filter.

\item\lmeth{sub}{filter}{ filter thing \&key \res}
Thing may be a number or another filter.

\item\lfun{shift-by-pi}{ filter \&key \res}
Defined on one-dimensional filters only.  Returns a filter with taps
that are frequency-modulated from the original filter taps (i.e.,
multiplies every other sample by -1).

\item\lfun{interpolate-hex-image}{ im \&key filter \res}
Does a simple interpolation from a hex image to a full-density
(rectangular) image.  Half of the samples in a hex image are zero
(every other pixel on a line, with lines staggered).
\end{description}


\subsection{Operations on Pyramids}

The pyramid module is loaded into OBVIUS by evaluating:
\begin{verbatim}
(obv-require :pyramid)
\end{verbatim}
Since pyramids are made up of lists of images, many of the image
processing functions described above are also defined for pyramids,
including {\tt mul}, {\tt sub}, {\tt div}, {\tt add}, {\tt
square-error}, {\tt abs-error}, {\tt linear-xform}, {\tt clip}, {\tt
abs-value}, {\tt square}, {\tt square-root}, {\tt zero!}, {\tt
point-operation}, {\tt apply-filter}, {\tt expand-filter}, {\tt blur},
{\tt greater-than}, {\tt greater-than-or-equal-to}, {\tt less-than},
{\tt less-than-or-equal-to}, {\tt minimum}, {\tt maximum}, {\tt
range}.

In addition, the following functions are particular for gaussian
pyramids, laplacian pyramids, and qmf-pyramids.  Take a look at
{\tt <obv>/tutorials/image-processing/pyramid-tutorial.lisp} for
examples of how to use these functions.

\begin{description}
\item\lfun{make-gaussian-pyramid}{ image \&key level display-type filter \res}
Make a gaussian-pyramid (a low-pass pyramid) from an image, and expand
the pyramid to the specified level.

\item\lfun{gaussian-pyramid-p}{ thing}

\item\lfun{make-laplacian-pyramid}{ image \&key level display-type
forward-filter inverse-filter \res}
Make a laplacian-pyramid (a bandpass pyramid) from an image, and
expand the pyramid to the specified level.

\item\lfun{laplacian-pyramid-p}{ thing}

\item\lfun{make-separable-qmf-pyramid}{ image \&key level display-type
lo-filt edge-handler \res}
Make a QMF-pyramid (a bandpass pyramid) from an image, and expands the
pyramid to the specified level.  Lo-filt is the kernel of an
appropriate low-pass filter.  The other filters are automatically
contructed from this kernel.

\item\lfun{qmf-pyramid-p}{ thing}

\item\lmeth{build}{pyramid}{ level}
Extend the pyramid to the specified level.

\item\lmeth{access}{pyramid}{ level \&key \res}
Returns the viewable that is at the specified level of the pyramid.

\item\lmeth{access-band}{pyramid}{ level band \&key \res}
Returns the viewable that is at the specified level (and band) of the
pyramid, i.e., use this to get the vertical band on the second level
of a qmf pyramid.

\item\lmeth{low-band}{pyramid}{}
Returns the low-pass band (top level) of the pyramid.

\item\lmeth{collapse}{pyramid}{}
See pyramid-tutorial for details.

\item\lmeth{height}{pyramid}{}
Returns the height (the number of levels built so far) of the
pyramid.
\end{description}




\subsection{Operations on Arrays}

\comment {EJC
* Need to fix methods to be \lmeth. Ugh.
* should we export displaced-row?
* where should covariance go? stats or matrix? correlation?
* what about simplex? Shuold there be a fitting module?
should regress go with stepit in fitting module?
* stats, stepit, regression.
}

Many of the image operations are also defined for arrays.  These
methods call C code if the arguments are all {\tt single-float}
arrays, or if they are all {\tt fixnum} arrays.  If they are some
other array type, or undeclared type, or if they are mixed array
types, then the operation is done in lisp (and is slower).  Some of
the operations only make sense for vectors or for 2d arrays. Some
image processing operations that are not yet implemented for
arrays.  

Below, a few routines are documented in detail since they are
not defined on images. At the end, a list of array operations 
that work like viewable operations is given.
\begin{description}
\item\lfun{almost-equal}{ thing1 thing2 \&key tolerance}
Compare two arrays and return the first array if 
the maximum difference between the two arrays is 
no more than {\tt tolerance}. Otherwise return nil.
The default tolerance is set in the global variable {\tt *tolerace*}.

\item\lfun{scalar-multiple}{ thing1 thing2 \&key tolerance}
Compare two arrays and return the first array if 
the first array is nearly a scalar multiple of the other.
This is decided as follows. The least-squares 
scaling coefficient between the arrays is computed, 
and the maximum difference between the first array
and the scaled copy of the second array is compared
to {\tt tolerance}. If the difference is no larger
than {\tt tolerance}, the first array is returned.
Otherwise nil is returned.
The default tolerance is set in the global variable {\tt *tolerace*}.

\item\lfun{randomize}{ array number \&key \res }
Return an array with the elements of {\tt array}
individually randomized by adding uniformly distributed
values between {\tt - number} and {\tt number}.

\item\lfun{shuffle}{ array \&key \res }
Return an array that contains the elements of {\tt array}
randomly reordered.
\end{description}

\begin{verbatim}
with-displaced-vectors
total-size dimensions
check-size rank z-dim y-dim x-dim similar copy
coerce-to-float print-values zero! fill! normalize
mean variance third-moment fourth-moment sum-of
product-of maximum minimum range mean-square-error
mean-abs-error max-abs-error point-minimum point-maximum
square-error abs-error add mul sub div negate square abs-value
linear-x-form square-root power natural-logarithm
point-operation periodic-point-operation quantize clip
circular-shift crop paste 
greater-than less-than equal-to
greater-than-or-equal-to less-than-or-equal-to 
covariance correlate array-magnitude
array-abs-value array-square-magnitude array-complex-phase
array-cross-product
\end{verbatim}


\subsection{Operations on Lists}
Many arithmetic operations are also define on lists.
They usually work just like the array or image equivalents.
Some operations (for example, {\tt mean})
will function on lists of viewables, arrays, etc.
Many only make sense for lists of numbers
(for example, {\tt minimum}).

Below, a few routines are documented in detail since they are
not defined on arrays. At the end, a list of list operations 
that work like array operations is given.
\begin{description}
\item\lfun{elements}{ array \&key \res }
Returns a list containing the elements of {\tt array}.
\end{description}

\begin{verbatim}
add sub mul div maximum minimum mean variance
square square-root natural-logarithm negate abs-value normalize
fill! dimensions total-size similar copy shuffle 
vector-length vector-distance randomize almost-equal scalar-multiple 
\end{verbatim}


\subsection{Operations on Matrices}

The matrix module may be loaded into OBVIUS by evaluating:
\begin{verbatim}
(obv-require 'matrix)
\end{verbatim}
The following is a list of matrix operations that are defined on
arrays and vectors.  Many of these methods are implemented only for
float arrays.  By convention, vectors are treated as row vectors.
Using a row vector in a place where a column vector is required will
generate a continuable error, where you will usually have the option to
transpose the row-vector.  To work with column vectors, you must use
Nx1 arrays.  Various utilities are provided to simplify row and column
operations and conversions.

\index{floating-point exceptions}
Note also that some of the C code called by these methods is not
protected against floating-point exceptions.  By default, Lisp will
generate a continuable error if a floating-point exception is
encountered while executing a foreign function.   To turn this
behavior off, evaluate {\tt (setf (enabled-floating-point-traps) nil)}.
Note that you will no longer receive {\em any} indication that these
errors have occured!

OBVIUS Matrix operations are divided into three categories:
basic matrix arithmetic, 
operations related to the SVD and other decompositions,
and utilities (such row and column operations).


\mysubsubsec{Matrix Arithmetic}

\begin{description}

\item\lmeth{matrix-mul}{array}{ thing1 thing2 \&key \res}
{\tt Thing1} and {\tt thing2} may be arrays or vectors (subject to the
row-vector restrictions described above).

\item\lmeth{matrix-transpose}{array}{ array \&key \res}

\item\lfun{matrix-transpose-mul}{array}{ thing1 thing2 \&key \res}
Multiplies the (tranpose of {\tt thing1}) with {\tt thing2}, without
actually having to compute the transpose.  {\tt Thing1} and 
{\tt thing2} may both be arrays, or one may be an array and the other a
vector (subject to the row-vector restrictions described above).

\item\lmeth{matrix-mul-transpose}{array}{ thing1 thing2 \&key \res}
Multiplies the {\tt thing1} times (tranpose of {\tt thing2}).  {\tt
Thing1} and {\tt thing2} may both be arrays, or one may be an array
and the other a vector (subject to the row-vector restrictions
described above).

\item\lmeth{outer-product}{vector}{ vector1 vector2 \&key \res}

\item\lmeth{cross-product}{vector}{ vector1 vector2 \&key \res}

\item\lmeth{dot-product}{vector}{ thing1 thing2 \&key \res}
Same as {\tt matrix-transpose-mul}, but returns a scalar.

\item\lmeth{vector-length}{vector}{ vector }
Euclidean length of the vector.

\item\lmeth{vector-length}{vector}{ vector1 vector1 }
Euclidean distance between vectors.

\item\lmeth{normalize}{vector}{ vector \&key \res}
Returns multiple values.  First, a vector of unit length in the
direction of the argument {\tt vector}.  
Second, the length of the argument {\tt vector}.

\item\lmeth{matrix-inverse}{array}{ array \&key dimension-limit
condition-number-limit  singular-value-limit \res}
Computes the Moore-Penrose left pseudo-inverse using the SVD.  
SVD returns $(W,U,V)$ such that $M = S U V^t$.  
Pseudo-inverse is $M^{\sharp} = V 1/W U^t$.  
Allows the calling function to restrict the dimensions of the
pseudo-inverse.
If {\tt dimension-limit} is non-nil (it must be a
whole number N), throw away all but the first N dimensions by setting
those singular values to 0.  If {\tt singular-value-limit} is non-nil (it
must be a number x), set to zero those singular values smaller than x.
Default {\tt singular-value-limit} is *machine-tolerance*.
If {\tt condition-number-limit} is non-nil (it must be a number x), 
set to zero those singular values with condition numbers larger than
x.
The values returned are the inverse matrix, 
the minimum singular value (not including any zeros),
the maximum singular value,
and the number of nonzero singular values used
(after the clipping of singular values as described above).

\item\lfun{determinant}{ matrix \&key ignore-zeros u v}
Uses SVD to compute determinant of a square matrix.  {\tt
:Ignore-zeros} is an option to ignore any singular values that are zero,
implicitly assuming that the matrix is nonsingular.  The
The arguments {\tt :u} and {\tt :v} may be used to pass u and v matrices
to this function (and on to the SVD), to avoid consing.

\item\lfun{symmetric-p}{ array} ***
\item\lfun{square-p}{ array} ***
\item\lfun{diagonal-p}{ array} ***
\item\lfun{identity-p}{ array} ***
\item\lfun{unitary-p}{ array} ***
\item\lfun{orthogonal-p}{ array} ***
\item\lfun{singular-p}{ array} ***

\item\lfun{identity-matrix}{ size \&key \res}
Returns a {\tt size} by {\tt size} identity matrix.

\item\lfun{diagonal-matrix}{ thing \&key \res}
{\tt Thing} may be either a list or a vector.  Returns a diagonal
matrix with entries from thing.

\item\lfun{diagonal}{ matrix \&key \res}
Returns a vector with the diagonal entries of {\tt matrix}
as its contents.

\item\lfun{matrix-trace}{ matrix }
Returns the sum of the  diagonal entries of {\tt matrix}.

\end{description}


\mysubsubsec{Functions Related to SVD and Other Decompositions}

\begin{description}

\item\lfun{singular-value-decomposition}{ matrix \&key u v s}
equivalent macro \lsym{svd} Computes the singular values and complete
orthogonal decomposition of a real rectangular matrix, $M$. The matrix
is decomposed into $U W V^t$, where $U$ and $V$ are orthogonal
matrices, that consist of the eigenvectors of $M M^t$ and $M^t M$
respectively.  The entries of $W$ are nonnegative, the singular
values, which are the square roots of the eigenvalues of $M^t M$,
ordered in decreasing sequence.  If $M$ is a square symmetric matrix,
the columns of $U$ are its eigenvectors and the elements of $S$ are
its eigenvalues.  The function returns the multiple values $(S,U,V)$
($S$ is returned as a vector of values on the diagonal of the $W$
matrix). Should the function fail to compute the decomposition nil is
returned and a warning is signalled.  The arguments {\tt :s}, {\tt :u}
and {\tt :v} can be passed to avoid consing new arrays.

\lfun{with-svd}{ (s u v) matrix . body}
*** not documented

\lfun{with-static-svd}{ (s u v) matrix . body}
*** not documented

\item\lfun{singular-values}{ matrix }
Returns a vector of the singular values of the matrix, in decreasing
order (see description of singular-value-decomposition).

\item\lfun{left-singular-matrix}{ matrix }
Returns the U matrix from the singular-value-decomposition.

\item\lfun{right-singular-matrix}{ matrix }
Returns the V matrix from the singular-value-decomposition.

\item\lfun{principal-components}{ matrix \&key dimension scale}
Returns multiple values.  First, the principal components of the rows
of {\tt matrix} in a matrix of the same dimension, with most important
components first.  Second, a vector with the principal values of {\tt
matrix}.  Third, a vector with the singular values of {\tt matrix}.
When {\tt scale} is t, the row of the principal components matrix are
scaled by the principal values. By default they have unit length.  The
{\tt dimension} argument allows you to restrict how many components
(rows) you want in the resulting component matrix,

\item\lfun{condition-number}{ array}
Returns square-root of quotient of minimum singular value divided by
maximum singular value.

\item\lfun{quadratic-decomposition}{ matrix \&key \res}
Given matrix $M$, returns a matrix $A$ such that $M = A A^t$.  Uses A
sufficient conditions is that $M$ be positive-definite symmetric. Uses
SVD.  Generates an error if the matrix cannot be decomposed in this
way.

\item\lfun{row-space}{ matrix \&key \res singular-value-limit}
Return an orthogonal basis for the space spanned by the rows of the
matrix. The basis is generated using the SVD. The rows
of the basis matrix returned are the principal components,
and are in order of decreasing singular values. 
Any basis rows with singular values below {\tt singular-value-limit} 
are not included in the basis. {\tt singular-value-limit} defaults to 0.0.

\item\lfun{row-null-space}{ matrix \&key \res singular-value-limit}
Return an orthogonal basis for the space of vectors
orthogonal to the rows of the matrix.
See {\tt row-space} for details.

\item\lfun{col-space}{ matrix \&key \res singular-value-limit}

\item\lfun{col-null-space}{ matrix \&key \res singular-value-limit}

\item\lfun{solve-eigenvalue}{ matrix \&key \res}
Find unit vector $x$ such that $x$ minimizes $x^t M x$, for square
symmetric matrix $M$.  Finds the eigenvector corresponding to the
smallest eigenvalue of $M$.  Uses SVD.

\item\lfun{qr-decomposition}{ matrix \&key q r}
equivalent macro \lsym{qrd} \\ This function computes the QR
decomposition of the matrix, such that matrix = q r, where q is
orthonormal and r is upper triangular.  The function returns a list of
q and r.  The arguments {\tt :q} and {\tt :r} may be used to pass q
and r matrices to this function, to avoid consing.

\item\lfun{row-space-qr}{ matrix \&key q r \res}
Uses QR-decomposition instead of SVD.

\item\lfun{col-space-qr}{ matrix \&key r \res}
Uses QR-decomposition instead of SVD.

\item\lfun{row-null-space-qr}{ matrix \&key q r tmp tol \res}
Uses QR-decomposition instead of SVD.

\item\lfun{col-null-space-qr}{ matrix \&key q r tmp tol \res}
Uses QR-decomposition instead of SVD.

\item\lfun{determinant-qr}{ matrix \&key ignore-zeros q r}
Uses QR-decomposition instead of SVD.

\end{description}

\mysubsubsec{Matrix Utilities}
Unless otherwise noted, all row operations have corresponding column
operation counterparts. Row and column indices begin at 0.

\begin{description}

\item\lfun{make-matrix}{ \&rest stuff }
Return a matrix whose contents is {\tt stuff}.
The latter may be numbers, lists of numbers,
lists of lists of numbers, vectors, or arrays.
An {\tt single-float} matrix of the appropriate size is made
and filled with {\tt stuff}. 
In the case of vectors and lists of numbers, 
the rows of the resulting array are filled in order.

\item\lfun{vectorize}{ matrix size x-offset y-offset}
Returns a vector whose contents is the entire contents
of {\tt matrix}. The result is displaced to the original matrix.
The keyword arguments allow you to displace the vector to any location
and control its size.

\item\lfun{columnize}{ matrix size x-offset y-offset}
Same as vectorize, except makes a {\tt size} by 1 displaced array.

\item\lfun{row}{ row matrix }
Returns a vector whose contents is a copy of 
the specified row of {\tt matrix}.

\item\lfun{displaced-row}{ row matrix }
Returns a vector {\em displaced to} the specified row of the matrix.
Hence, if the resulting vector is modified, 
the data in the original matrix will be modified.
This function has no column-operation counterpart.

\item\lfun{rows}{ row matrix }
Returns a list of vectors which are copies 
of the rows of {\tt matrix}.

\item\lfun{displaced-rows}{ row matrix }
Returns a list vectors displaced to the rows of {\tt matrix}.
This function has no column-operation counterpart.

\item\lfun{col}{ col matrix }
Returns an Nx1 matrix whose contents is a copy of 
the specified column of the {\tt matrix}.

\item\lfun{cols}{ col matrix }
Returns a list of copies of the columns of {\tt matrix}.

\item\lfun{row-dim}{ matrix }
Returns the number of rows of {\tt matrix}.

\item\lfun{col-dim}{ matrix }

\item\lmeth{add-rows}{matrix vector}{ matrix vector \&key \res }
Add a row vector to each row of a matrix.
Arguments can be given in either order.

\item\lmeth{add-rows}{vector matrix}{ vector matrix \&key \res }

\item\lmeth{add-cols}{matrix vector}{ matrix vector \&key \res }

\item\lmeth{add-cols}{vector matrix}{ vector matrix \&key \res }

\item\lmeth{sub-rows}{matrix vector}{ matrix vector \&key \res }
Subtract {\tt vector} from each row of {\tt matrix}.

\item\lfun{sub-rows}{vector matrix}{ vector matrix \&key \res }
Subtract a each row of {\tt matrix} from {\tt vector}.

\item\lmeth{sub-cols}{matrix vector}{ matrix vector \&key \res }

\item\lmeth{sub-cols}{vector matrix}{ vector matrix \&key \res }

\item\lfun{mean-rows}{ matrix \&key \res }
Returns the row vector whose contents is the mean of all the rows
of {\tt matrix}.

\item\lfun{mean-cols}{ matrix \&key \res }

\item\lfun{normalize-rows}{ matrix \&key \res }
Returns a matrix with the rows of {\tt matrix} independently
normalized to a vector length of 1.0.

\item\lfun{normalize-cols}{ matrix \&key \res }

\item\lfun{sum-rows}{ matrix \&key \res }
Returns the row vector whose contents is the sum of all the rows
of {\tt matrix}.

\item\lfun{sum-cols}{ matrix \&key \res }

\item\lfun{covariance-rows}{ matrix \&key \res sample}
Calculates the covariance matrix associated with a matrix.  The matrix
is treated as many multivariate observations, with each observation in
a row.  Hence, the ijth entry of the resulting covariance matrix is
the covariance between the ith and jth columns of {\tt matrix}.
If sample is t, the sample covariance is calculated,
using (n-1) rather than n as the divisor.

\item\lfun{covariance-cols}{ matrix \&key \res sample}

\item\lfun{shuffle-rows}{ matrix \&key \res }
Returns a matrix with the rows of {\tt matrix} randomly shuffled.
This function has no column-operation counterpart.

\item\lfun{swap-rows}{ matrix first-row second-row \&key \res }
Returns a matrix identical to {\tt matrix},
but with rows numbered {\tt first-row} and {\tt second-row} swapped.
This function has no column-operation counterpart.

\item\lfun{sort-rows}{ matrix predicate \&key key \res }
Returns a matrix containing the rows of {\tt matrix} 
sorted according to {\tt predicate}, using {\tt key}.
This arguments are analagous to the Common Lisp {\tt sort} function,
except that {\tt matrix} is not modified.
This function has no column-operation counterpart.

\item\lfun{paste-rows}{ matrices \&key \res }
Paste a list of matrices into one big matrix,
such that each row of the result is a row from one of the
source matrices. In other words, paste the matrices
on top of one another.

\item\lfun{paste-cols}{ matrices \&key \res }

\item\lfun{append-rows}{ \&rest matrices }
Like {\tt paste-rows}, except that the matrices
are passed directly as arguments, rather than as a list.

\item\lfun{append-cols}{ matrices \&key \res }

\end{description}




%%%%%%%%%%%%%%%%%%%%%%%%%%%%%%%%%%%%%%%%%%%%%%%%%%%%%%%%%%%%%%%%%%%%%%%%%%%%%%%%%%
\section{Data Analysis}
\label{sec:data-analysis}

OBVIUS has some basic tools for data analysis and simulation. 
These include:

\begin{description}
\item[Matrix Tools] Singular value decomposition and related
operations, and row and column operations 
for dealing with multivariate data.

\item[Statistical and Simulation Tools] 
Simple methods for sample statistics, 
probability distributions commonly used in modelling,
and routines for generating noise drawn from particular distributions.

\item[Model Fitting] Linear least squares regression and STEPIT.

\end{description}



\subsection{Matrix Tools}

Many multivariate data analysis problems can be handled by
treating data as rows or columns in matrices, 
and applying simple linear algebra techniques to these matrices.
The matrix module contains many tools for handling matrices.

The matrix module may be loaded into OBVIUS by evaluating:
\begin{verbatim}
(obv-require :matrix)
\end{verbatim}

The following is an outline of the matrix tools available in OBVIUS.

\begin{description}
\item[Matrix Algebra]
Most basic matrix algebra routines are available. For example:

\begin{verbatim}
;; Make a diagonal matrix, and take its inverse
(setq d (diagonal-matrix '(1 2)))
(setq d-1 (matrix-inverse d))

;; Make another matrix, and take the sum and the product
(setq m (make-matrix '((1 2) (3 4))))
(setq sum (add m d))
(setq product (matrix-mul m d))
\end{verbatim}

Please check Section \ref{sec:operations} for a detailed listing of
matrix algebra routines.

\item[SVD] The singular value decomposition is at the heart of many
numerical linear algebra algorithms.  Please check Section
\ref{sec:operations} for more detail on the OBVIUS svd implementation,
and related tools such as matrix inversion, subspace algorithms, other
matrix decompositions, and linear least-squares regression.

An example calculation follows.

\begin{verbatim}
;; Example: Take the SVD of a matrix
(setq matrix (make-matrix '((1 2) (3 4))))
(svd matrix)
\end{verbatim}

SVD returns the singular values of the matrix in an vector,
and the left- and right- singular matrices as multiple return values.

\item[Row and Column Operations]
OBVIUS has routines for doing simple row and column operations.
Section \ref{sec:operations} section documents these functions in
detail.  For example, you can access the rows of an example data
matrix as follows:

\begin{verbatim}
;; Create a matrix and look at its row vectors
(setq matrix (make-matrix '((1 2) (3 4) (5 6))))
(setq first-row (row 0 matrix))
(setq second-row (row 1 matrix))

;; Swap the first and second rows of the matrix
(swap-rows matrix 0 1)
(setq first-row (row 0 matrix))

;; Look at the covariance matrix of the variables in each row
(setq covariance (covariance-rows matrix))
\end{verbatim}

\end{description}


\subsection{Statistical and Simulation Tools}

The statistics module may be loaded into OBVIUS by evaluating:
\begin{verbatim}
(obv-require :statistics)
\end{verbatim}

The following is a short outline of the statistical and simulation
functions that are available.  These functions are defined in the
files {\bf gaussian-noise.lisp} and {\bf statistics.lisp}:
\begin {description}
\item[Sample statistics] Some simple methods are available for
standard statistics.  Example:
\begin{verbatim}
;; Make a couple of vectors with data
(setq matrix-1 (make-matrix '((1 2 3 8 10 4))))
(setq matrix-2 (make-matrix '((4 5 2 2 1 8))))

;; Evaluate some statistics about the mean
(setq mean (mean matrix-1))
(setq standard-error (standard-error matrix-1))

;; Variance etc.
(setq variance (variance matrix-1))
(setq covariance (covariance matrix-1 matrix-2))

;; Examine an invented chi-square statistic
(cumulative-chi-square (sum-of (square (sub matrix-1 matrix-2))) 3)
\end{verbatim}

\item[Standard Functions] Some commonly used probability distributions
and other functions useful for modeling data are provided.  These
include the Gaussian, Weibull, Poisson, Binomial, Chi-Square, F, T, and
others.

\item[Noise Generation] Several routines are useful 
for generating noise data for simulation.  For example, you can
generate an array filled with Poisson noise as follows:
\begin{verbatim}
;; Poisson noise example
(setq noise (make-array 100 :element-type 'single-float))
(setq lambda 7.0)
(dotimes (i (length noise))
  (setf (aref noise i) (float (poisson-noise lambda))))
(mean noise)
(variance noise)
\end{verbatim}
Other routines exist for Gaussian, Bernoulli, and uniform noise (look
through the files {\bf gaussian-noise.lisp} and {\bf statistics.lisp}).
\end{description}


\subsection{Curve Fitting}

Two types of curve fitting are available at present: linear least
squares regression, and Chandler's STEPIT general-purpose minimization
function.  These are best explained by example.  For more details,
please refer to section \ref{sec:operations} section, or examine the
source code directly.
\begin{description}

\item[Linear Least Squares Regression] Here is an example of how to
solve a (cooked) regression problem.
\begin{verbatim}
;; Set up a matrix with the predictor variable
(setq predictor (make-array '(10 3) :element-type 'single-float))

;; Put random noise into the matrix
(randomize predictor 1.0 :-> predictor)

;; Set the observed matrix to be a linear transformation of the
;; predictor
(setq transform (make-matrix '( (1 2 3) (4 1 2) (4 9 2))))
(setq observed (matrix-mul predictor transform))

;; Solve the regression problem: the answer should be
;; very close to the transformation matrix.
(regress observed predictor)
\end{verbatim}

\item[Nonlinear Curve Fitting with Stepit]
Chandler's STEPIT program is a fast, general purpose minimization
function. Conceptually, you give STEPIT a set of parameters to adjust,
and an error function that depends on the values of those parameters,
and STEPIT attempts to adjust the parameters so as to minimize the
error function. In detail, the required arguments to {\tt stepit-fit}
are:
\begin{description}
\item[error-function] function to be called with parameters
followed by other arguments (see below).
It is the return value of this function that stepit attempts to minimize. 
\item[initial-parameters]
starting point of parameters (a single-float vector or array).
\item[error-function-args]
list of arguments that are passed (after the parameters)
to the error function at each iteration. 
These arguments are passed to error-function one after another, not as a list.
\item[keywords] Detailed control of stepit's iterative process.
See the source code for more information.
\end{description}

A simple example follows that illustrates the use of the OBVIUS
interface to STEPIT.  You are seeking the entries of the vector that
is closest to a pre-set vector (the answer is obvious, the result
should be the same as the pre-set vector).  You set up the pre-set
vector, and an initial guess for the closest vector. Then, give STEPIT
the three required arguments:
\begin{verbatim}
;; Set up a dummy problem to be solved by stepit
(setq vector (make-matrix 1 2))
(setq initial-parameters (make-matrix 0 0))
(stepit-fit #'vector-distance initial-parameters (list vector))
\end{verbatim}
STEPIT maintains its own set of parameters, call them {\tt
stepit-parameters} that it adjusts after each call to the error
function.  At each iteration, the error function is called as follows:
\begin{verbatim}
(vector-distance stepit-parameters vector)
\end{verbatim}
STEPIT adjusts the entries of {\tt stepit-parameters} until this
function call is minimized. Obviously, STEPIT will figure out that the
best choice of {\tt stepit-parameters} is identical to {\tt vector}.

Althought this problem is not very interesting, most parameter-fitting
problems can be coded with this structure. STEPIT therefore provides a
general method for fitting models to data. Note that the length of
time required for a stepit fit grows exponentially with the number of
parameters, so if your models have more than about 5 parameters, be
prepared to wait!
\end{description}











%%%%%%%%%%%%%%%%%%%%%%%%%%%%%%%%%%%%%%%%%%%%%%%%%%%%%%%%%%%%%%%%%%%%%%%%%%%%%%%%%%

\section{User Interface}
\label{sec:userface}

This section describes the user interface provided for OBVIUS.

\subsection{The Mouse}
\label{sec:mouse}

Mouse clicks within OBVIUS panes are captured and interpreted by
OBVIUS.  WARNING: Some X11 or OpenWindows window managers override
the mouse bindings in obvius.  In particular, twm allows the user to
bind mouse-clicks in their .twmrc file.  If your mouse is not
responding -- check your window manager manual pages.

All mouse-handling code is in the file {\bf x-mouse.lisp}.  The
bindings can be altered within this file.  The mouse can be used to
cycle through the stacks of pictures, zoom, pan, and select images,
among other things.

The following is a list of the current mouse bindings, organized
according to the ``bucky'' keys (Shift, Control, and Meta).  All
actions are performed on the pane containing the mouse at the time of
the button click.

\begin{tabular}{|l|ccc|}
\hline
buckies & left button       & middle button  & right  button       \\
\hline
Raw	& select-viewable   & refresh        & select-pane/position-message    \\
Control & previous-picture  & next-picture   & move-to-here        \\
Shift 	& zoom-in           & zoom-out       & numerical-magnifier \\
Meta	& describe-viewable & picture-parameters & picture-menu    \\
C-Sh	& pop-picture       & center         & drag                \\
C-Sh-M	& destroy-viewable  & unbound        & hardcopy             \\
\hline
\end{tabular}

The following are picture-specific mouse bindings for gray pictures:

\begin{tabular}{|l|ccc|}
\hline
buckies & left button       & middle button  & right  button       \\
\hline
M-C	& boost-contrast  & reduce-contrast  & histogram    \\
M-Sh	& x-slice           & y-slice        & crop           \\
\hline
\end{tabular}

The following are picture-specific mouse bindings for flipbooks:

\begin{tabular}{|l|ccc|}
\hline
buckies & left button       & middle button  & right  button       \\
\hline
M-C	& previous-frame  & next-frame  & show-movie    \\
\hline
\end{tabular}


\begin{description}

\item [Raw] (no bucky keys):  
\begin{itemize}
\item Left: {\bf Select viewable}.  Selecting a viewable that is bound
to a symbol prints that symbol into an emacs buffer at current prompt
position (the ``point''), where it is then ready to be evaluated.  
Of course, if the viewables have symbols bound to them, you can simply
type the symbol name yourself.

If the selected viewable has a nil name or a string name, then OBVIUS
prints a more cryptic functional expression like (obvius::pid 1) that
will also evaluate to the viewable.  This allows easy, interactive
manipulation of viewable objects.  Examples are given in the OBVIUS
tutorial.

You can also access unnamed viewables via the variables *, **, and ***
that are bound to the last three values returned by lisp. Also, the
variables @, @@, and @@@ are bound to the last three viewables
selected with a raw left mouse-click.

This mouse-click also selects the pane as the current pane.

\item Center: {\bf Refresh}.  The top picture on the picture stack is
recomputed (if necessary) and then redisplayed.  Calls (refresh pane).

\item Right: {\bf Select pane/Position Message}.  The pane containing
the mouse is set to be the current pane and some information that
depends on the picture type is printed in the Emacs minibuffer.  For
example, clicking right on a grey picture of an image gives the
(x,y)-position and the floating-point intensity value of the {\em
image} (not the picture).  This information is updated dynamically if
you drag the mouse.  Note that many other mouse clicks will also
select the pane.
\end{itemize}

\item [Control]:
\begin{itemize}
\item  Left: {\bf Previous picture}.  The stack of pictures in the
pane is cycled.  The pane is selected.  Calls (cycle-stack pane nil).
\item  Middle: {\bf Next picture}. The stack of pictures in the pane
is cycled in the opposite direction as Control-Left.  This pane is
selected.  Calls (cycle-stack pane t).
\item  Right: {\bf Move to here}.  The picture on top of the current
pane is moved to the top of this pane.  Selects this pane.
\end{itemize}

\item [Shift]:
\begin{itemize}
\item  Left: {\bf Zoom in}.  Pixel-replication is used to produce an
enlarged version of the image.  The zoomed picture can be dragged
using C-Sh-middle.
\item Middle: {\bf Zoom out}.  This reduces the magnification of a
picture.  This will subsample below the resolution of the original
image.
\item  Right: Numerical magnifying glass  [Currently unimplemented].
\end{itemize}

\item [Meta]:
\begin{itemize} 
\item Left: {\bf Describe viewable}.  This prints out a description of the
viewable in the lisp listener. 
\item  Middle: {\bf Edit picture parameters}.  Pops up a dialog to view and edit
the parameters of the picture.
\item  Right: {\bf Picture menu}.  Pops up a menu to select (i.e., cycle to
the top) any picture.  Faster than using next-picture or
previous-picture if the pane has a large stack.
\end{itemize}

\item [Control-shift]:
\begin{itemize}
\item Left: {\bf Pop}.  This mouse click calls (pop-picture pane),
which is described in section \ref{sec:pictures}.  The picture on the
top of the picture stack is destroyed. The viewable is not explicitly
destroyed, although {\em it will no longer be accessible if it did not
have a symbol name slot}.
\item  Middle: {\bf Center}.  Recenters the picture in the pane if it
has been dragged.
\item  Right: {\bf Drag}.  Allows interactive re-positioning of
pictures within the pane.
\end{itemize}

\item [Control-meta-shift]:
\begin{itemize}
\item Left: {\bf Destroy}.  This mouse click calls the method destroy
on the viewable.  All static memory used by the viewable its pictures
is relinquished, the pictures are removed from the picture stacks, and if
the viewable has a symbol name, it is unbound.  If the viewable has
inferiors that are otherwise unaccessable, then the inferiors are also
destroyed.  If the viewable has superiors, then this causes a
continuable error that allows the user to destroy both the viewable
its superior.
\item Right: {\bf Hardcopy}.  Prints the picture to a postscript
printer.  The global parameter *default-printer* specifies which
printer it goes to.
\end{itemize}

\item [Meta-Shift]:  These bindings are {\em picture specific}!

\noindent
For gray pictures:
\begin{itemize}
\item  Left: {\bf X-slice}. This produces a new viewable, a
one-dimensional ``slice'' of a two-dimensional image, and pushes a
one-d-graph picture of it onto the stack of the current pane.  The
y-coordinate of the slice is selected by the position of the
mouse-click in the original picture.
\item  Middle: {\bf Y-slice}. Orthogonal to x-slice.
\item Right: {\bf Interactive Crop}. Hold down the mouse and drag to
define a square subregion of a picture.  Releasing mouse creates a new
viewable, cropped from the original.
\end{itemize}

\item [Meta-Control]:  These bindings are {\em picture specific}!

\noindent
For gray pictures:
\begin{itemize}
\item Left: {\bf Boost-contrast}.
\item Middle: {\bf Reduce-Contrast}.
\item  Right: {\bf Histogram}.  This computes a histogram viewable, with
a bin size which is controlled by the global parameter {\bf
*default-number-of-bins*}, and pushes a one-d-graph picture of it onto
the stack of the current pane.  This calls the function make-histogram
which is described in section \ref{sec:images}.
\end{itemize}

\noindent
For flipbook pictures:
\begin{itemize}
\item Left: {\bf Previous Frame}.  Cycle through flip-book picture of a sequence.
\item Middle: {\bf Next Frame}. Cycle through flip-book picture of a sequence.
\item Right: {\bf Display Sequence}.  Display all frames of a
sequence.  See Section \ref{sec:flipbook} for a description of the
parameters that control delay between frames, etc.  Any mouse button
will stop the display.
\end{itemize}

\end{description}


\subsection{The Control Panel}
\index{control panel}
The control panel displays status information that tells you what the
OBVIUS listener (top-level process) is doing.  It also displays
on-line, passive (i.e., you don't have to do anything to get it),
mouse documentation.  This means that when you move the mouse into an
OBVIUS pane, the relevant mouse bindings are echoed in the message
line; if you press a bucky key (shift, meta, or control), the
documentation is updated accordingly.

To create a control panel, use the command 
\lfun{make-control-panel}{\&key left right top bottom}
You may wish to put this in your {\tt .obvius} initialization file.
To destroy the control panel, call \ldef{destroy-control-panel}.

The control panel also has several menu buttons.  Holding the right
mouse button down on a menu button reveals the pull-down menu.  Some
of the menu choices bring up dialog boxes.  The dialog box is created
by releasing the mouse button on that choice.  Some of the menu
choices are set up to reveal sub-menus.  The sub-menus are revealed by
moving the mouse to the right while continuing to hold down the right
button.

Within a dialog box, click left to move the cursor around and type as
usual.  Many of the dialog boxes are interfaces to function calls.
The functions are executed by clicking either the {\tt Execute} button
or the {\tt OK} button.  When this is done OBVIUS prints an expression
in the *lisp* buffer and evaluates it.  Clicking left on an OBVIUS
pane selects a viewable for the function argument, by inserting it at
the cursor position in the dialog box.

At any given moment, there can only be a single function dialog box.
If you try to bring up another one, OBVIUS will destroy the first one.
You can add functions to the menus by adding them to particular global
variables in the file {\tt x-control-panel.lisp}.  FOr example, you
could add a new function {\tt foo} to the  ``Misc'' menu, but adding
the symbol {\tt 'foo} to the value of the variable
\lsym{obvius::*obvius-misc-functions*}. Changes to these
variables are {\em dynamically} incoporated into the menus.

\index{menus}
The menus on the control panel are:
\begin{description}

\item [Parameters]: This menu allows you to adjust parameters of the
current picture, the screen, the default slots of pictures and
viewables, and global parameters.  The items on the menu are described
in more detail below.

\item [Statistics]: Menu of operations for computing viewable
statistics (e.g., mean or variance of an image or image-sequence).
Each menu choice brings up a dialog box for entering the function
arguments.

\item [Arith Ops]: Menu of statistics operations (e.g., mean,
variance, maximum, minimum, etc.).

\item [Filter Ops]: Menu of filtering operations (e.g., apply-filter,
fft, etc.).

\item [Geom Ops]: Menu of geometric operations (e.g., crop, paste,
slice, flip-x, etc.).

\item [Compare]: Menu of comparison operations (e.g., greater-than,
less-than, mean-square-error, etc.).

\item [Synth]: Menu of synthetic image functions (e.g., make-impulse,
make-sine-grating, make-zone-plate, etc.).

\item [Matrix]: Menu of matrix operations (e.g., matrix-multiply,
singular-value-decomposition, etc.).

\item [Misc]: Menu of miscillaneous operations (e.g., print-values,
hardcopy, etc.), and a submenu of optional modules that may be loaded.
\end{description}

\subsection{Current Picture, Picture Defaults and Viewable Defaults menus}
\begin{description}
\item [Current Picture]: Brings up a dialog box to change parameters
of the current picture (picture on top of the current pane).  Buttons
at the bottom allow user to Apply the changes to the picture (the
display will be updated), revert to the current slot values, revert to
the default slot values, or quit the dialog.

\item [Viewables]: Reveals a sub-menu that allows you to change
the default parameters of a particular viewable class.  The parameters
for each viewable class are described in section \ref{sec:viewables}.
The default parameters are used to initialize a new viewable when it
is first created.  Changing the default parameter values with this
menu has no effect on viewables that have already been created.

\item [Pictures]: Reveals a sub-menu that allows you to change
default parameters of a particular picture class.  The parameters for
each picture class are described in section \ref{sec:viewables}.  The
default parameters are used to initialize a new picture when it is
first created.  Changing the default parameter values with this menu
has no effect on pictures that have already been created.  The
parameters of the current picture may be changed with the {\tt Current
Picture Parameters} menu (mentioned above).
\end{description}


\subsection{Current Screen Parameters}

The user can change the following screen parameters from the {\tt
Parameters} menu on the control panel (see above):
\begin{description}
\item [Gray-shades]: Number of grays to allocate in the color map.  If
OBVIUS fails to allocate the requested number, it tries again with a
smaller number.
\item [Gray-gamma]: Default gamma value in the allocated gray-shades.
\item [Gray-dither]: If gray-shades is less than gray-dither, then
gray pictures will automatically be dithered.
\item [Foreground]: Foreground color used for graphics (e.g., it is
the color used for each new graph pictures).
\item [Background]: Background color of each pane.
\end{description}

\subsection{Global Parameters}

There are a number of global parameters that affect how OBVIUS
operates.  The user can change the values of these globals from the
{\tt Parameters} menu on the control panel (see above).  The function
of each parameter is describe below:

\begin{description}
\item \lsym{*default-printer*}: Name of a PostScript printer.

\item \lsym{*max-print-vals*}: Maximum number of values which the
print-values method prints in a line of the image.  See print-values,
Section \ref{sec:images}.

\item \lsym{*y-print-range*} and \lsym{*x-print-range*}: Default range of
values in an image to be printed by the print-values function.  Should
be a list of two numbers between 0 and 1.  See print-values, Section
\ref{sec:images}.

\item \lsym{*auto-bind-loaded-images*}: If t, automatically binds images
loaded from files to symbols matching their filenames.  See
load-image, Section \ref{sec:datfile}.

\item \lsym{*auto-update-pictures*}: If non-nil, out-of-date pictures will
be updated when they are revealed.  If nil (the default), user must
explicitly refresh the pane to update an out-of-date picture.

\item \lsym{*auto-destroy-orphans*}: Slightly risky parameter causes
viewables to be destroyed when 1) their names are rebound if they have no
pictures or superiors, 2) their only picture is popped off of the pane
and they do not have a symbol name or superiors, 3) their only
superior is destroyed, and they do not have a symbol name or any
pictures.  These three cases are the situations in which OBVIUS can
prevent the creation of orphans.  Note that this does not prevent the
user from creating their own orphans (e.g., make a new image with no
name and don't display it).  See also section~\ref{sec:memory} on
memory management.

\item \lsym{*auto-display-viewables*}: If non-nil, viewables returned to the
top-level listener are automatically displayed on the currently
selected pane.  This is the normal mode of operation.

\item \lsym{*heap-growth-rate*}: Number of elements to expand the heap
when out of memory.  See Section \ref{sec:memory}.

\item \lsym{*auto-expand-heap*}: When non-nil, expand heap automatically
when out of memory.  See Section \ref{sec:memory}.

\item \lsym{*div-by-zero-result*}: Default result used by /-0 and div
functions when the divisor is zero.  See Section \ref{sec:operations}.
\end{description}


\subsection{Running OBVIUS on a remote host}

The networking capabilities of X Windows (or OpenWindows) make it
possible to open OBVIUS windows on a machine other than the one on
which the OBVIUS process resides. Like other X clients, OBVIUS
examines the value of the shell environment variable, DISPLAY, in
order to determine which screen to open its windows on. If your X
server finds more than one screen, you will be able to open obvius
panes on any of them using the {\tt :screen} keyword argument to {\bf
new-pane}. This will typically be the case if you have two monitors
attached to one workstation, or if your framebuffer has a separate
overlay plane.  The list of screens that OBVIUS knows about is kept in
the (unexported) global variable {\tt obvius::*screen-list*}.

Because X is a server-based window system, each bitblt requires that
data be transferred between the OBVIUS (lisp) process, and the X
server process. Displaying a picture for the first time will be
delayed by the time it takes for this transfer, but the remote
(server) copy of the data is retained as an X pixmap, and subsequent
redisplay, via refresh or cycling a pane, is free of this penalty.


\subsection{Screens and Windows}

OBVIUS maintains a list of {\bf screens} on which its panes reside.
This mechanism makes it possible to have OBVIUS panes on several
screens which may have different capabilities at the same time, and
leaves open the possibility of using different window systems on the
various screens. System-independent code which defines windows and
screens is in pane.lisp, while system-dependent window code for X is
in the file {\bf x-window.lisp}.

OBVIUS panes are described in section \ref{sec:pictures}.  These panes
are built on top of LispView window objects and thus inherit all of
the attributes of those windows.  Accordingly, panes can be moved,
resized, refreshed, hidden, closed, or opened in the same manner as
any other X11 window.  NOTE: if you are using X11 (as opposed to
OpenWindows), you should not destroy an OBVIUS pane from the window
manager as this will destroy the client process (i.e., Lisp)!!  Use
the OBVIUS destroy method to destroy a pane (this can be invoked from
the {\tt misc} menu on the control panel).

In LispView, mouse events are handled by a separate Lisp process.
Multiprocessing poses particular difficulties for OBVIUS because some
of the mouse clicks may result in lots of computation, or they may
need to destructively modify data structures that are being
simultaneously accessed by the main process.  OBVIUS has two different
solutions for handling these problems:
\begin{enumerate}
\item If the mouse click involves only modification of pictures,
the computation {\em is} performed by the mouse process.  To avoid
data structure collisions (e.g., the mouse process and main process
both trying to display something in the same pane), OBVIUS ``locks''
the pane being modified, using the macro
\lsym{with-locked-pane} to give exclusive access to one process (see
{\bf pane.lisp} for examples of use).

\item If the mouse click requires substantial computation or
creation of viewables, OBVIUS does not let the mouse process handle
the event directly.  Rather, OBVIUS hands off compute-intensive
evaluation to the main lisp process via an evaluation queue.  The
main lisp process polls the eval queue each time it goes through its
read-evel-print loop.  The upshot of this is that some mouse events
are handled immediately (e.g., cycle stack) whereas others are not
(e.g., make-slice).  If the event is going to be handled in a delayed
fashion, OBVIUS will notify you by putting up the status message
``Enqueued form: <form>''.
\end{enumerate}


\subsection{Emacs and OBVIUS}

The user interface of OBVIUS is greatly enhanced when it is run as a
sub-process of the Emacs editor.  The following capabilities are available:
\begin{itemize}
\item Quick entry of viewable arguments to commands [left
mouse click (select-viewable)].
\item Passive mouse documentation and system messages in the Emacs
minibuffer (the bottom line of the Emacs window) (Note: this is only
true in the absence of the contol panel).
\item Interactive Lisp shell buffer with command history mechanism.
\item Evaluation of lisp expressions from Emacs file buffers [{\tt C-c
e}], and In-package compilation of lisp functions from Emacs file
buffers [{\tt C-c c}].
\item Function argument lists [{\tt C-c C-a}], documentation strings [{\tt
C-c C-d}], symbol apropos [{\tt C-c C-h}], symbol description [{\tt
C-c C-i}], macro expansion [{\tt C-c C-m}], and source file editing
[{\tt C-c .}].
\item Filename completion for load-image requests [{\tt C-c l}].
\item Complete record of interaction with OBVIUS (in the form of an
Emacs buffer).
\end{itemize}

The extensions to Emacs that provide this functionality are contained
in the \abox{emacs-source} files.  For more information, see the file
\abox{emacs-source}/cl-shell.doc.

\subsection{Help}
\label{sec:help}

The following helpful functions are provided:
\begin{description}
\item ({\bf obvius-parameters} \&key doc) \\
Prints out a list of the system
parameters and their current values.  If the optional {\em doc}
argument is non-nil, the documentation of each parameter will also be
printed. Documentation of each parameter is also
available via a call to the Common Lisp documentation function:
(documentation '\abox{symbol} 'variable).
\item ({\bf obvius-commands}) \\
Prints out a list of all OBVIUS functions which are exported.
\item ({\bf obvius-variables}) \\ 
Currently not implemented.
\end{description}
In addition, the source code for any OBVIUS function may be found in
using the ``C-c.'' command.  Also, the Common Lisp function {\tt
(apropos <piece-of-symbol-or-string>)} will search for all symbols
containing the given symbol or substring.

\section{Hacking}
\label{sec:hacking}

This section is provided for users who want to write their own
extensions to OBVIUS.  As OBVIUS becomes larger and more people work
on it, it becomes increasingly important to observe the existing
programming conventions.  It is generally a good idea to use existing
code (copy it and modify it) as a prototype for whatever you are
doing.  In the following sections, we try to describe and motivate
these conventions.

%%%%%%%%%%%%%%%%%%%%%%%%%%%%%%%%%%%%%%%%%%%%%%%%%%%%%%%%%%%%%%%%%%%%%%%%%%
\subsection{Writing Additional Image Processing Functions}
\label{sec:writing-ops}

Since OBVIUS was originally conceived as an image processing package,
it contains a library of image processing functions in the file {\bf
imops.lisp}.  Here is a simple example:
\begin{tt} \begin{verbatim}
(defmethod add ((val number) (im image) &key ->)
  (with-result ((result ->) im 'add val im)
    (add (data im) val :-> (data result))
    result))
\end{verbatim} \end{tt}
Note the result keyword argument, \lsym{:->}.  All functions that
return viewables should take this keyword argument.  The standard
handling of this argument is described in
section~\ref{sec:operations}.  OBVIUS provides a macro, {\tt
with-result} that makes it easy to provide standard handling of this
result argument.

\mysubsubsec{The with-result macro}
The \lsym{with-result} macro should be used by {\em all} operations
returning viewables.  It provides a simple mechanism for creating
functions with uniform result argument handling.  The syntax is as
follows: 
\begin{verbatim}
(with-result ((res-sym res-arg) model . history)
   . body)
\end{verbatim}
The programmer should think of this macro as being very similar to the
Common Lisp {\tt let*} macro.  The forms of the {\tt body} are
executed within a lexical environment where the symbol {\tt res-sym}
is bound to a ``result'' viewable.  This result viewable is protected
from the garbage collector, and from auto-destruction of orphans
(i.e., it is placed on the \lsym{*protected-viewables*} list).  The
macro returns the value (or multiple-values) returned by the {\tt
body}.  The {\tt res-arg}, {\tt model}, and {\tt history} arguments
are used to construct the result viewable.

The {\tt model} argument must either be a plist of keyword initargs
specifying the type and slots of the result, or an existing viewable
which is used as a model for the result.  If a plist is used, the
first two elements should be the keyword \lsym{:class} and the symbol
name of a viewable class.  The remaining keyword/value pairs should be
those you would pass to {\tt make-instance} in order to create the
viewable.  That is, they are a set of initialization arguments
(initargs) for the object.

The {\tt res-arg} is typically the result argument, \res, as passed
by the user.  If this is an existing viewable, OBVIUS will check that
it is consistent with the specification given by the {\tt model} and
then bind {\tt res-sym} to it.  If it is a name (nil, a symbol, or a
string), then OBVIUS will create a new viewable according to the
specification of the {\tt model} and give it this name.

The {\tt history} argument is typically the symbol name of the function
({\tt 'add} in the example above) followed by the arguments to the
function (except for the result argument itself).  This list is saved
in the history slot of the viewable.  If no {\tt history} argument is
passed, then the history of the result viewable will be set according
to the operations performed in the {\tt body}. 

The macro {\tt with-result} calls the methods \lsym{set-result} and
\lsym{set-not-current}, both of which are described below in section
\ref{sec:new-viewable}.  It also calls
\lsym{set-history} to set the history slot of the resulting image
according to the {\tt history} argument.  Examples of usage can be
found in {\bf imops.lisp}.  Calling {\tt macroexpand-1} on these
examples may help to clarify the use of the macro.

\mysubsubsec{Temporary Viewables}
Often, you will need to generate intermediate viewables within the
body of a function.  These viewables are not needed once execution of
the function is completed: they should be destroyed so that their
static arrays are returned to the system for re-allocation.  OBVIUS
provides a macro for this purpose: \\
\lfun{with-local-viewables}{ image-list . body}
This macro has the same syntax as the Common Lisp {\tt let*} macro.
It performs a local binding (using {\tt let*}), executes the body
within an {\tt unwind-protect} and then (silently) destroys all
viewables that were locally bound, thus freeing any heap memory
belonging to those viewables.   The returned value (or multiple values)
is that of the {\tt body}.

It is an error to return a temporary viewable from the body of {\tt
with-local-viewables}.  You may, however, return inferior viewables of
temporary viewables (even if \lsym{*auto-destroy-orphans*} is
non-nil).  For example, you may create a temporary complex image by
computing an fft, and then return the real-part of this complex image.
The complex image will be destroyed, but the real-part will be
preserved.  See the sections on destroying viewables and orphans
(section~\ref{sec:organization}) and the next section
(section~\ref{sec:using-memory}) for a more complete explanation of
\lsym{*auto-destroy-orphans*}.

If you need to create temporary static arrays (i.e., arrays allocated from
OBVIUS's static heap), use the the macro {\tt with-static-arrays},
described in the next section.

\mysubsubsec{Efficient Looping}
In the {\tt add} method given above, we call another method on the
data arrays of the argument image and the result image.  The methods
on arrays are typically designed to call C functions, which run at
least 20-30\% on Sun SPARC machines.  Despite this speed improvement,
it is somewhat inconvenient to have to write C code every time you
want a new image operation.  It is often simpler just to write the
code in LISP.  Several macros, defined in the file {\bf
lucid-image-loops.lisp} are provided to make it easier to write new
image processing functions in LISP:
\begin{description}
\item\lfun{loop-over-image-pixels}{ image-list . body}
Provides efficient looping mechanism for images.  The syntax is like
the Common Lisp {\tt let*} macro.  
See examples in the
files {\bf lucid-image-loops.lisp}, and {\bf imops.lisp}.

\item\lfun{loop-over-image-positions}{ image-list (ypos xpos) . body}
Provides efficient looping mechanism for images, with access to the current
coordinates.  See examples in the
files {\bf lucid-image-loops.lisp}, and {\bf synth.lisp}.
\end{description}

\mysubsubsec{Calling Foreign Functions}
\index{foreign functions}
If you do need to call C code (or some other language) from OBVIUS,
you can look at the foreign function definitions given in {\tt
lucid-ffi.lisp}.  Communication between Lisp and C can be complicated,
and you should  be aware of a few difficulties:
\begin{itemize}
\item If you pass non-static data structures to C code, you must be
careful that the Lisp Garbage Collector is not invoked.  See the
discussion of this in section~\ref{sec:using-memory} on memory
management.
\item If the C code encounters a floating point exception (eg, underflow,
overflow, divide-by-zero, log-of-negative-number... ), then Lisp
\index{foreign functions,floating-point exceptions}
will (by default) generate a continuable error.  This is appropriate
if it only happens once, but if it happens in a loop over a large data
structure, it is disastrous!  The user can turn off this error
handling by evaluating {\tt (setf (enabled-floating-point-traps)
nil)}, but then they will not know when an exception has occured.
Much of the C code in obvius handles floating exceptions more
gracefully (see, for example, \lsym{internal-div}), returning the
number of exceptions encountered within the loop.
\end{itemize}

%%%%%%%%%%%%%%%%%%%%%%%%%%%%%%%%%%%%%%%%%%%%%%%%%%%%%%%%%%%%%%%%%%%%%%%%%%
\subsection{The Memory Manager}
\label{sec:using-memory}

OBVIUS objects that contain large arrays should make use of OBVIUS's
static array allocator, defined in the {\bf memory.lisp} file.  In
Lucid Common Lisp, the ``static'' storage area is a portion of memory
that is not managed by the Lisp garbage-collector.  The reason for
using static arrays (as opposed to normal ``consed'' arrays) is
three-fold:
\begin{enumerate}
\item 
Lucid requires that arrays passed to foreign functions are static, and
begin on even word boundaries.  
\index{foreign functions, passing arrays to}
\index{foreign functions, and garbage collection}
Technically, this is only a problem if
the Lisp garbage collector is invoked while in the foreign function.
In this case, the arrays will be relocated by the garbage collector,
and the foreign function will write into the wrong portions of memory.
This can situation can arise if the foreign function calls a Lisp
function (a ``callback'' ), {\em or
if the foreign function calls} \lsym{malloc}! 
\item We want to prevent large arrays from being garbage collected.
\item Keeping images together in memory improves virtual memory swapping behavior.
\end{enumerate}
The OBVIUS static array allocation function ({\tt allocate-array})
works by displacing sub-arrays into large static vectors.  OBVIUS
keeps track of the space that has been allocated, and provides a
function to de-allocate arrays ({\tt free-array}).  The disadvantage
of this explicit memory allocation scheme is that arrays allocated in
this manner must be freed by the user or the programmer.

The following memory allocator functions are exported:
\begin{description}

\item\lfun{allocate-array}{ dims \&key element-type initial-element}
This function is similar to the \lsym{malloc} function in C.  Returns
an array of the specified dimensions and type, allocated from OBVIUS's
static heap.  The args are the same as for Common Lisp {\tt
make-array}.  The array returned is actually displaced to a large
static array.  Nevertheless, it is guaranteed to begin on an even word
boundary.  Remember that these arrays are not garbage collected and so
must be explicitly freed.  If the memory allocator does not have a
large enough block of the appropriate type of static memory, and if
\lsym{*auto-expand-heap*} is nil, a continuable error is given,
allowing the user to expand the heap by \lsym{*heap-growth-rate*}
elements.  If {\tt *auto-expand-heap*} is non-nil, this expansion will
happen automatically.  NOTE that Lucid is not very well protected
against over-expanding the static memory area and may crash if you
expand beyond the available process memory space.

\item\lfun{free-array}{ arr}
This function is similar to the \lsym{free} function in C.  Frees
storage space for an array that has been allocated using
allocate-array.  After evaluating (free-array A), it will still be
possible to access A, but the contents may be overwritten at any time.
It is also possible to free sub-arrays of an allocated array.  This
function gives a warning if the argument is not an array, or if it was
not allocated from the static heap.

\item\lfun{with-static-arrays}{ arr-list . body}
This macro acts just like the Common Lisp {\tt let*} macro except that
all of the static-arrays bound by it are freed (silently) before
exiting.  If a let* was used, the local variables would become unbound
on exiting, but the static storage space would not be freed.

\item\lfun{expand-heap}{ type-or-heap \&optional size return-block}
Expand the static heap of the given type (can also pass the heap
itself) by the given number of elements.  Any valid {\tt
:element-type} argument for the argument of the Common Lisp make-array
function may be used as a type.  Note that image arrays are of type
{\tt 'single-float}.  Default value for {\tt size} is {\tt
*heap-growth-rate*}.  If {\tt return-block} is nil (the default),
returns t.  Otherwise returns the newly allocated static array.

\item\lfun{heap-status}{}
Print out information about the static array allocator.  For each type
of static heap, prints the number of used elements, the number of free
elements, the size of the largest block, and the number of contiguous
blocks (i.e. fragmentation).

\item\lfun{run-heap-diagnostics}{}
Use this to verify the integrity of the static heap memory allocator.
Prints out all free arrays, giving warnings if inconsistencies are
found.

\item\lfun{ogc}{ \&key verbose}
This is the OBVIUS garbage collector, which can only be called from
top-level.  The \lsym{scrounge!} macro provides a top-level interface
(it just calls ogc with keyword {\tt :verbose} non-nil).  When keyword
{\tt :verbose} is non-nil, it prints out information about the space
reclaimed, the current heap status, and the (preserved) named
viewables.  {\tt ogc} preserves all static arrays that are owned by
pictures on picture stacks, and {\em non-orphaned} viewables (see the
discussion in the next section).  That is, all viewables that are
displayed as pictures, all viewables that are bound to symbols in the
current package or the user package, the viewables on the list
\lsym{*protected-viewables*}, and all of the inferior and superior
viewables of the aformentioned viewables (phew!).  All arrays that are
not preserved are returned to the heap for reallocation.  Therefore,
it is VERY IMPORTANT that the pointers from pictures to viewables, and
from viewables to their superiors and inferiors be maintained.  In
addition, users should be discouraged from creating their own data
structures containing ``inaccessible'' viewables, since these will not
be preserved during a garbage collection.  ASIDE: If Lucid ever
provides a finalize-instance method (called when all references to a
CLOS object have been lost and it is about to be garbage collected) we
will be able to get rid of this stuff and just have the
finalize-instance on viewables free their static arrays!

\end{description}

\mysubsubsec{Orphans and OGC}
\index{orphans}
In order to keep track of which allocated arrays are needed, OBVIUS
contains a fairly large amount of code devoted to maintaining
bi-directional pointers between viewables and pictures, and between
viewables and their superiors and inferiors.  For example, when
viewables are destroyed, their static space is de-allocated, and their
pictures are destroyed.  At any given time, OBVIUS can access
viewables through their pictures (which must reside on panes), through
the Lisp symbol-table (if they have symbol names), and through their
inferiors and superiors.  Any viewable that is not accessible in this
manner is called an ``orphan''.  The OBVIUS garbage collector,
\lsym{ogc}, will reclaim all allocated memory of orphans.

There are three standard situations in which an existing viewable can
become an orphan.  These are described in the section on ``orphans''
(in section~\ref{sec:organization}).  If the parameter
\lsym{*auto-destroy-orphans*} is nil, then the static space belonging
to these orphans will remain allocated.  In this case, the only way to
recover the static space belonging to these orphans is by running the
OBVIUS garbage collector {\tt ogc} (described below).

If the global parameter \lsym{*auto-destroy-orphans*} is non-nil, the
orphans created in these three situations will be {\em automatically
destroyed}.  This is a little bit dangerous for new users: it is easy
to write code that that locally creates orphans that one later expects
to use.  For example, you create a temporary compound-viewable (e.g.,
an image-pair), you bind a local variable to one of its inferiors
(e.g., the x-component).  Later in the function, outside of the scope
of the {\tt with-local-viewables} macro, you perform computations on
the inferior.  But this inferior was destroyed upon exiting the {\tt
with-local-viewables} macro!  You can prevent such destruction of
orphans by 1) setting \lsym{*auto-destroy-orphans*} to nil, or 2)
pushing the orphan onto the list in the special variable
\lsym{*protected-viewables*}.  The \lsym{with-local-viewables} macro
places its return value(s) on the {\tt *protected-viewables*} (see the
definition, in {\bf viewable.lisp}.

It is important to understand, however, that setting
\lsym{*auto-destroy-orphans*} to a non-nil value only ensures that
non-orphan viewables that become orphaned will be destroyed.  {\em It
is still possible to create orphans.} For example, suppose the user
writes a function that returns a list of viewables.  If I call this
function, and I don't hold onto the list returned by the function,
then these viewables will be orphans: OBVIUS will have no way of
accessing them.  But OBVIUS cannot ``know'' about the creation of such
orphans and thus cannot automatically destroy them.  As before, the
only way to recover the static space belonging to these orphans is by
running the OBVIUS garbage collector {\tt ogc} (described below).

Note also that this is a double-edged sword: if I call the function
and bind the resulting list to a global variable, then these viewables
are not accessible by OBVIUS (and are thus considered orphans), but I
(the user) {\em still have access to them} (through the global
variable).  Calling {\tt ogc} will therefore {\em de-allocate the
storage belonging to these viewables}.  The user must therefore be
careful not to create data structures containing ``hidden'' (orphaned)
viewables.  To return multiple viewables from a function, you should
use the Common Lisp multiple-values facility.  See
section~\ref{sec:memory} on memory management for more information.

\mysubsubsec{Troubleshooting Memory Problems}
If you find that viewables are being destroyed or over-written when
you don't want them to be destroyed, there are severa things to
examine.  First, try running your code with the variable
\lsym{*auto-destroy-orphans*} set to nil.  If this fixes the problem,
then you must figure out where in your code you are ``orphan'ing''
viewables (see the previous section).  You should get warnings when
you orphan viewables and the \lsym{*auto-destroy-orphans*} is nil.

You may also be creating new viewables that are orphans upon creation.
In this case the problems would not occur until you run the OBVIUS
garbage collector (\lsym{ogc}).  Look for places in the code that
store viewables in top-level data structures that OBVIUS doesn't know
about (i.e., viewable-matrices are ok, but a list of viewables is
not).

Finally, if the viewables your are losing are {\em not} orphans, it
must be that OBVIUS does not know about the static arrays of those
viewables.  Make sure that the \lsym{static-arrays-of} method returns
alist of all of the appropriate arrays.  These are the arrays that
will be protected from garbage collection.

\index{foreign functions, and garbage collection}
Further memory problems may arise because of foreign function calls.
If you call a foreign function with non-static Lisp data structures,
and the garbage collector is activated while you are in the foreign
code, the Lisp data structures will be moved by the garbage collector.
The foreign function, however, will still have the pointer to the old
data structure and may continue to read or write data from the old
memory locations.
There are several situations in which the garbage collector might be
called while in foreign code:
\begin{enumerate}
\item if the foreign code performs callbacks (that is, calls a Lisp
function)
\item if the ephemeral-garbage-collection (egc) is enabled
\item if the foreign code calls {\tt malloc}
\end{enumerate}
Even worse, this behavior may be different depending on whether your
code is interpreted or compiled, and which which version of the
compiler (i.e., production or development) you used!!  If your
def-foreign function definitions are {\em interpreted}, egc will often
run while you are in the C code!

The programmer should be aware of these situations, since these type
of bugs are often not noticed when testing code on small examples.
Once the code is in place and being used, these problems are very
difficult to track down!  Here are some possible solutions to the
problem:
\begin{enumerate}
\item write lisp interface code that {\em always} passes static
objects to your foreign code.  You can use the OBVIUS
\lsym{allocate-array} function to allocate static arrays.
\item Avoid the three items above that might cause garbage collection in lisp
while executing foreign code.  This means that all temporary space
used in the C code must be {\em allocated by Common Lisp} and passed
in to the foreign code.
\item Wrapping a \lsym{with-interrupts-deferred} macro around calls to the
foreign code can prevent garbage collection due to the second and
third items above.
\end{enumerate}

%%%%%%%%%%%%%%%%%%%%%%%%%%%%%%%%%%%%%%%%%%%%%%%%%%%%%%%%%%%%%%%%%%%%%%%%%%
\subsection{Adding New Viewable Subclasses}
\label{sec:new-viewable}

Use the CLOS function {\tt defclass} or the OBVIUS function
\lsym{def-simple-class} to define new types of viewables.  {\tt
def-simple-class} just calls defclass with args to set up the
accessors automatically.  The following functions/methods must be
provided for each viewable (for an example, look at the file {\bf
image-pair.lisp}).  In some cases, the default methods (defined in the
file {\bf viewable.lisp}) inherited from the generic viewable class
will suffice.  Typically, viewable classes and their methods are
placed in a file called
\abox{viewable-class}.lisp.
\begin{description}

\item {\tt (def-simple-class <viewable-class> ...)}
The class name is usually exported.

\item\lfun{settable-parameters}{ viewable-class-name}
This method, which should dispatch on a symbol eql to the viewable
class name, should return a list of the class slots which are user
adjustable.  It is used by the user interface code to create the
dialog boxes for editing the viewable slots.  Note that it should
append a list of the settable slots for this class with the result of
a call to the inherited method, since the superclasses may also
provide settable slots.  The default method returns {\tt
`(display-type)}.

\item\lfun{initialize-instance}{ vbl \&rest initargs}
This is the standard CLOS initialization function.  Any static arrays
needed by the viewable should be allocated here using the
\lsym{allocate-array} function (described in the previous section).
This method must also fill the superiors-of slots of the inferior
viewables if the viewable is a compound viewable (ie.  contains other
viewables).

\item\lfun{set-result}{ vbl model}
This method is called by the \lsym{with-result} macro (which should be
used by all functions that return viewables) as described above in
section \ref{sec:writing-ops}.  If the programmer is planning to write
routines for performing computations on instances of the new viewable
class, they should provide set-result methods to standardize the
result argument (:\res) behavior. The default methods are defined in
the {\bf viewable.lisp} file. There are four standard method dispatch
cases:
\begin{enumerate}
\item Both arguments are viewables.  In this case, the method checks that 
the viewables are of the same type, and calls \lsym{check-size} to
make sure they are the same size.  The default method is usually
sufficient here.
\item The {\tt vbl} argument is a name (nil/symbol/string), {\tt model} is a viewable.  
This method creates a new viewable, as specified by the {\tt model},
with the given name.  It should copy relevant slots from the model
into the result.  This method may need to be redefined for viewable
subclasses.
\item The {\tt vbl} argument is a viewable, {\tt model} is a property list (plist).  In
this case, the user has passed a viewable as a result argument, and it
must be checked for consistency with the initargs specified by the
plist.  This should be redefined for viewable subclasses to do this
error-checking.
\item The {\tt vbl} argument is a name (nil/symbol/string), and the
{\tt model} is a plist.  A new viewable is created using the plist as
a list of initargs in a call to the CLOS function
\lsym{make-instance}.  The default method defined in {\bf
viewable.lisp} is usually sufficient for this case.
\end{enumerate}

\item\lfun{make-<viewable-class>}{ \&rest initargs}
The top-level constructor function for the viewable.  This function
creates an instance of the viewable and fills in its slots.  This is
redundant -- the user could just as well call the CLOS function
\lsym{make-instance}.  It is provided to make the initargs explicit
(see, for example, the definition of \lsym{make-image}).  It should
make use of the functionality captured by the {\tt
initialize-instance} and \lsym{set-result} methods, by calling
\lsym{with-result}.  This function should
be exported.

\item\lfun{<viewable-class>-p}{ thing}
The predicate for the class.  Typically a macro.  This should be exported.

\item\lfun{print-object}{ vbl stream}
Use this to modify the printed form of this class.

\item\lfun{print-values}{ vbl}
This should print out ``values'' of the viewable.  It need not be
defined for all viewable subclasses.

\item\lfun{check-size}{ vbl \&rest vbl-list}
This method is called by set-result (case 1) and should ensure
that a list of viewables has comparable ``dimensions'' so that point
operations performed on them will work.  The meaning of ``dimensions''
depends on the sub-class of viewable.  The method should return a
viewable if the sizes are consistent, and signal an error otherwise.
See example in image.lisp.  The default method does no checking and
returns the first argument.

\item\lfun{static-arrays-of}{ vbl}
A method that returns a list of the static arrays allocated when the
viewable was created.  The list should {\em not} include static arrays
belonging to the inferiors of the viewable.  These arrays will be
freed when the viewable is destroyed.  They are also the arrays that
are preserved when the OBVIUS garbage collector (\lsym{ogc}) is
called.  The default method returns an empty list (nil).

\item\lfun{set-not-current}{ vbl}
This method is called when the viewable when it is destructively
modified.  It is called by the \lsym{with-result} macro.  This method
does not usually have to be written as the inherited one is usually
sufficient.  The inherited method increments the {\tt current} slot
and calls {\tt set-not-current} on the superiors and pictures of the
viewable (see viewable.lisp).

\item\lfun{inferiors-of}{ vbl}
A method that returns a list of the inferiors of the viewable.  This
must be defined for all compound viewables: it is used by the
\lsym{destroy} and \lsym{ogc} functions. The default (inherited)
method returns nil, and is thus appropriate for atomic (non-compound)
viewables.

\item\lfun{notify-of-inferior-destruction}{ vbl}
A method that is called on a compound viewable when one of its
inferior viewables is being destroyed.  The default (inherited) method
signals a continuable error.  If continued, it destroys both the
superior viewable and the inferior (see viewable.lisp)!  Some
viewables may allow destruction of inferiors, but they should erase
all pointers to those inferiors so as not to confuse the system memory
manager.  See the examples in {\bf sequence.lisp} and {\bf
gaussian-pyramid.lisp}.

\item\lfun{notify-of-superior-destruction}{ inf-vbl sup-vbl}
This method is called on a viewable when one of its superiors is being
destroyed.  It need not be rewritten for viewable subclasses.  The
standard method removes the superior from the \lsym{superiors-of}
list, and if \lsym{*auto-destroy-orphans*} is non-nil and the {\tt
inf-vbl} has been orphaned, it will be destroyed.

The description of the functions and methods above should make it
clear that the objects in OBVIUS are often connected by bi-directional
pointers.  This bi-directionality is necessary for the proper
functioning of the OBVIUS garbage collector (\lsym{ogc}).
These must be maintained every time an object is created or
destroyed.  This is especially difficult in the case of compound
viewables such as complex images and image sequences.  In order to
localize the source of such problems, it is important that the
\lsym{initialize-instance} method set up these pointers properly, and
that viewables should only be destroyed using the standard destroy
function (defined in the file viewable.lisp).  The superiors-of and
pictures-of slots must be carefully maintained.  The superiors-of slot
is used by set-not-current, destroy, and the garbage collector.  The
inferiors-of slot (method) is used by destroy and the garbage
collector. The pictures-of method is used by destroy and the
garbage-collector.
 
\item\lmeth{present}{viewable}{ pic viewable pane}
Described in section~\ref{sec:new-picture}.

\item {\bf Inferior Accessors} \\
In general, compound viewables should provide accessor methods to
access their sub-viewables.  The convention is that these take a :\res
keyword argument.  If no result is passed, they return the
sub-viewable, in the style of a standard accessor function.  If a
result is passed, the sub-viewable is copied into that result. If a
symbol is passed, a new viewable (bound to that symbol) is created and
the sub-viewable is copied into it.  For examples, see the definition
of \lsym{frame} in {\bf sequence.lisp} or \lsym{x-component} in {\bf
image-pair.lisp}.

\item {\bf History} \\
The \lsym{history} slot of a viewable contains the a history object.
This object preserves information about how the viewable was created.
It keeps track of the symbol names of intermediat viewables, so that
the user can print out an abbreviated version of the history.  The
history of a viewable is set with the \lsym{set-history} function:
{\tt (set-history vbl \&rest hist-list)}.  This destructively replaces
viewable elements of hist-list by their histories.  This function is
normally called by the \lsym{with-result} macro.
\end{description}


%%%%%%%%%%%%%%%%%%%%%%%%%%%%%%%%%%%%%%%%%%%%%%%%%%%%%%%%%%%%%%%%%%%%%%%%%%
\subsection{Multi-Processing}
\label{sec:multi-processing}

Starting with version 2.2, OBVIUS takes advantage of Lucid's
multi-processing capabilities.  The main lisp process runs a top-level
``listener'' or ``read-eval-print-loop'' (\lsym{repl}), described
below.  The source code for this is in the file {\bf repl.lisp}.

A secondary process, the \lsym{event-dispatch-process} created by
LispView, handles window system events (mouse clicks, window exposure,
etc).  NOTE: if this process dies, you will no longer be able to use
the mouse.  If you encounter an error and are given a choice of
killing this process, AVOID DOING SO!

\mysubsubsec{Mouse Events}
In order to avoid data-access collisions between processes, OBVIUS
mouse events are handled in one of two ways:
\begin{enumerate}
\item  If the a mouse event
requires processing that will only effect the display (i.e., pictures
and panes), and the processing is relatively simple (e.g., requiring
less then one second), then that event is handled immediately by the
mouse process.  In order to ensure that no other process tries to
access the same windows, all functions that are to be executed in
immediate mode should be called within the context of the macro
\lsym{with-locked-pane}.  This macro sets up a ``lock'' on the a given
pane, such that other processes executing a {\tt
with-locked-pane} macro will be forced to wait before accessing the pane.
\item More extensive event handling is placed on a queue, for
execution by the listener's read-eval-print-loop (described below).
This queue is executed in FIFO (first in first out).  Execution will
be delayed if the listener is not at top-level.
\end{enumerate}
Note that if there is an error, the debugger inhibits multi-processing
and therefore mouse events will not be handled.

\mysubsubsec{Repl}
The OBVIUS read-eval-print-loop has several purposes:
\begin{itemize}
\item Reads forms from *standard-input*, evaluating them and printing
their value(s) to *standard-output*.  If the value(s) begin printed
are viewables and \lsym{*auto-display-viewables*} is non-nil, then
these viewables are automatically displayed on the current pane.
\item Sets the standard Common Lisp variables, {\tt *, **, ***, +, ++,
+++, -, /, //, ///}, as described in Steele.
\item Handles enqueued mouse events (see above).
\item Sends messages to the Status line in the OBVIUS control panel,
indicating when the listener is at top level, computing, or in the
debugger. 
\end{itemize}

\mysubsubsec{Adding Menus and Dialog Boxes}
\index{menus,modifying} \index{dialog boxes}
The menus on the control panel are cynamically computed at
``click-time''.  This makes their reponse time slower, but makes it
very easy to modify them.  

Adding new functions to the function menus is easy.  You need only add
the function name (a symbol) to one of the global variables {\tt
obv::*obvius-<type>-functions*} where {\tt <type> } is one of {\tt
\{stats, arith, geom, filter, compare, synth, matrix, misc\} }.  These
menus are recomputed at ``click-time'' from these variables, so
modifying will result in an immediate change in the menu.  Function
dialog boxes are computed automatically (although this is sometimes a
bit flakey).

The Global parameters dialog is also computed at ``click-time'', and
will contain all exported symbols beginning and ending with an
asterisk (\*) character.

Similarly, if you define a new viewable or picture sub-class, it will
be automatically added (in alphabetical order) to the ``Viewable
defaults'' or ``Picture defaults'' menus.

The ``Modules'' sub-menu (on the ``misc'' menu), is created on the fly
to contain a list of all currently unloaded modules.  If you load a
module, it will no longer appear on this menu.  If you define a new
module (by adding to the obv::*obvius-module-plist*), it will appear
on this menu.

%%%%%%%%%%%%%%%%%%%%%%%%%%%%%%%%%%%%%%%%%%%%%%%%%%%%%%%%%%%%%%%%%%%%%%%%%%
\subsection{Display of Pictures}
\label{sec:auto-display}

%% Figure depicting the calls used to manipulate pane stacks

\begin{figure}
\newlength{\oldunitlength}
\setlength{\oldunitlength}{\unitlength}
\setlength{\unitlength}{0.015in}

\begin{center}
\begin{picture}(400,270)(4,0)

\put(70,260){\oval(80,20)}
\put(70,260){\makebox(0,0){pop-picture}}
\put(76,247){\vector(3,-4){18}}  	%to destroy

\put(330,260){\oval(80,20)}
\put(330,260){\makebox(0,0){display}}
\put(330,243){\circle{14}}
\put(330,243){\makebox(0,0){?}}
\put(320,245){\vector(-3,-1){78}}  	%to move-picture
\put(320,237){\vector(-1,-1){64}}  	%to push-picture
\put(292,148){\line(2,5){34}}	  	%to draw-pane
%\put(296,158){\line(2,5){30}}	  	%to draw-pane
\put(292,148){\vector(-3,-1){75}}  	%to draw-pane

\put(100,210){\makebox(0,0){\framebox(80,20){destroy (picture)}}}
\put(108,197){\vector(1,-1){24}}  	%to remove-picture

\put(200,210){\oval(80,20)}
\put(200,210){\makebox(0,0){move-picture}}
\put(189,197){\vector(-2,-1){48}}  	%to remove-picture
\put(203,197){\vector(1,-1){24}}  	%to push-picture

\put(40,160){\oval(80,20)}
\put(40,160){\makebox(0,0){cycle-pane}}
\put(52,147){\vector(3,-1){106}} 	%to draw-pane

\put(135,160){\makebox(0,0){\framebox(80,20){remove-picture}}}
\put(144,147){\vector(4,-3){32}} 	%to draw-pane

\put(230,160){\makebox(0,0){\framebox(80,20){push-picture}}}
\put(219,147){\vector(-3,-4){18}} 	%to draw-pane

\put(360,160){\oval(80,20)}
\put(360,160){\makebox(0,0){refresh}}
\put(348,147){\vector(-3,-1){106}} 	%to draw-pane

\put(200,110){\makebox(0,0){\framebox(80,20){draw-pane}}}
\put(200,97){\vector(0,-1){24}}

\put(360,60){\oval(80,20)}
\put(360,60){\makebox(0,0){drag-picture}}
\put(348,47){\vector(-3,-1){75}} 	%to render

\put(200,60){\makebox(0,0){\framebox(80,20){present}}}
\put(187,47){\vector(-1,-1){24}} %to compute-picture
\put(213,47){\vector(1,-1){24}} %to render

\put(150,10){\makebox(0,0){\framebox(80,20){compute-picture}}}
\put(250,10){\makebox(0,0){\framebox(80,20){render}}}

\end{picture}
\end{center}

\caption{Chart indicating the function call hierarchy of functions
used to manipulate the picture stacks of OBVIUS panes.  Functions in
oval boxes are top-level functions.  Those in rectangular boxes are
internal.  The question mark enclosed in a circle indicates that the
{\protect\tt display} function calls one of the three indicated functions.}
\label{fig:display-calls}

\setlength{\unitlength}{\oldunitlength}
\end{figure}


The most important convention in OBVIUS is the chain of command for
display of viewables, and manipulation of picture-stacks.  This is
illustrated in figure~\ref{fig:display-calls}.  Typically, the
top-level listener loop (\lfun{repl}, described above) will call the
\lsym{display} function when viewable(s) are returned
to top level.  Other code (e.g. a function called from the mouse
process) can achieve the same behavior by calling the function
\lsym{print-top-level-values}.

Since OBVIUS allows destructive modification of viewables, it often
happens that a picture will become out-of-date with respect to its
viewable.  OBVIUS keeps track of this by maintaining a \lsym{current}
slot for viewables and pictures.  This slot holds a version number.
If the version number of the picture is equal to that of the viewable,
it is up-to-date.  Software that modifies a viewable should call
\lsym{set-not-current} (or use the \lsym{with-result} macro). 
{\tt Set-not-current} increments the \lsym{current} slot of the
viewable.  Below, decribe some of the
functions used to manipulate and display pictures of viewables, and
the use of the {\tt current} slot.
\begin{description}

\item\lfun{display}{ viewable \&optional display-type \&rest initargs \&key pane make-new}
This is the exported method for creating a picture of a viewable.
This assumes {\em only} that the viewable exists.  The default value
for the :pane keyword is the current pane.  If there are no panes, the
function creates one using new-pane.  The default value for the
:display-type keyword is the display-type slot of the viewable.  If
this :display-type keyword is nil, then nothing is displayed.
Otherwise, the value of the :display-type keyword must be a picture
class (or the name of one) suitable for passing to make-instance.
When the :display-type keyword is non-nil, OBVIUS searches through the
picture stacks for a picture of the specified type.  If no picture of
this type exists or if the {\tt :make-new} parameter is non-nil, a new
one is created using {\tt make-instance}.  The {\tt initargs} are
passed along to make-instance, so the user can specify initialization
arguements for the picture.  If a picture of this type exists on one
of the panes, it is moved to the top of the current pane, if the
variable \lsym{*preserve-picture-pane*} is non-nil.  Otherwise it is
moved to the top of its own pane.  The picture will be updated if it
is out of date with the viewable and the variable
\lsym{*auto-update-pictures*} is non-nil.  In either case, the picture
is put onto the pane using either {\tt push-picture} or {\tt
move-picture}.  If it is already displayed in the desired pane, {\tt
display} just calls \lsym{draw-pane}.

\item\lfun{refresh}{ \&optional (pane *current-pane*)}
This provides an entrypoint to the draw-pane method that {\em always}
re-computes the internal representation of the picture, whether or not
it is out-of-date with respect to the viewable.

\item\lfun{move-picture}{ pic \&optional (pane *current-pane*)}
This function performs bookkeeping manipulations of the picture-stack
of the pane and the pane-of the picture (i.e., it is responsible for
keeping consistent pointer references).  If the picture is being moved
to another screen, then its \lsym{system-dependent-frob} may have to
be converted to be consistent with the new type of screen.  It then
calls {\tt draw-pane} to update the screen.

\item\lfun{cycle-pane}{ pane \&optional (num 1)}
This function performs bookkeeping manipulations of the picture-stack
of the pane, rotating the list {\tt num} places (note: num can be
negative).  It then calls {\tt draw-pane} to update the display.

\item\lfun{pop-picture}{ pane}
This function calls \lsym{destroy} on the top (visible) picture of the
pane stack.

\item\lfun{draw-pane}{ pane \&key clear}
This method is the bottleneck for bringing the appearance of the pane
up-to-date with the picture on top of the stack.  In addition to being
called the functions illustrated in figure~\ref{fig:display-calls}, it
is called internally by the window system when it receives ``damage''
or ``exposure'' events.  If the {\tt
:clear} argument is non-nil, {\tt draw-pane} clears the pane and sets
the title-bar to the string {\tt "displaying ..."}.  When the pane is
not empty, {\tt draw-pane} calls the {\tt present} method (described
in the next section) on the picture, viewable and pane, and then sets
the title bar string.  If an error occurs in the call to {\tt
present}, {\tt draw-pane} will pop the bad picture from the
picture-stack when the user returns from the debugger.

\end{description}

%%%%%%%%%%%%%%%%%%%%%%%%%%%%%%%%%%%%%%%%%%%%%%%%%%%%%%%%%%%%%%%%%%%%%%%%%%
\subsection{Adding a New Picture Subclass}
\label{sec:new-picture}

Pictures are typically created by the {\tt display} function
(defined in the file {\bf viewable.lisp}).  Note
that joint methods on viewables and pictures are defined in the
picture file by convention since the viewable files are loaded first.

In creating a new picture type, the programmer should first write the
low level code that fill's the \lsym{system-dependent-frob} slot.
This slot holds a system-dependent representation of the picture that
can be efficiently ``rendered'' to the screen or printer.  For
example, the frob might be an X window system offscreen bitmap, or an
object containing lists of lines that need to be drawn.  This
abstraction also makes porting to other window systems or display
devices easier.

The \lsym{drawable} picture subclass, which provides a parent class
for \lsym{graph}s and \lsym{vector-field}s is a good example.  The
frob for this class will be either an offscreen bitmap object (called
a bltable), or an object containing a description of lines to be drawn
directly in the pane.  The \lsym{retain-bitmap} determines which of
these types of frob is used.  The code in {\bf graph.lisp} draws the
graph into the frob using \lsym{draw-line} and \lsym{draw-text}
commands, which are defined for both type of frob.  Note also that
these are described for \lsym{postscript-bltable} frobs, which are
used when sending output to  a postscript printer.

After writing the low-level code that draws the picture, the
programmer must connect the picture with the rest of OBVIUS.  In order
to do this, the following functions must be provided for each new
picture subclass (for examples, see the files {\bf picture.lisp},
{\bf gray.lisp}, and {\bf drawing.lisp}):
\begin{description}

\item\lfun{defclass}{ \abox{picture-class} ...}
The class name is usually exported.

\item\lfun{settable-parameters}{ picture-class-name}
This method, which should dispatch on a symbol equal to the picture
class name, should return a list of the class slots which are user
adjustable.  It is used by the user interface code to automatically
create dialog boxes for editing pictures.  Note that it should append
a list of the settable slots for this class with the result of a call
to the inherited method, since the superclasses may also provide
settable slots.

\item\lfun{reset-picture-defaults}{ pic vbl \&rest initargs}
This method should compute appropriate values for all of the picture
slots.  It is called by the CLOS method \lsym{shared-initialize} which
is called by both \lsym{initialize-instance} {\em and}
\lsym{reinitialize-instance}.  That is, it is called whenever the
slots of the object are being set.  In order to modify slots of a
picture, the user will typically either use a dialog box, or call
\lsym{setp}.  Both of these call {\tt reinitialize-instance}.  The
{\tt reset-picture-defaults} method should take care of all automatic
computation of parameters (e.g., if the user sets the parameter to
{\tt :auto}), and should also handle parameter dependencies (i.e.,
when the modification of one parameter requires adjustment of
another).  {\em It must return a modifed list of initargs.}  See, for
example, the method for gray pictures.

\item\lfun{set-not-current}{ pic}
Called when the underlying viewable is modified.  The default
(inherited) method puts a double asterisk (ala emacs) in the title bar
if the picture is at the top of a picture stack.  Specialized versions
may need to destroy certain internal data, or reset certain
parameters.  See, for example, the method on gray pictures.

\item\lfun{static-arrays-of}{ pic}
Returns the static arrays used by the picture.  Note that if any
of the arrays are shared with a viewable, these should {\em not} be
returned.  They should be returned only by the static-arrays-of method
on the viewable (for example, see the code for gray pictures of
images, which resets the :scale and :pedestal slots ).  The standard
method just returns the {\tt static-arrays-of} the \lsym{system-dependent-frob}.

\item\lfun{drag-picture}{ pic dy dx}
This method is called while the user is dragging the picture with the
mouse.  It should adjust the {\tt x-offset} and {\tt y-offset} slots
of the picture and redraw the picture in the pane.  The default one
does this by calling draw-pane, which is inefficient in most cases.
Therefore, efficient versions are typically written for subclasses.

\item\lfun{zoom-picture}{ pic factor y x}
Called by the zoom mouse clicks.  {\tt factor} is the amount to zoom
by, and the {\tt y} and {\tt x} parameters indicate the location of
the mouse when it was clicked.  The default version in {\bf
picture.lisp} is often sufficient.

\item\lfun{position-message}{ pic vbl pane x y}
This should present useful information to the user when they click the
mouse on the pane at the given coordinates.  It typically calls
\lsym{pane-coord-to-viewable-coord} to compute the viewable coordinates, and
then \lsym{status-message} to put the message into the OBVIUS status line.
The default version does nothing.

\item\lfun{frob-coord-to-viewable-coord}{ pic frob-y frob-x}
This should convert the coordinates of the frob (these are typically
the same as the pane coordinates, but with the offset and zoom
removed) to those of the viewable.  It is called by the 
\lsym{pane-coord-to-viewable-coord} method.

\item\lfun{title-bar-string}{ pic}
Returns a string to be put in the title bar of the pane displaying the
picture.  This should give to the user any information necessary for
proper interpretation of the visual display.  The default version
returns the name of the viewable.

\item\lmeth{present}{picture}{ picture viewable pane}
This is a triple-method on picture, viewable, and pane sub-classes
which does the actual displaying.  The pane dispatch is meant to allow
specialization for different types of screen.  The typical method
computes the internal system-dependent picture representation by
calling the {\tt compute-picture} method, and then calls {\tt render}
to display the picture in the pane.  See {\bf gray.lisp} for an example.

\item\lfun{compute-picture}{ picture viewable}
This method computes the window-system-dependent intermediate picture
representation (the ``system-dependent-frob'').  It typically calls a
system-dependent picture-specific method {\tt draw-\abox{thing}} to
actually fill the frob.  Note that {\tt compute-picture} is called by
flipbooks and pasteups on the sub-pictures.  See {\bf gray.lisp} and
{\bf x-blt.lisp} for an example of how this is done.

\item\lfun{render}{ pane frob y-offset x-offset zoom}
This method displays the system-dependent intermediate picture
representation (the ``frob'') in the pane, with the given x-offset,
y-offset and zoom.

\end{description}

%%%%%%%%%%%%%%%%%%%%%%%%%%%%%%%%%%%%%%%%%%%%%%%%%%%%%%%%%%%%%%%%%%%%%%%%%%
\subsection{Porting To Other Machines, Window Systems, or Lisp Implementations} 

[***** This section should be elaborated.  Describe the strategy of
the window-system interface layering.  In particular, the generic
screen, pane, and frob stuff.
*****]

Porting to another window system would require replacing the source
files beginning with an ``x-'' prefix, and would probably require a
moderate amount of work.  We have deliberately kept our use of
LispView simple, and have tried to provide a layer of generic
interface to the window system.  All references to LispView code are
package-prefixed with the lispview: prefix.  The generic OBVIUS window
system interface is accessed in the OBVIUS source files (e.g. {\bf
pane.lisp} and {\bf drawing.lisp}).  It relies on four
system-dependent types of object: screen, pane, drawable, bltable.
The screen and pane objects should inherit from the system-independent
versions defined in pane.lisp (see, for an example, the definitions of
x-screen and x-pane in x-window.lisp).  The function make-pane should
be provided.  The drawable and bltable objects do not inherit from any
system-independent objects, but make-bltable and make-drawable
functions must be provided.

As for porting to another Lisp implementation, there are three
Lucid-specific facilities which are heavily used in OBVIUS: 1) the
foreign function interface, and 2) the multi-processing capability.
We also make extensive use of Lucid's CLOS implementation, and 3) the
static array allocation facility.  Calls to lucid-specific functions
are package-prefixed with {\tt LCL:} (general lucid extensions), {\tt
CLOS:} (Lucid's implementation of CLOS), or {\tt MP:} (Lucid's
multi-processing facility).  Changing the syntax of the foreign
function calls (assuming the alternate Lisp Implementation provides a
foreign function interface) would probably be relatively simple.  The
multi-processing is actually only used indirectly through the LispView
interface to X.  Only the code in {\bf repl.lisp} relies on the
multi-processing capabilities of Lucid (which are modularized in the
MP package).  Porting LispView to run in another Lisp would probably
be difficult.  It might be easier to rewrite the window system
interface from scratch (see above).

Porting to other machines (assuming the X window system and Lucid
Common Lisp are both available) necessitates either porting LispView
to the new machine, or providing your own X windows interface.  Rumor
has it that Unipress is planning to port LispView to other hardware
platforms, but it is not certain when this will happen.


%\include{test}

\appendix

%%% Keep this file consistent with the INSTALL file in the top directory.

\section{Installation}	
\label{sec:install}

This section gives step-by-step instructions for installing OBVIUS on:

\begin{itemize}
\item{{\bf Sun workstations}}

With Lucid (Sun) Common Lisp 4.1, with CLOS and LispView.  Currently,
OBVIUS runs comfortably on Suns with 8 bit frame buffers (color or
gray-scale monitors), or monochrome (1bit) frame buffers.

\item{{\bf SGI workstations}}

With Lucid (SGI) Common Lisp 4.1 (beta) with CLOS.  In order to run
OBVIUS, you will need at least this version of Common Lisp from Lucid;
older versions like version 3.0.2+ will not work.  The current version
of OBVIUS for this platform was developed on both the Indigo and the
Personal Iris.  It supports both 8 and 24 bit frame buffers.

\end{itemize}


\subsection{Requirements}

You need a fairly large swap space (virtual memory, paging area) in
order to use OBVIUS to its fullest potential (we generally allot 50
Megabytes or more).  20 Megabytes is a bare minimum to get OBVIUS to
load; this amount of swap space will not allow you to do very much in
the way of image processing as the image storage space restrictions
will be overly stringent.

If you are accustomed to working with Lisp, these requirements will
not be very remarkable.  If they do seem somewhat voracious, remember
that if you do image processing with individual binary programs and
shell scripts, as many people do, you will end up storing all of your
results as files on the disk and may soon frequently require easily
this much storage space for your intermediate, temporary results.

The OBVIUS directory, which contains Lisp and C source and binary
files and documentation, occupies around 4Mb of disk space.  The
complete lisp image (including Lucid 4.0, CLOS, OBVIUS, possibly
LispView) will occupy at least 15Mb, so you should have that much disk
space to devote to OBVIUS.

In order to make use of the user interface features, you should also
have GNU emacs. You can obtain GNU emacs via anonymous ftp from
prep.ai.mit.edu.

OBVIUS 3.0 runs under the X window system.  X was developed at MIT and
there are public domain servers available for a number of different
architectures.  You can ftp a copy of X from expo.lcs.mit.edu.  Or,
you can get Openwindows (available from Sun), that includes both the X
and News window systems.

You also need LispView, an object-oriented lisp-interface to X, if you
are running OBVIUS on a Sun.  LispView is available from Sun
Microsystems with Sun (Lucid) Common Lisp, version 4.0.  LispView is
not required for the SGI platform as OBVIUS uses the native SGI
graphics library (GL).  The lisp-interface to the GL is included in
the OBVIUS distribution.


\subsection{Installation Instructions}

The symbol \abox{obv} will be used throughout this document to
indicate the pathname of the primary OBVIUS directory.

\begin{enumerate}

\item 
Decide where the OBVIUS source directory tree (approx. 3-5 Mbytes)
will live in your filesystem.  We will refer to this directory as
\abox{obv}.  On our machines, this is /usr/local/obvius. Also decide
where the OBVIUS executable binary file (the lisp image, approximately
15 Mbytes) will live. We'll refer to this directory as
\abox{bin-path}.  On our system, this is /usr/local/bin.  This
directory must be in each OBVIUS user's PATH environment variable.

\item  
Extract the OBVIUS directory tree from the tape or from the
compressed tarfile (you have probably already done this!):
\begin{itemize}
\item	{\tt mkdir \abox{obv}}
\item	{\tt cd \abox{obv}/..}
\item	If you are using a tape, insert the tape and wait for the
whirring to stop.  If you are starting from a compressed tar file
(obvius.tar.Z), put the tarfile in the current directory and
uncompress it using {\tt uncompress obvius.tar.Z}.
\item	For a tape, execute {\tt tar xvf /dev/nrst8}.  From a
tarfile, execute {\tt tar xvf obvius.tar}.
\end{itemize}
The distribution contains several source subdirectories (lisp-source,
c-source and emacs-source), a documentation subdirectory (doc), and a
set of top-level files (README, INSTALL, lucid-defsys.lisp, etc).  You
may move the individual source subdirectories anywhere in your file
system.  Also, decide where you want the Lisp and C binaries to live
(see next instruction).

\item	
In the directory \abox{obv}/c-source, there are two makefiles:
Makefile-sun4 (for sun 4's) and Makefile-sgi (for SGI's).  Edit these
makefiles to refer to the appropriate binary pathnames.  The default
is for the binaries to go in subdirectories sun4-bin and sgi-bin of
the \abox{obv} directory.  Also edit the {\tt INCLUDES} line in these
makefiles to refer to the appropriate pathnames for the Xlib.h and X.h
include files (this is required for the Sun platform only).  Create
the necessary binary directories using mkdir.

\item
In file \abox{obv}/site-paths.lisp, edit the definitions for
*obvius-directory-path*, *lisp-source-path*, *c-source-path*,
*binary-path*.  Usually these will be \abox{obv}/,
\abox{obv}/lisp-source/, \abox{obv}/c-source/, and the appropriate
binary subdirectory for the given machine (must be consistent with the
directory chosen in the previous item).  Also, edit the defvar for
*obvius-image-path*.

\item  
In file \abox{obv}/lucid-site-paths.lisp, edit the defvars for
*temp-ps-directory* (usually /tmp/), *default-printer* (should be the
name of a postscript printer), and *print-command-string* (this will
probably need to be changed depending on your printing configuration).

\item	
Edit the file \abox{obv}/lucid-site-init.lisp.  This file should load
any optional modules that you want to be part of the basic OBVIUS lisp
image at your site.  The list of modules appears in the
lucid-defsys.lisp file.

\item 
Start up a lisp world that includes CLOS.  Use ``lisp -n'' to start
lisp without loading your personal lisp-init.lisp file.

\item
If you are running on a Sun platform, load LispView by typing: {\tt
(load "<lispview>/lispview.sbin")}.  You can get this compiled
lispview file (lispview.sbin) via anonymous ftp from
white.stanford.edu.

If you are compiling LispView from scratch, you must add the following
lines to the file "build.lisp":
\begin{verbatim}
   ...
   (in-package "USER")

   ;; add these lines to get lispview to complie correctly
   (proclaim '(optimize (compilation-speed 0) (speed 3) (safety 1)))
   (proclaim '(notinline make-instance))
   ...
\end{verbatim}

\item 
To compile and load OBVIUS, type {\tt (load
"\abox{obv}/lucid-defsys")} and then type {\tt (obvius::run-obvius)}.
This takes a while.  Note that if you do not have a lot of swap space,
lisp may die during the compilation.  Just quit the lisp and start
again.  It will load the files that have already been compiled and
then compile the rest of them.

\item To save an OBVIUS lisp image:
\begin{itemize}
\item 
You should have already compiled all of the OBVIUS source (see
previous instruction), and tested to see that it works (take a look at
<obv>/obvius-tutorial.lisp and try evaluating some of the expressions).

\item Start up a lisp world that includes CLOS.

\item Load LispView if you are running on a Sun platform by typing
{\tt (load "\abox{lispview}/lispview.sbin")}.

\item 
Type {\tt (load "\abox{obv}/lucid-defsys")} and then
type {\tt (obvius::make-obvius)}.  WARNING: We have found that
attempting a ``disksave'' over an NFS connection sometimes fails.  If
your OBVIUS image does not work correctly, then try making it from
your file server.
\end{itemize}

\item
OBVIUS will run without Gnu Emacs, but there are many advantages to
running with it.  If you will be using emacs, you will need to install
the emacs extensions which are included in the tarfile in the
emacs-source subdirectory.  Copy the directory of emacs-extensions
whereever you like in your filesystem.  If you prefer, you can put
them in with your standard distribution.  We will refer to this
directory as \abox{emacs-extensions}.  On our machines, this is {\tt
"<obv>/emacs-source"}.  Add the following lines to your .emacs file:
\begin{verbatim}
(setq load-path (cons \abox{emacs-extensions} load-path)) \\
(setq *obvius-program* \abox{obvius-image}) \\
(autoload 'run-obvius  "cl-obvius" "" t) \\
\end{verbatim}
The symbol {\tt \abox{obvius-image}} should be the pathname for the lisp
image containing OBVIUS.  To start OBVIUS from a saved OBVIUS lisp
image, run emacs and type {\tt M-x run-obvius}.

If you have not yet made an OBVIUS lisp image, or want to run from the
compiled binary files, you may put the following definition in your
.emacs file:
\begin{verbatim}
(defun run-development-obvius ()
  (interactive)
  (load "cl-obvius")
  (let ((*cl-replacement-prompt*  "OBVIUS-3.0> "))
    (run-cl "<lisp-view>"))
  (setq default-directory "<obv>/lisp-source/") ;VISCI
  (cl-send-string "(load \"<obv>/lucid-defsys\")\n") ;VISCI
  (cl-send-string "(obvius::run-obvius)\n")
  (cl-add-obvius-key-bindings))	;add bindings to cl-shell-mode and lisp-mode
\end{verbatim}
where {\tt <lisp-view>} refers to the pathname of a lisp world
containing LispView and CLOS.  To run OBVIUS, you may type {\tt M-x
run-development-obvius} in Emacs.

\item 
If you want to use the same emacs environment for running Common Lisp,
you may want to put the following lines in your .emacs file as well:
\begin{verbatim}
(setq *cl-program* \abox{lisp-image}) \\
(autoload 'run-cl "cl-shell" "" t) \\
\end{verbatim}
where \abox{lisp-image} refers to the pathname for a lisp-image.  On
our machines, this is in {\tt /usr/local/bin/Lucid-4.0}.  To start up
Common Lisp inside emacs, type {\tt M-x run-cl}.  Documentation for
the cl-shell environment is in the file
\abox{emacs-extensions}/cl-shell.doc.  Further extensions to emacs are
included with the source, and examples of how to use these may be
found in the file {\tt <obv>/emacs-source/example.emacs}.

\end{enumerate}


\section{Release Notes}
\label{sec:release-notes}

\subsection{Version 2.1 Release Notes}

There have been several major changes since releases 1.0 and 1.2 of
OBVIUS.  We have tried to keep most of the changes internal (i.e.
non-exported), but some of them will affect the top-level user.  The
following are the major changes since release 1.2:

\subsubsection{General Changes}

\begin{itemize}

\item OBVIUS now runs under the X11 or OpenWindows window systems  exclusively.
We no longer support SunView or the Symbolics window systems.

\item OBVIUS now runs under Lucid 4.0.  This version of Lucid provides
an implementation of CLOS (we no longer use PCL), and supports
multi-processing.

\item OBVIUS now uses a standard interface to the window system called
``LispView''.  LispView was written at Sun, and is distributed with
Sun (Lucid) Common Lisp.  We have deliberately kept our interface to
LispView as simple as possible so that it should be easy to port
OBVIUS to other GUI platforms.  Note that LispView has a few minor
bugs when running under the twm window manager.

\end{itemize}

\subsubsection{Specific Notes for OBVIUS Users}

\begin{itemize}

\item The following global parameters have gone away: 
{\tt *SEQ-DISPLAY-TYPE*}, {\tt *DEFAULT-DF-SIZE*}, {\tt
*DEFAULT-NUMBER-OF-BINS*}, {\tt *AUTO-SCALE-IMAGES*}, {\tt
*GAMMA-CORRECT*}, {\tt *GRAY-SCALE*}, {\tt *GRAY-PEDESTAL*}, {\tt
*AUTO-SCALE-COMPOUND-IMAGES*}, {\tt *RETAIN-GRAPHS*}, {\tt
*INDEPENDENT-PARAMETERS*}, {\tt *DEFAULT-PASTEUP-FORMAT*}, {\tt
*AUTO-SCALE-VECTOR-FIELDS*}, {\tt *VECTOR-FIELD-DENSITY*}, {\tt
*VECTOR-FIELD-MAGNIFY*}, {\tt *VECTOR-FIELD-LENGTHEN*}, {\tt
*VECTOR-FIELD-SKIP*}, {\tt *VECTOR-FIELD-ARROW-HEADS*}.  In their
place, we provide a mechanism for the user to adjust the default
values of slots for CLOS classes. The function \lsym{set-default}
allows the user to set the default values of the slots of any class.
The arguments look like: {\tt (set-default <class-name>
<slot-name-or-initarg> <default-form>). } The function
\lsym{get-defaults} is provided to get a list of the current default
values for a class.  For example, to auto-scale gray pictures by
default, you would execute {\tt (set-default 'gray 'scale t)}.  To
auto-scale an existing instance of a gray picture, you would execute 
{\tt (setp scale t)}.

\item The user initialization file may now be called either
\lsym{obvius-init.lisp} or \lsym{.obvius}.  This allows you to use a ``.'' file if
your prefer, and allows old users to keep around their former startup
file (previous version of obvius did not use .obvius).  Also, the user
can now have an \lsym{obvius-window-init.lisp} or
\lsym{.obvius-windows} file.  This one will be loaded before the
window system is initialized and thus allows modification of the
behavior of window system resource allocation etc.  Default versions
of both of these files are included in the lisp-source directory.

\item All pictures now have a \lsym{:zoom} slot which you can setp.
Zoom factors are no longer limited to powers of two.  A zoom value of
one corresponds to the ``natural'' size of the picture.  If the zoom
value is set to :auto, the picture is automagically zoomed to fill the
pane.  If it is a list of length two, the picture is zoomed to these
dimensions.

\item The \lsym{:scale} in gray pictures now has a different meaning.  The
values in the image are {\em divided} by the scale (after subtracting off
the pedestal).  The resulting values should be in the range [0,1],
where 1 will correspond to the brightest intensity of the display, and
0 the darkest.  Values outside this range are clipped.

\item New picture type \lsym{dither} allows display of dithered images
on screens deeper than 1 bit.  It inherits from gray.

\item \lsym{one-d-graph} picture type is now called \lsym{graph}.

\item New picture type \lsym{drawing}.  This is now the parent class
for {\tt graph}s, {\tt vector-field}s, {\tt surface-plot}s, and {\tt
contour-plot}s.

\item New slots have been added to screens.  The \lsym{:gray-gamma} parameter
determines the gamma correction exponent for the gray colormap.  The
\lsym{:gray-shades} parameter corresponds to the number of gray colors in the
colormap.  The \lsym{:gray-dither} parameter determines whether OBVIUS dithers
float array values into gray pictures.  If this holds a number, OBVIUS
will dither if there are less gray shades than this number.  If t,
OBVIUS always dithers.  The \lsym{:foreground} should hold either a
keyword corresponding to one of the colors in {\tt
/usr/lib/X11/rgb.txt}, or a list of three numbers in the interval
[0,1] corresponding to red, green, blue components.  This color is
used as a default for graphs, vector-fields, etc.  The
\lsym{:background} slot is similar.  It determines the color used to
clear the panes.

\item Edge-handler arguments to \lsym{make-filter} have changed.  They are now
keywords instead of strings.  The current list is: {\tt nil (wrap),
:dont-compute, :zero, :repeat, :reflect1, :reflect2, :extend }.

\item The "point-operation" \index{point operations} 
macros ({\tt +. *. /.} etc) are now act like the corresponding Common
Lisp functions (documented in Steele, edition I).  They take any
number of arguments, along with a result arg.  New ones have been
defined.  Eventually, ALL of the numerical functions in Steele should
be created.  Partial list of new operations: {\tt fill!}, {\tt
make-ramp}, {\tt make-1-over-r}, {\tt make-r-squared}, {\tt expt.},
{\tt exp.}, {\tt log.}, {\tt mod.}, {\tt sin.}, {\tt cos.}, {\tt
realpart.}, {\tt imagpart.}, {\tt phase.}, {\tt incf},. {\tt decf.},
{\tt zero-crossings}, {\tt rect-to-polar} {\tt polar-to-rect}, {\tt
transpose}, {\tt max-abs-error}, {\tt minimum-location}, {\tt
point-minimum}, {\tt maximum-location}, {\tt point-maximum}, {\tt
skew}, {\tt kurtosis}.

\item The {\tt *auto-scrounge!} parameter has been eliminated.  A new
parameter, \lsym{*auto-destroy-orphans*}, has been added to allow
OBVIUS to destroy orphaned viewables.  This is described in
section~\ref{sec:organization}.

\item The macro \lsym{with-local-images} has been replaced by
\lsym{with-local-viewables}.  The behavior has been slightly modified:
1) the new macro is capable of returning multiple-values, 2) the macro
signals an error if you try to return one of the local viewables from
the body, 3) you may return inferiors of compound viewables -- these
will not be destroyed (even if the global symbol
\lsym{*auto-destroy-orphans*} is non-nil).

\item Emacs interface has been rewritten as a much cleaner package (in use
since 1/90).  Support for arglists, macroexpansion, describe, etc are
provided.  See the file emacs-source/cl-shell.doc for more
information.

\item Arguments to display have changed.  The new function is much
more powerful and flexible.  The syntax is as follows:
\begin{verbatim}
(display viewable &optional (display-type (display-type vbl))
                  &rest keys
                  &key 
		  (pane (or *current-pane* (new-pane)))
		  make-new
                  &allow-other-keys)
\end{verbatim}
The differences are: 1) \lsym{:display-type} is now an optional
parameter (it used to be a keyword).  2) The \lsym{:make-new}
parameter allows you to force OBVIUS to create a new picture of the
given type (otherwise, OBVIUS may just bring an existing picture of
the given type up to the top of the current pane).  3) You can pass
keywords corresponding to slot names in the picture (as in setp) to
set them up.  Example: {\tt (display <image> 'gray :scale 200
:pedestal 37)}

\item Arguments to \lsym{hardcopy} have changed.  They are now similar
to those for display:
\begin{verbatim}
(hardcopy viewable &optional (display-type (display-type vbl))
                   &rest keys
                   &key
		   path printer y-location x-location y-scale x-scale
		   invert-flag suppress-text make-new
                   &allow-other-keys)
\end{verbatim}
Also, hardcopy now does real postscript output (i.e. line drawing
commands) for drawing pictures (e.g. graphs, vector-fields,
surface-plots), so these will be drawn at the resolution of the
printer!

\item Adopted a convention for matrix multiply operations that vectors
correspond to row-vectors.  We used to automagically interpret vectors
as either row or column depending on the operation.  Now, you get a
continuable error if the method is expecting a column and you give it
a row (or vice versa).  To work with column vectors, make Nx1 arrays.
Or, use the functions vectorize and columnize to convert (via
displaced arrays) between row and columns.

\item Arguments to \lsym{make-sin-grating} have changed (see
description in Section \ref{sec:operations}).

\item Arguments to \lsym{make-histogram} and \lsym{entropy} have been
changed (see description in Section \ref{sec:operations}).

\item Keyword arguments to {\tt make-<vbl>} functions can now be any initargs that
you could pass to a call to make-instance.  They no longer take a
:\res keyword -- you should use the \lsym{:name} keyword to specify a
name.  The purpose of the functions is to 1) emphasize that some args
are required, 2) provide result arg behavior (including history, etc).

\item You can now create sequences of any type of viewable (not just images)
and can display flipbooks of most types of pictures.

\item Big speedup in most synthetic image generation code. (eg.
{\tt make-uniform-random-image, make-sin-grating}, etc).

\item Optional loadable modules: look at \lsym{*obvius-module-plist*}.
Current modules are {\tt :gaussian-pyramid :qmf-pyramid :overlay
:x-magnifier :matrix}.  Load these by executing {\tt
(obv-require <module-name>)}.

\item The \lsym{overlay} module allows you to display two graphs on one plot,
or overlay a one-bit image on top of a gray image.  See
section~\ref{sec:overlays}.

\item Mouse clicks are now received by a secondary process.  Some of
them are handled immediately (i.e., even when the top-level listener
is computing something).  Examples are the simple manipulations of
panes: cycle-pane, move-to-here, etc.  Other operations are put onto a
queue for evaluation by the top-level (listener) process.  See
section~\ref{sec:multi-processing} for more details.

\item Mouse bindings have changed slightly.  The right mouse button
now displays position messages dynamically as you drag the mouse.
Bindings with the M-Sh and M-C buckys down are now picture-specfic and
will change depending on the type of picture in the pane.  See
section~\ref{sec:mouse} for a listing of the bindings.

\item We now provide a more interactive user interface: a control
panel with menus.  To use it, execute {\tt (obv-require
:x-control-panel)} and then {\tt (make-control-panel)}.  Status
messages will appear in the control-panel if it exists, in the Emacs
mini-buffer (if running in cl-shell-mode) or not at all.  Dialog boxes
provide pop-up help: click M-left on an item for pop-up help
information (NOTE: due to a bug in LispView, this feature doesn't work
sometimes.  To ``reset'' it, click in a blank region of the dialog,
and then click M-left on the desired item immediately afterward).

\item New viewable, \lsym{polar-image} stores the magnitude and
complex-phase of a complex-image.  The functions magnitude and
complex-phase do the right thing on polar images.  The arithmetic
functions (e.g., {\tt add}, {\tt sub}, {\tt mul}, and {\tt div}) are
not yet defined correctly for {\tt polar-images}.  Convert them to
complex images first (using {\tt polar-to-complex}), and then do the
arithmetic.  \lsym{polar-to-rect} renamed \lsym{polar-to-complex}, and converts a
polar-image to complex-image.  Likewise, \lsym{rect-to-polar} renamed
\lsym{complex-to-polar}.

\item Defined nearly all image operations on arrays.  Redefined image
methods to call these array methods.

\item Defined nearly all image operations on discrete-functions.
Got rid of peak.  Now it is same as image operations, maximum and
maximum-location.

\item Changed args to \lsym{circular-shift}, \lsym{paste}, and
\lsym{crop} to be keywords.
Defined circular-shift, paste, and crop for vectors and arrays.  For
vectors, ignores the ``y'' keywords.

\item Added ``x'' ``y'' ``x-dim'' and ``y-dim'' keyword arguments to 
\lsym{correlate} for correlating over a restricted range of offsets.

\item Defined complex-discrete-functions.  All the operations for
complex-images are implemented for complex-discrete-functions,
including fft, magnitude, etc.

\item One-d-images can now be made by specifying only the 1 relevant
size, e.g., {\tt (make-sin-grating 64) } 
One-d-images are now stored as vectors (not as 2D arrays).

\end{itemize}

\subsubsection{Notes for OBVIUS Hackers (i.e. non-exported changes)}

\begin{itemize}

\item OBVIUS now uses LispView, an object-oriented Lisp-to-X interface
written by Sun.  Most of the old X/SunView cruftiness has been thrown
out.  The new interface is slower, but it is also more robust and
requires much less maintenance!

\item Lucid-4.0 contains a nearly-up-to-spec version of CLOS.  In
particular, defclass and initialize-instance now adhere to the spec.
We have defined a
simple macro called {\tt def-simple-class} which automatically creates
accessors, an initarg and an initform (nil) for each slot.
We have also moved most of the functionality of the {\tt make-<vbl>} functions
into initialize-instance methods, where it belongs!

\item To avoid multi-processing collisions, any function which might
be called directly from the mouse must be careful to modify the
contents of the pane only within a call to the macro
\lsym{with-locked-pane}.  Also, note that because of the
multi-processing in Lucid4.0, you can no longer compile when in the
debugger.  Alternatively, one can enqueue forms for execution by the
top-level loop using the function {\tt push-onto-eval-queue}.

\item The memory manager has been rewritten to be noticably faster,
more general, and more robust.  A diagnostic tool is provided to check
the integrity of the heap data structures:
\lsym{obvius::run-heap-diagnostics}. Eventually, we would like to
replace this with a Lucid-supported static memory allocator.  Two
global parameters are provided to adjust the behavior of the memory
manager. If \lsym{*auto-expand-heap*} is non-nil, the static heap is
automatically expanded when OBVIUS needs more memory.  The number of
elements by which to expand is determined by the variable \lsym{*heap-growth-rate*}.

\item The \lsym{with-result} macro replaces all 
\lsym{with-result-<vbl>} macros.  It returns the value of the last
expression in the body (including multiple values).  It has been
rewritten to use a model specification similar to what you would pass
to make-instance.  If you pass a model-plist, the first keyword must
be {\tt :class} followed by a CLOS class or the symbolic name of one.
Also, if you don't pass a history list, it will not modify the history
of the result.  This allows the programmer to let the history be that
resulting from the sub-calls made within a function.  See
sections~\ref{sec:writing-ops} and~\ref{sec:new-viewable}.

\item Auto-display is no longer done from the print-object method.  It
is now done directly by the top level loop (\lsym{repl}).  In a
multi-processing environment, it was too messy to determine whether
the printing was being done at ``top-level''.

\item The order of arguments in the methods \lsym{present} and
\lsym{set-result} has changed.

\item The keyword arguments used for the \lsym{set-result} method on
\lsym{viewable-sequence} are a bit different.  You can now pass a
\lsym{:sub-viewable-spec} which will be used to create the sub-viewables.
You must provide a {\tt :length} parameter with this.  

\item The \lsym{gray-lut} (gray lookup-table) in a colormap is now a vector of
length no more than 256 which indexes into the successive gray entries
in the colormap.  These gray values may be gamma-corrected.

\item Some variable names have changed:

\begin{tabular}{ccc}
status-line-message & {\tt -->} & status-message \\
draw-gray           & {\tt -->} & draw-float-array \\
draw-bitmap	    & {\tt -->} & draw-bit-array \\
new-array           & {\tt -->} & allocate-array \\
memory-usage        & {\tt -->} & heap-status \\
*growth-rate* 	    & {\tt -->} & *heap-growth-rate* \\
scrounge!           & {\tt -->} & ogc \\
\end{tabular}

In addition, the function \lsym{default-display-type} is no longer used.

\item Patches are loaded from the patches sub-directory.  Files in
this directory are expected to have filenames like patch-123.lisp.
They are loaded at startup time in numerical order, starting with the
filenumber specified by the parameter \lsym{*starting-patch-file*}.
See the file {\tt lisp-source/patches.lisp} for the code that does the
loading.  We will periodically put new patch files up for anonymous
ftp.  Users will typically have the choice of ftp'ing a whole new
source tree, with the patches installed, or just the patch  files.
The new source tree will have an appropriate {\tt
*starting-patch-file*}, so as not to load unnecessary patches.

\end{itemize}

\subsection{Version 2.2beta Release Notes}

Below is a listing of major changes/additions.  See the Changelog for
detailed changes.
\begin{itemize}
\item Color-images and color-pictures added.

\item Overlays totally rewritten.

\item Viewable-matrices added.  Sequences now inherit from
viewable-matrices. 

\item A huge number of matrix utilites added, and new versions of svd,
matrix-inverse, regrees, etc.

\item Stepit added.

\item List-ops added.

\item Statistics and numerical-recipes added.

\item Plot symbols added to graphs.

\item Polar-plots added.

\item Scatter-plots added.

\item For flipbooks and overlays, picture parameters dialog now allows
you to change parameters of the sub-pictures.

\item Generic-ops added.  When adding new ops, write the methods for
arrays and for generic viewables first.

\item eliminated gamma-correct from gray (and pasteup) pictures.
Wrote new gamma-correct method.

\item args to compile-load changed: Now  takes an :output-file keyword
like the Common Lisp compile-file command.

\end{itemize}


\subsection{Version 2.2 Release Notes}

Below is a listing of major changes/additions.  See the Changelog for
detailed changes.

\begin{itemize}
\item Additional keyword args :optimizations and
:brief-optimize-message have been added to compile-load.
:optimizations is a list that you would pass to (proclaim (optimize
...)).  :brief-optimize-message overrides the Lucid :optimize-message
arg and prints a one line synopsis of the current compiler
optimizations.

\item Arglists have been made more consistent for \lsym{crop},
\lsym{paste}, \lsym{make-slice}, and \lsym{circular-shift}.  
Old keywords were left in for back-compatibility.

\item Overlays totally rewritted again (see obvius-tutorial.lisp for
examples).  Overlays is no longer an optional module.

\item New version of surface plots, that can be printed using
hardcopy.

\item Contour plots added (an optional module).

\item Fast display-sequence code added.

\item Synthetic functions:  Eliminated \lsym{make-normal-random-image}.
Changed names of some functions:
\begin{tabular}{ccc}
\lsym{make-uniform-random-image} & {\tt -->} & make-uniform-noise \\
\lsym{make-gaussian-noise-image} & {\tt -->} & make-gaussian-noise \\
\lsym{make-random-dot-image} & {\tt -->} & make-random-dots \\
\end{tabular}

\item \lsym{load-image} on image sequences now checks for a {\tt
class} key.  Useful for image-pairs or color-images.

\item \lsym{fft} now takes a {\tt :dimensions} keyword.

\item Added display for filters.

\item Default args to \lsym{point-operation} have changed.  The
default binsize is now nil (=> call the function, rather than making a
lookup-table).  This is slower, but gives no sampling surprises in the
result!

\item Added :center, :post-center, and :pre-center keyword for fft
and power-spectrum. 

\item Added \lsym{rotate} and \lsym{resample}.

\item Added \lsym{round.} \lsym{truncate.} \lsym{floor.} methods

\item The \lsym{warp} code is no longer an optional module
(it is included in the source).

\item Pane manipulation functions now exported (useful from within
programs, or if mouse dies): {\tt pop-picture}, {\tt cycle-pane}, {\tt
get-viewable}.

\item Added function \lsym{next-pane}. 

\item Got rid of {\tt destroy-top-picture} (same as {\tt pop-picture}), and
{\tt destroy-pane} which is just {\tt (destroy *current-pane*)}.  Got
rid of {\tt destroy-top-viewable}.  Provided {\tt current-viewable}
instead, which returns viewable on top of current pane.

\item Added \lsym{purge!} function.

\item Fixed {\tt setf iref} and lots of other annoying little bugs.

\item Added \lsym{periodic-discrete-function}s and \lsym{log-discrete-function}s.

\end{itemize}

\section{Known Bugs}

This appendix is a partial list of known bugs in Obvius.  Other bugs
are listed in the file {\tt \abox{obv}/TODO.soon} and {\tt
\abox{obv}/TODO}.  They will be fixed in a future release of the
system.

\begin{description}

\item {\bf 1D/2D} images have some inconsistencies.  In particular,
many of the synthetic image generating functions break on
one-d-images, and you can not setf iref a one-d-image.  We plan to
extend Obvius so that images (and all viewables) can be any rank (like
arrays).

\item {\bf Hardcopy} still needs lots of work.

\item {\bf Multiple Screens} are not yet supported.

\item {\bf 24bit Displays} are not yet supported.

\item {\bf Pyramids} do not necessarily work correctly on
one-d-images.  Qmf-pyramids do not work at all.

\item {\bf Screen color} does not update all pictures, e.g., pasteups
and graph flipbooks will have the wrong background color.

\end{description}


%%% Index: generated automatically from the LaTeX-produced file
%%% main.idx, using the lisp code in index.lisp.
\clearpage\thispagestyle{empty}\cleardoublepage
\addcontentsline{toc}{section}{Index}
\makeatletter %for fancy stuff in index entries
%%% Index file: automatically generated from main.idx
%%% using the make-index function defined in index.lisp.

\begin{theindex}

\item control panel 78
\item datfile 17
\item dialog boxes 95
\item floating-point exceptions 63
\item foreign functions 86
\subitem  and garbage collection 87, 90
\subitem  passing arrays to 87
\subitem floating-point exceptions 45, 86
\item menus 78
\subitem modifying 95
\item orphans 12, 20, 88
\item point operations 107
\item tutorial 7
\item {\ptt *.} 45
\item {\ptt *auto-bind-loaded-images*} 80
\item {\ptt *auto-destroy-orphans*} 107, 108, 12, 13, 80, 85, 89, 93
\item {\ptt *auto-display-viewables*} 11, 80, 94
\item {\ptt *auto-expand-heap*} 110, 19, 81, 87
\item {\ptt *auto-update-pictures*} 15, 80, 97
\item {\ptt *current-pane*} 15, 16
\item {\ptt *default-printer*} 39, 80
\item {\ptt *div-by-zero-result*} 45, 81
\item {\ptt *heap-growth-rate*} 110, 20, 80, 87
\item {\ptt *max-print-vals*} 23, 80
\item {\ptt *obvius-module-plist*} 109
\item {\ptt *preserve-picture-pane*} 11, 97
\item {\ptt *print-command-string*} 39
\item {\ptt *protected-viewables*} 84, 88, 89
\item {\ptt *starting-patch-file*} 111
\item {\ptt *temp-ps-directory*} 39
\item {\ptt *x-print-range*} 23, 80
\item {\ptt *y-print-range*} 23, 80
\item {\ptt +.} 40, 44
\item {\ptt -.} 44, 45
\item {\ptt .obvius-windows} 106
\item {\ptt .obvius} 106
\item {\ptt /.} 45
\item {\ptt :->} 11, 40, 84
\item {\ptt :auto}
\subitem pedestal 30
\subitem scale 30
\subitem tick-step 35
\subitem x-range 35
\subitem zoom 16
\item {\ptt :background} 107
\item {\ptt :class} 85
\item {\ptt :destroyed} 12
\item {\ptt :display-type} 108
\item {\ptt :edge-handler} 26
\item {\ptt :foreground} 107
\item {\ptt :gray-dither} 107
\item {\ptt :gray-gamma} 107
\item {\ptt :gray-shades} 107
\item {\ptt :make-new} 108, {\bf 15}, 38
\item {\ptt :name} 109
\item {\ptt :reflect1} 26
\item {\ptt :reflect2} 26
\item {\ptt :repeat} 26
\item {\ptt :scale} 107
\item {\ptt :start-vector} 26
\item {\ptt :step-vector} 26
\item {\ptt :sub-viewable-spec} 111
\item {\ptt :zero} 26
\item {\ptt :zoom} 107
\item {\ptt <.} 48
\item {\ptt <=.} 48
\item {\ptt <viewable-class>-p} {\bf 92}
\item {\ptt =.} 48
\item {\ptt >.} 47
\item {\ptt >=.} 48
\item {\ptt abs-error} {\bf 48}
\item {\ptt abs-value}
\subitem {\pit image} {\bf 45}
\item {\ptt abs.} 45
\item {\ptt access-band} 27
\subitem {\pit pyramid} {\bf 61}
\item {\ptt access}
\subitem {\pit pyramid} {\bf 61}
\item {\ptt add-cols}
\subitem {\pit matrix vector} {\bf 68}
\subitem {\pit vector matrix} {\bf 69}
\item {\ptt add-rows}
\subitem {\pit matrix vector} {\bf 68}
\subitem {\pit vector matrix} {\bf 68}
\item {\ptt add} 41, 52
\subitem {\pit bit-image} {\bf 52}
\subitem {\pit filter} {\bf 60}
\subitem {\pit image} {\bf 44}
\item {\ptt allocate-array} {\bf 87}, 90, 91
\item {\ptt almost-equal} {\bf 62}
\item {\ptt and.} 52
\item {\ptt append-cols} {\bf 70}
\item {\ptt append-rows} {\bf 70}
\item {\ptt append-sequence} {\bf 56}
\item {\ptt append.} 56
\item {\ptt apply-filter} 26, 27
\subitem {\pit image} {\bf 50}
\item {\ptt arrow-heads} 34
\item {\ptt aspect-ratio} 35
\item {\ptt axis-color} 35, 36
\item {\ptt back-and-forth} 37
\item {\ptt background} 30
\item {\ptt bit-image-p} {\bf 51}
\item {\ptt bit-image} 25
\item {\ptt bitmap} {\bf 30}
\item {\ptt blur}
\subitem {\pit image} {\bf 51}
\item {\ptt border} 32
\item {\ptt box-color} 33
\item {\ptt box-p} 33
\item {\ptt build}
\subitem {\pit pyramid} {\bf 61}
\item {\ptt change-memory-management} {\bf 19}
\item {\ptt check-size} 91, {\bf 92}
\item {\ptt circular-shift} 110, 112
\subitem {\pit image} {\bf 46}
\item {\ptt clip}
\subitem {\pit image} {\bf 46}
\item {\ptt coerce-to-float} {\bf 52}
\item {\ptt col-dim} {\bf 68}
\subitem {\pit viewable-matrix} {\bf 58}
\item {\ptt col-null-space-qr} {\bf 67}
\item {\ptt col-null-space} {\bf 67}
\item {\ptt col-space-qr} {\bf 67}
\item {\ptt col-space} {\bf 67}
\item {\ptt collapse} 27
\subitem {\pit pyramid} {\bf 61}
\item {\ptt color-image-p} {\bf 57}
\item {\ptt color-image} {\bf 27}
\item {\ptt color-picture} {\bf 31}
\item {\ptt color} 33, 34, 35, 36
\item {\ptt cols} {\bf 68}
\item {\ptt columnize} {\bf 68}
\subitem {\pit viewable-matrix} {\bf 59}
\item {\ptt col} {\bf 68}
\item {\ptt complex-conjugate} {\bf 54}
\item {\ptt complex-image-p} {\bf 53}
\item {\ptt complex-image} {\bf 25}
\item {\ptt complex-phase}
\subitem {\pit polar-image} {\bf 54}
\item {\ptt complex-to-polar} 109, {\bf 54}
\item {\ptt compute-picture} {\bf 100}
\item {\ptt condition-number} {\bf 66}
\item {\ptt constructor} 13
\item {\ptt contour-plot} 33
\item {\ptt copy} 51
\subitem {\pit bit-image} {\bf 52}
\subitem {\pit filter} {\bf 60}
\subitem {\pit image} {\bf 42}
\item {\ptt correlate} 110, {\bf 48}
\item {\ptt covariance-cols} {\bf 69}
\item {\ptt covariance-rows} {\bf 69}
\item {\ptt crop} 110, 112
\subitem {\pit bit-image} {\bf 52}
\subitem {\pit image} {\bf 47}
\item {\ptt cross-product}
\subitem {\pit vector} {\bf 64}
\item {\ptt current-viewable} {\bf 15}
\item {\ptt current} 11, 95
\item {\ptt cycle-pane} {\bf 15}, {\bf 97}
\item {\ptt dct} {\bf 50}
\item {\ptt def-simple-class} 90
\item {\ptt default-display-type} 111
\item {\ptt defclass} {\bf 98}
\item {\ptt derivative} {\bf 50}
\item {\ptt destroy-control-panel} {\bf 78}
\item {\ptt destroy} 12, 92, 97
\item {\ptt determinant-qr} {\bf 67}
\item {\ptt determinant} {\bf 65}
\subitem {\pit viewable-matrix} {\bf 59}
\item {\ptt diagonal-image-matrix} {\bf 58}
\item {\ptt diagonal-matrix} {\bf 65}
\item {\ptt diagonal-p} {\bf 65}
\item {\ptt diagonal} {\bf 65}
\item {\ptt dimensions} 51
\subitem {\pit filter} {\bf 60}
\subitem {\pit image} {\bf 41}
\subitem {\pit sequence} {\bf 56}
\subitem {\pit viewable-matrix} {\bf 58}
\item {\ptt discrete-function} 24
\item {\ptt displaced-rows} {\bf 68}
\item {\ptt displaced-row} {\bf 68}
\item {\ptt display-type} 10, 15, 38
\item {\ptt display} 11, {\bf 15}, 38, 95, {\bf 97}
\item {\ptt distribution-mean-and-variance} {\bf 55}
\item {\ptt distribution-mean} {\bf 55}
\item {\ptt distribution-variance} {\bf 55}
\item {\ptt dither} 107, {\bf 30}
\item {\ptt div} 52
\subitem {\pit color-transform} {\bf 57}
\subitem {\pit image} {\bf 45}
\subitem {\pit rgb->yiq} {\bf 57}
\subitem {\pit yiq->rgb} {\bf 57}
\item {\ptt domain} {\bf 55}
\item {\ptt dot-product}
\subitem {\pit vector} {\bf 64}
\subitem {\pit viewable-matrix} {\bf 59}
\item {\ptt downsample} 51
\subitem {\pit image} {\bf 46}
\item {\ptt drag-picture} {\bf 99}
\item {\ptt draw-line} 98
\item {\ptt draw-pane} 97
\item {\ptt draw-text} 98
\item {\ptt drawable} 98
\item {\ptt drawing} 107
\item {\ptt elements} {\bf 63}
\item {\ptt entropy} 109, {\bf 49}
\item {\ptt equal-to} 51
\subitem {\pit image} {\bf 48}
\item {\ptt evaluate} {\bf 55}
\subitem {\pit discrete-function} {\bf 25}
\item {\ptt event-dispatch-process} 94
\item {\ptt expand-filter} 26, 27
\subitem {\pit image} {\bf 50}
\item {\ptt expand-heap} {\bf 20}, {\bf 87}
\item {\ptt expt.} 45
\item {\ptt fft} 113, 50
\subitem {\pit complex-image} {\bf 54}
\subitem {\pit image} {\bf 49}
\item {\ptt fill!}
\subitem {\pit image} {\bf 44}
\item {\ptt fill-symbol-p} 35, 36
\item {\ptt filter-p} {\bf 60}
\item {\ptt filter} {\bf 26}
\item {\ptt first-image} {\bf 53}, 54
\item {\ptt flip-x} {\bf 47}, 51
\item {\ptt flip-y} {\bf 47}, 51
\item {\ptt flipbook} {\bf 37}
\item {\ptt floor.} 113
\subitem {\pit image} {\bf 46}
\item {\ptt foreground} 30
\item {\ptt frame-delay} 37
\item {\ptt frame} 93
\subitem {\pit number viewable-sequence} {\bf 56}
\item {\ptt free-array} {\bf 87}
\item {\ptt free} 87
\item {\ptt frob-coord-to-viewable-coord} {\bf 99}
\item {\ptt gamma-correct}
\subitem {\pit image} {\bf 45}
\item {\ptt gauss-in}
\subitem {\pit image} {\bf 50}
\item {\ptt gauss-out}
\subitem {\pit image} {\bf 51}
\item {\ptt gaussian-pyramid-p} {\bf 61}
\item {\ptt gaussian-pyramid} {\bf 27}
\item {\ptt get-defaults} 106
\item {\ptt get-viewable} {\bf 16}
\item {\ptt getp} {\bf 16}
\item {\ptt graph-type} 34
\item {\ptt graph} 107, 26, {\bf 34}, 98
\item {\ptt gray-dither} 30
\item {\ptt gray-lut} 111
\item {\ptt gray-shades} 30
\item {\ptt gray} 26, 29
\item {\ptt greater-than-or-equal-to}
\subitem {\pit image} {\bf 48}
\item {\ptt greater-than} 51
\subitem {\pit image} {\bf 47}
\item {\ptt hardcopy} 108, 38
\item {\ptt heap-status} {\bf 20}, {\bf 88}
\item {\ptt height}
\subitem {\pit pyramid} {\bf 61}
\item {\ptt hilbert-transform} {\bf 50}
\item {\ptt histogram} 24
\item {\ptt history-sexp} 13
\item {\ptt history} 10, 93
\item {\ptt identity-matrix} {\bf 65}
\item {\ptt identity-p} {\bf 65}
\item {\ptt image-from-array} {\bf 42}
\item {\ptt image-from-bit-image} {\bf 52}
\item {\ptt image-list}
\subitem {\pit image-sequence} {\bf 56}
\item {\ptt image-matrix-p} {\bf 58}
\item {\ptt image-matrix} 28
\item {\ptt image-pair-p} {\bf 53}
\item {\ptt image-pair} {\bf 25}
\item {\ptt image-p} {\bf 41}
\item {\ptt image-sequence-p} {\bf 56}
\item {\ptt image-sequence} {\bf 25}, 28
\item {\ptt image} 21
\item {\ptt imaginary-part} {\bf 53}
\item {\ptt independent-parameters} 32, 37
\item {\ptt inferiors-of} {\bf 92}
\item {\ptt info-get} {\bf 11}
\item {\ptt info-list} 11
\item {\ptt info-set} {\bf 11}
\item {\ptt initialize-instance} {\bf 91}, 93, 98
\item {\ptt internal-div} 86
\item {\ptt interpolate-hex-image} {\bf 60}
\item {\ptt invert}
\subitem {\pit bit-image} {\bf 52}
\item {\ptt iref} 23, 51
\subitem efficiency 23
\subitem {\pit bit-image} {\bf 52}
\subitem {\pit image} {\bf 41}
\item {\ptt kurtosis} {\bf 49}
\item {\ptt laplacian-pyramid-p} {\bf 61}
\item {\ptt left-image} {\bf 53}
\item {\ptt left-singular-matrix} {\bf 66}
\item {\ptt length.} 56
\item {\ptt less-than-or-equal-to}
\subitem {\pit image} {\bf 48}
\item {\ptt less-than}
\subitem {\pit image} {\bf 48}
\item {\ptt line-width} 33, 35, 36
\item {\ptt linear-xform}
\subitem {\pit image} {\bf 45}
\item {\ptt load-image-sequence} {\bf 19}, {\bf 44}
\item {\ptt load-image} 113, 17, {\bf 18}, {\bf 44}
\item {\ptt log-discrete-function} 113, 24
\item {\ptt loop-over-image-pixels} 22, 23, {\bf 86}
\item {\ptt loop-over-image-positions} 23, {\bf 86}
\item {\ptt low-band}
\subitem {\pit pyramid} {\bf 61}
\item {\ptt magnitude}
\subitem {\pit image-pair} {\bf 54}
\subitem {\pit polar-image} {\bf 54}
\item {\ptt make-1-over-r} {\bf 43}
\item {\ptt make-<viewable-class>} {\bf 92}
\item {\ptt make-atan} {\bf 43}
\item {\ptt make-bit-image} {\bf 52}
\item {\ptt make-color-image} {\bf 57}
\item {\ptt make-complex-image} {\bf 53}
\item {\ptt make-control-panel} {\bf 78}
\item {\ptt make-discrete-function} {\bf 24}, {\bf 54}
\item {\ptt make-disc} {\bf 43}
\item {\ptt make-filter} 107, 26, {\bf 59}
\item {\ptt make-fractal} {\bf 43}
\item {\ptt make-gaussian-noise-image} 113
\item {\ptt make-gaussian-noise} {\bf 43}
\item {\ptt make-gaussian-pyramid} 27, {\bf 61}
\item {\ptt make-grating} {\bf 42}
\item {\ptt make-histogram} 109, {\bf 55}
\item {\ptt make-image-matrix} {\bf 58}
\item {\ptt make-image-pair} {\bf 53}
\item {\ptt make-image-sequence} {\bf 56}
\item {\ptt make-image} 22, {\bf 41}, 52, 92
\item {\ptt make-impulse} {\bf 42}
\item {\ptt make-instance} 91, 92
\item {\ptt make-iterated-image} {\bf 56}
\item {\ptt make-laplacian-pyramid} {\bf 61}
\item {\ptt make-left} {\bf 43}
\item {\ptt make-log-discrete-function} {\bf 24}
\item {\ptt make-matrix} {\bf 68}
\item {\ptt make-normal-random-image} 113
\item {\ptt make-periodic-discrete-function} {\bf 24}
\item {\ptt make-pinwheel} {\bf 43}
\item {\ptt make-polar-image} {\bf 53}
\item {\ptt make-r-squared} {\bf 43}
\item {\ptt make-ramp} {\bf 42}
\item {\ptt make-random-dot-image} 113
\item {\ptt make-random-dots} {\bf 43}
\item {\ptt make-saw-tooth-grating} {\bf 42}
\item {\ptt make-scatter}
\subitem {\pit image} {\bf 36}
\item {\ptt make-separable-filter} 26, {\bf 59}
\item {\ptt make-separable-qmf-pyramid} {\bf 61}
\item {\ptt make-sin-grating} 109, {\bf 43}
\item {\ptt make-slice} 112
\subitem {\pit image-pair} {\bf 53}
\subitem {\pit image} {\bf 51}
\subitem {\pit sequence} {\bf 57}
\item {\ptt make-soft-threshold-df} {\bf 55}
\item {\ptt make-square-grating} {\bf 42}
\item {\ptt make-synthetic-image-sequence} {\bf 57}
\item {\ptt make-synthetic-image} {\bf 42}
\item {\ptt make-top} {\bf 43}
\item {\ptt make-uniform-noise} {\bf 43}
\item {\ptt make-uniform-random-image} 113
\item {\ptt make-viewable-matrix} {\bf 58}
\item {\ptt make-viewable-sequence} {\bf 55}
\item {\ptt make-zone-plate} {\bf 43}
\item {\ptt malloc} 87
\item {\ptt map.} {\bf 57}
\item {\ptt matrix-inverse}
\subitem {\pit array} {\bf 64}
\subitem {\pit viewable-matrix} {\bf 59}
\item {\ptt matrix-mul-transpose}
\subitem {\pit array} {\bf 64}
\subitem {\pit viewable-matrix} {\bf 59}
\item {\ptt matrix-mul}
\subitem {\pit array} {\bf 64}
\subitem {\pit viewable-matrix} {\bf 59}
\item {\ptt matrix-trace} {\bf 65}
\item {\ptt matrix-transpose-mul} {\bf 64}
\subitem {\pit viewable-matrix} {\bf 59}
\item {\ptt matrix-transpose}
\subitem {\pit array} {\bf 64}
\subitem {\pit viewable-matrix} {\bf 59}
\item {\ptt max-abs-error} {\bf 48}
\item {\ptt max.} 49
\item {\ptt maximum-location}
\subitem {\pit image} {\bf 49}
\item {\ptt maximum} 36
\subitem {\pit image} {\bf 49}
\item {\ptt mean-abs-error} {\bf 48}
\item {\ptt mean-cols} {\bf 69}
\item {\ptt mean-rows} {\bf 69}
\item {\ptt mean-square-error} {\bf 48}
\item {\ptt mean}
\subitem {\pit image} {\bf 49}
\item {\ptt min.} 49
\item {\ptt minimum-location}
\subitem {\pit image} {\bf 49}
\item {\ptt minimum}
\subitem {\pit image} {\bf 49}
\item {\ptt move-picture} {\bf 97}
\item {\ptt mul} 51, 52
\subitem {\pit bit-image} {\bf 52}
\subitem {\pit filter} {\bf 60}
\subitem {\pit image} {\bf 45}
\item {\ptt name} 10
\item {\ptt natural-logarithm}
\subitem {\pit image} {\bf 45}
\item {\ptt negate}
\subitem {\pit filter} {\bf 60}
\subitem {\pit image} {\bf 45}
\item {\ptt new-pane} 13
\item {\ptt next-pane} 113, {\bf 15}
\item {\ptt normalize-cols} {\bf 69}
\item {\ptt normalize-rows} {\bf 69}
\item {\ptt normalize}
\subitem {\pit vector} {\bf 64}
\subitem {\pit viewable-matrix} {\bf 59}
\item {\ptt not.} 52
\item {\ptt notify-of-inferior-destruction} {\bf 92}
\item {\ptt notify-of-superior-destruction} {\bf 93}
\item {\ptt num-levels} 33
\item {\ptt obvius-init.lisp} 106
\item {\ptt obvius-window-init.lisp} 106
\item {\ptt obvius::*obvius-misc-functions*} 78
\item {\ptt obvius::run-heap-diagnostics} 110
\item {\ptt ogc} 12, {\bf 20}, {\bf 88}, 90, 92, 93
\item {\ptt one-d-graph} 107
\item {\ptt one-d-image-p} {\bf 51}
\item {\ptt one-d-image} 24
\item {\ptt or.} 52
\item {\ptt orthogonal-p} {\bf 65}
\item {\ptt outer-product}
\subitem {\pit vector} {\bf 64}
\subitem {\pit viewable-matrix} {\bf 59}
\item {\ptt overlay} 109, {\bf 38}
\item {\ptt pane-coord-to-viewable-coord} 99
\item {\ptt pane} 11, {\bf 13}
\item {\ptt paste-cols} {\bf 70}
\item {\ptt paste-rows} {\bf 70}
\item {\ptt pasteup-format} 32
\item {\ptt pasteup} 26, {\bf 31}
\item {\ptt paste} 110, 112, 51
\subitem {\pit bit-image} {\bf 52}
\subitem {\pit image} {\bf 47}
\item {\ptt pedestal} 14, 29, 30, 31, 32
\item {\ptt periodic-discrete-function} 113, 24
\item {\ptt periodic-point-operation}
\subitem {\pit image} {\bf 46}
\item {\ptt pictures-of} 10
\item {\ptt picture} 10, 11, {\bf 13}, 7
\item {\ptt plot-symbol} 35, 36
\item {\ptt point-maximum}
\subitem {\pit image} {\bf 48}
\item {\ptt point-minimum}
\subitem {\pit image} {\bf 48}
\item {\ptt point-operation} 113, 24
\subitem {\pit image} {\bf 45}
\item {\ptt polar-image} 109, {\bf 25}
\item {\ptt polar-plot} {\bf 35}
\item {\ptt polar-to-complex} 109, 52, {\bf 54}
\item {\ptt polar-to-rect} 109
\item {\ptt pop-picture} {\bf 16}, {\bf 97}
\item {\ptt position-message} {\bf 99}
\item {\ptt postscript-bltable} 98
\item {\ptt power-spectrum}
\subitem {\pit image} {\bf 50}
\item {\ptt power}
\subitem {\pit image} {\bf 45}
\item {\ptt present} 111
\subitem {\pit picture} {\bf 100}
\subitem {\pit viewable} {\bf 93}
\item {\ptt principal-components} {\bf 66}
\item {\ptt print-object} {\bf 92}
\item {\ptt print-top-level-values} 95
\item {\ptt print-values} 23, {\bf 92}
\subitem {\pit image} {\bf 41}, {\bf 53}
\item {\ptt purge!} 113, 12
\item {\ptt qmf-pyramid-p} {\bf 61}
\item {\ptt qr-decomposition} {\bf 67}
\item {\ptt qrd} 67
\item {\ptt quadratic-decomposition} {\bf 66}
\item {\ptt quantize}
\subitem {\pit image} {\bf 46}
\item {\ptt randomize} {\bf 62}
\item {\ptt range} {\bf 55}
\subitem {\pit image} {\bf 49}
\item {\ptt rank} {\bf 41}
\item {\ptt real-part} {\bf 54}
\item {\ptt rect-to-polar} 109
\item {\ptt reduce.} {\bf 57}
\item {\ptt refresh} {\bf 15}, {\bf 97}
\item {\ptt reinitialize-instance} 98
\item {\ptt render} {\bf 100}
\item {\ptt repl} 111, 94, {\bf 95}
\item {\ptt resample} 113
\subitem {\pit image} {\bf 47}
\item {\ptt reset-picture-defaults} {\bf 98}
\item {\ptt retain-bitmap} 33, 35, 98
\item {\ptt rgb-bits} 31
\item {\ptt right-image} {\bf 53}
\item {\ptt right-singular-matrix} {\bf 66}
\item {\ptt room} {\bf 19}
\item {\ptt rotate} 113
\subitem {\pit image} {\bf 47}
\item {\ptt round.} 113
\subitem {\pit image} {\bf 46}
\item {\ptt row-dim} {\bf 68}
\subitem {\pit viewable-matrix} {\bf 58}
\item {\ptt row-null-space-qr} {\bf 67}
\item {\ptt row-null-space} {\bf 67}
\item {\ptt row-space-qr} {\bf 67}
\item {\ptt row-space} {\bf 67}
\item {\ptt rows} {\bf 68}
\item {\ptt row} {\bf 68}
\item {\ptt run-heap-diagnostics} {\bf 88}
\item {\ptt save-image} 17
\subitem {\pit image-sequence} {\bf 19}, {\bf 44}
\subitem {\pit image} {\bf 18}, {\bf 44}
\item {\ptt save-picture} {\bf 44}
\item {\ptt scalar-multiple} {\bf 62}
\item {\ptt scale} 14, 29, 30, 31, 32, 34
\item {\ptt scrounge!} 20, 88
\item {\ptt second-image} {\bf 53}, 54
\item {\ptt separable-filter-p} {\bf 60}
\item {\ptt separable-filter} {\bf 26}
\item {\ptt seq-delay} 37
\item {\ptt sequence-length} {\bf 56}
\item {\ptt set-default} 106
\item {\ptt set-history} 85, 93
\item {\ptt set-not-current} 11, 85, {\bf 92}, 95, {\bf 99}
\item {\ptt set-result} 111, 85, {\bf 91}, 92
\item {\ptt setp} 14, {\bf 16}, 29, 34, 99
\item {\ptt settable-parameters} {\bf 91}, {\bf 98}
\item {\ptt shared-initialize} 98
\item {\ptt shift-by-pi} {\bf 60}
\item {\ptt shuffle-rows} {\bf 69}
\item {\ptt shuffle} {\bf 62}
\item {\ptt side-by-side} {\bf 47}, 51
\item {\ptt similar} 51
\subitem {\pit bit-image} {\bf 52}
\subitem {\pit image} {\bf 41}
\item {\ptt singular-p} {\bf 65}
\item {\ptt singular-value-decomposition} {\bf 66}
\item {\ptt singular-values} {\bf 66}
\item {\ptt size}
\subitem {\pit viewable-matrix} {\bf 58}
\item {\ptt skew} {\bf 49}
\item {\ptt skip} 33, 34
\item {\ptt slice} 24
\item {\ptt solve-eigenvalue} {\bf 67}
\item {\ptt sort-rows} {\bf 70}
\item {\ptt sqr.} 45
\item {\ptt sqrt.} 45
\item {\ptt sqr} 45
\item {\ptt square-error} {\bf 48}
\item {\ptt square-magnitude} {\bf 54}
\item {\ptt square-p} {\bf 65}
\item {\ptt square-root}
\subitem {\pit image} {\bf 45}
\item {\ptt square}
\subitem {\pit image} {\bf 45}
\item {\ptt static-arrays-of} 90, {\bf 92}, {\bf 99}
\item {\ptt status-message} 99
\item {\ptt sub-cols}
\subitem {\pit matrix vector} {\bf 69}
\subitem {\pit vector matrix} {\bf 69}
\item {\ptt sub-display-type} 37
\item {\ptt sub-rows} {\bf 69}
\subitem {\pit matrix vector} {\bf 69}
\item {\ptt sub-sequence} {\bf 56}
\item {\ptt subsample} 51
\subitem {\pit image} {\bf 47}
\item {\ptt sub} 52
\subitem {\pit filter} {\bf 60}
\subitem {\pit image} {\bf 44}
\item {\ptt sum-cols} {\bf 69}
\item {\ptt sum-of} 60
\item {\ptt sum-rows} {\bf 69}
\item {\ptt superiors-of} 10, 93
\item {\ptt surface-plot} 32
\item {\ptt svd} 66
\item {\ptt swap-rows} {\bf 70}
\item {\ptt switch-components} {\bf 54}
\item {\ptt symbol-size} 35, 36
\item {\ptt symmetric-p} {\bf 65}
\item {\ptt system-dependent-frob} 97, 98, 99
\item {\ptt title-bar-string} {\bf 99}
\item {\ptt total-size} 51
\subitem {\pit image} {\bf 41}
\subitem {\pit viewable-matrix} {\bf 59}
\item {\ptt transpose} {\bf 47}, 51
\item {\ptt truncate.} 113
\item {\ptt unitary-p} {\bf 65}
\item {\ptt upsample} 51
\subitem {\pit image} {\bf 46}
\item {\ptt variance} {\bf 49}
\item {\ptt vector-field} 26, {\bf 34}, 98
\item {\ptt vector-length}
\subitem {\pit vector} {\bf 64}
\item {\ptt vectorize} {\bf 68}
\subitem {\pit viewable-matrix} {\bf 59}
\item {\ptt viewable-list}
\subitem {\pit viewable-sequence} {\bf 56}
\item {\ptt viewable-matrix-p} {\bf 58}
\item {\ptt viewable-matrix} 28
\item {\ptt viewable-sequence-p} {\bf 56}
\item {\ptt viewable-sequence} 111, {\bf 25}
\item {\ptt viewable} {\bf 10}, 7
\subitem auto display 11
\subitem destructive modification 11
\subitem slots of 10
\item {\ptt volume} {\bf 60}
\item {\ptt warp} 113
\subitem {\pit image} {\bf 47}
\item {\ptt with-interrupts-deferred} 90
\item {\ptt with-local-images} 108
\item {\ptt with-local-viewables} 108, {\bf 85}, 89
\item {\ptt with-locked-pane} 110, 82, 94
\item {\ptt with-result-<vbl>} 110
\item {\ptt with-result} 110, 84, 91, 92, 93, 95
\item {\ptt with-static-arrays} {\bf 87}
\item {\ptt with-static-svd} {\bf 66}
\item {\ptt with-svd} {\bf 66}
\item {\ptt x-axis} 35
\item {\ptt x-component} {\bf 53}, 93
\item {\ptt x-dim} 51
\subitem {\pit bit-image} {\bf 51}
\subitem {\pit filter} {\bf 59}
\subitem {\pit image} {\bf 41}
\subitem {\pit sequence} {\bf 56}
\subitem {\pit viewable-matrix} {\bf 58}
\item {\ptt x-label} 35
\item {\ptt x-offset} 30
\item {\ptt x-range} 35
\item {\ptt y-axis} 35
\item {\ptt y-component} {\bf 53}
\item {\ptt y-dim} 51
\subitem {\pit bit-image} {\bf 52}
\subitem {\pit filter} {\bf 60}
\subitem {\pit image} {\bf 41}
\subitem {\pit sequence} {\bf 56}
\subitem {\pit viewable-matrix} {\bf 58}
\item {\ptt y-label} 35
\item {\ptt y-offset} 30
\item {\ptt y-range} 35
\item {\ptt y-tick-step} 35
\item {\ptt z-max} 33
\item {\ptt z-min} 33
\item {\ptt z-size} 32
\item {\ptt zero!}
\subitem {\pit image} {\bf 44}
\item {\ptt zero-crossings} {\bf 48}, 51
\item {\ptt zoom-picture} {\bf 99}
\item {\ptt zoom} 30, 32, 33, 34, 35, 37
\subitem slot 16

\end{theindex}


\end{document}
