%%%%%%%%%%%%%%%%%%%%%%%%%%%%%%%%%%%%%%%%%%%%%%%%%%%%%%%%%%%%%%%%%%%%%%%%%%%%%%%%%%
\section{Data Analysis}
\label{sec:data-analysis}

OBVIUS has some basic tools for data analysis and simulation. 
These include:

\begin{description}
\item[Matrix Tools] Singular value decomposition and related
operations, and row and column operations 
for dealing with multivariate data.

\item[Statistical and Simulation Tools] 
Simple methods for sample statistics, 
probability distributions commonly used in modelling,
and routines for generating noise drawn from particular distributions.

\item[Model Fitting] Linear least squares regression and STEPIT.

\end{description}



\subsection{Matrix Tools}

Many multivariate data analysis problems can be handled by
treating data as rows or columns in matrices, 
and applying simple linear algebra techniques to these matrices.
The matrix module contains many tools for handling matrices.

The matrix module may be loaded into OBVIUS by evaluating:
\begin{verbatim}
(obv-require :matrix)
\end{verbatim}

The following is an outline of the matrix tools available in OBVIUS.

\begin{description}
\item[Matrix Algebra]
Most basic matrix algebra routines are available. For example:

\begin{verbatim}
;; Make a diagonal matrix, and take its inverse
(setq d (diagonal-matrix '(1 2)))
(setq d-1 (matrix-inverse d))

;; Make another matrix, and take the sum and the product
(setq m (make-matrix '((1 2) (3 4))))
(setq sum (add m d))
(setq product (matrix-mul m d))
\end{verbatim}

Please check Section \ref{sec:operations} for a detailed listing of
matrix algebra routines.

\item[SVD] The singular value decomposition is at the heart of many
numerical linear algebra algorithms.  Please check Section
\ref{sec:operations} for more detail on the OBVIUS svd implementation,
and related tools such as matrix inversion, subspace algorithms, other
matrix decompositions, and linear least-squares regression.

An example calculation follows.

\begin{verbatim}
;; Example: Take the SVD of a matrix
(setq matrix (make-matrix '((1 2) (3 4))))
(svd matrix)
\end{verbatim}

SVD returns the singular values of the matrix in an vector,
and the left- and right- singular matrices as multiple return values.

\item[Row and Column Operations]
OBVIUS has routines for doing simple row and column operations.
Section \ref{sec:operations} section documents these functions in
detail.  For example, you can access the rows of an example data
matrix as follows:

\begin{verbatim}
;; Create a matrix and look at its row vectors
(setq matrix (make-matrix '((1 2) (3 4) (5 6))))
(setq first-row (row 0 matrix))
(setq second-row (row 1 matrix))

;; Swap the first and second rows of the matrix
(swap-rows matrix 0 1)
(setq first-row (row 0 matrix))

;; Look at the covariance matrix of the variables in each row
(setq covariance (covariance-rows matrix))
\end{verbatim}

\end{description}


\subsection{Statistical and Simulation Tools}

The statistics module may be loaded into OBVIUS by evaluating:
\begin{verbatim}
(obv-require :statistics)
\end{verbatim}

The following is a short outline of the statistical and simulation
functions that are available.  These functions are defined in the
files {\bf gaussian-noise.lisp} and {\bf statistics.lisp}:
\begin {description}
\item[Sample statistics] Some simple methods are available for
standard statistics.  Example:
\begin{verbatim}
;; Make a couple of vectors with data
(setq matrix-1 (make-matrix '((1 2 3 8 10 4))))
(setq matrix-2 (make-matrix '((4 5 2 2 1 8))))

;; Evaluate some statistics about the mean
(setq mean (mean matrix-1))
(setq standard-error (standard-error matrix-1))

;; Variance etc.
(setq variance (variance matrix-1))
(setq covariance (covariance matrix-1 matrix-2))

;; Examine an invented chi-square statistic
(cumulative-chi-square (sum-of (square (sub matrix-1 matrix-2))) 3)
\end{verbatim}

\item[Standard Functions] Some commonly used probability distributions
and other functions useful for modeling data are provided.  These
include the Gaussian, Weibull, Poisson, Binomial, Chi-Square, F, T, and
others.

\item[Noise Generation] Several routines are useful 
for generating noise data for simulation.  For example, you can
generate an array filled with Poisson noise as follows:
\begin{verbatim}
;; Poisson noise example
(setq noise (make-array 100 :element-type 'single-float))
(setq lambda 7.0)
(dotimes (i (length noise))
  (setf (aref noise i) (float (poisson-noise lambda))))
(mean noise)
(variance noise)
\end{verbatim}
Other routines exist for Gaussian, Bernoulli, and uniform noise (look
through the files {\bf gaussian-noise.lisp} and {\bf statistics.lisp}).
\end{description}


\subsection{Curve Fitting}

Two types of curve fitting are available at present: linear least
squares regression, and Chandler's STEPIT general-purpose minimization
function.  These are best explained by example.  For more details,
please refer to section \ref{sec:operations} section, or examine the
source code directly.
\begin{description}

\item[Linear Least Squares Regression] Here is an example of how to
solve a (cooked) regression problem.
\begin{verbatim}
;; Set up a matrix with the predictor variable
(setq predictor (make-array '(10 3) :element-type 'single-float))

;; Put random noise into the matrix
(randomize predictor 1.0 :-> predictor)

;; Set the observed matrix to be a linear transformation of the
;; predictor
(setq transform (make-matrix '( (1 2 3) (4 1 2) (4 9 2))))
(setq observed (matrix-mul predictor transform))

;; Solve the regression problem: the answer should be
;; very close to the transformation matrix.
(regress observed predictor)
\end{verbatim}

\item[Nonlinear Curve Fitting with Stepit]
Chandler's STEPIT program is a fast, general purpose minimization
function. Conceptually, you give STEPIT a set of parameters to adjust,
and an error function that depends on the values of those parameters,
and STEPIT attempts to adjust the parameters so as to minimize the
error function. In detail, the required arguments to {\tt stepit-fit}
are:
\begin{description}
\item[error-function] function to be called with parameters
followed by other arguments (see below).
It is the return value of this function that stepit attempts to minimize. 
\item[initial-parameters]
starting point of parameters (a single-float vector or array).
\item[error-function-args]
list of arguments that are passed (after the parameters)
to the error function at each iteration. 
These arguments are passed to error-function one after another, not as a list.
\item[keywords] Detailed control of stepit's iterative process.
See the source code for more information.
\end{description}

A simple example follows that illustrates the use of the OBVIUS
interface to STEPIT.  You are seeking the entries of the vector that
is closest to a pre-set vector (the answer is obvious, the result
should be the same as the pre-set vector).  You set up the pre-set
vector, and an initial guess for the closest vector. Then, give STEPIT
the three required arguments:
\begin{verbatim}
;; Set up a dummy problem to be solved by stepit
(setq vector (make-matrix 1 2))
(setq initial-parameters (make-matrix 0 0))
(stepit-fit #'vector-distance initial-parameters (list vector))
\end{verbatim}
STEPIT maintains its own set of parameters, call them {\tt
stepit-parameters} that it adjusts after each call to the error
function.  At each iteration, the error function is called as follows:
\begin{verbatim}
(vector-distance stepit-parameters vector)
\end{verbatim}
STEPIT adjusts the entries of {\tt stepit-parameters} until this
function call is minimized. Obviously, STEPIT will figure out that the
best choice of {\tt stepit-parameters} is identical to {\tt vector}.

Althought this problem is not very interesting, most parameter-fitting
problems can be coded with this structure. STEPIT therefore provides a
general method for fitting models to data. Note that the length of
time required for a stepit fit grows exponentially with the number of
parameters, so if your models have more than about 5 parameters, be
prepared to wait!
\end{description}










