%%%%%%%%%%%%%%%%%%%%%%%%%%%%%%%%%%%%%%%%%%%%%%%%%%%%%%%%%%%%%%%%%%%%%%%%%%%%%%%%%%
\section{Pictures}

This section describes the various picture subclasses.  To add new
picture classes to OBVIUS, see section \ref{sec:hacking} of this
manual.  The currently defined picture classes are listed below.  Not
every picture is applicable to every viewable.  For example, {\tt
image}s may be displayed as {\tt gray} or {\tt surface-plot} pictures,
but not as {\tt graph}s.  The indentation indicates inheritance of
classes, and the viewables to which each picture type applies are
listed in parentheses.
\begin{tabbing}
gray \= (image) \\ 
\> pasteup \= (image-matrix, image-sequence, color-image, gaussian-pyramid, \\
\> \> laplacian-pyramid, image-pair, complex-image, polar-image) \\
\> dither (image) \\
color-picture (color-image) \\
bitmap (bit-image) \\
drawing \\
\> graph (image, discrete-function, histogram) \\
\> vector-field (image-pair) \\
\> surface-plot (image) \\
\> contour-plot (image) \\
\> polar-plot (discrete-function, image-pair, viewable-sequence) \\
\> scatter-plot (image-pair) \\
flipbook (viewable-sequence, image-sequence) \\
overlay (viewable-sequence) \\
\end{tabbing} 

%%%%%%%%%%%%%%%%%%%%%%%%%%%%%%%%%%%%%%%%%%%%%%%%%%%%%%%%%%%%%%%%%%%%%%%%%%%%%%%%%%

\subsection{Display}
\label{sec:display}

Pictures of a given type may be created using the {\tt display}
function, which is described in section~\ref{sec:pictures}.

\subsection{Gray Pictures}
\label{sec:gray}

The \lsym{gray} picture subclass is the usual display type for images.
Each gray picture has several parameters that control its computation
from the underlying image.  The parameter values of a {\tt gray}
picture may be changed using the {\tt Current Picture} dialog from the
{\tt Parameters} menu on the control panel.  The parameter values may
also be reset using the \lsym{setp} macro (described in section
\ref{sec:organization}).

The transformation from a (floating point) image pixel value to a gray
lookup table index is given by:
\begin{displaymath}
{\rm round}\left( \frac{x - p}{s} \right),
\end{displaymath}
where $p$ is the picture's \lsym{pedestal} parameter, $s$ is the
\lsym{scale} parameter.

The size of the gray lookup table is determined by the
\lsym{gray-shades} slot of the screen.  The gray lookup table is a
vector of indices into the screen colormap.  The indices are ordered,
from darkest to lightest.  The sequence of gray shades that they refer
to may not be linear: the {\tt gray-gamma} parameter of the screen
controls the set of gray values that are allocated.

OBVIUS will make use of the gray shades available.  The screen
parameter \lsym{gray-dither} is used to decide whether or not to
dither the gray pictures.  If its value is a number, and if there are
fewer than \lsym{gray-dither} grays in the color map, then OBVIUS will
automatically dither the picture.  For example, if there are 8 gray
shades allocated in the color map, and if \lsym{gray-dither} is
greater than 8, then OBVIUS will dither the viewable values using a
simple error-propagating dither algorithm into those 8 gray shades.
If the value of \lsym{gray-dither} is nil, then OBVIUS will never
dither.  If it is non-nil (but not a number), then OBVIUS will always
dither.  All of the screen parameters mentioned ({\tt gray-shades},
{\tt gray-dither}, {\tt gray-gamma}), may be adjusted from the {\tt
Current Screen} item on the {\tt Parameters} menu.

If the value of \lsym{scale} is set to {\tt :auto}\index{{\ptt :auto},scale},
then OBVIUS automaticaly uses the range (max minus min) of the image.
If the value of \lsym{pedestal} is set to {\tt
:auto}\index{{\ptt :auto},pedestal}, then OBVIUS automaticaly uses the
minimum of the image.  Since these are stored on the info list, they
only need be computed once.

In addition, each gray picture has \lsym{zoom}, \lsym{y-offset}, and
\lsym{x-offset} parameters (inherited from the general picture class)
that are described in Section \ref{sec:organization}.  

%%%%%%%%%%%%%%%%%%%%%%%%%%%%%%%%%%%%%%%%%%%%%%%%%%%%%%%%%%%%%%%%%%%%%%%%%%%%%%%%%%

\subsection{Dither Pictures of Images}
\label{sec:dither}

A \ldef{dither} picture is like a {\tt gray} picture, except that
OBVIUS forces it to only use two shades of gray (i.e., black and
white).  The dither class inherits from the gray class.  This is a
separate class for two reasons: 1) The user can request an image to be
displayed in 1-bit dithered format, even on a deeper screen, and 2)
the internal representation for dithers is more efficient (it uses a
1-bit deep bltable).

%%%%%%%%%%%%%%%%%%%%%%%%%%%%%%%%%%%%%%%%%%%%%%%%%%%%%%%%%%%%%%%%%%%%%%%%%%%%%%%%%%

\subsection{Bitmaps}
\label{sec:bitmap}

The \ldef{bitmap} picture subclass, defined in the file {\bf
gray.lisp}, are used for displaying {\tt bit-images}.  The parameters
of a {\tt bitmap} are:
\begin{itemize}
\item \lsym{foreground} specifies the foreground color (e.g., :red or
:green or any other color supported by LispView).
\item \lsym{background} specifies the background color (e.g., :red or
:green or any other color supported by LispView).
\end{itemize}

%%%%%%%%%%%%%%%%%%%%%%%%%%%%%%%%%%%%%%%%%%%%%%%%%%%%%%%%%%%%%%%%%%%%%%%%%%%%%%%%%%

\subsection{Color Pictures}
\label{sec:color-picture}

The \ldef{color-picture} class is defined in the file {\bf
color-picture.lisp}, and may be used for displaying {\tt
color-images}.  The {\tt color-picture} interprets the sub-images of a
{\tt color-image} as corresponding to R, G, and B electron beam gun
values.  The parameters of a color picture are:
\begin{itemize}
\item \lsym{scale} analogous to grays and pasteups.
\item \lsym{pedestal} analogous to grays and pasteups.
\end{itemize}
Example:
\begin{verbatim}
(progn
  (obv-require :color)
  (load-image "/images/clown")
  (change-class clown 'color-image))
(display clown 'color-picture)
\end{verbatim}
In this example, {\tt clown} is a datfile directory with three data
files corresponding to R, G, and B.  {\tt Load-image} loads it as an
{\tt image-sequence} by default, so we use {\tt change-class} to
convert it to a {\tt color-image}.  Then it can be displayed as a {\tt
color-picture}.

Unfortunately, LispView (and consequently OBVIUS) does not (yet)
support 24bit displays, so {\tt color-pictures} are dithered.  The
parameter \lsym{rgb-bits} specifies how many bits to use for dithering
each band.  This parameter may be set in the {\tt Current Screen}
dialog from the {\tt Parameters} button in the control panel.  

%%%%%%%%%%%%%%%%%%%%%%%%%%%%%%%%%%%%%%%%%%%%%%%%%%%%%%%%%%%%%%%%%%%%%%%%%%%%%%%%%%

\subsection{Pasteups}
\label{sec:pasteup}

The \ldef{pasteup} picture subclass are used for displaying {\tt
image-matrices}, {\tt image-sequences}, {\tt color-images}, {\tt
gaussian-pyramids}, {\tt laplacian-pyramids}, {\tt image-pairs}, {\tt
complex-images}, and {\tt polar-images}.  {\tt Pasteups} are defined
in the file {\bf pasteup.lisp}.

A \lsym{pasteup} is a way of laying out (pasting up) a number of
pictures so that they can all be seen simultaneously.  The {\tt
pasteup} class inherits from the {\tt gray} class, and thus these
objects have all of the parameters of {\tt grays}.  In addition, there
is an {\tt independent-parameters} slot that forces all of the
individual sub-pictures to have indentical rescaling parameters.  If
{\tt independent-parameters} is non-nil, then the images that make up
the {\tt image-sequence} are rescaled independently.  Otherwise, all
of the images are rescaled with the same {\tt scale} and {\tt
pedestal}.

The parameter values of a {\tt pasteup} picture may be changed using
the {\tt Current Picture} dialog from the {\tt Parameters} menu, or
using the {\tt setp} macro.  The default values (the values used to
initialize a new picture) for these parameters may be changed using
the {\tt Picture Defaults} menu.  The parameters are:
\begin{itemize}
\item \lsym{independent-parameters}.  If non-nil, then the images are
rescaled independently.

\item \lsym{scale} is ignored if {\tt independent-parameters} is
non-nil.  {\tt Scale} is the same as for gray pictures of images, if
{\tt independent-parameters} is nil.  That is, if {\tt
independent-parameters} is nil and {\tt scale} is not a number, then
OBVIUS automatically uses:
\begin{verbatim}
(range image)
\end{verbatim}

\item \lsym{pedestal} is ignored if {\tt independent-parameters} is
non-nil.  {\tt Scale} is the same as for gray pictures of images, if
{\tt independent-parameters} is nil.  That is, if {\tt
independent-parameters} is nil and {\tt scale} is not a number, then
OBVIUS automatically uses:
\begin{verbatim}
(minimum image)
\end{verbatim}

\item \lsym{border} specifies the spacing between pictures in the
pasteup.

\item \lsym{pasteup-format} may be either {\tt :horizontal}, {\tt
:vertical}, or {\tt :square}, specifying the layout of the
sub-pictures.

\item \lsym{zoom} see Section \ref{sec:pictures}.
\end{itemize}

%%%%%%%%%%%%%%%%%%%%%%%%%%%%%%%%%%%%%%%%%%%%%%%%%%%%%%%%%%%%%%%%%%%%%%%%%%%%%%%%%%

\subsection{Surface Plots}
\label{sec:surface-plot}

The \lsym{surface-plot}  pictures are perspective wire-frame views
of image intensity data (i.e, they are plots of image intensity as a
function of image position).  They are defined in the file {\bf
drawing.lisp}, and the code for creating them is in {\bf
surface-plot.lisp}.

Surface plots have a number of parameters that control the
transformation from the image pixel (world) coordinates to the
display.  The parameter values of a {\tt surface-plot} picture may be
changed using the {\tt Current Picture} dialog from the {\tt
Parameters} menu, or using the {\tt setp} macro.  The default values
(the values used to initialize a new picture) for these parameters may
be changed using the {\tt Picture Defaults} menu.  The parameters are:
\begin{itemize}

\item {\tt Theta} and {\tt phi} specify the orientation of the viewer
relative to the center pixel of the image data.  The viewer always
points toward the center of the intensity surface.  {\tt theta} is the
angle (counter-clockwise) from the y-axis.  {\tt phi} is the elevation
(aximuthal angle) above the x-y plane.

\item \lsym{z-size} determines the height of the surface relative to
its base.  The height (in world coordinates) will be the product of
{\tt z-size} and the average of the image dimensions.

\item {\tt r} specifies the distance of the camera from the center of
the image in world coordinates.  It also controls the perspective
distortion, since OBVIUS attempts to maintain an image-plane size
consistent with the {\tt zoom} parameter.  Thus, {\tt focal-length = r
* zoom}.  If the value is not a number, then OBVIUS automatically
uses:
\[
2\sqrt{\left(\frac{x}{2}\right)^2 + \left(\frac{y}{2}\right)^2 + 
		\left(\frac{z}{2}\right)^2},
\]
where $x$ and $y$ are the image dimensions, and $z$ is the product of
the {\tt z-size} parameter and the average of the image dimensions.
This is twice the minimal safe distance: the camera will be far enough
away not to hit the surface.

\item {\tt Zoom} determines the approximate size of the surface plot (see Section
\ref{sec:organization}.  It is not exact due to perspective distortion.

\item {\tt X-step} and {\tt y-step} specify the sampling rate for the
wireframe plot in x- and y- directions respectively.  If the value is
not a positive integer, then automatically uses:
\begin{verbatim}
(round (/ (x-dim image) 20))
(round (/ (y-dim image) 20))
\end{verbatim}

\item \lsym{box-p} (either {\tt T} or {\tt nil}) specify whether or
not to draw a 3D box around the plot.

\lsym{box-color} the color to use for the box (e.g., :red or :green or
any other color supported by LispView).

\item {\tt retain-bitmap} (either {\tt T} or {\tt nil}) specifies
whether to draw the surface plot to an off-screen bitmap, or redraw it
each time it is displayed.
\end{itemize}

%%%%%%%%%%%%%%%%%%%%%%%%%%%%%%%%%%%%%%%%%%%%%%%%%%%%%%%%%%%%%%%%%%%%%%%%%%%%%%%%%%

\subsection{Contour Plots}
\label{sec:contour-plot}

The \lsym{contour-plot} pictures are defined in the file {\bf
contour.lisp}.  The code works by constructing 3D triangles from
sampling the image in a grid.  Then for each z-value, it constructs a
curve containing segments each of which represents the intersection of
the constant z-value plane with a particular triangle.  So in the end,
the contour contains a list of curves.  This is a lot of computation,
so you should be careful to keep the {\tt num-levels} parameter
relatively small and the {\tt skip} parameter relatively large.

The parameters for contour-plots are:
\begin{itemize}
\item \lsym{z-min} and \lsym{z-max}: are Z-values for the smallest and
largest contour levels.

\item \lsym{num-levels}: Total number of contour levels.

\item \lsym{skip}: The subsampling factor.  If not a
number, the sampling will be chosen automatically.

\item \lsym{retain-bitmap}: (either {\tt T} or {\tt nil}) specifies
whether to keep a copy of the bitmap, or redraw it each time it is
displayed.

\item \lsym{zoom}: determines the size of the graph.  A zoom factor of
1 indicates that the samples of the underlying array should be plotted
one pixel apart in the x-direction.  The height of the graph is
determined relative to the width using the {\tt aspect-ratio} slot.

\item \lsym{line-width} specifies the width of the graph lines (bars)
drawn, in pixels.

\item \lsym{color} specifies the color of the vectors (e.g., :red,
:green, or any other color supported by LispView).
\end{itemize}


%%%%%%%%%%%%%%%%%%%%%%%%%%%%%%%%%%%%%%%%%%%%%%%%%%%%%%%%%%%%%%%%%%%%%%%%%%%%%%%%%%

\subsection{Vector Fields}
\label{sec:vector-field}

\ldef{vector-field} pictures, defined in the file {\bf
drawing.lisp}, may be used to display {\tt image-pairs}.  The first
image in the pair is interpreted as the y-component and the second is
interpreted as the x-component.

The parameter values of a {\tt vector-field} picture may be changed
using the {\tt Current Picture} dialog from the {\tt Parameters} menu,
or using the {\tt setp} macro.  The default values (the values used to
initialize a new picture) for these parameters may be changed using
the {\tt Picture Defaults} menu.  The parameters are:
\begin{itemize}
\item \lsym{skip} determines the subsampling factor.  If not a number, the
sampling will be chosen automatically to give a reasonable vector
density for the given vector-field size.  The default density can be set
from the {\tt Picture Parameters} menu.

\item \lsym{zoom} determines the overall size of the vector-field
picture, relative to the size of the underlying viewable.  If nil,
then the vector field will be the size of the underlying viewable
(i.e., they would line up correctly if the vector field were superposed
on the image).  If t, the zoom will be determined to make the overall
vector field be the size of the pane.  If a dimension list, this will
be the size of the vector field.  

\item \lsym{scale} determines the length of the vectors.  The vector
magnitude will be divided (as for gray) by the scale and the resulting
number will specify the length of the vector, in units corresponding
to the distance between the vector bases.

\item \lsym{arrow-heads} If a number, draws vectors with arrow heads of
that length.  If not a nubmer, draws oriented line segments without
arrow heads (e.g., for displaying orientation fields rather than flow
fields).  With arrow heads, the tail of the vector will correspond to
the appropriate pixel location in the image-pair.  Without arrow
heads, the {\em middle} of the line will correspond to the appropriate
pixel location in the image-pair.

\item \lsym{color} specifies the color of the vectors (e.g., :red,
:green, or any other color supported by LispView).
\end{itemize}


%%%%%%%%%%%%%%%%%%%%%%%%%%%%%%%%%%%%%%%%%%%%%%%%%%%%%%%%%%%%%%%%%%%%%%%%%%%%%%%%%%

\subsection{Graph Pictures}
\label{sec:graph}

The \ldef{graph} pictures are defined in the files {\bf drawing.lisp} and
{\bf graph.lisp}.  The parameter values of a \lsym{graph} picture may
be changed using the {\tt Current Picture} dialog from the {\tt
Parameters} menu, or using the \lsym{setp} macro.  The default values
(the values used to initialize a new picture) for these parameters may
be changed using the {\tt Picture Defaults} menu.  The parameters are:
\begin{itemize}
\item \lsym{graph-type} may be either {\tt :line}, {\tt :point}, 
or {\tt :bar}.  For most viewables, the automatic choice is {\tt
:line}.  For {\tt histograms}, the automatic choice is {\tt :bar}.

\item \lsym{x-range} is a list of length 2 specifying the range of
values covered by the x-axis.  If this is {\tt
:auto}\index{{\ptt :auto},x-range}, then OBVIUS automatically uses {\tt (list
0 (total-size image))}.

\item \lsym{y-range} is a list of length 2 specifying the range of
the y-axis.If this is {\tt :auto}, then OBVIUS
automatically uses {\tt (list (minimum image) (maximum image))}.

\item \lsym{y-axis} is the x-coordinate position of the y-axis.  If
this is nil, no y-axis is drawn.

\item \lsym{x-axis} is the y-coordinate position of the x-axis.  If
this is nil, no x-axis is drawn.

\item \lsym{y-tick-step} and  \lsym{y-tick-step} are the distances 
(in graph coordinates) between tick marks on the y- and x-axes,
respectively.  If set to {\tt :auto}\index{{\ptt :auto},tick-step}, OBVIUS
will choose this to display between 4 and 10 ticks on the axis.

\item \lsym{y-label} \lsym{x-label} [currently not used].

\item \lsym{aspect-ratio} determines the physical (i.e., in screen
pixel coordinates) ratio of height to width for the graphing region.

\item \lsym{retain-bitmap} (either {\tt T} or {\tt nil}) specifies
whether to keep a copy of the bitmap, or redraw it each time it is
displayed.

\item \lsym{zoom} determines the size of the graph.  A zoom factor of
1 indicates that the samples of the underlying array should be plotted
one pixel apart in the x-direction.  The height of the graph is
determined relative to the width using the {\tt aspect-ratio} slot.

\item \lsym{line-width} specifies the width of the graph lines (bars)
drawn, in pixels.

\item \lsym{axis-color} specifies color of the axes (e.g., :red,
:green, or any number of other colors defined by LispView).

\item \lsym{color} color of the graph and plot-symbols (e.g., :red,
:green,or any number of other colors defined by LispView).

\item \lsym{plot-symbol} Either {\tt :circle}, {\tt :square}, or {\tt
nil} (for no plot symbol).

\item \lsym{symbol-size} specifies size of plot symbol, must be
integer between 1 and 10.

\item \lsym{fill-symbol-p} specifies whether or not plot symbol is
filled, must be {\tt T} or {\tt nil}.
\end{itemize}

%%%%%%%%%%%%%%%%%%%%%%%%%%%%%%%%%%%%%%%%%%%%%%%%%%%%%%%%%%%%%%%%%%%%%%%%%%%%%%%%%%

\subsection{Polar Plots}
\label{sec:polar-plots}

The \ldef{polar-plot} pictures, defined in the file {\bf drawing.lisp},
may be used to display {\tt discrete-functions}.  They may also be
used to display {\tt image-pairs} or {\tt viewable-sequences} made up
of pairs of discrete-functions.  

For a single {\tt discrete-function}, the angular coordinate of the
plot is specified (in radians) by the {\tt origin} and {\tt increment}
of the {\tt discrete-function}.  The radial coordinate is specified by
the {\tt discrete-functions}'s data array.  An example is given in the
OBVIUS tutorial.

For a pair (either {\tt image-pair} or {\tt viewable-sequence} pair),
the first viewable of the pair is interpreted as the magnitude (radial
component) and the second is interpreted as the angle (in radians).
Example: 
\begin{verbatim}
(display (make-viewable-sequence
	  (list (make-discrete-function '(lambda (x) (* 100.0 (cos x))) 
                                        0.0 2-pi :size 63)
		(make-discrete-function '(lambda (x) (* 2.0 x)) 
                                        0.0 2-pi :size 63)))
	 'polar-plot)
\end{verbatim}

The parameters of a {\tt polar-plot} are:
\begin{itemize}
\item \lsym{maximum} The range of values plotted radially are between
0 and maximum.

\item \lsym{line-width} specifies the width of the graph lines (bars)
drawn, in pixels.

\item \lsym{axis-color} *** not implemented yet

\item \lsym{color} color of the graph and plot-symbols (e.g., :red,
:green,or any number of other colors defined by LispView).

\item \lsym{plot-symbol} Either {\tt :circle}, {\tt :square}, or {\tt
nil} (for no plot symbol).

\item \lsym{symbol-size} specifies size of plot symbol, must be
integer between 1 and 10.

\item \lsym{fill-symbol-p} specifies whether or not plot symbol is
filled, must be {\tt T} or {\tt nil}.
\end{itemize}


%%%%%%%%%%%%%%%%%%%%%%%%%%%%%%%%%%%%%%%%%%%%%%%%%%%%%%%%%%%%%%%%%%%%%%%%%%%%%%%%%%

\subsection{Scatter Plots}
\label{sec:scatter-plot}

The {\tt scatter-plot} pictures, defined in the file {\bf
drawing.lisp}, may be used to display data that is stored in an {\tt
image-pair}.  The parameters of a {\tt scatter-plot} are the same as
those of a graph.  An example of displaying a scatter-plot is given in
the OBVIUS tutorial.  The {\tt scatter-plot} method may be also used
as a shorthand for generating {\tt scatter-plots}:
\begin{description}
\item\lmeth{make-scatter}{image}{thing1 thing2 \&rest initargs}
Thing1 and thing2 may be images, vectors, or lists.  Makes a
scatter-frob a simple viewable that displays itself as a scatter plot.
\end{description}


%%%%%%%%%%%%%%%%%%%%%%%%%%%%%%%%%%%%%%%%%%%%%%%%%%%%%%%%%%%%%%%%%%%%%%%%%%%%%%%%%%

\subsection{Flipbooks}
\label{sec:flipbook}

The \ldef{flipbook}s are the standard display type for {\tt
viewable-sequences} and {\tt image-sequences}.  {\tt Flipbooks} are
defined in the file {\bf flipbook.lisp}.  {\tt Flipbooks} can be
thought of as a list of sub-pictures.  For example, {\tt flipbooks} of
{\tt image-sequences} contain a list of {\tt gray} pictures.

Mouse bindings are provided to cycle through the subpictures in the
flipbook (see section~\ref{sec:mouse}).  {\tt Flipbook} pictures have
a number of parameters that control the delay between frames when the
pictures are played back.  There is also a way to change the scaling
parameters of the {\tt gray} pictures that make up the {\tt flipbook}.

The parameter values of a {\tt flipbook} picture may be changed using
the {\tt Current Picture} dialog from the {\tt Parameters} menu, or
using the {\tt setp} macro.  The default values (the values used to
initialize a new picture) for these parameters may be changed using
the {\tt Picture Defaults} menu.  The parameters are:
\begin{itemize}
\item \lsym{frame-delay}.  Delay between frames (in approximately
1/60th secs).

\item \lsym{seq-delay}.  Delay between each repeat (in approximately
1/60th secs).

\item \lsym{back-and-forth}.  When t, the flip-book is shown forward
and back during each repeat.

\item \lsym{independent-parameters}.  If non-nil, then the images are
rescaled independently (see above description of {\tt pasteup}
pictures).

\item \lsym{zoom}. Described in section~\ref{sec:pictures}.

\item \lsym{sub-display-type} specifies the display types for the
pictures that make up the {\tt flipbook}.  If T, then OBVIUS chooses
the display type automatically.  For {\tt flipbooks} of {\tt
image-sequences}, the automatic choice is for the sub-pictures to be
{\tt grays}.
\end{itemize}

The {\tt setp} macro may be used to set the parameters of the
sub-pictures that make up a {\tt flipbook}.  For example, 
\begin{verbatim}
(setp scale 100.0)
\end{verbatim}
If {\tt independent-parameters} is non-nil, then this will affect only
the sub-picture that is currently being displayed.  If {\tt
independent-parameters} is nil, then this will change the scale for
all of the sub-pictues.

The parameters for the individual sub-pictures may be set
independently if {\tt independent-parameters} is non-nil.  The
parameters may be set using the {\tt Picture Parameters} dialog in
conjunction with the {\tt previous-frame} (C-M-left) and {\tt
next-frame} (C-M-middle) mouse bindings.  Bring up the current picture
dialog (e.g., from the {\tt Parameters} button on the control panel),
and change the picture parameters of the current sub-picture.  Use the
mouse bindings to change the current sub-picture.  The dialog box is
automatically updated to give the parameter settings for the new
sub-picture.  Then go ahead and change the parameters of the new
sub-picture.

%%%%%%%%%%%%%%%%%%%%%%%%%%%%%%%%%%%%%%%%%%%%%%%%%%%%%%%%%%%%%%%%%%%%%%%%%%%%%%%%%%

\subsection{Overlays}
\label{sec:overlays}

Some {\tt viewable-sequences} can be displayed as \ldef{overlay}
pictures.  The {\tt overlay} class is defined in the file {\bf
overlay.lisp}.  Currently, overlays are supported only for overlaying:
(1) (1) {\tt graphs} of {\tt discrete-functions}, (2) {\tt graphs} of
{\tt one-d-images}, (3) {\tt bitmaps} of {\tt bit-images}, and (4) for
overlaying {\tt bitmaps} of {\tt bit-images} on top of a {\tt gray} of
an {\tt image}.  Examples are given in the OBVIUS tutorial showing how
to display {\tt overlays}.  For more examples, look in the {\bf
overlay.lisp} file.

The parameters for the individual sub-pictures in the overlay may be
set independently using the {\tt Picture Parameters} dialog in
conjunction with the {\tt previous-frame} (C-M-left) and {\tt
next-frame} (C-M-middle) mouse bindings.  Bring up the current picture
dialog (e.g., from the {\tt Parameters} button on the control panel),
and change the picture parameters of the current sub-picture.  Use the
mouse bindings to change the current sub-picture.  The dialog box is
automatically updated to give the parameter settings for the new
sub-picture.  Then go ahead and change the parameters of the new
sub-picture.

%%%%%%%%%%%%%%%%%%%%%%%%%%%%%%%%%%%%%%%%%%%%%%%%%%%%%%%%%%%%%%%%%%%%%%%%%%%%%%%%%%

\subsection{Hardcopy}
\label{sec:hardcopy}

The function \lsym{hardcopy} produces postscript output of OBVIUS
pictures:
\begin{verbatim}
(hardcopy viewable
          &optional display-type
          &key path printer
          x-location y-location x-scale y-scale
          invert-flag suppress-text make-new
	  &allow-other-keys)
\end{verbatim}
\noindent
Note that the syntax is quite similar to that of  the \lsym{display}
function described in section~\ref{sec:display}.
Produces a postscript version of the viewable.  The parameters for
{\tt hardcopy} are similar to those of \lsym{display}: any additional
keywords are used to specify initargs of the picture.  The optional
parameter
\lsym{display-type} specifies the picture class for the postscript
picture.  The function searches for a picture of that display-type in
one of the panes (starting with the top of the current pane).  If one
is found, then it is copied to make the postscript picture.  If none
is found, then it makes a new picture using default parameters.
Additional keywords may be passed to the hardcopy function (similar to
the display function) to override the parameters of the on-screen
picture.  By default, hardcopy passes the keyword-value pair (:zoom
:auto) so that the resulting picture will fill the page. The argument
\lsym{:make-new} will force OBVIUS to make a new picture using
default-parameters.

The keyword argument {\tt :path} specifies a file pathname.  If this
parameter is nil (the default), OBVIUS writes the postscript to a
temporary file, ships it to a printer, and then deletes the temporary
file.  The arguments {\tt :x-location} and :y-location determine the
location of the picture on the page (default is top-left corner).  The
arguments {\tt :x-scale} and {\tt :y-scale} specify by how much to
rescale the size of the the picture (default is 1).  Hardcopy prints
the name of the viewable on the page just below where the picture
appears (unless the argument {\tt :suppress-text} is non-nil, in
which case nothing is printed).  The keyword {\tt :printer} specifies
which printer to use (default is the global {\tt *default-printer*}).

The mouse binding, C-M-Sh-right, calls hardcopy to print a picture to
the default printer.

The following global variables must be set for {\tt hardcopy} to work
properly:
\begin{itemize}
\item \lsym{*temp-ps-directory*} -- Directory to hold temporary
postscript files before they are sent to the printer.  Typical value
is something like: {\tt "/tmp/"}.

\item \lsym{*default-printer*} -- Name of postscript printer, to be
inserted into the {\tt *print-command-string*} (e.g., {\tt "laserwriter1"}).

\item \lsym{*print-command-string*} -- Command string for
sending a postscript file to a printer.  This is used as a format
string (i.e., an argument to the Common Lisp {\tt format} function)
with two arguments: the filename and the printer name.  That is, this
string will be used in the following fashion:
\begin{verbatim}
(format nil *print-command-string* <file> <printer>)
\end{verbatim}
where {\tt <printer>} will default to the value of {\tt
*default-printer*}.  On our system, the value of {\tt
*print-command-string*} is {\tt "cat ~A | lpr -P~A ; rm ~@*~A"}.  The
{\tt ~@*~A} construct allows us to process the filename twice (see
Steele, Common Lisp: The Language).  The result of this call to {\tt
format} will then be run as a shell program.  Thus, the our example
string will cat the first argument (filename) to the {\tt lpr} command
with the appropriate printer argument (second argument), and then call
{\tt rm} on the first argument (deleting the temporary postscript
file).
\end{itemize}

